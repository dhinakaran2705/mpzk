\section{Relevant prior art: an overview} 
We will present our $\DPZK$ protocol \dpname{} by first explaining the single prover version of our protocol. 
\dnote{adapt this part based on the summary of the DPZK derived from the existing single prover versions}

Before that, we present an overview of the techniques underlying our interactive protocol
with succinct argument size and efficient verification with sublinear public
key operations and sublinear field operations in expectation. In our
presentation of the core ideas, we first give an informal overview of the
existing techniques, and then discuss our specific modifications to them.

Our starting point is the Interactive PCP argument presented in Ligero\cite{Ligero2017}. The
aforementioned protocol uses elementary techniques, is concretely
efficient being devoid of public key operations, and achieves an argument
size which is square-root in size of the arithmetic circuit. Our core technical
contribution is to use homomorphic commitments and inner product arguments to 
further reduce the size of the ``queries'' to the witness oracle. The
reduced access to the witness oracle comes with another surprising benefit: The
verifier restricts its computations to those directly relevant to the parts of
witness being revealed. This allows us to achieve sublinear verification on
average. By carefully restricting the sub-protocols using expensive public key
operations to asymptotically smaller circuits (much smaller for practically
relevant instantiations), we keep the number of public key operations for the
verifier to be strictly sublinear.
    
\subsection{Arguments for Arithmetic Circuits}
Our protocol natively supports showing the satisfiability of the $\bbF$-arithmetic circuit for a finite field $\bbF$ which
is an $\npol$-complete language. For ease of presentation, we produce an argument for the statement ``$\exists \wit, \text{ s.t } \C(\wit)=1$'' for an $\bbF$-arithmetic circuit $\C$. This can be easily modified to yield arguments for the more typical $\npol$ language $\calL_C = \{\stmt :\exists \wit \text{ s.t. } \C(\stmt,\wit)=1\}$, which we will briefly discuss later. The satisfiability of an arithmetic circuit $\C$ reduces to proving existence of vectors $x,y,z,\wit$ such that:
$z = x \circ y$,
$x = A \wit$,
$y = B \wit$,
$z = C \wit$ and
$P_{add} \wit = 0$ for public matrices $A,B,C$ and $P_{add}$ (depending on $\C$). Thus, broadly in the IPCP setting we have the following protocol:
\begin{enumerate}
\item {\bf Oracle Setup}: The prover sets up witness oracle for $x,y,z,\wit$ and provides query access to the verifier. 
\item {\bf Linear Checks}: The verifier runs a subprotocol with the prover to verify the linear constraints $x=A \wit$, $y=B \wit$, $z=C \wit$, $P_{add} \wit = 0$.
\item {\bf Quadratic Check}: The verifier runs a subprotocol with the prover to verify $z=x\circ y$.
\end{enumerate}

\subsection{Interactive Protocols for RS-code Oracles}
\subsubsection{Reed Solomon Witness Encoding}
The linear and quadratic constraints discussed above do not naively admit a sublinear query, i.e, the verifier needs to access complete vectors to be convinced with high probability. Error correcting codes have been used to encode the witness in PCP constructions to enable verification with sublinear query. We discuss two such encodings based on Reed-Solomon codes which have been used in recent constructions \cite{Ligero2017}, \cite{Aurora}, \cite{STARK2019}.

To encode a vector $x\in \bbF^N$, one specifies two domains $G,H\subseteq \bbF$. We will call $G$ as {\em interpolation} domain and $H$ as {\em evaluation} domain. The encoding in \cite{Aurora} encodes the vector $x$ as a single Reed-Solomon codeword. This is done by first constructing a polynomial $\hat{f}^x$ which interpolates the vector $x$ on $G$, and then computing its evaluations $\langle \hat{f}^x(\eta) \rangle_{\alpha\in H}$ on points in $H$. The sizes of domains $G$ and $H$ need to be $\Omega(N)$ in the above encoding. For a vector $x$, we will use the notation $\hat{f}^x$ to denote the polynomial interpolating $x$ on $G$, and $f^x$ to denote the vector of evaluations of $\hat{f}^x$ on $H$. Thus in the above scheme, $f^x$ is an encoding of the vector $x$.


An alternative encoding used in \cite{Ligero2017} encodes parts of a vector separately, and thus the encoded vector corresponds to a set of Reed-Solomon codewords, or a single codeword of an Interleaved Reed-Solomon code(see Definition \ref{defn:interleaved}). More specifically, one chooses integers $m$ and $\ell$ such that $m\ell > N$ and domains $G$ and $H$ of size $\Omega(\ell)$. The vector $x\in \bbF^N$ is written as
$x=(x_1|\cdots|x_m)$ where $x_i\in \bbF^l$ for all $i\in [m]$. The vector $x$ is encoded as $(f^x_1,\ldots,f^x_m)$ where $f^x_i$ encodes $x_i$ as described before, i.e $f^x_i=\langle \hat{f}^x_i(\eta)\rangle_{\alpha\in H}$ where the polynomial $\hat{f}^x_i$ interpolates the vector $x_i$ on $G$. 

\begin{comment}
\subsubsection{Encoding:}\label{subsec: encode} 
If $|x|=ml$ then read $x$ as 
$$x=
\begin{bmatrix}
x_{11} & x_{12} & \ldots & x_{1l}\\
x_{21} & x_{22} & \ldots & x_{2l}\\
& \vdots\\
x_{m1} & x_{m2} & \ldots & x_{ml}
\end{bmatrix}
$$	
Construct polynomials $\hat{f}^x_i(\cdot)$ of deg $k$ such that $\hat{f}^x_i(\zeta_j)=x_{ij}$ $\forall i\in [m], j\in [l]$ where $k>l$ and $l+t=k$.

Define 
$$ \oracle =
\begin{bmatrix}
u_{11} & u_{12} & \ldots & u_{1n}\\
u_{21} & u_{22} & \ldots & u_{2n}\\
& \vdots\\
u_{m1} & u_{m2} & \ldots & u_{mn}
\end{bmatrix}
$$
where $u_{ij}= \hat{f}^x_i(\eta_j)$ $\forall i\in[m], j\in[n]$ $n>k$. $\bm{\zeta}=\{\zeta_1,\ldots,\zeta_l\}$ we will call it interpolation domain and $\bm{\eta} = \{\eta_1,\ldots,\eta_n\}$ we will call it evaluation domain. 

Let $L$ be the set of codewords and the above linear code has distance $d$. Then a correctly computed $\oracle_x$ is in $L^m$.

\subsubsection{Commitment:}\label{subsec: commit} Then compute the commitment of $\oracle_x$. 
\pnote{Elaborate how to do the commitment.}
Let $\cm_x=(c_1,\ldots c_n)$ where 
$c_j= \com( \begin{bmatrix}
u_{1j} & u_{2j} & \ldots & u_{mj}
\end{bmatrix}^T)$ $\forall j\in [n]$

\paragraph{\textbf{Proximity Check:}}\label{sec:proximity} In this check $\prover$ convinces $\verifier$ that $\cm_x$ is the commitment of a correct codeword matrix $\oracle_x$ which is encoding of a matrix $x$. If $\oracle_x$ is malformed, i.e. if it is sufficiently far away from a correct codeword matrix, then $\prover$ passes the following check with $\negl(t)$ probability.
\begin{enumerate}
	\item $\prover \rightarrow \verifier :$ $\prover$ sends $cm$ to $\verifier$.
	
	\item $\verifier \rightarrow \prover :$ $\verifier$ as a challenge picks $\gamma \in \bbF^m$ uniformly at random and sends it to $\prover$.
	
	\item $\prover \rightarrow \verifier :$ $\prover$ computes $u=\gamma^T\oracle_x$ and sends $u$ to $\verifier$.
	
	%\pnote{ instead of using $w = \gamma^T \oracle_{x}$, using $u$, to avoid conflict with witness.}
	\item $\verifier \rightarrow \prover :$ $\verifier$ picks a random subset $Q\subseteq [n]$ such that $|Q|=t$ and sends $Q$ to $\prover$.
	
	\item $\prover \leftrightarrow \verifier: $ $\prover$ and $\verifier$ run a subprotocol to prove the innerproduct arguement for the following statement $\innp{\gamma}{\oracle_x[\cdot,j_u]}=u_{j_u}$ $\forall j_u\in Q$
	
	\item If $\verifier$ accepts the innerproduct arguement in step 5, then checks if $w\in L$. If yes then $\verifier$ outputs accept else outputs reject.
\end{enumerate}

Note that the subprotocol for innerproduct arguement has the proof of knowledge property, proof given in \cite{Bulletproofs}. So there is an expected $\ppt$ extractor $\extrac_{innp}$ for the innerproduct argument, which either outputs the witness or breaks the binding property of the committment, if the arguement is accepted.
\begin{theorem}\label{theo:proximity}
	For $e < \frac{d}{3} $, if $\innp{\prover^*(\cm, \calU^*)}{\verifier(\cm)} $ is true, then there is an expected $\ppt$ $\extrac^{\prover^*}(\cm) \rightarrow \calU^*$ such that with overwhelming probability $\calU^*$ satisfies one of the following events: 
	\begin{itemize}
		\item $\com(\calU^*) = \cm$ and $ d(\calU^*, L^m) < e$
		\item $\extrac$ breaks the binding of the commitment. 
	\end{itemize} 
	
\end{theorem}

\begin{proof}
	Now we will design a $\ppt$ $\extrac$ for the above protocol.
	In step 4 $\extrac$ emulates $\verifier$ and picks $Q$ uniformly at random, then if innerproduct arguement is accepted then run $\extrac_{innp}$, in polytime with overwhelming probability. $\extrac_{innp}$ outputs either $\calU^*[\cdot,j_u]$ $\forall j_u\in Q$ or breaks the binding property of the commitment scheme, and stores the indices of $Q$ in a set $S$. then rewinds to the step 4 and then again picks $Q$ uniformly at random, $\extrac$ keeps extracting $\calU^*[\cdot,j_u]$ and updates $S$ by including only the new indices repeating until $|S|=n$. That is $\extrac$ extracts the whole $\calU^*$.
	
	Now $\extrac$ checks computes $d(\calU^*, L^m)$. Probability that $d(\calU^*,L^m) > e$ is $\negl(\lambda)$, by soundness analysis in \cite{Ligero2017}.
	
	Therefore $\extrac$ outputs $\calU^*$ whcih satisfies $\com(\calU^*) = \cm$ such that $d(\calU^*, L^m) < e$ with very high probability or it breaks the binding of the commitment scheme.
	
	Now we need to prove that the number of rewinding is polynomial.
	
	To get the bound on the expected number of rewindings consider $t=1$. 
	
	Let $X$ be the discrete random variable that represents the number of purchases until each of the $n$ column is picked at least once.
	
	Let $X_i$ be the discrete random variable that represents the number of rewindings after the $(i-1)^{th}$ distinct column to select the $i^{th}$ distinct column. As a base case, $X_1=1$, because the first column selected will always be distinct.
	
	By Linearity of expectation, $E[X]=\sum_{i=1}^{n}E[X_i]$
	
	After the $(i-1)^{th}$ distinct column is picked, there are $n-i+1$ columns remaining to be picked. Let $A_i$ be the event that one of those columns is picked in the next rewinding. Then $Pr(A_i)=\frac{n-i+1}{n}$.
	
	$X_i$ follows a geometric distribution (of trials). It's expected value is
	$$E[X_i]= \frac{1}{Pr(A_i)} = \frac{n}{n-i+1}$$
	From before, $E[X]$ is equal to the sum of all these expectations: i.e.
	$$E[X] = \sum\limits_{i=1}^{n} \frac{n}{n-i+1} = n\sum\limits_{i=1}^{n} \frac{1}{i} \approx n \log n$$
	Which is polynomial in $n$ and so polynomial in security parameter.
	%\pnote{Remaining analysis is to check the expected number of rewinding required to obtain $\oracle$.}
\end{proof}
\end{comment}
%------------------------------------------------------------------------------------------------------------------------------------


\subsubsection{Linear Check}
	Without loss of generality, we will discuss an argument for proving linear constraints of the form $Ax=0$ where $A$ is a public matrix and $x$ is a (secret) vector. The case $Ax=y$ is easily transformed to the required form by defining $A'=[A|-I]$ and $x'=(x,y)$, and running the protocol on $A'$ and $x'$. Assume that $x\in \bbF^N$ and $A\in \bbF^{N\times N}$. To achieve sublinear query complexity, we do two things: (i) we make a high probability reduction to the problem of proving $\innp{r^TA}{x}=0$ where $r\sample \bbF^N$ is sent by the verifier (ii) encode the witness $x$ using Reed-Solomon code as we described earlier.
	
	We first consider the encoding used in \cite{Aurora} which encodes $x$ as a single codeword $f^x$. The prover then provides oracle access to $f^x$. Let $\hat{r}\in\bbF[x]$ be the polynomial that interpolates the vector $r^TA$ on $G$. Then the linear check reduces to checking $\sum_{\zeta\in G}\hat{r}(\zeta) \cdot \hat{f}^x(\zeta) = 0$. In \cite{Aurora}, the aforementioned identity is checked with query size of $O(\log|H|)$ using especially developed	sumcheck protocol for univariate polynomials. The verifier still incurs $O(N)$ work to compute the encoding $\hat{r}$ for the size $N$ vector $r^TA$. 
	
	We now consider the interleaved encoding of the witness. Looking ahead, the interleaved encoding will have a direct impact on the circuit share complexity of the distributed proof generation in \name{}. Here, we write the vector $x\in \bbF^N$ as $x=(x_1|\cdots|x_m)$ where each $x_i\in \bbF^\ell$ for some $m\ell > N$, and $x_m$ padded as necessary. For suitable domains $G$ and $H$, we interpolate each chunk of the vector on $G$ 
	separately via polynomials $\hat{f}^x_1,\ldots,\hat{f}^x_m$. Similarly we write $r^TA=(r_1,\ldots,r_m)$ with $r_i\in \bbF^n$ and construct polynomials $\hat{r}_i$	for $i\in [m]$. The inner product check $\innp{r^TA}{x}=0$ is then equivalent to checking $\sum_{i\in [m]}\sum_{\zeta\in G}\hat{r}_i(\zeta)\hat{f}^x_i(\zeta)=0$. The	latter is checked by the prover sending the polynomial $\hat{p}=\sum_{i\in
	[m]}\hat{r}_i.\hat{f}^x_i$ to the verifier, and verifier checking $\sum_{\zeta\in G}\hat{p}(\zeta)=0$. Choosing $m,\ell\approx O(N^\frac{1}{2})$ this incurs square-root communication from the prover. How can the verifier be sure that $\hat{p}$ was indeed computed correctly from the witness polynomials $\hat{f}^i_x$?. As we formally prove in the later sections, this can be accomplished by the verifier querying the polynomial evaluations (oracles) at a constant number of locations, say $\eta_1,\ldots,\eta_q$. The verifier then checks
	that $\hat{p}$ is consistent at the queried locations by verifying $\hat{p}(\eta_j)=\sum_{i\in [m]}\hat{r}_i(\eta_j)\hat{f}^x_i(\eta_j)$ for
	all $j\in [q]$. This incurs a total query complexity of $q.m=O(N^\frac{1}{2})$.	
	From the perspective of verifier's efficiency, it only needs to compute evaluations of polynomials $\hat{r}_i$, $i\in [m]$ for $\eta\in \{\eta_1,\ldots,\eta_q\}$. %%We will show	that this can be done in expected sublinear time, after a one time $O(||A||)$  pre-processing. We will also discuss why a similar pre-processing does not work for the earlier encoding scheme.


%---------------------------------------------------------------------------------------------------------------------------------------------------
\begin{comment} 
\paragraph{\textbf{Linear Check:}}\label{sec: linearity} In this check, there is a publicly known matrix $A \in \bbF^{ml\times ml}$ and public vector $b\in \bbF^{ml}$, and $\prover$ claims that he knows a $x$ such that $Ax=b$ is true. Without loss of generality we can assume that the vector $b$ is the zero vector i.e. $b=\bm{0}$ $\prover$ starts with encoding the $x$ using interleaved $\RS$ code which is described above and gets $\oracle_x$, then he computes the commitment $\cm_x$ in the same way mentioned above. Now the following check ensures that if a maformed $\calU$ is used for $\oracle$, which is not far away from a correct encoding of $x$ for which $Ax\neq b$, then $\verifier$ outputs reject with very high probability.
\begin{enumerate}
	\item $\prover \rightarrow \verifier: $ $\prover$ sends $\cm_x$ to $\verifier$.
	\item $\verifier \rightarrow \prover: $ $\verifier$ picks a random $r\in \bbF^{ml}$ and sends that $r$ to $\prover$.
	\item $\prover$ and $\verifier$ compute $R=r^TA$, which is a vector of size $ml$. Read $R$ in a matrix form where first $l$ elements of $R$ forms the first row, next $l$ elements forms the second row, similary $R$ will have $m$ rows. Then they construct polynomials $R_i(\cdot)$ of degree $<l$ such that $R_i(\zeta_j)=R_{ij}$ $\forall i\in [m], j\in [l]$. 
	\item $\prover \rightarrow \verifier: $  $\prover$ computes a polynomial $p(\cdot)=\sum_{i\in[m]} ( R_i(\cdot)\cdot \hat{f}^x_i(\cdot))$ and sends $p(\cdot)$ to $\verifier$.
	\item $\verifier \rightarrow \prover: $ $\verifier$ samples $t$ distinct  indices $j_1,\ldots,j_t$ from the set $[n]$ independently at random and sends the indices to $\prover$.
	\item $\prover \leftrightarrow \verifier: $ $\prover$ and $\verifier$ run a subprotocol to prove the innerproduct arguement for the following statement $\innp{R_{j_u}}{\oracle_x[\cdot,j_u]} = p(\eta_{j_u})$ for all $j_u$ queried in the above step by $\verifier$, where $R_{j_u}=(R_1(\eta_{j_u}),\ldots,R_m(\eta_{j_u})$ and $\oracle_x[\cdot,j_u]$ denotes the $m-$length vector $(\oracle_x[1,j_u],\ldots, \oracle_x[m,j_u])$. $\verifier$ proceeds if the arguments succeed for all $u \in [t]$.
	\item $\verifier$ also checks that $\sum_{j\in[l]} p(\zeta_j)=0$.
	\item $\verifier$ accepts if all the above checks are succeed.	  
\end{enumerate}

\begin{theorem}
	For $e < \frac{d}{3} $, $\innp{\prover^*(\cm, \calU^*, A, b)}{\verifier(\cm, A, b)} \rightarrow 1 $ then there is an expected $\ppt$ $\extrac^{\prover^*}(\cm) \rightarrow \calU^*$ such that with overwhelming probability $\calU^*$ satisfies one of the following events:
	\begin{itemize}
		\item $\com(\calU^*)=\cm$ and $d(\oracle_x,L^m) > e $
		\item $\com(\calU^*)=\cm$ and $d(\oracle_x, l^m)\leq e \text{ and } x = \dec(\oracle_x)$ satisfies $ Ax = b$
		\item $\extrac$ breaks the commitment scheme.
	\end{itemize}
\end{theorem}

\begin{proof}
	We have a $\ppt$ extractor, $\extrac_{innp}$, for the inner product arguement for the statement $\innp{\bm{a}}{\bm{b}}=c$ where $\bm{a},\bm{b}$ are private, which can either extract $\bm{a}, \bm{b}$ or breaks the binding property of the commitment scheme with overwhelming probability. Now we will use $\extrac_{innp}$ to design a $\ppt$ extractor $\extrac$ which can extract the witness($\calU^*$) for which the arguement in the above protocol is accepted.
	
	$\extrac$ emulates $\verifier$'s role in the protocol till step 5, then calls $\extrac_{innp}$ to get $\calU^*[\cdot,j_u]$ or collision for the commitment. $\extrac$ stores the indices in a set say $S$ and rewind the prover to step 5 and picks $t$ indices again uniformly at random again and follows the above procedure. $\extrac$ keeps rewinding till $|S|=n$.
	
	If $|S|=n$, then $\extrac$ has the whole $\calU^*$.
	
	Then $\extrac$ computes $d(\calU^*, L^m)$:
	\begin{itemize}
		\item if $d(\calU^*,L^m) > e$ then $\extrac$ outputs $\calU^*$, which satisfies $\com(\calU^*)=\cm$ otherwise gets a collision for the binding property.
		
		\item if $\d(\calU^*,L^m) \leq e$ then by soundness analysis in \cite{Ligero2017}, the nearest codeword of $\calU^*$ decodes to $x$ such that $Ax\neq b$ has $\negl(\lambda)$ probabillity. That means nearest codeword of $\calU^*$ decodes to $x$ which satisfies $Ax=b$ with very high probability and $\com(\calU^*)=\cm$.
	\end{itemize}
	
	Similar analysis of Theorem:~\ref{theo:proximity} proves that $\extrac$ requires polynomially many rewinding to extract the witness $x$.
\end{proof}
\end{comment} 
%-------------------------------------------------------------------------------------------------------------------------------------

\subsubsection{Quadratic Check}
The quadratic check involves the prover convincing the verifier that $x\circ y=z$ by providing oracle access to the vectors $x,y$ and $z$. Again, we
consider the encoding by parts we discussed for the linear check. We write $x=(x_1|\cdots|x_m)$, $y=(y_1|\cdots|y_m)$ and $z=(z_1|\cdots|z_m)$ and
construct polynomials $\hat{f}^x_i$, $\hat{f}^y_i$ and $\hat{f}^z_i$ for $i\in [m]$ as before. The quadratic check then reduces to showing that
$\hat{f}^x_i(\zeta).\hat{f}^y_i(\zeta)-\hat{f}^z_i(\zeta)=0$ for all $i\in [m]$ and $\zeta\in G$. With high probability the above can be checked by
verifier sending a random vector $r\sample \bbF^m$ to the prover, and prover sending the polynomial $\hat{p}=\sum_{i\in [m] } r_i (\hat{f}^x_i .\hat{f}^y_i - \hat{f}^z_i)$ to the verifier. The verifier checks that $\hat{p}(\zeta)=0$ for all $\zeta\in G$. The verifier also checks that $\hat{p}$ is correctly computed from the oracles by querying the oracles at small number of points $\eta_1,\ldots,\eta_q$ and checking that $\hat{p}(\eta_j)=\sum_{i\in [m]}r_i(\hat{f}^x_i(\eta_j).\hat{f}^y_i(\eta_j)-\hat{f}^z_i(\eta_j))$ for $j\in [q]$.
%-----------------------------------------------------------------------------------------------------------------------------------------------
\begin{comment}
\paragraph{\textbf{Quadratic Check:}}\label{sec: quadratic} In this check $\prover$ claims that he knows $x,y,z\in\bbF^{ml}$ such that $x \circ y = z$. $\prover$ starts with encoding each of $x, y, z$ and obtains $\oracle_x, \oracle_y, \oracle_z$. Then $\prover$ performs column wise commitments and gets $\cm_x, \cm_y, \cm_z$. The following check ensures that if any of the $\oracle_x, \oracle_y, \oracle_z$ is malformed, which is not far away from a correct encoding of any of the $x, y, z$, then $\verifier$ rejects with very high probability.
\begin{enumerate}
	\item $\prover \rightarrow \verifier: $ $\prover$ sends $\cm_x, \cm_y, \cm_z$ to $\verifier$.
	
	\item $\verifier \rightarrow \prover: $ $\verifier$ picks $r$ uniformly at random from $\bbF^{m}$ and sends $r$ to $\prover$.
	
	\item $\prover \rightarrow \verifier: $ $\prover$ computes the polynomial $p(\cdot)= \sum_{i\in [m]} [r_i\cdot (\hat{f}^x_i(\cdot)\cdot \hat{f}^y_i(\cdot) - \hat{f}^z_i(\cdot))] $ and sends $p(\cdot)$ to $\verifier$. 
	
	\item $\verifier \rightarrow \prover: $ $\verifier$ sends $t$ randomly sampled indices $Q=\{j_u\}_{u\in[t]}$ from $[n]$.
	
	\item $\prover \leftrightarrow \verifier: $ $\prover$ and $\verifier$ run a subprotocol to prove the innerproduct arguement for the following statement $\innp{r\circ \oracle_x[\cdot,j_u]}{\oracle_y[\cdot,j_u]} - \innp{r}{\oracle_z[\cdot,j_u]} = p(\eta_{j_u})$ $\forall u\in[t]$. 
	
	There are two inner products in the above statement and the prover does not want to reveal the values of the individual inner products. Hence, \name2D combine them into a single inner product relation.
	
	For each $u\in[t]$, $\prover$ runs the following inner product arguement with $\verifier$:
	$$\innp{(r\circ \oracle_x[\cdot,j_u]||r)}{(\oracle_y[\cdot,j_u]||-\oracle_z[\cdot,j_u])} = p(\eta_{j_u})$$
	To facilitate this, $\oracle_x, \oracle_y$ and $\oracle_z$ should have been committed with different independently chosen sets of generators. And, the set of generators used for committing $r$ (to be concatenated with $r \circ \oracle_x [\cdot, j_u])$ should also be independent of the above three sets of generators. $\verifier$ proceeds if the arguements succeed for all $u\in[t]$.
	\item $\verifier$ checks if $p(\zeta_j)=0$ $\forall j\in[l]$. If yes then accepts, else rejects.
\end{enumerate}

\begin{theorem}
	For $ e < \frac{d}{3}, $ if $\innp{\prover^*(\cm_1, \cm_2, \cm_3 , \calU^*_1	, \calU^*_2, \calU^*_3)}{\verifier(\cm_1, \cm_2, \cm_3)} \rightarrow 1$ then there is an expected $\ppt$  $\extrac^{\prover^*}(\cm_i)\rightarrow \calU^*_i$ $\forall i\in [3]$ such that with overwhelming probability $\forall i \in [3]$, $\calU^*_i$ satisfies one of the following event: 
	\begin{itemize}
		\item $\com(\calU^*_i) = \cm_i$ and $(\vee_{i=1}^{3} d(\calU^*_i, L^m)> e)$ 
		\item $\com(\calU^*_i) = \cm_i$ and $\wedge_{i=1}{3} d(\calU^*_i,L^m)\leq e$ and $ x_i = \dec(\calU_i)$, where $\calU_i$ is the nearest codeword to $\calU^*_i$, for all $i\in [3]$, such that $x_1 \circ x_2 = x_3$
		\item $\extrac$ breaks the binding of the commitment scheme.
	\end{itemize} 
	
\end{theorem}
\begin{proof}
	We have a $\ppt$ extractor, $\extrac_{innp}$, for the inner product arguement for the statement $\innp{\bm{a}}{\bm{b}}=c$ where $\bm{a},\bm{b}$ are private, which can extract either $\bm{a}, \bm{b}$ or breaks binding property of the commitment scheme with overwhelming probability. Now we will use $\extrac_{innp}$ to design an expected $\ppt$ extractor $\extrac$ which can extract $\calU^*_1, \cal^*_2, \cal^*_3$ for which the arguemnet is accepted in the above protocol.
	
	$\extrac$ emulates $\verifier$'s role in the protocol till step 4, then calls $\extrac_{innp}$ to get $(r\circ \calU^*_1[\cdot,j_u]||r)$ and $(\calU^*_2[\cdot,j_u]||-\calU^*_2[\cdot,j_u])$ forall $u\in [t]$ or breaks the binding property of the commitment scheme. $\extrac$ stores all the indices in a set say $S$. From the output of $\extrac_{innp}$, $\extrac$ computes $\calU^*_1[\cdot,j_u], \calU^*_2[\cdot,j_u],\calU^*_3[\cdot,j_u]$. $\extrac$ rewinds $\prover$ to step 4 and again picks a random $Q$. $\extrac$ keeps repeating the above process till $|S|=n$.
	
	If $|S|=n$, then $\extrac$ has the whole of $\calU^*_1, \calU^*_2, \calU^*_3$.
	
	Then $\extrac$ checks if $\com(\calU^*_i)=\cm_i$, if no, then outputs a collision for the binding of the commitment scheme, else computes $d(\calU^*_i, L^m)$ for all $i \in [3]$:
	
	\begin{itemize}
		\item if $\vee_{i=1}^{3} d(\calU^*_i, L^m) > e$, then outputs $\calU^*_i \forall i\in [3]$.
		
		\item if $\wedge_{i=1}^3 d(\calU^*_i, L^m) \leq e$, then by soundness analysis in \cite{Ligero2017}, the nearest codewords of $\calU^*_1, \calU^*_2, \calU^*_3$ decode to $x, y, z$ repesctively such that probability that $x \circ y \neq z$ is $\negl(\lambda)$. That means the decoded values of the nearest codewords of $\calU^*_1, \calU^*_2, \calU^*_3$ satisfy $x\circ y = z$ with very high probability.
	\end{itemize}
	
	Similar analysis of Theorem~\ref{theo:proximity} proves that $\extrac$ requires polynomially many rewinding to extract the witness $x, y, z$.
\end{proof}

%-------------------------------------------------------------------------------------------------------------------------------------
\subsection{\name2D{}: A SNARK protocol with $O(|\C|^{1/c})$ proof size}
\paragraph{\textbf{Complete protocol:}} Let $L$ be a language in $\NP$ and $\stmt$ is an instance of $L$. Let $\prover$ be a prover that claims that the instance $\stmt$ is true, i.e. $\prover$ has a witness $\wit$ such that there is deterministic circuit $\C$ such that $\C(\stmt,\wit)=1$ iff $\stmt \in L$. 

Now $\prover$ wants to convince a verifier $\verifier$ that $\stmt\in L$ without revealing any information about the witness $\wit$. 

To do that $\prover$ gives a proof that $\C$ on input $(\stmt, \wit)$ is correctly executed and output 1. In other words $\prover$ proves that gate by gate evaluation is correctly done on an input which has a public part known to both $\prover$ and $\verifier$, and a private part which is only known to $\prover$. 
Note that $\C$ and $x$ both known to $\prover$ and $\verifier$. So without loss of generality we can assume that $\stmt$ is hardcoded in $\C$.

$\prover$ constructs the extended witness $\extwit$ in the following way: \\
Let $\C:\bbF^{n_i}\rightarrow \bbF$ such that $\prover$ has private input $\wit = (\wit_1,\ldots, \wit_{n_i})$ such that $\C(\wit)=1$.

Define the extended witness $\extwit = (\wit_1,\ldots,\wit_{n_i}, \beta_1,\ldots, \beta_s) \in \bbF^{ml}$, where $\beta_i$ is the output of the $i^{th}$ gate evaluating $\C(\wit)$, and $s$ is the number of gates in $\C$ and $ml>n_i + s$. $\prover$ defines a system of constraints that contains the following constraint for every multiplication gate g in the circuit $\C$ $$\beta_{a}.\beta_{b}-\beta{c}=0$$
and for every addition gate, the constraint 
$$\beta_a + \beta_b - \beta_c = 0$$
Where $\beta_a$, $\beta_b$ are the input values to the gate g and $\beta_c$ is the output value in the extended witness. For the output gate include the constraint $\beta_a + \beta_b - 1 = 0$ if the final gate is an addition gate, and $\beta_a\cdot \beta_b - 1 = 0$ if the final gate is an multiplication gate. 

$\prover$ constructs vectors $x,y,z \in \bbF^{ml}$ where the $j^{th}$ entry of $x,y$ and $z$ contains the values $\beta_a, \beta_b$ and $\beta_c$ corresponding to the $j^{th}$ multiplication gate in $\extwit$.

$\prover$ and $\verifier$ construct matrices $A, B$ and $C \in \bbF^{ml \times ml}$ such that 
$$x = A \extwit, y= B \extwit, z= C \extwit$$

Finally it constructs $P_{add} \in \bbF^{ml \times ml}$ such that the $j^{th}$ position of $P_{add} \extwit$ equals $\beta_a + \beta_b - \beta_c$ where $a, b$ and $c$ correspond to the $j^{th}$ addition gate of the circuit in $\extwit$.

$\prover$ encodes $\extwit , x, y , z$ using the defined encoding in sec:~\ref{subsec: encode} and gets $\oracle_{\extwit}, \oracle_x, \oracle_y$ and $\oracle_z$ and commit each of them using Pedersen vector commitment described in ~\ref{subsec: commit} and gets $\cm_{\extwit}, \cm_x, \cm_y, \cm_z$. To commit, pick different set of generators for each $\oracle_{\extwit}, \oracle_x, \oracle_y, \oracle_z$, which will give the property that commitment of $\oracle_{\extwit||x}$ is $\cm_{\extwit} \circ \cm_x$. Where $\extwit||x$ means that a new matrix is formed by adjoining columns of $\extwit$ followed by the columns of $x$. 

In the following protocol, $\verifier$ picks a random index set $Q\subseteq [n]$ of size $t$ and uses this $Q$ in all the followimg subprotocols.
\begin{enumerate}
	\item $\prover \leftrightarrow \verifier: $ $\prover$ and $\verifier$ run the subprotocol ~\ref{sec:proximity} for proximity check for the matrix $\extwit||x||y||z$. Which is encoded to the matrix $\oracle{\extwit||x||y||z}$, and $\cm_{\extwit||x||y||z}$ is the corresponding commitment.
	
	\item $\prover \leftrightarrow \verifier: $ $\prover$ and $\verifier$ run the subprotocol ~\ref{sec: linearity} for the linearity check for the public matrix $P_{add}$ of dimension $ml\times ml$, and the public vector is the 0-vector of length $ml$. This check ensures that the addition gates are corretly evaluated.
	
	\item $\prover \leftrightarrow \verifier:$ $\prover$ and $\verifier$ run the subprotocol ~\ref{sec: linearity} for linearity check for the public matrix $[A|-I]$ where $I$ is the identity matrix of dimension $ml \times ml$, and the public vector is the 0 vector of length $2ml$. This check ensures that $x$ is correctly computed from $\extwit$.
	
	\item $\prover \leftrightarrow \verifier:$ $\prover$ and $\verifier$ run the subprotocol ~\ref{sec: linearity} for linearity check for the public matrix $[B|-I]$ where $I$ is the identity matrix of dimension $ml \times ml$, and the public vector is the 0 vector of length $2ml$.This check ensures that $y$ is correctly computed from $\extwit$.
	
	\item $\prover \leftrightarrow \verifier:$ $\prover$ and $\verifier$ run the subprotocol ~\ref{sec: linearity} for linearity check for the public matrix $[C|-I]$ where $I$ is the identity matrix of dimension $ml \times ml$, and the public vector is the 0 vector of length $2ml$.This check ensures that $z$ is correctly computed from $\extwit$.
	
	\item $\prover \leftrightarrow \verifier:$ $\prover$ and $\verifier$ run the subprotocol ~\ref{sec: quadratic} for quadratic check to prove that $x \circ y = z$.
	
	\item If $\prover$ passes all the above check, then $\verifier$ accepts the arguement, else rejects.
\end{enumerate}

Note that if $\prover$ executed all the steps correctly, then it passes all the checks and so $\verifier$ accepts the proof. So completeness holds for the protocol.

To prove the proof of knowledge we will design an extractor which will output a witness if the arguement is accepted.

\begin{theorem}
	For $e < \frac{d}{3}$, if $\innp{\prover^*(\cm_{\extwit||x||y||z}, \calU^*)}{\verifier(\cm_{\extwit||x||y||z})} \rightarrow 1$, then there is an expected $\ppt$ $\extrac^{\prover^*}(\cm_{\extwit||x||y||z}) \rightarrow \calU^*$ such that with overwhelming probability $\calU^*$ satisfies one of the following events:
	\begin{itemize}
		\item $\com(\calU^*)=\cm_{\extwit||x||y||z}$ and $A\extwit = x \wedge B \extwit = y \wedge C \extwit = z \wedge x\circ y =z \wedge P_{add} \extwit = 0$
		\item $\extrac$ breaks the binding property of the commitment scheme.
		
	\end{itemize}
	Let $\cm$ is the commitment of $\calU$, used in above protocol by $\prover$. If $\verifier$ accepts the arguement generated by $\prover$, then there is an expected $\ppt$ extrator $\extrac$ having rewinding access to $\prover$, with polynomially many rewindings either outputs a correct $\extwit$ or breaks the binding property of the commitment scheme.
\end{theorem}
\begin{proof}
	We have extractors for the proximity check, linearity check and quadratic check, say $\extrac_{prox}, \extrac_{lin}, \extrac_{quad}$ are the extractors respectively. Using these extractors we will design  $\extrac$ for the complete protocol. 
	
	$\extrac$ emulates the role of the verifier and starts the protocol. 
	
	In the first step it calls $\extrac_{prox}$ and $\calU_1^*$.
	
	After executing the first step it calls $\extrac_{lin}$ and gets $\calU_2^*$. If $\calU_1^* = \calU_2^*$, then outputs collision and terminates.
	
	After executing the second if it is not terminated, it calls $\extrac_{lin}$ in the step 3, 4, 5 and gets $\calU_3^*$. 
	
	In the above 2 steps it gets collision that breaks the binding of the commitment otherwise proceeds.
	
	In this step $\extrac$ calls $\extrac_{quad}$, by concatanating the output of $\extrac_{quad}$ construct final matrix which equates with $\calU_1^* (=\calU_2^*=\calU_3^*)$. If not then that gives break of the binding property of the commitment scheme. Otherwise $\calU^*=\calU_1^*$ should satisfy the following:
	\begin{itemize}
		\item $\com(\calU*)=\cm_{\extwit}\cdot\cm_x\cdot\cm_y\cdot\cm_z$.
		\item $d(\calU^*,L^{4m}) < e$ and let $\calU$ is the closest codeword of $\calU^*$. Define $\oracle_{\extwit}$ to be the first $m$ rows of $\calU$
		
		Define $\oracle_{x}$ to be the $(m+1)^{th}$ rows to $2m^{th}$ rows of $\calU$
		
		Define $\oracle_{y}$ to be the $(2m+1)^{th}$ rows to $3m^{th}$ rows of $\calU$
		
		Define $\oracle_{z}$ to be the $(3m+1)^{th}$ rows to $4m^{th}$ rows of $\calU$
		
		Let $w= \dec(\oracle_{\extwit}), x=\dec(\oracle_x), y=\dec(\oracle_y), z=\dec(\oracle_z)$ such that
		
		$A\extwit = x \wedge B \extwit = y \wedge C \extwit = z \wedge x\circ y =z \wedge P_{add} \extwit = 0$.
	\end{itemize}
	Therefore $\extwit$ is a correct witness. Since all the above extractors use polynomial number of rewindings, $\extrac$ uses polynomial number of rewindings.		
\end{proof}
%--------------------------------------------------------------------------------------------------------------------------------------
\subsection{Zero - Knowledge}
Above defined three subprotocols are not zero-knowledge inherently. But converting them into zero-knowledge is easy. 

Consider the proximity check: In this subprotocol, $\verifier$ is learning $u = \gamma^T\oracle_{x}$, whihc he can't compute on his own. To prevent that modify $\oracle_{x}$ in the following way: include a random codeword in $(m+1)^{th}$ row, which blinds $u$ and makes $u$ a random codeword. 
In the remaining part, $\prover$ sends $\cm$, hiding property of the commitment scheme ensures that it does reveal any information.
Innerproduct proofs are given for $t$ columns. Instead of giving the proof, If $\prover$ opens $t$ columns of $\oracle_{x}$, still it does not reveal any information about $x$, since $t$ is smaller than the degrees of the polynomials used in $\enc$.

Consider the linear check: In the first step $\prover$ sends $\cm_x$ to $\verifier$. The hiding property of the commitment scheme ensures that $\verifier$ is getting no information about $\calU^*_x$ or $x$.
Then $\prover$ sends $p(\cdot) = \sum_{i\in[m]} R_i \cdot \hat{f}^x_i(\cdot)$ to check that if $\sum_{j\in [l]} p(\zeta_j) = 0$. But $\verifier$ instead of learning whether $\sum_{j\in[l]} p(\zeta_j) = 0$ or not, gets the complete polynomial $p(\cdot)$. To avoid leaking additional information we need to blind $p(\cdot)$ by adding a blinding polynomial $p_{blind}(\cdot)$ of degree $< k + l - 1$ such that $\sum_{j\in[l]} p_{blind}(\zeta_j) = 0$. Include a new row to $\oracle_x$ at the end where $j^{th}$ entry of the row is $p_{blind}(\eta_j)$ $\forall j\in [n]$.
Since number of inner product proofs is less than $t$, no information is leaked.

Consider the quadratic check: In the first step commitments do not leak any information about the witness or it's encoded values.
Then $\prover$ sends $p(\cdot) = \sum_{i\in[m]} [r_i\cdot (\hat{f}^x_i(\cdot)\cdot \hat{f}^y_i(\cdot) - \hat{f}^z_i(\cdot)]$. $\verifier$ is allowed to learn only if $p(\zeta_j)=0$ or not for all $j\in[l]$. To avoid leaking more information about $p(\cdot)$ we need to blind it. To do that pick a random polynomial $p_{blind}(\cdot)$ such that $p_{blind}(\zeta_j) = 0$ $\forall j\in [l]$. accordingly update $\oracle_x, \oracle_y,\oracle_z$ in the following way: pick there random codewords which are encodings of zeros, and append one of them each at the last of $\oracle_{x}, \oracle_{y}, \oracle_{z}$.
Inner product arguement is same as Proximity and linear check, it does not require any changes. 

\paragraph{Deisgning the simulator for the complete protocol: } In an actual execution of the protocol generates a transcript of the form:
\begin{align*}
\tau = \{ 
& \cm_{\extwit||x||y||z},\\
& \gamma(\in_R \bbF^{m}), r_1(\in_R \bbF^{ml}), r_2(\in_R \bbF^m), \\ 
& (u' = u+u_{blind}), (q^{lin}(\cdot) = p^{lin}(\cdot)+p^{lin}_{blind}(\cdot)), (q^{quad}(\cdot) = p^{quad}(\cdot) + p^{quad}_{blind}(\cdot)),\\
& Q (|Q|=t),\\
& \text{ inner product proof for } \\
& (\innp{\gamma}{\oracle_{\extwit||x||y||z}[\cdot, Q]} = u_Q, \innp{R_Q}{\oracle_{\extwit||x||y||z}[\cdot, Q]}=q^{lin}(\eta_Q),\\
& \innp{(r\circ \oracle_x[\cdot,j_u]||r)}{(\oracle_y[\cdot,j_u]||-\oracle_z[\cdot,j_u])} = q^{quad}(\eta_{j_u})
\}
\end{align*}
Consider a protocol, which is same as above protocol with the difference that $\prover$ instead of proving the inner product arguements opens the corresponding columns of $\oracle$. If this new protocol has zero knowledge property then our protocol also have zero knowledge property, since in our protocol whatever $\verifier$ can compute, $\verifier$ of the new protocol can also compute. It is easy to prove this by reduction.

Now we will prove that the new protocol is zero-knowledge. It will have the transcript of the following form:
\begin{align*}
\tau' = \{
& \cm_{\extwit||x||y||z},\\
& \gamma(\in_R \bbF^{m}), r_1(\in_R \bbF^{ml}), r_2(\in_R \bbF^m), \\ 
& (u' = u+u_{blind}), (q^{lin}(\cdot) = p^{lin}(\cdot)+p^{lin}_{blind}(\cdot)), (q^{quad}(\cdot) = p^{quad}(\cdot) + p^{quad}_{blind}(\cdot)),\\
& Q (|Q|=t),\\
& \oracle_{\extwit}[\cdot, Q], \oracle_{x}[\cdot, Q], \oracle_{y}[\cdot, Q], \oracle_{z}[\cdot, Q]
\}
\end{align*}

Let $\Sim$ be the simulator. $\Sim$ does the following:
\begin{itemize}
	\item  picks a random subste $Q$ of size $t$.
	\item  uniformly at random chooses $\gamma \in \bbF^m$.
	\item  uniformly at random chooses $r_1 \in \bbF^{ml}$ and $r_2 \in \bbF^m$.
	\item  chooses $t$ columns for $\oracle_{\extwit||x||y||z}$ according to the indices of $Q$.
	\item  computes the commitment of columns indexed by $Q$ for $\oracle_{\extwit||x||y||z}$, and for remaining positions picks uniform values from the range of $\com$, that fixes $\cm_{\extwit||x||y||z}$.
	\item  computes components of $u'$ indexed by $Q$ using $\oracle_{\extwit||x||y||z}$ and $\gamma$. Out of $n$ for remaining $n-t$ picks values for $u'$ in such a way that $u'$ is a valid coedword.
	\item  picks a random polynomial $q^{lin}(\cdot)$ such that degree is $<k+l-1$ and $\sum_{j\in [l]} q^{lin}(\zeta_j) = 0$.
	\item  picks a random polynomial $q^{quad}(\cdot)$ such that degree is $<2k-1$ and $q^{quad}(\zeta_j) = 0$ $\forall j\in [l]$.
\end{itemize} 
Then $\Sim$ outputs a transcript $\tau''$ which is computationally indistinguishable from $\tau'$. Therefore the new protocol has zero knowledge property, and hence \name2D has zero-knowledge property.

\end{comment}
%-------------------------------------------------------------------------------------------------------------------------------------

\subsubsection{Proximity Test}
The correctness of the previous two checks, namely the linear check and the
quadratic check rely on the fact that the witness oracles are evaluations of
``low'' degree polynomials. There are several known low degree tests for polynomials
from PCP literature. The protocol in \cite{Aurora} uses a recent test for
proximity by Ben Sesson et al.\cite{IOPP_FRI2018} with particularly efficient
prover and $O(\log d)$ query complexity for polynomials of degree at most $d$. 
We use a variant of proximity test from \cite{Ligero2017}, adapting it to work
with homomorphic commitments of the RS-encoded oracles and reducing the query
complexity.


[DPZKShortProof]
1. The use for commitments --- 2. Explain the 2D version  

[DPZKQuickVerify]
3. The need for 3D --- 4. Our 3D version.


\section{DPZK with reduced proof size}
1. The use for commitments --- 2. Explain the 2D version  
An brief recall on how our paradigm enables any improvement in single prover ZK protocol directly translate to the DP version.\\

The first part of \name{} is obtaining the proof size of $|\C|^{1/c}$ for any $c \geq 2$.
We will make use of homomorphic commitments and inner product arguments to remove the linear dependence of the proof size on $m$ in \cite{Ligero2017}. A brief summary of our idea follows. We will commit to the vector $\{\hat{f}^x_i(\eta)\}_{i \in [m]}$ using the homomorphic vector commitments. The vectors corresponding to each $\eta$ will be committed separately. Here, instead of ``opening'' these vectors to the verifier, the prover in \name{} will provide an inner product argument on the vectors.
\dnote{provide a short comparison of this protocol with the existing works here.}.

\subsection{\name2D{}: A SNARK protocol with $O(|\C|^{1/c})$ proof size}

We will now describe \name2D{} in detail. The protocol explained in this section will have $O(|\C|^{1/c})$ proof size and $O(|\C|^{1-1/c})$ public key operations (and $O(|\C|)$ overall complexity) for the verifier. We  explain the interactive version of our protocol \name2D{}. \\

We reduce the query complexity and verifier efficiency in the interactive protocol by using homomorphic vector commitments over the RS encoded witness oracles. Let $x \in \bbF^{|\C|}$ be the witness vector. Let $m$ and $\ell$ be integers such that $m\ell\geq |\C|$. We choose ordered domains $G=\{\zeta_1,\ldots,\zeta_\ell\}$ and $H=\{\eta_1,\ldots,\eta_n\}$. We then write the vectors $x$ as $x = (x_{1},\ldots,x_{m})$ where each $x_{i}\in \bbF^\ell$ for $i \in [m]$. Let $\hat{f}^x_{i}$ be the polynomial interpolating the vector $x_{i}$ on $G$ and let $f^x_{i}$ denote the corresponding evaluation of $\hat{f}^x_{i}$ on $H$. We define the RS-encoded witness $\rsoracle\in \bbF^{m\times n}$ as $\rsoracle[i,j]=\hat{f}^x_{i}(\eta_j)$ for $ i\in [m]$ and $j\in [n]$. We now construct a commitment oracle $\comoracle$ from $\rsoracle$.

.\pnote{start}
\subsubsection{Encoding:}\label{subsec: encode} 
If $|x|=ml$ then read $x$ as 
$$x=
\begin{bmatrix}
x_{11} & x_{12} & \ldots & x_{1l}\\
x_{21} & x_{22} & \ldots & x_{2l}\\
& \vdots\\
x_{m1} & x_{m2} & \ldots & x_{ml}
\end{bmatrix}
$$	
Construct polynomials $\hat{f}^x_i(\cdot)$ of deg $k$ such that $\hat{f}^x_i(\zeta_j)=x_{ij}$ $\forall i\in [m], j\in [l]$ where $k>l$ and $l+t=k$.

Define 
$$ \oracle =
\begin{bmatrix}
u_{11} & u_{12} & \ldots & u_{1n}\\
u_{21} & u_{22} & \ldots & u_{2n}\\
& \vdots\\
u_{m1} & u_{m2} & \ldots & u_{mn}
\end{bmatrix}
$$
where $u_{ij}= \hat{f}^x_i(\eta_j)$ $\forall i\in[m], j\in[n]$ $n>k$. $\bm{\zeta}=\{\zeta_1,\ldots,\zeta_l\}$ we will call it interpolation domain and $\bm{\eta} = \{\eta_1,\ldots,\eta_n\}$ we will call it evaluation domain. 

Let $L$ be the set of codewords and the above linear code has distance $d$. Then a correctly computed $\oracle_x$ is in $L^m$.
.\pnote{end}
 
\subsubsection{Commitment Oracle}
Throughout, we assume $\bbF$ is a prime field. Let $\com$ denote the Pedersen vector commitment scheme over $\bbF^m$ with randomness space as $\bbF$ and commitment space as group $\bbG$ with independent generators $g_1,\ldots,g_m, h$. Define $c_{j} = \com(\rsoracle[\cdot,j],\delta_{j})$, $j\in [n]$ where the notation $X[\cdot,j]$ denotes the $m$-length vector $(X[1,j],\ldots,X[m,j])$ and $\delta_{j}$ denotes the randomness for computing the commitment $c_{j}$. We define the oracle $\comoracle$ as $\comoracle[j]=c_{j}$. The oracle $\comoracle$ answers queries of the type $Q\subseteq [n]$, responding with elements $\comoracle[j]$ for $j\in Q$.\footnote{Defining the commitment vector as $\comoracle$ might seem superfluous here. But, the usage of $\comoracle$ will play a prominent role in our 3D protocol. We just define the notation here for the uniformity in our descriptions of the 2D and the 3D versions.} 
%We will use $\comoracle$ as the witness
%oracle, and adapt the subprotocols for checking linear constraints, quadratic
%constraints and proximity to this oracle.

.\pnote{start}

\subsubsection{Commitment:}\label{subsec: commit} Then compute the commitment of $\oracle_x$. 
\pnote{Elaborate how to do the commitment.}
Let $\cm_x=(c_1,\ldots c_n)$ where 
$c_j= \com( \begin{bmatrix}
u_{1j} & u_{2j} & \ldots & u_{mj}
\end{bmatrix}^T)$ $\forall j\in [n]$

.\pnote{end}

\subsubsection{Linear Check with Commitment Oracle}\label{sec:lincheck2D}
The linear check $Ax=0$ can be reduced to checking $\innp{r^TA}{x}=0$, where the verifier samples a random $r\sample \bbF^{ml}$ and sends it to the prover. As in \cite{Ligero2017}, the prover and verifer write $r$ as $(r_{1}|\cdots|r_{m})$ where each $r_i\in \bbF^\ell$. Both the prover and the
verifier also compute degree $<\ell$ polynomials $\hat{r}_{i}$ interpolating the vector $r_{i}$ on $G$. The required check in terms of polynomials can be expressed as:
\begin{equation}\label{eq:lincheck2D}
\sum_{\zeta\in G}\sum_{j\in [m]}
\hat{r}_{i}(\zeta) \cdot \hat{f}^x_{i}(\zeta) = 0.
\end{equation}
The prover computes the polynomial $\hat{p}=\sum_{i\in[m]}\hat{r}_i \cdot \hat{f}^x_i$. This polynomial $\hat{p}$ of degree $< k+\ell-1$ is sent to the verifier who checks $\sum_{\zeta\in G}\hat{p}(\zeta)=0$. As in the Linear check for \cite{Ligero2017}, the verifier needs to check if the polynomial $\hat{p}$ is correctly computed from the witness oracle $\comoracle$ to guard against dishonest provers. Following \cite{Ligero2017}, it is enough for the verifier to query the polynomial evaluations at a constant number of locations, say $\eta_1,\ldots,\eta_q$ and then check that $\hat{p}$ is consistent at the queried locations. We observe that the verifier only uses these queried values to prove some inner-product relations with other (publicly-known) vectors. Hence, our idea is to make the prover use inner-product arguments to prove the consistency of $\hat{p}$ to the verifier.

We will now formally describe the Linear check for our protocol \name2d{}. It will check that a purported commitment oracle $\comoracle$ encodes witness $x$ satisfying the constraint $Ax=0$ for a public matrix $A$. As before, we assume $x\in \bbF^{|\C|}$ and $A\in \bbF^{|\C|\times |\C|}$ and
$|\C|=m\ell$ for some positive integers $m$ and $\ell$. We further assume that the prover has RS-encoded oracle $\rsoracle$ which opens to the commitment $\comoracle$ and is $e$-close to the interleaved code $L_1^{m}$. The prover and the verifier interact as follows: 
\begin{enumerate}[{\rm 1.}]
\item $\verifier\rightarrow\prover$: The verifier sends a random $r\sample \bbF^N$ to the prover.

\item Both $\verifier$ and $\prover$ compute $R = r^T A$, then interpolate polynomials $R^i:i\in [m]$ in the evaluation domain $G$ such that $R^i$ interpolates the vector $(R_{(i-1)n+1},\ldots,R_{in})$ for $i\in [m]$. 

\item $\prover\rightarrow\verifier$: The prover computes polynomial  $p(\cdot)=\sum_{i\in [m]}R^i(\cdot)\hat{f}^x_{i}(\cdot)$ and sends $p$ to the verifier.

\item $\verifier\rightarrow\prover$: The verifier samples $t$ distinct indices $j_1,\ldots,j_t$ from the set $[n]$ independently at random and sends the indices to the prover.

\item $\prover\leftrightarrow\verifier$: The prover wants to prove that $\sum_{i\in [m]} R^i(\eta_{j_u}) \cdot \rsoracle[i, j_u] = p(\eta_{j_u})$ for each $u \in [t]$.
That is, $\prover$ wants to prove the inner product $\innp{R_{j_u}}{\rsoracle[\cdot,j_u]} = p(\eta_{j_u})$, where $R_{j_u} = (R^1(\eta_{j_u}), \ldots, R^m(\eta_{j_u}))$ and $\rsoracle[\cdot, j_u]$ denotes the $m$-length vector $(\rsoracle[1, j_u], \ldots, \rsoracle[m, j_u])$.
To do so, $\prover$ runs an inner-product argument for each $u \in [t]$ with the verifier. The verifier proceeds if the arguments succeed for all $u \in [t]$.

\item The verifier also checks that $\sum_{k\in [\ell]}p(\zeta_k)=0$.
%\begin{itemize}
%\item It checks $\sum_{k\in [\ell]}p(\zeta_k)=0$.
%\item It checks consistency of $\rsoracle[\cdot, j_u]$ with commitment oracle $\comoracle$ for $u\in [t]$
%using the inner-product arguments. 
%\end{itemize}
\item The verifier accepts if all the above checks succeed.
\end{enumerate}
.\dnote{self: write the formal lemma.}.
\dnote{The proof should be straightforward when using zk inner-product arguments, right?}
-----------------------------------------------------------------------------------------------------------------------------------
\paragraph{\textbf{Linear Check:}}\label{sec: linearity} In this check, there is a publicly known matrix $A \in \bbF^{ml\times ml}$ and public vector $b\in \bbF^{ml}$, and $\prover$ claims that he knows a $x$ such that $Ax=b$ is true. Without loss of generality we can assume that the vector $b$ is the zero vector i.e. $b=\bm{0}$ $\prover$ starts with encoding the $x$ using interleaved $\RS$ code which is described above and gets $\oracle_x$, then he computes the commitment $\cm_x$ in the same way mentioned above. Now the following check ensures that if a maformed $\calU$ is used for $\oracle$, which is not far away from a correct encoding of $x$ for which $Ax\neq b$, then $\verifier$ outputs reject with very high probability.
\begin{enumerate}
	\item $\prover \rightarrow \verifier: $ $\prover$ sends $\cm_x$ to $\verifier$.
	\item $\verifier \rightarrow \prover: $ $\verifier$ picks a random $r\in \bbF^{ml}$ and sends that $r$ to $\prover$.
	\item $\prover$ and $\verifier$ compute $R=r^TA$, which is a vector of size $ml$. Read $R$ in a matrix form where first $l$ elements of $R$ forms the first row, next $l$ elements forms the second row, similary $R$ will have $m$ rows. Then they construct polynomials $R_i(\cdot)$ of degree $<l$ such that $R_i(\zeta_j)=R_{ij}$ $\forall i\in [m], j\in [l]$. 
	\item $\prover \rightarrow \verifier: $  $\prover$ computes a polynomial $p(\cdot)=\sum_{i\in[m]} ( R_i(\cdot)\cdot \hat{f}^x_i(\cdot))$ and sends $p(\cdot)$ to $\verifier$.
	\item $\verifier \rightarrow \prover: $ $\verifier$ samples $t$ distinct  indices $j_1,\ldots,j_t$ from the set $[n]$ independently at random and sends the indices to $\prover$.
	\item $\prover \leftrightarrow \verifier: $ $\prover$ and $\verifier$ run a subprotocol to prove the innerproduct arguement for the following statement $\innp{R_{j_u}}{\oracle_x[\cdot,j_u]} = p(\eta_{j_u})$ for all $j_u$ queried in the above step by $\verifier$, where $R_{j_u}=(R_1(\eta_{j_u}),\ldots,R_m(\eta_{j_u})$ and $\oracle_x[\cdot,j_u]$ denotes the $m-$length vector $(\oracle_x[1,j_u],\ldots, \oracle_x[m,j_u])$. $\verifier$ proceeds if the arguments succeed for all $u \in [t]$.
	\item $\verifier$ also checks that $\sum_{j\in[l]} p(\zeta_j)=0$.
	\item $\verifier$ accepts if all the above checks are succeed.	  
\end{enumerate}

\begin{theorem}
	For $e < \frac{d}{3} $, $\innp{\prover^*(\cm, \calU^*, A, b)}{\verifier(\cm, A, b)} \rightarrow 1 $ then there is an expected $\ppt$ $\extrac^{\prover^*}(\cm) \rightarrow \calU^*$ such that with overwhelming probability $\calU^*$ satisfies one of the following events:
	\begin{itemize}
		\item $\com(\calU^*)=\cm$ and $d(\oracle_x,L^m) > e $
		\item $\com(\calU^*)=\cm$ and $d(\oracle_x, l^m)\leq e \text{ and } x = \dec(\oracle_x)$ satisfies $ Ax = b$
		\item $\extrac$ breaks the commitment scheme.
	\end{itemize}
\end{theorem}

\begin{proof}
	We have a $\ppt$ extractor, $\extrac_{innp}$, for the inner product arguement for the statement $\innp{\bm{a}}{\bm{b}}=c$ where $\bm{a},\bm{b}$ are private, which can either extract $\bm{a}, \bm{b}$ or breaks the binding property of the commitment scheme with overwhelming probability. Now we will use $\extrac_{innp}$ to design a $\ppt$ extractor $\extrac$ which can extract the witness($\calU^*$) for which the arguement in the above protocol is accepted.
	
	$\extrac$ emulates $\verifier$'s role in the protocol till step 5, then calls $\extrac_{innp}$ to get $\calU^*[\cdot,j_u]$ or collision for the commitment. $\extrac$ stores the indices in a set say $S$ and rewind the prover to step 5 and picks $t$ indices again uniformly at random again and follows the above procedure. $\extrac$ keeps rewinding till $|S|=n$.
	
	If $|S|=n$, then $\extrac$ has the whole $\calU^*$.
	
	Then $\extrac$ computes $d(\calU^*, L^m)$:
	\begin{itemize}
		\item if $d(\calU^*,L^m) > e$ then $\extrac$ outputs $\calU^*$, which satisfies $\com(\calU^*)=\cm$ otherwise gets a collision for the binding property.
		
		\item if $\d(\calU^*,L^m) \leq e$ then by soundness analysis in \cite{Ligero2017}, the nearest codeword of $\calU^*$ decodes to $x$ such that $Ax\neq b$ has $\negl(\lambda)$ probabillity. That means nearest codeword of $\calU^*$ decodes to $x$ which satisfies $Ax=b$ with very high probability and $\com(\calU^*)=\cm$.
	\end{itemize}
	
	Similar analysis of Theorem:~\ref{theo:proximity} proves that $\extrac$ requires polynomially many rewinding to extract the witness $x$.
\end{proof}
-----------------------------------------------------------------------------------------------------------------------------
\subsubsection{Quadratic Check}\label{sec:quadcheck2D}
The quadratic check involves the prover convincing the verifier that $x\circ y=z$ by providing oracle access to the vectors $x,y$ and $z$. Our protocol follows from \cite{Ligero2017}, except for the final check by the verifier. To check that $\hat{p}$ is correctly computed from the oracles, the verifier queries the oracles at small number of points $\eta_1,\ldots,\eta_q$. The prover responds with an inner-product proof that proves to the verifier that $\hat{p}(\eta_j)=\sum_{i\in [m]}r_i(\hat{f}^x_i(\eta_j) \cdot \hat{f}^y_i(\eta_j)-\hat{f}^z_i(\eta_j))$ for $j\in [q]$.\\

We now formally describe the interactive oracle protocol for checking the relation $x\circ y = z$ for vectors $x,y,z\in \bbF^|\C|$. Let $\rsoracle_x, \rsoracle_y$ and $\rsoracle_z$ denote the encodings of vectors $x$, $y$ and $z$ respectively via the RS code. Let $\comoracle_x,\comoracle_y$ and $\comoracle_z$ denote the respective commitment oracles. The prover and the verifier interact as follows:

\begin{enumerate}[{\rm 1.}]
\item $\verifier\rightarrow\prover$: The verifier sends a challenge $r\sample \bbF^m$ to the prover.

\item $\prover\rightarrow\verifier$: The prover computes the polynomial $p(\cdot)=\sum_{i\in [m]} r_i(\hat{f}^i_x(\cdot) \cdot \hat{f}^i_y(\cdot) - \hat{f}^i_z(\cdot))$.
and sends it to the verifier.

\item $\verifier\rightarrow\prover$: The verifier sends $t$ randomly sampled indexes $\{j_u\}_{u\in [t]}$ from $[n]$. 

\item $\prover\leftrightarrow\verifier$: The prover wants to prove that 
$\sum_{i\in [m]} r_i(\rsoracle_x[i,j_u] \cdot \rsoracle_y[i,j_u]-\rsoracle_z[i,j_u] = p(\eta_{j_u})$ for each $u \in [t]$.
This is equivalent to the $\prover$ proving $\innp{r \circ \rsoracle_x[\cdot,j_u]}{\rsoracle_y[\cdot,j_u]} - \innp{r}{\rsoracle_z[\cdot,j_u]} = p(\eta_{j_u})$. There are two inner products in the above statement and the prover does not want to reveal the values of the individual inner products. Hence, \name2D{} combines them in to a single inner product relation.
For each $u \in [t]$, $\prover$ runs the following inner-product argument with the verifier: 
\[
\left\langle{\left( (r \circ \rsoracle_x[\cdot,j_u] \, || \, r \right)} , {\left( \rsoracle_y[\cdot,j_u] \, || \, -\rsoracle_z[\cdot,j_u] \right)} \right\rangle = p(\eta_{j_u})
\]
To facilitate this, $\comoracle_x,\comoracle_y$ and $\comoracle_z$ should have been committed with different independently chosen sets of generators.\footnote{By independent, we mean that the prover does not know the Diffie-Hellman relation between any pair of generators.} And, the set of generators used for committing $r$ (to be concatenated with $r \circ \rsoracle_x[\cdot,j_u])$) should also be independent of the above three sets of generators.
The verifier proceeds if the arguments succeed for all $u \in [t]$.

\item The verifier also checks that $\sum_{k\in [\ell]}p(\zeta_k)=0$ and accepts if it succeeds.
%\begin{itemize}
%\item It checks $\sum_{k\in [\ell]}p(\zeta_k)=0$.
%\item It checks consistency of $\rsoracle[\cdot, j_u]$ with commitment oracle $\comoracle$ for $u\in [t]$
%using the inner-product arguments. 
%\end{itemize}
\end{enumerate}
--------------------------------------------------------------------------------------------------------------------

\paragraph{\textbf{Quadratic Check:}}\label{sec: quadratic} In this check $\prover$ claims that he knows $x,y,z\in\bbF^{ml}$ such that $x \circ y = z$. $\prover$ starts with encoding each of $x, y, z$ and obtains $\oracle_x, \oracle_y, \oracle_z$. Then $\prover$ performs column wise commitments and gets $\cm_x, \cm_y, \cm_z$. The following check ensures that if any of the $\oracle_x, \oracle_y, \oracle_z$ is malformed, which is not far away from a correct encoding of any of the $x, y, z$, then $\verifier$ rejects with very high probability.
\begin{enumerate}
	\item $\prover \rightarrow \verifier: $ $\prover$ sends $\cm_x, \cm_y, \cm_z$ to $\verifier$.
	
	\item $\verifier \rightarrow \prover: $ $\verifier$ picks $r$ uniformly at random from $\bbF^{m}$ and sends $r$ to $\prover$.
	
	\item $\prover \rightarrow \verifier: $ $\prover$ computes the polynomial $p(\cdot)= \sum_{i\in [m]} [r_i\cdot (\hat{f}^x_i(\cdot)\cdot \hat{f}^y_i(\cdot) - \hat{f}^z_i(\cdot))] $ and sends $p(\cdot)$ to $\verifier$. 
	
	\item $\verifier \rightarrow \prover: $ $\verifier$ sends $t$ randomly sampled indices $Q=\{j_u\}_{u\in[t]}$ from $[n]$.
	
	\item $\prover \leftrightarrow \verifier: $ $\prover$ and $\verifier$ run a subprotocol to prove the innerproduct arguement for the following statement $\innp{r\circ \oracle_x[\cdot,j_u]}{\oracle_y[\cdot,j_u]} - \innp{r}{\oracle_z[\cdot,j_u]} = p(\eta_{j_u})$ $\forall u\in[t]$. 
	
	There are two inner products in the above statement and the prover does not want to reveal the values of the individual inner products. Hence, \name2D combine them into a single inner product relation.
	
	For each $u\in[t]$, $\prover$ runs the following inner product arguement with $\verifier$:
	$$\innp{(r\circ \oracle_x[\cdot,j_u]||r)}{(\oracle_y[\cdot,j_u]||-\oracle_z[\cdot,j_u])} = p(\eta_{j_u})$$
	To facilitate this, $\oracle_x, \oracle_y$ and $\oracle_z$ should have been committed with different independently chosen sets of generators. And, the set of generators used for committing $r$ (to be concatenated with $r \circ \oracle_x [\cdot, j_u])$ should also be independent of the above three sets of generators. $\verifier$ proceeds if the arguements succeed for all $u\in[t]$.
	\item $\verifier$ checks if $p(\zeta_j)=0$ $\forall j\in[l]$. If yes then accepts, else rejects.
\end{enumerate}

\begin{theorem}
	For $ e < \frac{d}{3}, $ if $\innp{\prover^*(\cm_1, \cm_2, \cm_3 , \calU^*_1	, \calU^*_2, \calU^*_3)}{\verifier(\cm_1, \cm_2, \cm_3)} \rightarrow 1$ then there is an expected $\ppt$  $\extrac^{\prover^*}(\cm_i)\rightarrow \calU^*_i$ $\forall i\in [3]$ such that with overwhelming probability $\forall i \in [3]$, $\calU^*_i$ satisfies one of the following event: 
	\begin{itemize}
		\item $\com(\calU^*_i) = \cm_i$ and $(\vee_{i=1}^{3} d(\calU^*_i, L^m)> e)$ 
		\item $\com(\calU^*_i) = \cm_i$ and $\wedge_{i=1}{3} d(\calU^*_i,L^m)\leq e$ and $ x_i = \dec(\calU_i)$, where $\calU_i$ is the nearest codeword to $\calU^*_i$, for all $i\in [3]$, such that $x_1 \circ x_2 = x_3$
		\item $\extrac$ breaks the binding of the commitment scheme.
	\end{itemize} 
	
\end{theorem}
\begin{proof}
	We have a $\ppt$ extractor, $\extrac_{innp}$, for the inner product arguement for the statement $\innp{\bm{a}}{\bm{b}}=c$ where $\bm{a},\bm{b}$ are private, which can extract either $\bm{a}, \bm{b}$ or breaks binding property of the commitment scheme with overwhelming probability. Now we will use $\extrac_{innp}$ to design an expected $\ppt$ extractor $\extrac$ which can extract $\calU^*_1, \cal^*_2, \cal^*_3$ for which the arguemnet is accepted in the above protocol.
	
	$\extrac$ emulates $\verifier$'s role in the protocol till step 4, then calls $\extrac_{innp}$ to get $(r\circ \calU^*_1[\cdot,j_u]||r)$ and $(\calU^*_2[\cdot,j_u]||-\calU^*_2[\cdot,j_u])$ forall $u\in [t]$ or breaks the binding property of the commitment scheme. $\extrac$ stores all the indices in a set say $S$. From the output of $\extrac_{innp}$, $\extrac$ computes $\calU^*_1[\cdot,j_u], \calU^*_2[\cdot,j_u],\calU^*_3[\cdot,j_u]$. $\extrac$ rewinds $\prover$ to step 4 and again picks a random $Q$. $\extrac$ keeps repeating the above process till $|S|=n$.
	
	If $|S|=n$, then $\extrac$ has the whole of $\calU^*_1, \calU^*_2, \calU^*_3$.
	
	Then $\extrac$ checks if $\com(\calU^*_i)=\cm_i$, if no, then outputs a collision for the binding of the commitment scheme, else computes $d(\calU^*_i, L^m)$ for all $i \in [3]$:
	
	\begin{itemize}
		\item if $\vee_{i=1}^{3} d(\calU^*_i, L^m) > e$, then outputs $\calU^*_i \forall i\in [3]$.
		
		\item if $\wedge_{i=1}^3 d(\calU^*_i, L^m) \leq e$, then by soundness analysis in \cite{Ligero2017}, the nearest codewords of $\calU^*_1, \calU^*_2, \calU^*_3$ decode to $x, y, z$ repesctively such that probability that $x \circ y \neq z$ is $\negl(\lambda)$. That means the decoded values of the nearest codewords of $\calU^*_1, \calU^*_2, \calU^*_3$ satisfy $x\circ y = z$ with very high probability.
	\end{itemize}
	
	Similar analysis of Theorem~\ref{theo:proximity} proves that $\extrac$ requires polynomially many rewinding to extract the witness $x, y, z$.
\end{proof}

----------------------------------------------------------------------------------------------------------------------------




\subsubsection{Proximity Protocol}\label{sec:proximity2D}
We describe a protocol for ``proximity'' of a purported codeword to the interleaved code. Let $U\in \bbF^{m\times n}$ denote the purported codeword and let $e< d/3$ denote the proximity parameter. 
\dnote{To Nitin: why did you not use the notation $\rsoracle$ for $U$? Is it that only those $U$s which satisfy the proximity test become $\rsoracle$?}
In its first message the prover computes commitments $c_1,\ldots,c_n$ to the columns of $U$ and sends it to verifier. Thereafter, the prover and the verifier interact as
follows:
\begin{enumerate}[{\rm 1.}]
\item $\prover\rightarrow \verifier$: Commitments $c_1,\ldots,c_n$ to the columns of $U$.
\item $\verifier\rightarrow \prover$: Verifier samples $\gamma \sample \bbF^n$ and sends to the prover.
\item $\prover\rightarrow \verifier$: The prover computes $u=\gamma^T U$ and sends it to the verifier. 
\item $\verifier\rightarrow \prover$: Verifier sends $t$ randomly sampled indexes $\{j_u\}_{u\in [t]}$ from $[n]$ to the prover.
\item $\prover\leftrightarrow\verifier$: The prover has to now prove to the verifier that $u_{j_u} = \gamma^T U[\cdot, j_u]$, for all $u \in [t]$. The prover and the verifier will run the inner-product argument to check $\innp{\gamma}{U[\cdot, j_u]} = w_{j_u}$ for all $u \in [t]$. The verifier proceeds if the checks succeed.
\item The verifier also checks if  $w\in L$ and accepts if it succeeds.
\end{enumerate}
.\pnote{conflict in notations: $u = \gamma^T U$ and $u \in [t]$}
---------------------------------------------------------------------------------------------------------------------------

\paragraph{\textbf{Proximity Check:}}\label{sec:proximity} In this check $\prover$ convinces $\verifier$ that $\cm_x$ is the commitment of a correct codeword matrix $\oracle_x$ which is encoding of a matrix $x$. If $\oracle_x$ is malformed, i.e. if it is sufficiently far away from a correct codeword matrix, then $\prover$ passes the following check with $\negl(t)$ probability.
\begin{enumerate}
	\item $\prover \rightarrow \verifier :$ $\prover$ sends $cm$ to $\verifier$.
	
	\item $\verifier \rightarrow \prover :$ $\verifier$ as a challenge picks $\gamma \in \bbF^m$ uniformly at random and sends it to $\prover$.
	
	\item $\prover \rightarrow \verifier :$ $\prover$ computes $u=\gamma^T\oracle_x$ and sends $u$ to $\verifier$.
	
	%\pnote{ instead of using $w = \gamma^T \oracle_{x}$, using $u$, to avoid conflict with witness.}
	\item $\verifier \rightarrow \prover :$ $\verifier$ picks a random subset $Q\subseteq [n]$ such that $|Q|=t$ and sends $Q$ to $\prover$.
	
	\item $\prover \leftrightarrow \verifier: $ $\prover$ and $\verifier$ run a subprotocol to prove the innerproduct arguement for the following statement $\innp{\gamma}{\oracle_x[\cdot,j_u]}=u_{j_u}$ $\forall j_u\in Q$
	
	\item If $\verifier$ accepts the innerproduct arguement in step 5, then checks if $w\in L$. If yes then $\verifier$ outputs accept else outputs reject.
\end{enumerate}

Note that the subprotocol for innerproduct arguement has the proof of knowledge property, proof given in \cite{Bulletproofs}. So there is an expected $\ppt$ extractor $\extrac_{innp}$ for the innerproduct argument, which either outputs the witness or breaks the binding property of the committment, if the arguement is accepted.
\begin{theorem}\label{theo:proximity}
	For $e < \frac{d}{3} $, if $\innp{\prover^*(\cm, \calU^*)}{\verifier(\cm)} $ is true, then there is an expected $\ppt$ $\extrac^{\prover^*}(\cm) \rightarrow \calU^*$ such that with overwhelming probability $\calU^*$ satisfies one of the following events: 
	\begin{itemize}
		\item $\com(\calU^*) = \cm$ and $ d(\calU^*, L^m) < e$
		\item $\extrac$ breaks the binding of the commitment. 
	\end{itemize} 
	
\end{theorem}

\begin{proof}
	Now we will design a $\ppt$ $\extrac$ for the above protocol.
	In step 4 $\extrac$ emulates $\verifier$ and picks $Q$ uniformly at random, then if innerproduct arguement is accepted then run $\extrac_{innp}$, in polytime with overwhelming probability. $\extrac_{innp}$ outputs either $\calU^*[\cdot,j_u]$ $\forall j_u\in Q$ or breaks the binding property of the commitment scheme, and stores the indices of $Q$ in a set $S$. then rewinds to the step 4 and then again picks $Q$ uniformly at random, $\extrac$ keeps extracting $\calU^*[\cdot,j_u]$ and updates $S$ by including only the new indices repeating until $|S|=n$. That is $\extrac$ extracts the whole $\calU^*$.
	
	Now $\extrac$ checks computes $d(\calU^*, L^m)$. Probability that $d(\calU^*,L^m) > e$ is $\negl(\lambda)$, by soundness analysis in \cite{Ligero2017}.
	
	Therefore $\extrac$ outputs $\calU^*$ whcih satisfies $\com(\calU^*) = \cm$ such that $d(\calU^*, L^m) < e$ with very high probability or it breaks the binding of the commitment scheme.
	
	Now we need to prove that the number of rewinding is polynomial.
	
	To get the bound on the expected number of rewindings consider $t=1$. 
	
	Let $X$ be the discrete random variable that represents the number of purchases until each of the $n$ column is picked at least once.
	
	Let $X_i$ be the discrete random variable that represents the number of rewindings after the $(i-1)^{th}$ distinct column to select the $i^{th}$ distinct column. As a base case, $X_1=1$, because the first column selected will always be distinct.
	
	By Linearity of expectation, $E[X]=\sum_{i=1}^{n}E[X_i]$
	
	After the $(i-1)^{th}$ distinct column is picked, there are $n-i+1$ columns remaining to be picked. Let $A_i$ be the event that one of those columns is picked in the next rewinding. Then $Pr(A_i)=\frac{n-i+1}{n}$.
	
	$X_i$ follows a geometric distribution (of trials). It's expected value is
	$$E[X_i]= \frac{1}{Pr(A_i)} = \frac{n}{n-i+1}$$
	From before, $E[X]$ is equal to the sum of all these expectations: i.e.
	$$E[X] = \sum\limits_{i=1}^{n} \frac{n}{n-i+1} = n\sum\limits_{i=1}^{n} \frac{1}{i} \approx n \log n$$
	Which is polynomial in $n$ and so polynomial in security parameter.
	%\pnote{Remaining analysis is to check the expected number of rewinding required to obtain $\oracle$.}
\end{proof}

-------------------------------------------------------------------------------------------------------

\paragraph{\textbf{Complete protocol:}} Let $L$ be a language in $\NP$ and $\stmt$ is an instance of $L$. Let $\prover$ be a prover that claims that the instance $\stmt$ is true, i.e. $\prover$ has a witness $\wit$ such that there is deterministic circuit $\C$ such that $\C(\stmt,\wit)=1$ iff $\stmt \in L$. 

Now $\prover$ wants to convince a verifier $\verifier$ that $\stmt\in L$ without revealing any information about the witness $\wit$. 

To do that $\prover$ gives a proof that $\C$ on input $(\stmt, \wit)$ is correctly executed and output 1. In other words $\prover$ proves that gate by gate evaluation is correctly done on an input which has a public part known to both $\prover$ and $\verifier$, and a private part which is only known to $\prover$. 
Note that $\C$ and $x$ both known to $\prover$ and $\verifier$. So without loss of generality we can assume that $\stmt$ is hardcoded in $\C$.

$\prover$ constructs the extended witness $\extwit$ in the following way: \\
Let $\C:\bbF^{n_i}\rightarrow \bbF$ such that $\prover$ has private input $\wit = (\wit_1,\ldots, \wit_{n_i})$ such that $\C(\wit)=1$.

Define the extended witness $\extwit = (\wit_1,\ldots,\wit_{n_i}, \beta_1,\ldots, \beta_s) \in \bbF^{ml}$, where $\beta_i$ is the output of the $i^{th}$ gate evaluating $\C(\wit)$, and $s$ is the number of gates in $\C$ and $ml>n_i + s$. $\prover$ defines a system of constraints that contains the following constraint for every multiplication gate g in the circuit $\C$ $$\beta_{a}.\beta_{b}-\beta{c}=0$$
and for every addition gate, the constraint 
$$\beta_a + \beta_b - \beta_c = 0$$
Where $\beta_a$, $\beta_b$ are the input values to the gate g and $\beta_c$ is the output value in the extended witness. For the output gate include the constraint $\beta_a + \beta_b - 1 = 0$ if the final gate is an addition gate, and $\beta_a\cdot \beta_b - 1 = 0$ if the final gate is an multiplication gate. 

$\prover$ constructs vectors $x,y,z \in \bbF^{ml}$ where the $j^{th}$ entry of $x,y$ and $z$ contains the values $\beta_a, \beta_b$ and $\beta_c$ corresponding to the $j^{th}$ multiplication gate in $\extwit$.

$\prover$ and $\verifier$ construct matrices $A, B$ and $C \in \bbF^{ml \times ml}$ such that 
$$x = A \extwit, y= B \extwit, z= C \extwit$$

Finally it constructs $P_{add} \in \bbF^{ml \times ml}$ such that the $j^{th}$ position of $P_{add} \extwit$ equals $\beta_a + \beta_b - \beta_c$ where $a, b$ and $c$ correspond to the $j^{th}$ addition gate of the circuit in $\extwit$.

$\prover$ encodes $\extwit , x, y , z$ using the defined encoding in sec:~\ref{subsec: encode} and gets $\oracle_{\extwit}, \oracle_x, \oracle_y$ and $\oracle_z$ and commit each of them using Pedersen vector commitment described in ~\ref{subsec: commit} and gets $\cm_{\extwit}, \cm_x, \cm_y, \cm_z$. To commit, pick different set of generators for each $\oracle_{\extwit}, \oracle_x, \oracle_y, \oracle_z$, which will give the property that commitment of $\oracle_{\extwit||x}$ is $\cm_{\extwit} \circ \cm_x$. Where $\extwit||x$ means that a new matrix is formed by adjoining columns of $\extwit$ followed by the columns of $x$. 

In the following protocol, $\verifier$ picks a random index set $Q\subseteq [n]$ of size $t$ and uses this $Q$ in all the followimg subprotocols.
\begin{enumerate}
	\item $\prover \leftrightarrow \verifier: $ $\prover$ and $\verifier$ run the subprotocol ~\ref{sec:proximity} for proximity check for the matrix $\extwit||x||y||z$. Which is encoded to the matrix $\oracle{\extwit||x||y||z}$, and $\cm_{\extwit||x||y||z}$ is the corresponding commitment.
	
	\item $\prover \leftrightarrow \verifier: $ $\prover$ and $\verifier$ run the subprotocol ~\ref{sec: linearity} for the linearity check for the public matrix $P_{add}$ of dimension $ml\times ml$, and the public vector is the 0-vector of length $ml$. This check ensures that the addition gates are corretly evaluated.
	
	\item $\prover \leftrightarrow \verifier:$ $\prover$ and $\verifier$ run the subprotocol ~\ref{sec: linearity} for linearity check for the public matrix $[A|-I]$ where $I$ is the identity matrix of dimension $ml \times ml$, and the public vector is the 0 vector of length $2ml$. This check ensures that $x$ is correctly computed from $\extwit$.
	
	\item $\prover \leftrightarrow \verifier:$ $\prover$ and $\verifier$ run the subprotocol ~\ref{sec: linearity} for linearity check for the public matrix $[B|-I]$ where $I$ is the identity matrix of dimension $ml \times ml$, and the public vector is the 0 vector of length $2ml$.This check ensures that $y$ is correctly computed from $\extwit$.
	
	\item $\prover \leftrightarrow \verifier:$ $\prover$ and $\verifier$ run the subprotocol ~\ref{sec: linearity} for linearity check for the public matrix $[C|-I]$ where $I$ is the identity matrix of dimension $ml \times ml$, and the public vector is the 0 vector of length $2ml$.This check ensures that $z$ is correctly computed from $\extwit$.
	
	\item $\prover \leftrightarrow \verifier:$ $\prover$ and $\verifier$ run the subprotocol ~\ref{sec: quadratic} for quadratic check to prove that $x \circ y = z$.
	
	\item If $\prover$ passes all the above check, then $\verifier$ accepts the arguement, else rejects.
\end{enumerate}

Note that if $\prover$ executed all the steps correctly, then it passes all the checks and so $\verifier$ accepts the proof. So completeness holds for the protocol.

To prove the proof of knowledge we will design an extractor which will output a witness if the arguement is accepted.

\begin{theorem}
	For $e < \frac{d}{3}$, if $\innp{\prover^*(\cm_{\extwit||x||y||z}, \calU^*)}{\verifier(\cm_{\extwit||x||y||z})} \rightarrow 1$, then there is an expected $\ppt$ $\extrac^{\prover^*}(\cm_{\extwit||x||y||z}) \rightarrow \calU^*$ such that with overwhelming probability $\calU^*$ satisfies one of the following events:
	\begin{itemize}
		\item $\com(\calU^*)=\cm_{\extwit||x||y||z}$ and $A\extwit = x \wedge B \extwit = y \wedge C \extwit = z \wedge x\circ y =z \wedge P_{add} \extwit = 0$
		\item $\extrac$ breaks the binding property of the commitment scheme.
		
	\end{itemize}
	Let $\cm$ is the commitment of $\calU$, used in above protocol by $\prover$. If $\verifier$ accepts the arguement generated by $\prover$, then there is an expected $\ppt$ extrator $\extrac$ having rewinding access to $\prover$, with polynomially many rewindings either outputs a correct $\extwit$ or breaks the binding property of the commitment scheme.
\end{theorem}
\begin{proof}
	We have extractors for the proximity check, linearity check and quadratic check, say $\extrac_{prox}, \extrac_{lin}, \extrac_{quad}$ are the extractors respectively. Using these extractors we will design  $\extrac$ for the complete protocol. 
	
	$\extrac$ emulates the role of the verifier and starts the protocol. 
	
	In the first step it calls $\extrac_{prox}$ and $\calU_1^*$.
	
	After executing the first step it calls $\extrac_{lin}$ and gets $\calU_2^*$. If $\calU_1^* = \calU_2^*$, then outputs collision and terminates.
	
	After executing the second if it is not terminated, it calls $\extrac_{lin}$ in the step 3, 4, 5 and gets $\calU_3^*$. 
	
	In the above 2 steps it gets collision that breaks the binding of the commitment otherwise proceeds.
	
	In this step $\extrac$ calls $\extrac_{quad}$, by concatanating the output of $\extrac_{quad}$ construct final matrix which equates with $\calU_1^* (=\calU_2^*=\calU_3^*)$. If not then that gives break of the binding property of the commitment scheme. Otherwise $\calU^*=\calU_1^*$ should satisfy the following:
	\begin{itemize}
		\item $\com(\calU*)=\cm_{\extwit}\cdot\cm_x\cdot\cm_y\cdot\cm_z$.
		\item $d(\calU^*,L^{4m}) < e$ and let $\calU$ is the closest codeword of $\calU^*$. Define $\oracle_{\extwit}$ to be the first $m$ rows of $\calU$
		
		Define $\oracle_{x}$ to be the $(m+1)^{th}$ rows to $2m^{th}$ rows of $\calU$
		
		Define $\oracle_{y}$ to be the $(2m+1)^{th}$ rows to $3m^{th}$ rows of $\calU$
		
		Define $\oracle_{z}$ to be the $(3m+1)^{th}$ rows to $4m^{th}$ rows of $\calU$
		
		Let $w= \dec(\oracle_{\extwit}), x=\dec(\oracle_x), y=\dec(\oracle_y), z=\dec(\oracle_z)$ such that
		
		$A\extwit = x \wedge B \extwit = y \wedge C \extwit = z \wedge x\circ y =z \wedge P_{add} \extwit = 0$.
	\end{itemize}
	Therefore $\extwit$ is a correct witness. Since all the above extractors use polynomial number of rewindings, $\extrac$ uses polynomial number of rewindings.		
\end{proof}
%--------------------------------------------------------------------------------------------------------------------------------------
\subsection{Zero - Knowledge}
Above defined three subprotocols are not zero-knowledge inherently. But converting them into zero-knowledge is easy. 

Consider the proximity check: In this subprotocol, $\verifier$ is learning $u = \gamma^T\oracle_{x}$, whihc he can't compute on his own. To prevent that modify $\oracle_{x}$ in the following way: include a random codeword in $(m+1)^{th}$ row, which blinds $u$ and makes $u$ a random codeword. 
In the remaining part, $\prover$ sends $\cm$, hiding property of the commitment scheme ensures that it does reveal any information.
Innerproduct proofs are given for $t$ columns. Instead of giving the proof, If $\prover$ opens $t$ columns of $\oracle_{x}$, still it does not reveal any information about $x$, since $t$ is smaller than the degrees of the polynomials used in $\enc$.

Consider the linear check: In the first step $\prover$ sends $\cm_x$ to $\verifier$. The hiding property of the commitment scheme ensures that $\verifier$ is getting no information about $\calU^*_x$ or $x$.
Then $\prover$ sends $p(\cdot) = \sum_{i\in[m]} R_i \cdot \hat{f}^x_i(\cdot)$ to check that if $\sum_{j\in [l]} p(\zeta_j) = 0$. But $\verifier$ instead of learning whether $\sum_{j\in[l]} p(\zeta_j) = 0$ or not, gets the complete polynomial $p(\cdot)$. To avoid leaking additional information we need to blind $p(\cdot)$ by adding a blinding polynomial $p_{blind}(\cdot)$ of degree $< k + l - 1$ such that $\sum_{j\in[l]} p_{blind}(\zeta_j) = 0$. Include a new row to $\oracle_x$ at the end where $j^{th}$ entry of the row is $p_{blind}(\eta_j)$ $\forall j\in [n]$.
Since number of inner product proofs is less than $t$, no information is leaked.

Consider the quadratic check: In the first step commitments do not leak any information about the witness or it's encoded values.
Then $\prover$ sends $p(\cdot) = \sum_{i\in[m]} [r_i\cdot (\hat{f}^x_i(\cdot)\cdot \hat{f}^y_i(\cdot) - \hat{f}^z_i(\cdot)]$. $\verifier$ is allowed to learn only if $p(\zeta_j)=0$ or not for all $j\in[l]$. To avoid leaking more information about $p(\cdot)$ we need to blind it. To do that pick a random polynomial $p_{blind}(\cdot)$ such that $p_{blind}(\zeta_j) = 0$ $\forall j\in [l]$. accordingly update $\oracle_x, \oracle_y,\oracle_z$ in the following way: pick there random codewords which are encodings of zeros, and append one of them each at the last of $\oracle_{x}, \oracle_{y}, \oracle_{z}$.
Inner product arguement is same as Proximity and linear check, it does not require any changes. 

\paragraph{Deisgning the simulator for the complete protocol: } In an actual execution of the protocol generates a transcript of the form:
\begin{align*}
\tau = \{ 
& \cm_{\extwit||x||y||z},\\
& \gamma(\in_R \bbF^{m}), r_1(\in_R \bbF^{ml}), r_2(\in_R \bbF^m), \\ 
& (u' = u+u_{blind}), (q^{lin}(\cdot) = p^{lin}(\cdot)+p^{lin}_{blind}(\cdot)), (q^{quad}(\cdot) = p^{quad}(\cdot) + p^{quad}_{blind}(\cdot)),\\
& Q (|Q|=t),\\
& \text{ inner product proof for } \\
& (\innp{\gamma}{\oracle_{\extwit||x||y||z}[\cdot, Q]} = u_Q, \innp{R_Q}{\oracle_{\extwit||x||y||z}[\cdot, Q]}=q^{lin}(\eta_Q),\\
& \innp{(r\circ \oracle_x[\cdot,j_u]||r)}{(\oracle_y[\cdot,j_u]||-\oracle_z[\cdot,j_u])} = q^{quad}(\eta_{j_u})
\}
\end{align*}
Consider a protocol, which is same as above protocol with the difference that $\prover$ instead of proving the inner product arguements opens the corresponding columns of $\oracle$. If this new protocol has zero knowledge property then our protocol also have zero knowledge property, since in our protocol whatever $\verifier$ can compute, $\verifier$ of the new protocol can also compute. It is easy to prove this by reduction.

Now we will prove that the new protocol is zero-knowledge. It will have the transcript of the following form:
\begin{align*}
\tau' = \{
& \cm_{\extwit||x||y||z},\\
& \gamma(\in_R \bbF^{m}), r_1(\in_R \bbF^{ml}), r_2(\in_R \bbF^m), \\ 
& (u' = u+u_{blind}), (q^{lin}(\cdot) = p^{lin}(\cdot)+p^{lin}_{blind}(\cdot)), (q^{quad}(\cdot) = p^{quad}(\cdot) + p^{quad}_{blind}(\cdot)),\\
& Q (|Q|=t),\\
& \oracle_{\extwit}[\cdot, Q], \oracle_{x}[\cdot, Q], \oracle_{y}[\cdot, Q], \oracle_{z}[\cdot, Q]
\}
\end{align*}

Let $\Sim$ be the simulator. $\Sim$ does the following:
\begin{itemize}
	\item  picks a random subste $Q$ of size $t$.
	\item  uniformly at random chooses $\gamma \in \bbF^m$.
	\item  uniformly at random chooses $r_1 \in \bbF^{ml}$ and $r_2 \in \bbF^m$.
	\item  chooses $t$ columns for $\oracle_{\extwit||x||y||z}$ according to the indices of $Q$.
	\item  computes the commitment of columns indexed by $Q$ for $\oracle_{\extwit||x||y||z}$, and for remaining positions picks uniform values from the range of $\com$, that fixes $\cm_{\extwit||x||y||z}$.
	\item  computes components of $u'$ indexed by $Q$ using $\oracle_{\extwit||x||y||z}$ and $\gamma$. Out of $n$ for remaining $n-t$ picks values for $u'$ in such a way that $u'$ is a valid coedword.
	\item  picks a random polynomial $q^{lin}(\cdot)$ such that degree is $<k+l-1$ and $\sum_{j\in [l]} q^{lin}(\zeta_j) = 0$.
	\item  picks a random polynomial $q^{quad}(\cdot)$ such that degree is $<2k-1$ and $q^{quad}(\zeta_j) = 0$ $\forall j\in [l]$.
\end{itemize} 
Then $\Sim$ outputs a transcript $\tau''$ which is computationally indistinguishable from $\tau'$. Therefore the new protocol has zero knowledge property, and hence \name2D has zero-knowledge property.

----------------------------------------------------------------------------------------------------------------------


\subsection{The pitfalls of 2D}
The need for 3D