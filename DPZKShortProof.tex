%\section{Setting up \name{}} 
\section{MPC Friendly Zero Knowledge Argument from Interactive PCP}
In this section, we describe a sub-linear interactive zero knowledge argument as
an Interactive PCP (IPCP). The IPCP we describe admits an efficient MPC to
convert it to a publicly-verifiable, non-interactive argument of knowledge,
using the ``BCS'' transformation in \cite{BCS16}. To serve as a
baseline for comparison, we also describe naive extensions of existing protocols
such as Bulletproofs \cite{bulletproofs}, Spartan \cite{spartan} to the
distributed prover setting. 

Our protocol proves membership in the $\npol$ complete
language specified by {\em rank one constraint system} (R1CS). An R1CS over a
field $\FF$ is specified by $M\times N$ matrices $A,B$ and $C$, where the
associated language $\mc{L}(A,B,C)$ consists of $\wit\in \FF^N$ such that
$A\wit\circ B\wit=C\wit$. We follow the broad outline of the interactive PCP
construction in \cite{ligero} which includes: (i) encoding the witness via suitable
error correcting code, (ii) checking linear relations on the witness, with
sub-linear access to the witness encoding, (iii) checking quadratic relations on
the witness with sub-linear access to witness encoding and (iv) a proximity
protocol to check that witness is correctly encoded. However, one runs into 
challenges in converting the IPCP to a non-interactive
argument when the witness is distributed across several provers. The naive
approach of provers sharing their encoded witnesses with an aggregator, which
constructs the oracle breaches privacy among the provers, as these encodings
only support {\em bounded independence}, i.e, they maintain privacy only when a
bounded part of the witness is revealed. We overcome this, and several other
technical challenges in our construction. The core techniques behind our
construction are summarized below:
\begin{itemize}
\item We use homomorphic commitment over witness encoding and provide oracle
access to the commitment. This facilitates oracle realization through
aggregation.
\item We design protocols for checking linear constraints, quadratic constraints
and proximity to correct encoding with oracle access to the commitment.
\item We come up with new witness encoding scheme, to facilitate smaller
argument size, and reduce communication between the provers during proof
generation.
\end{itemize}
While our use of homomorphic commitments introduces expensive cryptographic
operations, we keep their usage strictly sub-linear. In particular, for an
argument size of $O(N^{1/c})$, verification and aggregation incur $O(N^{1-2/c})$
exponentiations, while the prover incurs $O(N/\log N)$ exponentiations. We also
compare the concrete performance of our standalone SNARK with state of the art in Section
\ref{sec:snarkcomparison}.
