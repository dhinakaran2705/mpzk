\section{Relevant prior art: an overview} 
We will present our DPZK protocol \dpname{} by first explaining the single prover version of our protocol. 
\dnote{adapt this part based on the summary of the DPZK derived from the existing single prover versions}

Before that, we present an overview of the techniques underlying our interactive protocol
with succinct argument size and efficient verification with sublinear public
key operations and sublinear field operations in expectation. In our
presentation of the core ideas, we first give an informal overview of the
existing techniques, and then discuss our specific modifications to them.

Our starting point is the Interactive PCP argument presented in Ligero\cite{Ligero2017}. The
aforementioned protocol uses elementary techniques, is concretely
efficient being devoid of public key operations, and achieves an argument
size which is square-root in size of the arithmetic circuit. Our core technical
contribution is to use homomorphic commitments and inner product arguments to 
further reduce the size of the ``queries'' to the witness oracle. The
reduced access to the witness oracle comes with another surprising benefit: The
verifier restricts its computations to those directly relevant to the parts of
witness being revealed. This allows us to achieve sublinear verification on
average. By carefully restricting the sub-protocols using expensive public key
operations to asymptotically smaller circuits (much smaller for practically
relevant instantiations), we keep the number of public key operations for the
verifier to be strictly sublinear.
    
\subsection{Arguments for Arithmetic Circuits}
Our protocol natively supports showing the 
satisfiability of the $\FF$-arithmetic circuit for a finite field $\FF$ which
is an $\npol$-complete language. For ease of presentation, we produce an argument for the
statement ``$\exists x, \text{ s.t } C(x)=1$'' for an $\FF$-arithmetic circuit
$C$. This can be easily modified to
yield arguments for the more typical $\npol$ language $\mc{L}_C = \{x:\exists w
\text{ s.t. } C(x,w)=1\}$, which we will briefly discuss later. The
satisfiability of an arithmetic circuit $C$ reduces to proving existence of
vectors $x,y,z,w$ such that:
$z = x\circ y$,
$x = P_x w$,
$y = P_y w$,
$z = P_z w$ and
$P_{add} w = 0$ for public matrices $P_x,P_y$, $P_z$ and $P_{add}$ (depending on
$C$). Thus, broadly in the IPCP setting we have the following protocol:
\begin{enumerate}
\item {\bf Oracle Setup}: The prover sets up witness oracle
$x,y,z,w$ and provides query access to the verifier. 
\item {\bf Linear Checks}: The verifier runs a subprotocol with the prover to verify
the linear constraints $x=P_x w$, $y=P_y w$, $z=P_z w$, $P_{add} w = 0$.
\item {\bf Quadratic Check}: The verifier runs a subprotocol with the prover to
verify $z=x\circ y$.
\end{enumerate}

\subsection{Interactive Protocols for RS-code Oracles}
\subsubsection{Reed Solomon Witness Encoding}
The linear and quadratic constraints discussed above do not naively admit a
sublinear query, i.e, the verifier needs to access complete vectors to
be convinced with high probability. Error correcting codes have been used to
encode the witness in PCP constructions to enable verification with sublinear
query. We discuss two such encodings based on Reed-Solomon codes which have
been used in recent constructions \cite{Ligero2017,Aurora,STARK2019}.

To encode a vector $x\in \FF^N$, one specifies two domains $G,H\subseteq \FF$.
We will call $G$ as {\em interpolation} domain and $H$ as {\em evaluation} domain.
The encoding in \cite{Aurora} encodes the vector $x$ as a single Reed-Solomon
codeword. This is done by first constructing a polynomial $\hat{f}^x$ which
interpolates the vector $x$ on $G$, and then computing its evaluations
$\langle \hat{f}^x(\alpha) \rangle_{\alpha\in H}$ on points in $H$. The sizes
of domains $G$ and $H$ need to be $\Omega(N)$ in the above encoding. For a
vector $x$, we will use the notation $\hat{f}^x$ to denote the polynomial
interpolating $x$ on $G$, and $f^x$ to denote the vector of evaluations of
$\hat{f}^x$ on $H$. Thus in the above scheme, $f^x$ is an encoding of the
vector $x$.

An alternative encoding used in \cite{Ligero2017} encodes parts of a vector
separately, and thus the encoded vector corresponds to a set of Reed-Solomon
codewords, or a single codeword of an Interleaved Reed-Solomon code(see
Definition \ref{defn:interleaved}). More specifically, one chooses integers $m$
and $\ell$ such that $m\ell > N$ and domains $G$ and $H$ of size
$\Omega(\ell)$. The vector $x\in \FF^N$ is written as
$x=(x_1|\cdots|x_m)$ where $x_i\in \FF^l$ for all $i\in [m]$. The vector $x$ is
encoded as $(f^x_1,\ldots,f^x_m)$ where $f^x_i$ encodes $x_i$ as described
before, i.e $f^x_i=\langle \hat{f}^x_i(\alpha)\rangle_{\alpha\in H}$ where the
polynomial $\hat{f}^x_i$ interpolates the vector $x_i$ on $G$. 

\subsubsection{Linear Check}
Without loss of generality, we will discuss an argument for proving linear constraints
of the form $Ax=0$ where $A$ is a public matrix and $x$ is a (secret) vector. The
case $Ax=y$ is easily transformed to the required form by defining
$A'=[A|-I]$ and $x'=(x,y)$, and running the protocol on $A'$ and
$x'$. Assume that $x\in \FF^N$ and $A\in \FF^{N\times N}$. To achieve sublinear query
complexity, we do two things: (i) we make a high probability reduction to the
problem of proving $\innp{r^TA}{x}=0$ where $r\sample \FF^N$ is sent by
the verifier (ii) encode the witness $x$ using Reed-Solomon code as
we described earlier.

We first consider the encoding used in \cite{Aurora} which encodes $x$ as a
single codeword $f^x$. The prover then provides oracle access to $f^x$. Let
$\hat{r}\in\FF[x]$ be the polynomial that interpolates the vector $r^TA$ on $G$.
Then the linear check reduces to checking $\sum_{\alpha\in
G}\hat{r}(\alpha) \cdot \hat{f}^x(\alpha) = 0$. In \cite{Aurora}, the aforementioned
identity is checked with query size of $O(\log|H|)$ using especially developed
sumcheck protocol for univariate polynomials. The verifier still incurs $O(N)$
work to compute the encoding $\hat{r}$ for the size $N$ vector $r^TA$. 

We now consider the interleaved encoding of the witness. 
Looking ahead, the interleaved encoding will have a direct impact on the circuit share complexity 
of the distributed proof generation in \name{}.
Here, we write the vector $x\in \FF^N$ as $x=(x_1|\cdots|x_m)$
where each $x_i\in \FF^\ell$ for some $m\ell > N$, and $x_m$ padded as necessary. For
suitable domains $G$ and $H$, we interpolate each chunk of the vector on $G$ 
separately via 
polynomials $\hat{f}^x_1,\ldots,\hat{f}^x_m$. Similarly we write
$r^TA=(r_1,\ldots,r_m)$ with $r_i\in \FF^n$ and construct polynomials $\hat{r}_i$
for $i\in [m]$. The inner product check $\innp{r^TA}{x}=0$ is then equivalent to checking
$\sum_{i\in [m]}\sum_{\alpha\in G}\hat{r}_i(\alpha)\hat{f}^x_i(\alpha)=0$. The
latter is checked by the prover sending the polynomial $\hat{p}=\sum_{i\in
[m]}\hat{r}_i.\hat{f}^x_i$ to the verifier, and verifier checking
$\sum_{\alpha\in H}\hat{p}(\alpha)=0$. Choosing $m,\ell\approx O(N^\frac{1}{2})$ this
incurs square-root communication from the prover. How can the verifier be
sure that $\hat{p}$ was indeed computed correctly from the witness polynomials 
$\hat{f}^i_x$?. As we formally prove in the later sections, this can be
accomplished by the verifier querying the polynomial evaluations (oracles) at a constant
number of locations, say $\alpha_1,\ldots,\alpha_q$. The verifier then checks
that $\hat{p}$ is consistent at the queried locations by verifying
$\hat{p}(\alpha_j)=\sum_{i\in [m]}\hat{r}_i(\alpha_j)\hat{f}^x_i(\alpha_j)$ for
all $j\in [q]$. This incurs a total query complexity of $q.m=O(N^\frac{1}{2})$.
From the perspective of 
verifier's efficiency, it only needs to compute evaluations of
polynomials 
$\hat{r}_i$, $i\in [m]$ for $\alpha\in \{\alpha_1,\ldots,\alpha_q\}$. We will show
that this can be done in expected sublinear time, after a one time $O(||
A||)$ 
pre-processing. We will also discuss why a similar pre-processing does not work
for the earlier encoding scheme.

\subsubsection{Quadratic Check}
The quadratic check involves the prover convincing the verifier that $x\circ
y=z$ by providing oracle access to the vectors $x,y$ and $z$. Again, we
consider the encoding by parts we discussed for the linear check. We write
$x=(x_1|\cdots|x_m)$, $y=(y_1|\cdots|y_m)$ and $z=(z_1|\cdots|z_m)$ and
construct polynomials $\hat{f}^x_i$, $\hat{f}^y_i$ and $\hat{f}^z_i$ for $i\in
[m]$ as before. The quadratic check then reduces to showing that
$\hat{f}^x_i(\alpha).\hat{f}^y_i(\alpha)-\hat{f}^z_i(\alpha)=0$ for all $i\in
[m]$ and $\alpha\in G$. With high probability the above can be checked by
verifier sending a random vector $r\sample \FF^m$ to the prover, and prover
sending the polynomial $\hat{p}=\sum_{i\in
[m]}r_i(\hat{f}^x_i.\hat{f}^y_i-\hat{f}^z_i)$ to the verifier. The verifier
checks that $\hat{p}(\alpha)=0$ for all $\alpha\in G$. The verifier also checks
that $\hat{p}$ is correctly computed from the oracles by querying the oracles
at small number of points $\alpha_1,\ldots,\alpha_q$ and checking that
$\hat{p}(\alpha_j)=\sum_{i\in
[m]}r_i(\hat{f}^x_i(\alpha_j).\hat{f}^y_i(\alpha_j)-\hat{f}^z_i(\alpha_j))$ for
$j\in [q]$.

\subsubsection{Proximity Test}
The correctness of the previous two checks, namely the linear check and the
quadratic check rely on the fact that the witness oracles are evaluations of
``low'' degree polynomials. There are several known low degree tests for polynomials
from PCP literature. The protocol in \cite{Aurora} uses a recent test for
proximity by Ben Sesson et al.\cite{IOPP_FRI2018} with particularly efficient
prover and $O(\log d)$ query complexity for polynomials of degree at most $d$. 
We use a variant of proximity test from \cite{Ligero2017}, adapting it to work
with homomorphic commitments of the RS-encoded oracles and reducing the query
complexity.


[DPZKShortProof]
1. The use for commitments --- 2. Explain the 2D version  

[DPZKQuickVerify]
3. The need for 3D --- 4. Our 3D version.


\section{DPZK with reduced proof size}
1. The use for commitments --- 2. Explain the 2D version  
An brief recall on how our paradigm enables any improvement in single prover ZK protocol directly translate to the DP version.\\

The first part of \name{} is obtaining the proof size of $N^{1/c}$ for any $c \geq 2$.
We will make use of homomorphic commitments and inner product arguments to remove the linear dependence of the proof size on $m$ in \cite{ligero}. A brief summary of our idea follows. We will commit to the vector $\{\hat{f}^x_i(\alpha)\}_{i \in [m]}$ using the homomorphic vector commitments. The vectors corresponding to each $\alpha$ will be committed separately. Here, instead of ``opening'' these vectors to the verifier, the prover in \name{} will provide an inner product argument on the vectors.

\dnote{provide a short comparison of this protocol with the existing works here.}

\subsection{\name2D{}: A SNARK protocol with $O(N^{1/c})$ proof size}

We will now describe \name2D{} in detail. The protocol explained in this section will have $O(N^{1/c})$ proof size and $O(N^{1-1/c})$ public key operations (and $O(N)$ overall complexity) for the verifier. We will first explain the interactive version of our protocol \name2D{}. \\

We reduce the query complexity and verifier efficiency in the interactive protocol by
using homomorphic vector commitments over the RS encoded witness oracles. Let
$x\in \FF^N$ be the witness vector. Let $m$ and $\ell$ be integers such that
$m\ell\geq N$. We choose ordered domains $G=\{\zeta_1,\ldots,\zeta_\ell\}$ and
$H=\{\eta_1,\ldots,\eta_n\}$. We then write the vectors $x$ as $x =
(x_{1},\ldots,x_{m})$ where each $x_{i}\in \FF^\ell$ for $i \in [m]$. 
Let $\hat{f}^x_{i}$ be the polynomial interpolating the vector $x_{i}$
on $G$ and let $f^x_{i}$ denote the corresponding evaluation of
$\hat{f}^x_{i}$ on $H$. We define the RS-encoded witness $\rsoracle\in
\FF^{m\times n}$ as $\rsoracle[i,j]=\hat{f}^x_{i}(\eta_j)$ for $i\in
[m]$ and $j\in [n]$. We now construct a commitment oracle $\comoracle$ from $\rsoracle$.

\subsubsection{Commitment Oracle}
Throughout, we assume $\FF$ is a prime field. Let $\comm$ denote the Pedersen
vector commitment scheme over $\FF^m$ with randomness space as $\FF$ and
commitment space as group $\GG$ with independent generators $g_1,\ldots,g_m, h$. 
Define $c_{j} = \comm(\rsoracle[\cdot,j],\delta_{j})$, $j\in [n]$ where the notation $X[\cdot,j]$
denotes the $m$-length vector $(X[1,j],\ldots,X[m,j])$ and $\delta_{j}$
denotes the randomness for computing the commitment $c_{j}$. We
define the oracle $\comoracle$ as $\comoracle[j]=c_{j}$. The oracle
$\comoracle$ answers queries of the type $Q\subseteq [n]$, responding with
elements $\comoracle[j]$ for $j\in Q$.\footnote{Defining the commitment vector as 
$\comoracle$ might seem superfluous here. But, the usage of $\comoracle$ will play 
a prominent role in our 3D protocol. We just define the notation here for the uniformity 
in our descriptions of the 2D and the 3D versions.} 
%We will use $\comoracle$ as the witness
%oracle, and adapt the subprotocols for checking linear constraints, quadratic
%constraints and proximity to this oracle.

\subsubsection{Linear Check with Commitment Oracle}\label{sec:lincheck2D}
The linear check $Ax=0$ can be reduced to checking
$\innp{r^TA}{x}=0$, where the verifier samples a random $r\sample \FF^{mn}$
and sends it to the prover. As in \cite{ligero}, the prover and verifer write $r$ as
$(r_{1}|\cdots|r_{m})$ where each $r_i\in \FF^\ell$. Both the prover and the
verifier also compute degree $<\ell$ polynomials $\hat{r}_{i}$ interpolating
the vector $r_{i}$ on $G$. The required check in terms of polynomials can be
expressed as:
\begin{equation}\label{eq:lincheck2D}
\sum_{\alpha\in G}\sum_{j\in [m]}
\hat{r}_{i}(\alpha) \cdot \hat{f}^x_{i}(\alpha) = 0.
\end{equation}
The prover computes the polynomial $\hat{p}=\sum_{i\in[m]}\hat{r}_i \cdot \hat{f}^x_i$. 
This polynomial $\hat{p}$ of degree $< k+\ell-1$ is sent to the verifier who checks
$\sum_{\alpha\in H}\hat{p}(\alpha)=0$. 
As in the Linear check for \cite{ligero}, the verifier needs to check if
the polynomial $\hat{p}$ is correctly computed from the witness oracle $\comoracle$
to guard against dishonest provers.
Following \cite{ligero}, it is enough for the verifier to query the polynomial evaluations at a constant
number of locations, say $\alpha_1,\ldots,\alpha_q$ and then check
that $\hat{p}$ is consistent at the queried locations. We observe that the verifier only
uses these queried values to prove some inner-product relations with other 
(publicly-known) vectors. Hence, our idea is to make the prover 
use inner-product arguments to prove the consistency of $\hat{p}$ to the verifier.

We will now formally describe the Linear check for our protocol \name2d{}. It will check that a purported commitment oracle
$\comoracle$ encodes witness $x$ satisfying the constraint $Ax=0$ for a public
matrix $A$. As before, we assume $x\in \FF^N$ and $A\in \FF^{N\times N}$ and
$N=m\ell$ for some positive integers $m$ and $\ell$. We further assume that
the prover has RS-encoded oracle $\rsoracle$ which opens to the commitment
$\comoracle$ and is $e$-close to the interleaved code $L_1^{m}$. 
The prover and the verifier interact as follows:
\begin{enumerate}[{\rm 1.}]
\item $\verifier\rightarrow\prover$: The verifier sends a random $r\sample
\FF^N$ to the prover.

\item Both $\verifier$ and $\prover$ interpolate polynomials $R^i:i\in [m]$ such that $R^i$ interpolates the
vector $(r_{(i-1)n+1},\ldots,r_{in})$ for $i\in [m]$. 

\item $\prover\rightarrow\verifier$: The prover computes polynomial 
$p(\cdot)=\sum_{i\in [m]}R^i(\cdot)\hat{f}^x_{i}(\cdot)$ and sends $p$ to the verifier.

\item $\verifier\rightarrow\prover$: The verifier samples $t$ distinct indices
$j_1,\ldots,j_t$ from the set $[n]$ independently at random and sends the indices to the prover.

\item $\prover\leftrightarrow\verifier$: The prover wants to prove that 
$\sum_{i\in [m]} R^i(\eta_{j_u}) \cdot \rsoracle[i, j_u] = p(\eta_{j_u})$ for each $u \in [t]$.
That is, $\prover$ wants to prove the inner product $\innp{R_{j_u}}{\rsoracle[\cdot,j_u]} = p(\eta_{j_u})$,
where $R_{j_u} = (R^1(\eta_{j_u}), \ldots, R^m(\eta_{j_u}))$ and 
$\rsoracle[\cdot, j_u]$ denotes the $m$-length vector $(\rsoracle[1, j_u], \ldots, \rsoracle[m, j_u])$.
To do so, $\prover$ runs an inner-product argument for each $u \in [t]$ with the verifier. The verifier proceeds if the arguments succeed for all $u \in [t]$.

\item The verifier also checks that $\sum_{k\in [\ell]}p(\zeta_k)=0$.
%\begin{itemize}
%\item It checks $\sum_{k\in [\ell]}p(\zeta_k)=0$.
%\item It checks consistency of $\rsoracle[\cdot, j_u]$ with commitment oracle $\comoracle$ for $u\in [t]$
%using the inner-product arguments. 
%\end{itemize}
\item The verifier accepts if all the above checks succeed.
\end{enumerate}
\dnote{self: write the formal lemma.}
.\dnote{The proof should be straightforward when using zk inner-product arguments, right?}


\subsubsection{Quadratic Check}\label{sec:quadcheck2D}
The quadratic check involves the prover convincing the verifier that $x\circ y=z$ 
by providing oracle access to the vectors $x,y$ and $z$. Our protocol follows from
\cite{ligero}, except for the final check by the verifier.
To check that $\hat{p}$ is correctly computed from the oracles, the verifier queries 
the oracles at small number of points $\alpha_1,\ldots,\alpha_q$. The prover responds
with an inner-product proof that proves to the verifier that $\hat{p}(\alpha_j)=\sum_{i\in
[m]}r_i(\hat{f}^x_i(\alpha_j) \cdot \hat{f}^y_i(\alpha_j)-\hat{f}^z_i(\alpha_j))$ for $j\in [q]$.\\

We now formally describe the interactive oracle protocol for checking the relation
$x\circ y = z$ for vectors $x,y,z\in \FF^N$. Let $\rsoracle_x,\rsoracle_y$ and
$\rsoracle_z$ denote the encodings of vectors $x$, $y$ and $z$ respectively via
the RS code. Let $\comoracle_x,\comoracle_y$ and $\comoracle_z$ denote
the respective commitment oracles. The prover and the verifier interact as
follows:

\begin{enumerate}[{\rm 1.}]
\item $\verifier\rightarrow\prover$: The verifier sends a challenge $r\sample
\FF^p$ to the prover.
\item $\prover\rightarrow\verifier$: The prover computes the polynomial $p_j(.)=\sum_{i\in [m]}
r_i(\hat{f}^i_x(\cdot) \cdot \hat{f}^i_y(\cdot) - \hat{f}^i_z(\cdot))$.
and sends it to the verifier.

\item $\verifier\rightarrow\prover$: The verifier sends $t$ randomly sampled indexes $\{j_u\}_{u\in [t]}$ from $[n]$. 

\item $\prover\leftrightarrow\verifier$: The prover wants to prove that 
$\sum_{i\in [m]} r_i(\rsoracle_x[i,j_u] \cdot \rsoracle_y[i,j_u]-\rsoracle_z[i,j_u] = p(\eta_{j_u})$ for each $u \in [t]$.
That is, $\prover$ wants to prove the inner product $\innp{r \circ \rsoracle_x[\cdot,j_u]}{\rsoracle_y[\cdot,j_u]} - \innp{r}{\rsoracle_z[\cdot,j_u]} = p(\eta_{j_u})$.
To do so, for each $u \in [t]$, $\prover$ runs the following inner-product argument with the verifier: 
\[
\innp{(r \circ \rsoracle_x[\cdot,j_u] || r}{\rsoracle_y[\cdot,j_u] || -\rsoracle_z[\cdot,j_u] } = p(\eta_{j_u})
\]
To facilitate this, $\comoracle_x,\comoracle_y$ and $\comoracle_z$ should have been committed with different independently chosen sets of generators.\footnote{By independent, we mean that the prover does not know the Diffie-Hellman relation between any pair of generators.} And, the set of generators used for committing $r$ (to be concatenated with $r \circ \rsoracle_x[\cdot,j_u])$) should also be independent of the above three sets of generators.
The verifier proceeds if the arguments succeed for all $u \in [t]$.

\item The verifier also checks that $\sum_{k\in [\ell]}p(\zeta_k)=0$ and accepts if it succeeds.
%\begin{itemize}
%\item It checks $\sum_{k\in [\ell]}p(\zeta_k)=0$.
%\item It checks consistency of $\rsoracle[\cdot, j_u]$ with commitment oracle $\comoracle$ for $u\in [t]$
%using the inner-product arguments. 
%\end{itemize}
\end{enumerate}






\subsubsection{Proximity Protocol}\label{sec:proximity2D}
We describe a protocol for ``proximity'' of a purported codeword to the
interleaved code. Let $U\in \FF^{n_2\times n_1}$ denote the purported codeword and let
$e< d/3$ denote the proximity parameter. 
\dnote{To Nitin: why did you not use the notation $\rsoracle$ for $U$? Is it that only those $U$s which satisfy the proximity test become $\rsoracle$?}
In its first message the
prover computes commitments $c_1,\ldots,c_n$ to the columns of $U$ and
sends it to verifier. Thereafter, the prover and the verifier interact as
follows:
\begin{enumerate}[{\rm 1.}]
\item $\prover\rightarrow \verifier$: Commitments $c_1,\ldots,c_n$ to the columns of $U$.
\item $\verifier\rightarrow \prover$: Verifier samples $\gamma \sample \FF^n$ and sends to the prover.
\item $\prover\rightarrow \verifier$: The prover computes $w=\gamma^T U$ and sends it to the verifier. 
\item $\verifier\rightarrow \prover$: Verifier sends $t$ randomly sampled indexes $\{j_u\}_{u\in [t]}$ from $[n]$ to the prover.
\item $\prover\leftrightarrow\verifier$: The prover has to now prove to the verifier that $w_{j_u} = \gamma^T U[\cdot, j_u]$, for all $u \in [t]$. The prover and the verifier will run the inner-product argument to check $\ip{\gamma}{U[\cdot, j_u]} = w_{j_u}$ for all $u \in [t]$. The verifier proceeds if the checks succeed.
\item The verifier also checks if  $w\in L$ and accepts if it succeeds.
\end{enumerate}

\subsection{The pitfalls of 2D}
The need for 3D