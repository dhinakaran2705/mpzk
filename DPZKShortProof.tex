%\section{Setting up \name{}} 
\section{MPC Friendly Zero Knowledge Arguments using Homomorphic Oracles}
There have been recent constructions of efficient zero knowledge arguments which
augment PCP based proofs with interaction. These include the {\em Interactive
Oracle Proof} based constructions in \cite{ligero,aurora}. To achieve sublinear
argument size, the IOP framework provides verifier with only an oracle access to
the prover's messages. The abstraction of oracle is realized in Random Oracle
Model (ROM) using collision resistant hash functions (CRH). Under existing framework,
it is not clear how to realize oracle access to a prover message, which is
distributed across several provers. We address this by introducing {\em
homomorphic oracles}. In this setting, we provide oracle access to a
homomorphic commitment over the encoded witness. Such an oracle access can be
realized by an aggregating party $\Ag$, which aggregates commitments from the
provers, and uses CRH to realize the oracle. 

We illustrate the usage of homomorphic oracle to construct a distributed prover
variant of the zero knowledge argument in \cite{ligero}, called
$\mathsf{Ligero}$. The $\mathsf{Ligero}$ protocol is a special case of IOP,
where only the first prover message (encoding of witness) is an oracle, while
the remaining messages can be read by the verifier in entirety. We give a brief
overview of the construction in \cite{Ligero}, and then describe our variant.

\noindent{$\mathsf{Ligero}$} provides sublinear zero knowledge argument for existence of
$\wit\in \FF^N$ such that $A\wit\circ B\wit=C\wit$ for some $M\times N$ public
matrices $A,B$ and $C$. To accomplish this, the construction in \cite{ligero}
describes following key steps:
\begin{enumerate}[{\rm 1.}]
\item encoding of the witness $\wit$ as $\ewit$ using error correcting codes.
The verifier is allowed oracle access to the encoded witness. The encoding
satisfies {\em bounded independence}, i.e, a verifier making bounded number of
oracle queries does not learn anything about the witness.
\item An interactive protocol for the linear relation $A\wit=b$ for public $A$
and $b$. The protocol involves the verifier interacting with the prover, and
making oracle queries. 
\item An interactive protocol for the quadratic relation $\wit_x\circ
\wit_y=\wit_z$ given oracle access to the encodings of $\wit_x,\wit_y$ and
$\wit_z$. Again, the verifier interacts with the prover and makes oracle
queries. 
\item An interactive protocol to check if the oracle is well formed.
\end{enumerate}
The above protocols involve prover sending certain polynomials as messages, and
verifier checking identities over the polynomials. We first describe how to
accomplish above steps using homomorphic oracle by using inner product arguments
over committed vectors. Specifically, we use the commitment scheme and inner
product argument from \cite{Bulletproofs}. This yeilds a zero knowledge
argument, which we call $\name$. Using the properties of inner product argument,
we obtain a wider range of tradeoff between argument size and prover's and
verifier's complexity at the expense of introducing expensive cryptographic
operations. The following lemma summarizes the complexity measures achieved by
our construction.
\begin{lemma}
Properties of the 2D construction.
\end{lemma}

