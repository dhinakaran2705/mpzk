\appendix


\subsection{Protocols}\label{app:protocolboxes}
In the protocol listings (Figures \ref{fig:quadcheck} -- \ref{fig:distquadcheck}), {$\mathsf{pp}$ denotes the public parameters consisting of: 
{\small$\mathsf{pp} = (\FF, \GG, \rsc{\eta}{n,\ell}, \rsc{\alpha}{h,m},\rsc{\eta}{n,s+\ell-1},\rsc{\eta}{n,2\ell-1}, \rsc{\alpha}{h,2m-1}, \bm{g},h)$}
%Linear Check of Graphene
\begin{comment}
\begin{figure}[h!]
	{\footnotesize
		%\centering
		\begin{framed}
			\noindent{$\linearcheck(\mathsf{pp},A\in \mc{M}_{M,N},b\in \FF^M,[\pi];\ewit)$}: 			%\pnote{why $\wit$ is not part of the witness of linear check protocol?}
			
			\noindent{\bf Relation}: $\ewit=\open(\pi)\wedge A\wit=b$ for $\wit=\dec(\ewit)$.
			
			\begin{enumerate}[{\rm 1.}]
				\item $\verifier\rightarrow\prover$: $\rho\sample \FF^p$.
				\item $\prover$ computes: $=\sum_{i\in [p]}\rho_i\ewit[i,\cdot,\cdot]$, 
				commitments $\tilde{c}_1,\ldots,\tilde{c}_\ell$ as in Section ~\ref{sec:matrixcommitment}.
				\item $\prover\rightarrow\verifier$: $\tilde{\bm{c}}=(\tilde{c}_1,\ldots,\tilde{c}_\ell)$.
				\item $\verifier\rightarrow\prover$: $r\sample \FF^M$.
				\item $\prover\leftrightarrow\verifier$ compute: Polynomials $R^i$, $i\in [p]$ interpolating $R=r^TA$
				as in Section ~\ref{sec:lincheck}. 
				\item $\prover$ computes: Matrix $P$ from $R$ and $\ewit$ as described in Section ~\ref{sec:lincheck}. Samples $P_0\sample \FF^{2m-1}$, $\omega_0\sample \FF$ and $c_0\gets \comm(P_0,\omega_0)$.
				Computes commitments $c_1,\ldots,c_{s+\ell-1}$ from $P$.
				\item $\prover\rightarrow\verifier$: $c_0,c_1,\ldots,c_{s+\ell-1}$.
				\item $\verifier\rightarrow\prover$: $Q=\{(j_u,k_u):u\in [t]\}$ for $(j_u,k_u)\sample [h]\times [n]$ for $u\in [t]$.
				\item $\verifier\rightarrow\pi$: $\{k_u:u\in [t]\}$.
				\item $\prover\rightarrow\verifier$: $\ewit[\cdot,j_u,k_u]$ for $u\in [t]$.
				\item $\pi\rightarrow\verifier$: $\pi[\cdot,k_u]$ for $u\in [t]$.
				\item $\verifier\rightarrow\prover$: $\delta\sample \FF^p$, $\beta\sample \FF\backslash \{0\}$. 
				\item $\prover$ and $\verifier$ run inner product arguments to check:
				\begin{enumerate}
					\item $\innerproduct(\mathsf{pp},\bm{1}_{j_u}^T\Lambda_{h,2m-1},\mathsf{cm}_{k_u},v_u;\overline{P}[\cdot,k_u])$ 
					for $u\in [t]$ where $\mathsf{cm}_{k_u}=\prod_{a=1}^{s+\ell-1}c_a^{\Lambda^T_{n,s+\ell-1}[a,k_u]}$, 
					$v_u=\sum_{i=1}^pR^i(\alpha_{j_u},\eta_{k_u})U[i,j_u,k_u]$ (check consistency of $P$ with $\pi$).
					\item $\innerproduct(\mathsf{pp},1^m||0^{m-1},\mathsf{cm},r^Tb;z)$ where $z=\beta P_0+\overline{P}\varphi$ and $\mathsf{cm}=c_0^{\beta}\cdot \prod_{a=1}^{s+\ell-1}c_k^{\varphi_k}$ (check the condition $r^TAw=r^Tb$).
					\item $\innerproduct(\mathsf{pp},\bm{1}_{j_u}^T\Lambda_{h,m},C_u,\innp{\delta}{X_u})$ for $u\in [t]$ 
					where $C_u=\prod_{i=1}^p(\pi[i,k_u])^{\delta_i}$ (consistency of $X_u$ with $\pi$). 
				\end{enumerate}
				\item $\verifier$ checks: $\prod_{a=1}^\ell(\tilde{c}_a)^{\Lambda^T_{n,\ell}[a,k_u]}=\prod_{i=1}^p(\pi[i,k_u])^{\rho_i}$ for $u\in [t]$ (check proximity of $\ewit$ to $\mc{W}_1$).
				\item $\verifier$ accepts if all the checks succeed.
			\end{enumerate}
		\end{framed}
		\caption{Linear Check Protocol}
		\label{fig:linearcheck}
	}
\end{figure}
\end{comment}
%%%%%%%%%%%%%%%%%%%%%%%%%%%%%%%%%%%%%%%%
%%%%%%%%%%%%%%%%%%%%%%%%%%%%%%%%%%%%%%
%Quadratic Check of Graphene
\begin{figure}[h!]
	{\footnotesize
		%\centering
		\begin{framed}
			\noindent{$\quadcheck(\mathsf{pp},[\pi_x],[\pi_y],[\pi_z];\ewit_x, \ewit_y, \ewit_z)$}:
			
			\noindent{\bf Relation}: $\ewit_a=\open(\pi_a)$ for $a\in \{x,y,z\}$,
			$\wit_x \circ \wit_y = \wit_z$ where  $\wit_a=\dec(\ewit_a)$ for $a\in \{x,y,z\}$.
			
			\begin{enumerate}[{\rm 1.}]
				\item $\verifier\rightarrow\prover$: $\rho\sample \FF^{3p}$.
				\item $\prover$ computes: (a) $\tilde{\ewit}=\sum_{i=1}^p[\rho_i\ewit_x[i,\cdot,\cdot]+
				\rho_{p+i}\ewit_y[i,\cdot,\cdot]+\rho_{2p+i}\ewit_z[i,\cdot,\cdot]]$, (b)
				commitments $\tilde{c}_1,\ldots,\tilde{c}_\ell$ as $\tilde{c}_k = \prod_{i=1}^{p} (\pi_x[i,k])^{\rho_i}\cdot (\pi_y[i,k])^{\rho_{p+i}}\cdot(\pi_z[i,k])^{\rho_{2p+i}}$ $\forall k\in[\ell]$.
				\item $\prover\rightarrow\verifier$: $\tilde{\bm{c}}=(\tilde{c}_1,\ldots,\tilde{c}_\ell)$.
				\item $\verifier\rightarrow\prover$: $r\sample \FF^p$.
				%\item $\prover\leftrightarrow\verifier$ compute: Polynomials $R^i$, $i\in [p]$ interpolating $R=r^TA$ as in Section ~\ref{sec:quadcheck}. 
				\item $\prover$ computes: (a) $p_j(\cdot) = \sum_{i\in[p]} r_i[Q^i_x(\alpha_j,\cdot)Q^i_y(\alpha_j,\cdot) - Q^i_z(\alpha_j,\cdot)]$ $\forall j\in [h]$
				(b) matrix $P$ such that $P[j,k] = p_j(\eta_k)$ as described in Section ~\ref{sec:quadcheck}, (c)
				computes commitments $c_1,\ldots,c_{2\ell}$ from $P$. $\prover$ also
				samples $P_0\sample \FF^{2m-1}$ with $P_0[j]=0^m$ for $j\in[m]$ and computes commitment $c_0$
				to $P_0$. 
				\item $\prover\rightarrow\verifier$: $c_0$, $c_1,\ldots,c_{2\ell-1}$.
				\item $\verifier\rightarrow\prover$: $Q=\{(j_u,k_u):u\in [t]\}$ for 	$Q\sample [h]\times [n]$, $u\in [t]$ and $\tau \sample \FF^s$, $\gamma \sample \FF^m$.
				\item $\verifier\rightarrow\pi$: $\{k_u:u\in [t]\}$.
				\item $\prover\rightarrow\verifier$: $X_u=\ewit_x[\cdot,j_u,k_u]$ , $Y_u=\ewit_y[\cdot,j_u,k_u]$ and $Z_u=\ewit_z[\cdot,j_u,k_u]$ for $u\in [t]$.
				
				\item $\pi\rightarrow\verifier$: $\pi[\cdot,k_u]$ for $u\in [t]$.
				\item $\verifier\rightarrow\prover$: $\delta\sample
				\FF^p$, $\beta_x\sample \FF$, $\beta_y\sample \FF$, $\beta_z\sample \FF$,
				$\beta\sample \FF\backslash \{0\}$.
				\item $\prover$ computes:
				$V_u=\sum_{i=1}^p\delta_i\big(\beta_x\ewit_x[i,\cdot,k_u]+\beta_y\ewit_y[i,\cdot,k_u]+\beta_z\ewit[i,\cdot,k_u]\big)$.
				%$\beta \FF\backslash \{0\}$. 
				\item $\prover \text{ and }\verifier$ compute:
				$W_u=\beta_xX_u+\beta_yY_u+\beta_zZ_u$ for $u\in [t]$.
				$T_u=(C_u)^{\beta_x}\cdot(D_u)^{\beta_y}\cdot(E_u)^{\beta_z}$, for $u\in [t]$ where
				$C_u=\prod_{i=1}^{p}(\pi_x[i,k_u])^{\delta_i}$, $D_u=\prod_{i=1}^{p}(\pi_y[i,k_u])^{\delta_i}$
				and $E_u=\prod_{i=1}^{p}(\pi_y[i,k_u])^{\delta_i}$.
				\item $\prover$ and $\verifier$ run inner-product arguments to check:
				\begin{enumerate}
					\item $\innerproduct(\mathsf{pp},\bm{1}_{j_u}^T\Lambda_{h,2m-1},\mathsf{cm}_{k_u},v_u;\overline{P}[\cdot,k_u])$ for $u\in [t]$ where $\mathsf{cm}_{k_u}=\prod_{a=1}^{2\ell-1}(c_a)^{\Lambda_{n,2\ell-1}^T[a,k_u]}$, 
					$v_u=\sum_{i=1}^p r_i[X_u[i]\cdot Y_u[i] - Z_u[i]]$ (check consistency of $P$ with $\pi$).
					\item $\innerproduct(\mathsf{pp},\gamma||0^{m-1},\mathsf{cm},0;z)$
					where $z=\beta P_0 +\overline{P}\varphi$, $\varphi = \Phi^T\tau$ and
					$\mathsf{cm} =  (c_0)^{\beta}\cdot\prod_{a=1}^{2\ell-1} (c_a)^{\varphi_a}$ %(check the condition $r^TAw = r^Tb$).
					\item
					$\innerproduct(\mathsf{pp},\bm{1}_{j_u}^T\Lambda_{h,m},T_u,w_u;\overline{V}_u])$, where $\overline{V}_u$ stands for first $m$ entries of $V_u$ and $w_u = \innp{\delta}{W_u}$ (consistency of $X_u, Y_u, Z_u$ with $\pi$). 
				\end{enumerate}
				\item $\verifier$ checks proximity of $\ewit_x,\ewit_y$
				and $\ewit_z$ according to Eqn \eqref{eq:combinedproximity}.
				\item $\verifier$ accepts if all the checks succeed.
			\end{enumerate}
		\end{framed}
		\caption{Quadratic Check Protocol}
		\label{fig:quadcheck}
	}
\end{figure}

%%%%%%%%%%%%%%%%%%%%%%%%%%%%%%%%%%%%%%%%
%Distributed Linear Check of DP-Graphene

\begin{figure}[h]
	{\footnotesize
		%\centering
		\begin{framed}
			\noindent{$\distlinearcheck(\mathsf{pp}, \, A\in \mc{M}_{M,N}, \, b\in \FF^M, \, [\pi];\, \shr{\ewit}, \, \shr{0^{2m-1}})$}:
			%\pnote{why $\wit$ is not part of the witness of linear check protocol?}
			
			\noindent{\bf Relation}: $\ewit=\open(\pi)\wedge A\wit=b$ for $\shr{\wit}=\dec(\shr{\ewit})$ for all $\xi \in [\Num]$ and $\sum_{\xi\in[\Num]} \shr{\wit} = \wit$.
			
			\begin{enumerate}[{\rm 1.}]
				\item $\verifier\rightarrow\distprover$: $\rho\sample \FF^p$.
				\item $\distprover$ computes: $\shr{\tilde{\ewit}} = \sum_{i\in [p]}\rho_i\shr{\ewit}[i,\cdot,\cdot]$, 
				commitments $\shr{\tilde{c}_1},\ldots,\shr{\tilde{c}_\ell}$ as $\shr{\tilde{c}}_k = \prod_{i\in[p]}(\shr{\comoracle[i,k]})^{\rho_i}$ $\forall k\in[\ell]$
				\item $\distprover\rightarrow\Ag$: $\shr{\tilde{\bm{c}}} = (\shr{\tilde{c}_1},\ldots,\shr{\tilde{c}_\ell})$.
				\item  {$\Ag\rightarrow\verifier$: $\tilde{\bm{c}}=\combine(\shr{\tilde{\bm{c}}}$.} %(\tilde{c}_1,\ldots,\tilde{c}_\ell)$.
				\item $\verifier\rightarrow\distprover$: $r\sample \FF^M$.
				\item $\distprover \text{ and }\verifier$ compute: Polynomials $R^i$, $i\in [p]$ interpolating $R=r^TA$
				as in Section ~\ref{sec:lincheck}. 
				\item $\distprover$ computes: Matrix $\shr{P}$ from $R$ and $\shr{\ewit}$ as described in Section ~\ref{sec:lincheck}. Samples $\shr{P_0}\sample \FF^{2m-1}$, $\shr{\omega_0} \sample \FF$ and $\shr{c_0}\gets \comm(\shr{P_0},\shr{\omega_0})$,  and $\shr{d_0}\gets \comm(\shr{0^{2m-1}},\shr{o})$ where $\shr{o} \sample \FF$.
				Computes commitments $\shr{c_1},\ldots,\shr{c_{s+\ell-1}}$ from $\shr{P}$.
				\item $\distprover\rightarrow\Ag$: $\shr{c_0} ,\shr{c_1} ,\ldots, \shr{c_{s+\ell-1}}, \shr{d_0}$.
				\item  {$\Ag\rightarrow\verifier$: $c_k = \combine(\shr{c_k}) \, \forall k\in [s+\ell-1]$ and sends $c_0,c_1,\ldots,c_{s+\ell-1}$.}
				\item $\verifier\rightarrow\distprover$: $Q=\{(j_u,k_u):u\in [t]\}$ for %$(j_u,k_u)
				$Q\sample [h]\times [n]$ for $u\in [t]$.
				\item $\verifier\rightarrow\pi$: $\{k_u:u\in [t]\}$.
				\item $\distprover\rightarrow\Ag$: $\shr{X_u}=\shr{\ewit}[\cdot,j_u,k_u],  \shr{P_u} = \shr{P}[\cdot,k_u]$ for $u\in [t]$.
				\item  {$\Ag\rightarrow\verifier$: $X_u = \sum_{\xi\in[\Num]} \shr{X_u}, P_u= \sum_{\xi\in [\Num]} \shr{P_u}$ and sends ${X_u}$ for $u\in [t]$.}
				\item $\pi\rightarrow\verifier$: $\pi[\cdot,k_u]$ for $u\in [t]$.
				\item $\verifier\rightarrow\distprover$: $\delta\sample \FF^p$, $\beta\sample \FF\backslash \{0\}$. 
				\item $\distprover\rightarrow\Ag$: $\shr{z} = \beta\shr{P_0} + \shr{\overline{P}}\varphi + \shr{0^{2m-1}}$, $\shr{V_u} = \sum_{i\in[p]} \delta_i \shr{\ewit}[i,\cdot,k_u]$ and sends $\shr{z}$, $\shr{V_u}$.
				\item  {$\Ag$ computes $z=\sum_{\xi\in[\Num]} \shr{z}$, $V_u=\sum_{\xi\in[\Num]} \shr{V_u}$}.
				\item $\Ag$ and $\verifier$ run inner product arguments to check:
				\begin{enumerate}
					\item $\innerproduct(\mathsf{pp},\bm{1}_{j_u}^T\Lambda_{h,2m-1},\mathsf{cm}_u,v_u;\overline{P}[\cdot,k_u])$ 
					%\pnote{We can use $P[1:2m, k_u]$ to denote the first 2m entries of $P[\cdot,k_u]$ as $\overline{P}$ is the submatrix after cropping both the directions, $h$ and $n$.}
					for $u\in [t]$ where $\mathsf{cm}_u=\prod_{a=1}^{s+\ell-1}(c_a)^{\Lambda_{n,s+\ell-1}[a,k_u]}$, 
					$v_u=\sum_{i=1}^pR^i(\alpha_{j_u},\eta_{k_u})X_u[i]$ (check consistency of $P$ with $\pi$).
					\item $\innerproduct(\mathsf{pp},1^m||0^{m-1},\mathsf{cm},r^Tb;z)$ where $z=\beta P_0+\overline{P}\varphi$ and $\mathsf{cm}= (c_0)^{\beta}\cdot\prod_{a=1}^{s+\ell-1}(c_a)^{\varphi_a}$ (check the condition $r^TAw=r^Tb$).
					\item $\innerproduct(\mathsf{pp},\bm{1}_{j_u}^T\Lambda_{h,m},C_u,w_u;V_u)$ for $u\in [t]$ 
					where $C_u=\prod_{i=1}^p(\pi[i,k_u])^{\delta_i}$ and $w_u= \sum_{i\in[p]} \delta_i X_u[i]$(consistency of $X_u$ with $\pi$). 
				\end{enumerate}
				\item $\verifier$ checks: $\prod_{a=1}^\ell(\tilde{c}_a)^{\Lambda^T_{n,\ell}[a,k_u]}=\prod_{i=1}^p(\pi[i,k_u])^{\rho_i}$ for $u\in [t]$ (check proximity of $\ewit$ to $\mc{W}_1$).
			\end{enumerate}
		\end{framed}
		\caption{Distributed Linear Check Protocol}
		\label{fig:distlincheck}
	}
\end{figure}
%%%%%%%%%%%%%%%%%%%%%%%%%%
%Distributed Quadratic Check of DP-Graphene
\begin{figure}[h]
	{\footnotesize
		%\centering
		\begin{framed}
			\noindent{$\distquadcheck(\mathsf{pp}, [\pi_x], [\pi_y], [\pi_z];\, \shr{\ewit_x}, \shr{\ewit_y}, \shr{\ewit_z}\, \shr{0^{2m}})$}:
			%\pnote{why $\wit$ is not part of the witness of linear check protocol?}
			
			\noindent{\bf Relation}: $[\ewit_x||\ewit_y||\ewit_z]=\open(\pi)\wedge \wit_x \circ \wit_y = \wit_z$ for $\shr{\wit_a}=\dec(\shr{\ewit_a})$ for all $\xi \in [\Num]$ and $\sum_{\xi\in[\Num]} \shr{\wit_a} = \wit_a$ $\forall a\in \{x,y,z\}$.
			
			\begin{enumerate}[{\rm 1.}]
				\item $\verifier\rightarrow\distprover$: $\rho\sample \FF^{3p}$.
				\item $\distprover$ computes: $\shr{\tilde{\ewit}} =\sum_{i=1}^p[\rho_i\shr{\ewit_x}[i,\cdot,\cdot]+
				\rho_{p+i}\shr{\ewit_y}[i,\cdot,\cdot]+\rho_{2p+i}
				\shr{\ewit_z}[i,\cdot,\cdot]]$, 
				commitments $\shr{\tilde{c}_1},\ldots,\shr{\tilde{c}_\ell}$ as $\shr{\tilde{c}}_k = \prod_{i=1}^{p} (\shr{\comoracle_x}[i,k])^{\rho_i}\cdot (\shr{\comoracle_y}[i,k])^{\rho_{p+i}}\cdot(\shr{\comoracle_z}[i,k])^{\rho_{2p+i}}$ $\forall k\in[\ell]$.
				\item $\distprover\rightarrow\Ag$: $\shr{\tilde{\bm{c}}} = (\shr{\tilde{c}_1},\ldots,\shr{\tilde{c}_\ell})$.
				\item  {$\Ag\rightarrow\verifier$: $\tilde{\bm{c}}=\combine(\shr{\tilde{\bm{c}}})$.} %(\tilde{c}_1,\ldots,\tilde{c}_\ell)$.
				\item $\verifier\rightarrow\distprover$: $r\sample \FF^p$.
				\item Provers $\distprover$ run the MPC: $\shr{\ewit_x.\ewit_y}\leftarrow
				\mathsf{Mult}(\shr{\ewit_x},\shr{\ewit_y})$ to obtain shares of the hadamard product of the encodings.
				\item $\distprover$ computes: Matrix $\shr{P}$ from $r$ and $\shr{\ewit_x.\ewit_y}, \, \shr{\ewit_z}$ as described in Section ~\ref{sec:quadcheck}. Samples $\shr{P_0}\sample \FF^{2m-1}$ such that $P_0[j]=0 \forall j\in[m]$, $\shr{\omega_0} \sample \FF$ and $\shr{c_0}\gets \comm(\shr{P_0},\shr{\omega_0})$,  and 
				$\shr{d_0}\gets \comm(\shr{0^{2m-1}},\shr{o})$ where $\shr{o} \sample \FF$.
				Computes commitments $\shr{c_1},\ldots,\shr{c_{2\ell-1}}$ from $\shr{P}$.
				\item $\distprover\rightarrow\Ag$: $\shr{c_0} ,\shr{c_1} ,\ldots, \shr{c_{2\ell-1}}, \shr{d_0}$.
				\item  {$\Ag\rightarrow\verifier$: $c_k = \combine(\shr{c_k}) \, \forall k\in [2\ell-1]$ and sends $c_0,c_1,\ldots,c_{2\ell-1}$.}
				\item $\verifier\rightarrow\distprover$: $Q=\{(j_u,k_u):u\in [t]\}$ for %$(j_u,k_u)
				$Q\sample [h]\times [n]$ for $u\in [t]$. And $\tau \sample \FF^s, \, \gamma \sample \FF^m, \beta\sample \FF^\ast$.
				\item $\verifier\rightarrow\pi$: $\{k_u:u\in [t]\}$.
				\item $\distprover\rightarrow\Ag$: 
				%\begin{itemize}
					 $\shr{X_u}=\shr{\ewit_x}[\cdot,j_u,k_u]$,
					 $\shr{Y_u}=\shr{\ewit_y}[\cdot,j_u,k_u]$,
					 $\shr{Z_u}=\shr{\ewit_z}[\cdot,j_u,k_u]$,
					 $\shr{P_u} = \shr{P}[\cdot,k_u]$
				%\end{itemize} 
				 for $u\in [t]$.
				\item  {$\Ag\rightarrow\verifier$: $A_u = \sum_{\xi\in[\Num]} \shr{A_u}$ where $A\in \{X, Y, Z, P\}$ and sends ${X_u, Y_u, Z_u}$ for $u\in [t]$.}
				\item $\pi\rightarrow\verifier$: $\pi[\cdot,k_u]$ for $u\in [t]$.
				\item $\verifier\rightarrow\distprover$: $\delta\sample \FF^p$, $\beta_x\sample \FF$, $\beta_y\sample \FF$, $\beta_z\sample \FF$,
				$\beta\sample \FF\backslash \{0\}$. 
				
				\item $\distprover$ computes:
				$\shr{V_u}=\sum_{i=1}^p\delta_i\big(\beta_x\shr{\ewit_x}[i,\cdot,k_u]+\beta_y\shr{\ewit_y}[i,\cdot,k_u]+\beta_z\shr{\ewit}[i,\cdot,k_u]\big)$ and $\shr{z} =  \beta\cdot P_0 + \shr{\overline{P}}\varphi + \shr{0^{2m-1}}$.
				\item $\distprover\rightarrow\Ag$:  sends $\shr{z}, \shr{V_u}$.
				\item  {$\Ag$ computes $z=\sum_{\xi\in[\Num]} \shr{z}$} and $V_u = \sum_{\xi\in[\Num]} \shr{V_u}$.
				\item $\Ag \text{ and } \verifier$: Both compute $W_u = \beta_x X_u + \beta_y Y_u + \beta_z Z_u$ for $u\in[t]$. $T_u = (C_u)^{\beta_x}\cdot(D_u)^{\beta_y}\cdot(E_u)^{\beta_z}$, for $u\in [t]$ where
				$C_u=\prod_{i=1}^{p}(\pi_x[i,k_u])^{\delta_i}$, $D_u=\prod_{i=1}^{p}(\pi_y[i,k_u])^{\delta_i}$
				and $E_u=\prod_{i=1}^{p}(\pi_y[i,k_u])^{\delta_i}$.
				\item $\Ag$ and $\verifier$ run inner product arguments to check:
				\begin{enumerate}
					\item $\innerproduct(\mathsf{pp},\bm{1}_{j_u}^T\Lambda_{h,2m-1},\mathsf{cm}_{k_u},v_u;\overline{P}[\cdot,k_u])$ for $u\in [t]$ where $\mathsf{cm}_{k_u}=c_0^{\beta}\prod_{a=1}^{2\ell-1}(c_a)^{\Lambda_{n,2\ell-1}^T[a,k_u]}$, 
					$v_u=\sum_{i=1}^p r_i[X_u[i]\cdot Y_u[i] - Z_u[i]]$ (check consistency of $P$ with $\pi$).
					\item $\innerproduct(\mathsf{pp},\gamma||0^{m-1},\mathsf{cm},0;z)$ where $z=\beta\cdot P_0 + \overline{P}\times \varphi$ and $\varphi = \Phi^T\tau$ and $\mathsf{cm} = \prod_{a=1}^{2\ell-1} (c_a)^{\varphi_a}$ %(check the condition $r^TAw = r^Tb$).
					\item
					$\innerproduct(\mathsf{pp},\bm{1}_{j_u}^T\Lambda_{h,m},T_u,w_u;\overline{V}_u)$, where $\overline{V}_u$ stands for first $m$ entries of $V_u$ and $w_u = \innp{\delta}{W_u}$ (consistency of $X_u, Y_u, Z_u$ with $\pi$).
				\end{enumerate}
				\item $\verifier$ checks proximity of $\ewit_x,\ewit_y$ and $\ewit_z$ according to the Equation \eqref{eq:combinedproximity}.
			\end{enumerate}
		\end{framed}
		\caption{Distributed Quadratic Check Protocol}
		\label{fig:distquadcheck}
	}
\end{figure}

\begin{comment}
\subsection{Interleaved Codes: Alternative interpretation} \label{app:AltILC}
For concreteness, let $\FF$ be the finite field $\FF_q$, consisting of $q$
elements. Let $\HH$ denote the finite field consisting of $q^m$ elements. From
standard algebra, the field $\HH$ is specified by an irreducible polynomial $h(x)\in \FF[x]$ of
degree $m$ as follows:
\begin{itemize}
\item The elements of $\HH$ are polynomials $a(x)$ over $\FF$ of degree at most
$m-1$.
\item The field operations $(+,\cdot)$ on $\HH$ are defined as corresponding
polynomial operations modulo $h(x)$, i.e, on $\HH$ we define $a(x)+b(x) =
(a(x)+b(x)) \text{ mod } h(x)$ and $a(x)\cdot b(x) = (a(x)\cdot b(x)) \text{ mod
} h(x)$. 
\end{itemize}
We can identify the elements of field $\HH$ naturally with the set $\FF^m$ via
the bijection $f:\FF^m\leftrightarrow \HH$ given by
$f(a_0,\ldots,a_{m-1})=\sum_{i=0}a_ix^i$. For any $a,b\in \FF^m$ and $\alpha\in
\FF$, it is seen that $f(a+b)=f(a)+f(b)$ and $f(\alpha a)=\alpha f(a)$. We can
extend $f$ to the bijection $\tilde{f}$ between the set $\mc{M}_{m,n}$ of $m\times n$
matrices over $\FF$ and $\HH^n$ by defining:
\begin{equation}\label{lem:bijection}
\tilde{f}(A) = \big(f(A[\cdot,1]),\ldots,f(A[\cdot,n])\big)
\end{equation}
We have the following:
\begin{lemma}[Correspondence Lemma]\label{lem:correspondence}
Let $L$ be an $[n,k,d]$ code over the field $\FF=\FF_q$, and let $\mc{C} :=
\ric{L}{m}$ be the row interleaved code of $L$. Then $\tilde{f}(\mc{C})$ is an
$[n,k,d]$ code over the field $\HH$, where $\HH=\FF_{q^m}\cong \FF^m$. Moreover,
the generator matrix $\mc{G}$ of $L$ is also the generator matrix of
$\tilde{f}(\mc{C})$. We note that in the latter case $\mc{G}$ is viewed as a
matrix over the field $\HH$.
\end{lemma}
\begin{proof}
Let $\tilde{\mc{C}}$ denote the linear code over $\HH$ given by $\tilde{\mc{C}} :=
\{\mc{G}x: x\in \HH^k\}$. Clearly, $\tilde{\mc{C}}$ is a linear code over $\HH$.
Further it can be seen that $\tilde{\mc{C}}$ is an $[n,k,\ast]$ linear code because
$\mathsf{rank}_{\FF}(\calG)=\mathsf{rank}_{\HH}(\calG)$. This is because $\calG$
contains entries from the subfield $\FF$, and so the determinant of a submatrix of $\calG$
vanishes over $\HH$ if and only if it vanishes over the subfield $\FF$. Now,
consider $A\in \mc{C}$. Then there exists $m\times k$ matrix $P$ such that for all $j\in [n]$, 
we have $A[\cdot,j]=\sum_{i\in [m]}\calG[i,j]P[\cdot,i]$. Therefore, for all
$j\in [m]$,  we have
\begin{align*}
f(A[\cdot,j]) &= f\big(\sum_{i=1}^m \calG[i,j]P[\cdot,i]\big) =
	 \sum_{i=1}^k \calG[i,j]f(P[\cdot,i]) \\
	&= \sum_{i=1}^k \calG[i,j]p_i(x) \text{ where } p_i(x)=f(P[\cdot,i]).
\end{align*}
From the above, it can be seen that:
\[ \tilde{f}(A)=(f(A[\cdot,1]),\ldots,f(A[\cdot,n]))=\calG
[p_1(x),\ldots,p_k(x)]^T \]
 where we view $\tilde{f}(A)$ as a column vector. Thus
$\tilde{f}(\mc{C})\subseteq \tilde{\mc{C}}$. Similarly, one can show that
$\tilde{\mc{C}}\subseteq \tilde{f}(\mc{C})$ and thus conclude
$\tilde{f}(\mc{C})=\tilde{\mc{C}}$. Using similar arguments we can also show
that the code $\tilde{f}(\mc{C})=\tilde{\mc{C}}$ also has the same parity check
matrix as the code $L$, and hence has the same minimum distance $d$. We skip the
proof as it is standard. 
\end{proof}
\end{comment}
% removing proof of Lemma bicdecoding
\begin{comment}
%\subsection{Proof of Lemma~\ref{lem:bicdecoding} }\label{prooflemmaRScode}
\begin{proof}[Proof of Lemma ~\ref{lem:bicdecoding}]\label{prooflemmaRScode}
	Let $U_1 \in \mc{C}_1$  such that $\dham_1(U^\ast, U_1) \leq \dham(U^\ast, U) \forall U\in \mc{C}_1$ and let $\dham_1(U^\ast, U_1) = \beta_1$ with $\beta_1<e_1$
	Therefore $U^\ast$ differs from $U_1$ in $\beta_1$ many columns, i.e. in the remaining $n-\beta_1$ columns $U^\ast$ and $U_1$ are identical.
	So, there are columns $S_1 = \{j_{s} : s\in [n - \beta_1] \}$ for which $U^\ast[i,j] = U_1[i,j]$ $\forall j\in S$.
	Let $U_2 \in \mc{C}_2$  such that $\dham_2(U^\ast, U_2) \leq \dham(U^\ast, U) \forall U\in \mc{C}_2$ and let $\dham_1(U^\ast, U_2) = \beta_2$ with $\beta_2<e_2$
	Therefore $U^\ast$ differs from $U_2$ in $\beta_2$ many rows, i.e. in the remaining $n-\beta_1$ rows $U^\ast$ and $U_2$ are identical.
	So, there are rows $S_2 = \{i_{t} : t\in [h - \beta_2] \}$ for which $U^\ast[i,j] = U_2[i,j]$ $\forall i\in T$.
	
	$U_1\in \mc{C}_1 \implies \exists$ polynomials $p_1(\cdot), \ldots, p_h(\cdot)$ of degree $< \ell$
	
	$U_2\in \mc{C}_2 \implies \exists$ polynomials $q_1(\cdot), \ldots, q_n(\cdot)$ of degree $< m$
	
	Such that $p_i(\eta_j) = U_1[i,j] $ and $ q_i(\alpha_j) = U_2[j,i]$
	
	Therefore, $U_1[i,j] = U_2[i,j]$ $\forall i\in S_1, j\in S_2$. Choose a bivariate polynomial $Q(x,y)$ such that $deg_x(Q)<\ell $ and $deg_y(Q)<m$ and $Q(\eta_{i_s}, \alpha_{j_t}) = U_1[i_s,j_t] = U_2[i_s,j_t] \forall (i, j)\in S_1\times S_2$. 
	
	Defne $U = Q(\eta_i,\alpha_j)$ $\forall i\in[h], j\in[n]$, Then $U \in L_1\otimes L_2$ and $U^\ast[i,j] = U[i,j]$ $\forall (i,j) \in S_1 \times S_2$.
	
	And $|S_1|= n-\beta_1 > n-e_1$ and $|S_2|= h-\beta_2 > h-e_2$.
	This proves the above claim.	
\end{proof}
\end{comment}

\subsection{Proof of Proposition \ref{lem:3dcompression}} \label{app:ProofofLem3dcompression}
We need several observations to prove the proposition, which we present next.
Throughout this section, let $L$ be a linear $[n,k,d]$ code over a field $\HH$ and let $\FF\subseteq \HH$ be a subfield.  Let $m\geq 1$ be an integer, and let $\mc{C}$ denote the row interleaved code $\ric{L}{m}$. For a matrix $U\in \HH^{m\times n}$ and a vector $u_0\in \HH^n$, let $\aff{u_0}{U}$ denote the affine space as:
{
\begin{equation}\label{eq:affspace}
\aff{u_0}{U} := \{u_0+r^TU: r\in \FF^m\}
\end{equation}}
Note that in the above, the scalars in the linear combination come from $\FF$.

The following result was proved in \cite{ligero} for the setting $\HH=\FF$.
For completeness, we present an adaptation of the proof to the setting where
$\FF$ is a subfield of $\HH$.
\begin{lemma}\label{lem:farpoint}
Let $L$ and $\mc{C}$ be codes as defined, and let $e$ be a positive integer such that $e+2\leq |\FF|$. Then for any $u_0\in \HH^n$ and any $U^\ast\in \HH^{m\times n}$ such that $d(U^\ast,\mc{C})>e$, there exists $v\in \aff{u_0}{U^\ast}$ such that $d(v,L)>e$.
\end{lemma} 
\begin{proof}
For sake of contradiction, assume that $d(u,L)\leq e$ for all $u\in
\aff{u_0}{U^\ast}$. Let $x$ be the point in $\aff{u_0}{U^\ast}$ such that
$d(x,L)$ is maximum. By assumption $d(x,L)\leq e$. Let $v\in L$ be such that
$\Delta(x,v)=d(x,L)$. Let $E\subseteq [n]$ be the set of positions where $x$ and
$v$ differ. Since $d(U^\ast,\mc{C})>e$, there exists row $R$ of $U^\ast$ and there is some 
position $j\in [n]\backslash E$ 
such that $R_j\neq 0$. Let
$\alpha_1,\ldots,\alpha_{e+1}$ be distinct non zero elements in $\FF$. Let $E_k$
for $k=1,\ldots,e+1$ denote the set of positions where $x+\alpha_kR$ and $v$
differ. Fix a position $i\in E$. Then there exists at most one $k\in [e+1]$ such
that $i\not\in E_k$. By pegion hole principle, there exists $k\in [e+1]$ such
that $E\subseteq E_k$. We also observe that since $\alpha_k\neq 0$, $j\in E_k$.
Thus $d(x+\alpha_k,v)>d(x,v)$, contradicting the choice of $x$. This proves the
lemma.   
\end{proof}

Next we prove a result about lines with respect to linear codes. The result was
proved in \cite{ligero} for the case $\HH=\FF$ and $e<d/4$. The authors in
\cite{ligero} conjectured the result for $e<d/3$. Here we prove the result for
$e<d/3$ when $\FF$ can be an arbitrary subfield of $\HH$.

\begin{lemma}[Affine Line]\label{lem:affineline}
Let $L$ be the linear code as defined. Define an affine line $\ell_{u,v}$ in $\HH^n$ as $\ell_{u,v} := \{u + \alpha v:\alpha\in \FF\}$ for $u,v\in \HH^n$. Then for $e < d/3$ and any affine line $\ell_{u,v}$:
\begin{enumerate}[{\rm (i)}]
\item $d(x,L)\leq e$ for all $x\in \ell_{u,v}$, or
\item $d(x,L)> e$ for at most $d$ points in $\ell_{u,v}$.
\end{enumerate}
\end{lemma}
\begin{proof}
%\nnote{Candidate for Appendix: Putting it here just to have all content for the moment}
In this proof, let $\delta(x,y)$ denote the set of positions where the vectors
$x$ and $y$ differ, and let $\wt{x}$ denote $|\delta(x,\bm{0})|$. The above is equivalent to
proving that if there exist $d+1$ distinct points $X =
\{x_1,\ldots,x_{d+1}\}\subseteq \ell_{u,v}$ such
that $d(x_i,L)\leq e$ for all $i\in [d+1]$, then $d(x,L)\leq e$ for all $x\in
\ell_{u,v}$. Assume that there exists such a set $X$ of $d+1$ points. Let $\ell_i$ denote the point in $L$ closest to $x_i$ for $i\in
[d+1]$. Set $\eta_i=x_i-l_i$ for all $i$ and let
$\bm{\eta}=\{\eta_1,\ldots,\eta_{d+1}\}$. By assumption we have
$\wt{\eta_i}\leq e$ for all $i\in [d+1]$. Since $\ell_{u,v}$ is contained in affine
span of any two distinct points on it, we have tuples
$\{(\alpha_i,\beta_i)\}_{i\in [d+1]}d$ such that $\alpha_i + \beta_i=1$ and $x_i=\alpha_ix_1 +
\beta_ix_2$ for $i\in [d+1]$. Note that $(\alpha_1,\beta_1)=(1,0)$ and
$(\alpha_2,\beta_2)=(0,1)$. Observe that $\alpha_i$'s and $\beta_i$'s must be
distinct for all $i\in [d+1]$. We call $X$ as {\em degenerate} if there exist
$i\neq j$ satisfying $\alpha_j=\gamma\alpha_i$ and $\beta_j=\gamma\beta_j$ for
some $\gamma\in \FF\backslash {0}$. We consider two cases:

\noindent{\em $X$ is degenerate}: Degeneracy implies there exist $i\neq j$ such that $x_i=\gamma x_j$ for
some $\gamma\in \FF\backslash\{0\}$. In this case we have $0 =
\frac{1}{1-\gamma}x_i -\frac{\gamma}{1-\gamma}x_j\in \ell_{u,v}$. This implies
$\ell_{u,v}=\{\alpha x_i: \alpha\in \FF\}$
and hence $d(x,L)=d(x_i,L)\leq e$ for all $x\in \ell_{u,v}$. Thus the statement
of the Lemma holds in this case.\smallskip 

\noindent{\em X is not degenerate}: We first prove
that $\ell_i=\alpha_i\ell_1+\beta_i\ell_2$ and
$\eta_i=\alpha_i\eta_1+\beta_i\eta_2$ for
all $i\in [d+1]$. We have
\begin{align*}
\ell_i+\eta_i& = x_i = \alpha_ix_1 + \beta_ix_2 
    = \alpha_i(\ell_1+\eta_1) + \beta_i(\ell_2 + \eta_2) \\
    &= (\alpha_i\ell_1 + \beta_i\ell_2) + (\alpha_i\eta_1 + \beta_i\eta_2)
\end{align*}
Rearranging we have,
$
\ell_i - (\alpha_i\ell_1 + \beta_i\ell_2) = \alpha_i\eta_1 +
\beta_i\eta_2 - \eta_i
$.
In the above equation we see that LHS is a vector in $L$. Further
$\wt{\alpha_i\eta_1 + \beta_i\eta_2 - \eta_i}\leq
\wt{\eta_1}+\wt{\eta_2}+\wt{\eta_i}\leq 3e < d$. Thus the LHS must be equal to
$0$ and hence $\ell_i = \alpha_i\ell_1 + \beta_i\ell_2$ and
$\eta_i=\alpha_i\eta_1+\beta_i\eta_2$. Observe that any $x^\ast\in \ell_{u,v}$ can
be written as $x^\ast=\alpha^\ast x_1+\beta^\ast x_2$. Thus
$d(x^\ast,L)=d(\alpha^\ast x_1+\beta^\ast x_2,L)\leq \wt{\alpha^\ast
\eta_1+\beta^\ast \eta_2}\leq |E|$ where $E$ denotes the set
$\delta(\eta_1,0)\cup \delta(\eta_2,0)$. Our final effort will be to show that
$|E|\leq e$.

\noindent{\it Claim}: $|E|\leq e$ where $E = \delta(\eta_1,0)\cup
\delta(\eta_2,0)$. We consider the partition of $E$ into sets
$E_0=\delta(\eta_1,0)\backslash \delta(\eta_2,0)$,
$E_1=\delta(\eta_2,0)\backslash \delta(\eta_1,0)$ and
$E_{01}=\delta(\eta_1,0)\cap \delta(\eta_2,0)$. Let $t=|E|$. Consider a $t\times (d+1)$ matrix $M=(m_{ij})$
where $m_{ij}=0$ if $j^{th}$ coordinate ($\eta_i^j$) of $\eta_i$ is zero, and $m_{ij}=1$
otherwise. We will show that each row of $M$ has at most one $0$. Assume that
there exists $i$ such that $m_{ip}=0$ and $m_{iq}=0$ for $p\neq q$. We consider
three cases:
\begin{itemize}
\item If $i\in E_0$, the above condition implies $\alpha_p\eta_1^i =
\alpha_q\eta_1^i=0$, or $\alpha_p=\alpha_q=0$ as $\eta_1^i\neq 0$ for $i\in E_0$.
This contradicts the fact that $X$ is not degenerate.
\item The case $i\in E_1$ is similar to above.
\item If $i\in E_{01}$ we have
$\alpha_p\eta_1^i+\beta_p\eta_2^i=\alpha_q\eta_1^i+\beta_q\eta_2^i=0$ which
implies $\alpha_p/\beta_p=\alpha_q/\beta_q=-\eta_2^i/\eta_1^i$, or
$\alpha_p/\alpha_q = \beta_p/\beta_q$ which contradicts the fact that $X$ is
not degenerate. Note that all denominators can be argued to be non-zero for $i\in E_{01}$. 
\end{itemize}

From the above, we conclude that each row has at least $d$ entries as $1$.
Counting by columns, we have that each column has at most $e$ entries as $1$
(since $\wt{\eta_i}\leq e$ for all $i\in [d]$. Comparing the lower and upper
bounds on the number of $1$ entries in the matrix we have $td \leq e(d+1)$
which implies $t \leq e + e/d < e + 1$. Thus $t\leq e$, as we wanted to show.
This completes the proof.

\end{proof}

The following result underlies proximity protocols in our work and in
\cite{ligero}. Intuitively the result states that if a matrix is far away from the code $\mc{C}$, a random linear combination of its rows is far away from a codeword in $L$, and thus the proximity of the matrix to $\mc{C}$ may be tested by testing the proximity of a random linear combination of its rows to $L$.

\begin{lemma}[Affine Subspace]\label{lem:affinesubspace}
Let the codes $L$ and $\mc{C}$ be as defined and $e<d/3$ be an integer. Let $U\in \HH^{m\times n}$ be a matrix such that $d(U,\mc{C})>e$. Then for any $u_0\in \HH^n$, $\prob{d(u_0+r^TU,L)\leq e}\leq d/|\FF|$ for uniformly sampled $r\sample \FF^m$.
\end{lemma}
\begin{proof}
From Lemma \ref{lem:farpoint}, there exists $u\in \aff{u_0}{U}$ such that $d(u,L)>e$. Now we can write $\aff{u_0}{U}$ as union of affine lines passing through $u$. Applying lemma \ref{lem:affineline} to each line, we see that at most $d$ points $x$ on each affine line satisfy $d(x,L)\leq e$. Thus, a randomly sampled point $x$ in $\aff{u_0}{U}$, equivalently obtained as $u_0+r^TU$ for a randomly sampled vector $r\in \FF^m$ satisfies $d(x,L)$ with probability at most $d/|\FF|$.
\end{proof}

We are now in a position to prove Proposition \ref{lem:3dcompression}.
\begin{proof}[Proof of Proposition \ref{lem:3dcompression}]
Let $\HH$ denote the field $\FF^m$. Then $u_0$ can be viewed as a point in
$\HH^n$. Similarly $U$ can be viewed as $p\times n$ matrix over $\HH$.
We consider $\mc{C}$ as $[n,k,d]$ code over $\HH$ and $\mc{C}^p$ as the
interleaved code of $\mc{C}$ over the field $\HH$. Then by applying Lemma \ref{lem:affinesubspace}
with $\HH=\FF^m$ and codes $\mc{C}$ and $\mc{C}^p$ in place of codes $L$ and
$\mc{C}$, we have the desired bound.
\end{proof}

\subsection{Proof of knowledge for Linear Check}\label{app:linear_soundness}
\begin{proof}[Proof-Sketch of Lemma ~\ref{lem:linearcheck_sound}]
Suppose the oracle $\pi$ commits to $\ewit$. In full version, we show how
 this $\ewit$ can be ``extracted'' by the appropriate extractor. Note that $\ewit\in
(\mc{W}_2)^p$ as a commitment implicitly corresponds to such a matrix. Let $e<(n-\ell)/3$ be a parameter. First we show that 
an adversarial prover succeeds with negligible probability if $d(\ewit,(\mc{W}_1)^p)>e$. Second, we
show that for $d(\ewit,(\mc{W}_1)^p)\leq e$, the prover succeeds with negligible probability when
$A\wit\neq b$ where $\wit=\dec(\ewit)$. Consider the case when $d(\ewit,(\mc{W}_1)^p)>e$. Then for
$\tilde{\ewit}=\sum_{i\in [p]}\rho_i\ewit[i,\cdot,\cdot]$, by Proposition ~\ref{lem:3dcompression},
 we have $d(\tilde{\ewit},\mc{C}_1)>e$ with probability $1-o(1)$. Let $\tilde{\bm{c}}=(\tilde{c}_1,
\ldots,\tilde{c}_\ell)$ be the commitments to $\tilde{\ewit}$ sent by the prover
(Step 3 in Figure ~\ref{fig:linearcheck}). Define 
the vector $\tilde{\bm{C}}=(\tilde{C}_1,\ldots,\tilde{C}_n)$ where
$\tilde{C}_k=\prod_{a=1}^\ell (\tilde{c}_a)^{\Lambda_{n,\ell}^T[a,k]}$ for $k\in [n]$.
 Let $\hat{\bm{C}}=(\hat{C}_1,\ldots,\hat{C}_n)$ where
$\hat{C}_k=\prod_{i=1}^p(\pi[i,k])^{\rho_i}$. Now if
$\Delta(\tilde{\bm{C}},\hat{\bm{C}})>e$, we see that the prover succeeds in the
proximity check equation \eqref{eq:proxchecks} with probability at most $(1-e/n)^t$. 
In the complete
proof, we show that when $\Delta(\tilde{\bm{C}},\hat{\bm{C}})\leq e$, 
while $d(\tilde{\ewit},\mc{C}_1)>e$, the prover knows two openings to a commitment.
Thus an adversarial prover succeeds with probability at most $(1-e/n)^t$ when
$d(\ewit,(\mc{W}_1)^p)>e$.

We now consider the case when $d(\ewit,(\mc{W}_1)^p)\leq e$. From Lemma
~\ref{lem:bicdecoding}, there exists (unique) $\ewit^*\in (\mc{W})^p$
such that $\Delta_1(\ewit,\ewit^*)\leq e$.
Let $\wit=\dec(\ewit)=\dec(\ewit^*)$. We consider the prover's success
probability when $A\wit\neq b$, and thus with overwhelming probability $r^TA \wit \neq
r^Tb$. Let $P^*$ denote the correctly computed intermediate matrix from $\ewit^*$ and let
$\hat{P}$ denote the correctly computed intermediate matrix from $\ewit$. We note that
$\Delta_1(\hat{P},P^*)\leq e$. Let $c_1,\ldots,c_{s+\ell-1}$ be the commitments
to the intermediate matrix sent by the prover. If these commitments correspond to a
matrix $P=P^*$, the inner product check
$\innerproduct(\mathsf{pp},[1^m||0^{m-1}],\mathsf{cm},r^Tb)$ fails when using the
commitment $\mathsf{cm}=\prod_{k=1}^{s+\ell-1}c_k^{\varphi_k}$ for
$\varphi=\Phi\times [1^s]$. This is because $\innp{1^m||0^{m-1}}{P^*[1:2m-1,1:s+\ell-1]\times
\varphi}=r^TA\wit\neq r^Tb$, %where $\overline{P^*}$ denotes the $(2m-1)\times (s+\ell-1)$ submatrix of $P^*$,
as discussed in the protocol. If the
commitments correspond  to a matrix $P\neq P^*$, we have
$\Delta_1(P,P^*)\geq n-s-\ell$ by distance property of the code
$\rsc{\eta}{n, s+\ell-1}$. 
(We note that a prover implicitly commits to a matrix in
$\rsc{\eta}{n,s+\ell-1}\otimes \rsc{\alpha}{h,2m-1}$).
Thus there exists a set $E$ of at least $n-s-\ell-e$ columns, such that for
$k\in E$, $\hat{P}[\cdot,k]=P^*[\cdot,k]\neq P[\cdot,k]$. The checks
$\innp{\bm{1}_{j_u}}{\ewit[i,\cdot,k_u]}=X_u[i]$ for $i\in [p]$ and $u\in [t]$ 
force the prover to provide vectors $X_u=\ewit[\cdot,j_u,k_u]$ with
overwhelming probability. Then the consistency check succeeds for the 
uniformly sampled query point $(j_u,k_u)$ when:
\[ P[j_u,k_u] = \sum_{i\in [p]}R^i(\alpha_{j_u},\eta_{k_u})X_u[i] =
\hat{P}[j_u,k_u] \]
For $k_u\in E$, the above holds when the distinct codewords $P^*[\cdot,k_u]$ and
$P[\cdot,k_u]$ in $\rsc{\alpha}{h,2m-1}$ agree on the position $j_u$, which happens with 
probability at most $2m/h$. Thus probability $\prob{\mc{E}_u}$ that above query succeeds is bounded by:
\begin{align*}
\prob{\mc{E}_u}&\leq \frac{s+\ell+e}{n} + \frac{n-s-\ell-e}{n}\cdot\frac{2m}{h}
\\
&= \frac{2m}{h}+\left(1-\frac{2m}{h}\right)\left(\frac{s+\ell+e}{n}\right)
\end{align*}
The above probability is smaller than a constant $\epsilon<1$ for suitable
choices of parameters. Hence, the overall probability of prover's success is
$\negl(\secpar)$ for $t=O(\secpar)$.

\smallskip
\paragraph{Extraction}: Let $\extrac_{ip}$ be the extractor of the inner product argument which takes, in expectation, $\Pip(n)$ amount of time, where $\Pip(\cdot)$ is polynomial, to output the private vector of length or breaking the binding of the commitment scheme. We will design an extractor $\extrac$ for $\linearcheck$, which uses $\extrac_{ip}$.

\smallskip
If $\prover^\ast$ fails in the proximity check (step 12) then $\verifier$ outputs reject, and so the extractor $\extrac$ terminates with $\abort$.
If $\prover^\ast$ succeeds in the proximity check (step 12) then $\ewit\in \mc{W}_1$ and $\extrac$ proceeds in the following way:

\smallskip
$\extrac$ plays the role of the verifier and rewinds the provers polynomially many times if required.

\smallskip
$\extrac$ runs the protocol till step 8, sends $Q$ and receives $\ewit[\cdot, j_u,k_u]$. %Then it rewinds $\prover^\ast$ and sends new random $Q$. $\extrac$ keeps rewinding till $k_u$'s of all the $Q$s covered $[n]$. In expectation it takes $O(n\log(n))$ rewinds.

$\extrac$ picks random $\delta\in \FF^p$ and $\beta\in\FF^\ast$ and proceeds to run the inner product arguments. $\extrac$ uses $\extrac_{ip}$ and gets:
\begin{itemize}
	\item  $\overline{P}[\cdot,k_u]$ in expected time $\Pip(m)$, $\forall u\in[t]$
	\item $z=\beta P_0 + \overline{P}\varphi$ in expected time $\Pip(m)$
	\item $V_u = \sum_{i\in[p]} \delta_i\ewit[i,\cdot,k_u]$ in expected time $\Pip(m)$, $\forall u\in[t]$
\end{itemize}
Then $\extrac$ rewinds $\delta, \beta$ $O(p\log(p))$ times and gets $P_0$ from $z$ and from $V_u$, $\extrac$ gets $\ewit[\cdot,\cdot,k_u]$. $\extrac$ checks $\ewit[\cdot,j_u,k_u]$ received in step 8, matches with the extracted $\ewit[\cdot,\cdot,k_u]$'s $j_u$ position or not. If does not, then $\extrac$ gets two opening of $\pi[\cdot,k_u]$ and outputs $\abort$, otherwise $\extrac$ proceeds in the following way: $\extrac$ rewinds $\prover^\ast$ and sends uniformly random $Q$ and keeps repeating untill all the $Q$'s together cover all the columns. It takes $n\log(n)$ rewindings in expectation. Then $\extrac$ extracts the whole $\ewit$ and checks if all the vectors are consistent or not. If not, that gives the break for the binding of the commitment scheme and if consistent, then it decodes $\ewit$ and outputs correct witness $\wit$. 
\smallskip
Therefore, expected Run time of $\extrac$ is $O(n\log(n)(O(p\log(p))(\Pip(m)) + \Pip(m) + \Pip(m))) $, which is polynomial in the size of the circuit.
\end{proof}

\subsection{Proof of knowledge of Quadratic Check }\label{app:quad_soundness}
\begin{proof}[Proof-Sketch of Lemma ~\ref{lem:quadcheck_sound}]
	Suppose the oracle $\pi$ commits to $\ewit_x, \ewit_y, \ewit_z$. In full version, we show how
	this $\ewit_x, \ewit_y, \ewit_z$ can be ``extracted'' by the appropriate extractor. Note that $\ewit_x||\ewit_y||\ewit_z\in
	(\mc{W}_2)^{3p}$, where $\ewit = \ewit_x||\ewit_y||\ewit_z$ is the juxtaposing along $p$ direction, as a commitment implicitly corresponds to such a matrix. Let $e<(n-\ell)/3$ be a parameter. First we show that 
	an adversarial prover succeeds with negligible probability if $d(\ewit,(\mc{W}_1)^{3p})>e$. Second, we
	show that for $d(\ewit,(\mc{W}_1)^{3p})\leq e$, the prover succeeds with negligible probability when
	$\wit_x\circ\wit_y\neq\wit_z$ where $\wit_a=\dec(\ewit_a)$ for $a\in\{x,y,z\}$. Consider the case when $d(\ewit,(\mc{W}_1)^{3p})>e$. Then for
	$\tilde{\ewit}=\sum_{i\in [3p]}\rho_i\ewit[i,\cdot,\cdot]$, by Proposition ~\ref{lem:3dcompression},
	we have $d(\tilde{\ewit},\mc{C}_1)>e$ with probability $1-o(1)$. Let $\tilde{\bm{c}}=(\tilde{c}_1,
	\ldots,\tilde{c}_\ell)$ be the commitments to $\tilde{\ewit}$ sent by the prover
	(Step 3 in Figure ~\ref{fig:quadcheck}). Define 
	the vector $\tilde{\bm{C}}=(\tilde{C}_1,\ldots,\tilde{C}_n)$ where
	$\tilde{C}_k=\prod_{a=1}^\ell (\tilde{c}_a)^{\Lambda_{n,\ell}^T[a,k]}$ for $k\in [n]$.
	Let $\hat{\bm{C}}=(\hat{C}_1,\ldots,\hat{C}_n)$ where
	$\hat{C}_k=\prod_{i=1}^{3p}(\pi[i,k])^{\rho_i}$. Now if
	$\Delta(\tilde{\bm{C}},\hat{\bm{C}})>e$, we see that the prover succeeds in the
	proximity check equation \eqref{eq:proxchecks} with probability at most $(1-e/n)^t$. 
	In the complete
	proof, we show that when $\Delta(\tilde{\bm{C}},\hat{\bm{C}})\leq e$, 
	while $d(\tilde{\ewit},\mc{C}_1)>e$, the prover knows two openings to a commitment.
	Thus an adversarial prover succeeds with probability at most $(1-e/n)^t$ when
	$d(\ewit,(\mc{W}_1)^{3p})>e$.
	
	We now consider the case when $d(\ewit,(\mc{W}_1)^{3p})\leq e$. From Lemma
	~\ref{lem:bicdecoding}, there exists (unique) $\ewit^*\in (\mc{W})^{3p}$
	such that $\Delta_1(\ewit,\ewit^*)\leq e$.
	Let $\wit_a=\dec(\ewit_a)=\dec(\ewit^*_a)$. We consider the prover's success
	probability when $\wit_x\circ \wit_y = \wit_z$, and thus with overwhelming probability $\sum_{i\in[p]} r_i\cdot(\wit_x[i,\cdot,\cdot]\cdot\wit_y[i,\cdot,\cdot]-\wit_z[i,\cdot,\cdot]) = [\bm{0}]^{ms}$. Let $P^*$ denote the correctly computed intermediate matrix from $\ewit^*$ and let
	$\hat{P}$ denote the correctly computed intermediate matrix from $\ewit$. We note that
	$\Delta_1(\hat{P},P^*)\leq e$. Let $c_1,\ldots,c_{2\ell-1}$ be the commitments
	to the intermediate matrix sent by the prover. If these commitments correspond to a
	matrix $P=P^*$, the inner product check
	$\innerproduct(\mathsf{pp},[\gamma||0^{m-1}],\mathsf{cm},0)$ fails when using the
	commitment $\mathsf{cm}=\prod_{k=1}^{2\ell-1}c_k^{\varphi_k}$ for
	$\varphi=\Phi\times \tau$. This is because $\innp{\gamma||0^{m-1}}{P^*[1:2m-1,1:2\ell-1]\times
		\varphi}=\sum_{j\in[m]} \gamma_j\sum_{k\in[s]} \tau_k\sum_{i\in[p]}r_i[\wit_x[i,j,k]\cdot\wit_y[i,j,k]-\wit_z[i,j,k]] \neq 0$, %where $\overline{P^*}$ denotes the $(2m-1)\times (s+\ell-1)$ submatrix of $P^*$,
	as discussed in the protocol. If the
	commitments correspond  to a matrix $P\neq P^*$, we have
	$\Delta_1(P,P^*)\geq n-2\ell$ by distance property of the code
	$\rsc{\eta}{n, 2\ell-1}$. 
	(We note that a prover implicitly commits to a matrix in
	$\rsc{\eta}{n,2\ell-1}\otimes \rsc{\alpha}{h,2m-1}$).
	Thus there exists a set $E$ of at least $n-2\ell-e$ columns, such that for
	$k\in E$, $\hat{P}[\cdot,k]=P^*[\cdot,k]\neq P[\cdot,k]$. The checks
	$\innp{\bm{1}_{j_u}}{W_u}=\sum_{i\in[p]}\delta_i(\beta_x X_u + \beta_y Y_u + \beta_z Z_u)$ for $i\in [p]$ and $u\in [t]$ 
	force the prover to provide vectors $X_u=\ewit_x[\cdot,j_u,k_u], Y_u=\ewit_y[\cdot,j_u,k_u],
	Z_u=\ewit_z[\cdot,j_u,k_u]$ with
	overwhelming probability. Then the consistency check succeeds for the 
	uniformly sampled query point $(j_u,k_u)$ when:
	\[ P[j_u,k_u] = \sum_{i\in [p]}r_i(X_u[i]\cdot Y_u[i]-Z_u[i]) =
	\hat{P}[j_u,k_u] \]
	For $k_u\in E$, the above holds when the distinct codewords $P^*[\cdot,k_u]$ and
	$P[\cdot,k_u]$ in $\rsc{\alpha}{h,2m-1}$ agree on the position $j_u$, which happens with 
	probability at most $2m/h$. Thus probability $\prob{\mc{E}_u}$ that above query succeeds is bounded by:
	\begin{align*}
	\prob{\mc{E}_u}&\leq \frac{2\ell+e}{n} + \frac{n-2\ell-e}{n}\cdot\frac{2m}{h}
	\\
	&= \frac{2m}{h}+\left(1-\frac{2m}{h}\right)\left(\frac{2\ell+e}{n}\right)
	\end{align*}
	The above probability is smaller than a constant $\epsilon<1$ for suitable
	choices of parameters. Hence, the overall probability of prover's success is
	$\negl(\secpar)$ for $t=O(\secpar)$.
	
	\smallskip
	\paragraph{Extraction}: Let $\extrac_{ip}$ be the extractor of the inner product argument which takes, in expectation, $\Pip(n)$ amount of time, where $\Pip(\cdot)$ is polynomial, to output the private vector of length or breaking the binding of the commitment scheme. We will design an extractor $\extrac$ for $\quadcheck$, which uses $\extrac_{ip}$.
	
	\smallskip
	If $\prover^\ast$ fails in the proximity check (step 15) then $\verifier$ outputs reject, and so the extractor $\extrac$ terminates with $\abort$.
	If $\prover^\ast$ succeeds in the proximity check (step 15) then $\ewit\in \mc{W}_1$ and $\extrac$ proceeds in the following way:
	
	\smallskip
	$\extrac$ plays the role of the verifier and rewinds the provers polynomially many times if required.
	
	\smallskip
	$\extrac$ runs the protocol till step 7, sends $Q,\gamma, \tau$ and receives $\ewit_x[\cdot, j_u,k_u], \ewit_y[\cdot, j_u,k_u], \ewit_z[\cdot,j_u,k_u]$. %Then it rewinds $\prover^\ast$ and sends new random $Q$. $\extrac$ keeps rewinding till $k_u$'s of all the $Q$s covered $[n]$. In expectation it takes $O(n\log(n))$ rewinds.
	
	$\extrac$ picks random $\delta\in \FF^p$, $\beta_x,\beta_y,\beta_z\in\FF$ and $\beta\in\FF^\ast$ and proceeds to run the inner product arguments. $\extrac$ uses $\extrac_{ip}$ and gets:
	\begin{itemize}
		\item  $\overline{P}[\cdot,k_u]$ in expected time $\Pip(m)$, $\forall u\in[t]$
		\item $z=\beta P_0 + \overline{P}\varphi$ in expected time $\Pip(m)$
		\item $V_u = \sum_{i\in[p]} \delta_i(\beta_x\ewit_x[i,\cdot,k_u]+\beta_y\ewit_y[i,\cdot,k_u]+\beta_z\ewit_z[i,\cdot,k_u])$ in expected time $\Pip(m)$, $\forall u\in[t]$
	\end{itemize}
	Then $\extrac$ rewinds $\delta, \beta_x,\beta_y,\beta_z,\beta$ $O(p\log(p))$ times and gets $V_u$, $\extrac$ gets $\ewit_x[\cdot,\cdot,k_u], \ewit_y[\cdot,\cdot,k_u], \ewit_z[\cdot,\cdot,k_u]$ by solving a system of equation with $3p$ unknowns. $\extrac$ checks $\ewit_a[\cdot,j_u,k_u]$ received in step 7, matches with the extracted $\ewit_a[\cdot,\cdot,k_u]$'s $j_u$ position or not, for $a\in\{x,y,z\}$. If does not, then $\extrac$ gets two opening of $\pi_a[\cdot,k_u]$, for some $a\in\{x,y,z\}$ and outputs $\abort$, otherwise $\extrac$ proceeds in the following way: $\extrac$ rewinds $\prover^\ast$ and sends uniformly random $Q$ and keeps repeating untill all the $Q$'s together cover all the columns. It takes $n\log(n)$ rewindings in expectation. Then $\extrac$ extracts the whole $\ewit_x,\ewit_y,\ewit_z$ and checks if all the vectors are consistent or not. If not, that gives the break for the binding of the commitment scheme and if consistent, then it decodes $\ewit_a$ and outputs correct witness $\wit_a$ for $a\in\{x,y,z\}$. 
	\smallskip
	Therefore, expected Run time of $\extrac$ is $O(n\log(n)(O(p\log(p))(\Pip(m)) + \Pip(m) + \Pip(m))) $, which is polynomial in the size of the circuit.
\end{proof}
%%%%%%%%%%%%%%%%%%%%%%%%%%%%%%%%%%%%%%%%%%%%%%%
\subsection{Proof of soundness of Distributed Linear Check}
\begin{lemma}[Soundness with Extraction]\label{lem:distlinearcheck_sound}
	For all polynomially bounded adversary $\Adv$ controlling all the provers $\prover_1, \ldots, \prover_{\Num}$ and all $\pi\in\GG^{p\times n}$, $A\in\FF^{M\times N}, b\in\FF^M$, there exists an polynomial time extractor $\extrac$ with rewinding access to the transcript $\tr$ between the aggregator $\Ag$ and the verifier $\verifier$ such that $\extrac$ either breaks the commitment binding or outputs a witness with overwhelming probability whenever $\Adv$ succeeds, i.e.,
	
	{\small
		\begin{align*}
		\condprob{\begin{array}{c}
			\ewit=\open(\pi)\wedge \\
			A\wit=b
			\end{array}
		}{
			\begin{array}{c}
			\sigma\gets \gen(\secparam) \\
			\ewit\gets \extr^{\mathsf{tr}}(\bm{x},\sigma) \\
			\wit\gets \dec(\ewit)
			\end{array}}\geq
		\epsilon(\Adv)-\kappa_{lc}(\secpar)
		\end{align*} 
	}
where $\epsilon(\Adv):= \condprob{\langle \Adv(\bm{x},\sigma),\verifier^\pi(\bm{x},\sigma)\rangle=1}{\sigma\gets \gen(\secparam)}$ denotes the success probability of $\Adv$, $\kappa_{lc}$ denotes a negligible function, and $\bm{x}$ denotes the tuple $(A,b,M,N)$.
\end{lemma}

%%%%%%%%%%%%%%%%%%%%%%%%%%%%%%%%%%%%%%%%%%%%%%%%%%%%%%%%%%%
\subsection{Proof of soundness of Distributed Quadratic Check}
\begin{lemma}[soundness with extraction]\label{lem:distquadcheck_sound}
	For all polynomoially bounded adversary $\Adv$, controlling all the provers $\prover_1, \ldots, \prover_{\Num}$ and $\pi\in\GG^{3p\times n}$, there exists an expected polynomial time extractor $\extrac$ with rewinding access to the transcript $\tr$ between the aggregator $\Ag$ and the verifier $\verifier$ such that $\extrac$ either breaks the commitment binding, or it outputs a witness with overwhelming probability whenever $\Adv$ succeeds, i.e.,
	
	{\footnotesize
		\begin{align*}
		\condprob{
			\begin{array}{c}
			{[}\ewit_x||\ewit_y||\ewit_z{]}=\open(\pi)\wedge \\
			\wit_z=\wit_x\circ\wit_y
			\end{array}
		}{
			\begin{array}{c}
			\sigma %\sample \gen(\secparam) \\
			\leftarrow \gen(\secparam) \\
			{[}\ewit_x||\ewit_y||\ewit_z{]}%\sample \extr^{\mc{O}}(\sigma)\\
			\leftarrow \extr^{\mathsf{tr}}(\sigma) \\ 
			\wit_a=\dec(\ewit_a), a\in \{x,y,z\}
			\end{array}
		}\\
		\geq \epsilon(\Adv) - \kappa_{\rm qd}(\secpar)
		\end{align*}
	}
	for some negligible function $\kappa_{qd}$. In the above, $\epsilon(\Adv)$
	denotes the success probability of the prover $P^\ast$ as before.
\end{lemma}
%%%%%%%%%%%%%%%%%%%%%%%%%%%%%%%%%%%%%%%%%%%%%%%%%%%%%%%%%%%

\section{cryptographic preliminaries}
\begin{comment}
\subsection{Commitment scheme} \label{app:DefCommitment}
\begin{definition}\label{defn:commscheme}
 A pair of $\ppt$ algorithms $(\gen,\com)$ constitute a non-interactive commitment scheme if $\sigma\sample \gen(\secparam)$ consists of description of sets $\calM_\sigma$ (message space), $\calR_\sigma$ (randomness space), $\calC_\sigma$ (commitment space) and an efficiently computable function $\com_\sigma: \calM_\sigma\times \calR_\sigma\rightarrow \calC_\sigma$ which is {\em hiding} and {\em binding} as defined later.
\end{definition}

For $x\in \calM_\sigma$, we generate a {\em commitment} of $x$ as $\com_\sigma(x,r)$ where $r\sample \calR_\sigma$ is drawn uniformly at random. For ease of notation, we simply use $\com$ instead of $\com_\sigma$ and use $\com(x)$ to denote the random variable corresponding to commitment of $x$. 

\begin{definition}[Hiding Commitment]\label{defn:hidingcomm}
A commitment scheme $(\gen,\com)$ is called perfectly {\em hiding}  if for all adversaries $\adv$, the following probability is  $1/2$:
{\small
\begin{align*}
\condprob{b=b'}{
\begin{array}{l}
\sigma\gets \gen(\secparam); \\
(x_0,x_1)\in \calM^2_p\gets \adv(\sigma); \\
b\sample \bitset; c\gets \com(x_b);\\
b'\gets \adv(\secparam,\sigma,c)
\end{array}
}
\end{align*}
}
\end{definition}

\begin{definition}[Binding Commitment]\label{defn:bindingcomm}
A commitment scheme $(\gen,\com)$ is called {\em binding} if for all $\ppt$  $\adv$, 
{\small
\begin{align*}
\condprob{
	\begin{array}{c}
	\com_\sigma(x_0,r_0)=\com_\sigma(x_1,r_1)\\
	\wedge x_0\neq x_1
\end{array}}{
\begin{array}{l}
\sigma \gets \gen(\secparam) \\
x_0,x_1,r_0,r_1 \gets \adv(\secparam,\sigma)
\end{array}
}\\
 < \negl(\lambda)
\end{align*}
}

\end{definition}

\subsection{Inter-product Arguments} \label{inner-product-instan}
\noindent{\em Logarithmic Inner-product Argument}: In this setting, we have $\calM=\bbZ^n_p$, $\calR=\bbZ_p$, $\calC=\bbG$ where $\bbG$ is group of prime order $p$. The algorithm $\gen$ samples generators $g_1,\ldots,g_n$, $h$ $\gets \bbG$. 
The commitment is a Pederson vector commitment  given by $\com({\bf x } ,r ) = h^r \cdot \prod_{i=1}^n {g_i}^{x_i}$ where ${\bf x}=(x_1,\ldots,x_n)$. 
We use the inner-product argument $(\prover,\verifier)$ from Bootle et. al in \cite{bulletproofs} for the commitment scheme $(\gen,\com)$. The interactive protocol $(\prover,\verifier)$ is a $O(\log n)$ round protocol with argument size $O(\log n)$. The verifier $\viplog$ performs $O(n)$ operations in $\bbZ_p$ and $O(n)$ operations in $\bbG$.\smallskip


\noindent{\em Square-root Inner-product Argument}: In this setting we use the same commitment scheme as above. For the inner product argument we use the interactive protocol $(\pipsq,\vipsq)$ from \cite{InnerProductDLS}. The construction in \cite{InnerProductDLS} gives a $5$-round protocol with total communication complexity $O(\sqrt{n})$. Here, the verifier $\vipsq$ performs $O(n)$ operations in $\bbZ_p$ and $O(\sqrt{n})$ operations in $\bbG$.
\end{comment}

\begin{comment}
\subsection{Forking Lemma and Knowledge Soundness}
We use the Forking Lemma from \cite{InnerProductDLS,bulletproofs} to describe 
expected polynomial time knowledge extractors for our protocols. Let
$(\prover,\verifier)$ be a public coin $(2\mu+1)$-move interactive protocol with
challenges $x_1,\ldots,x_\mu$ in sequence. Let $n_i\geq 1$ for $1\leq i\leq
\mu$. We call a collection of $n=\prod_{i=1}^\mu n_i$ accepting transcripts to be
$(n_1,\ldots,n_\mu)$-tree of accepting transcripts, if the challenges are
organized in the tree format that we describe now: \commentA{the following didn't make sense. what does a path from root to a leaf node signify? one sequence of challenges for an accepting execution? at any rate this portion needs a clean write-up} The root of the tree is
labelled with the statement being proved. Each node of depth $i<\mu$ has exactly
$n_i$ children, each labelled with a distinct value of the $i^{th}$ challenge
$x_i$. We call a $\ppt$ algorithm $\chi$ to be a {\em witness extraction
algorithm} if $\chi$ can extract a witness $w$ to the statement, given an
appropriate tree of accepting transcripts. This can be seen as a generalization
of the notion of special soundness for Sigma protocols with $n=2$ and $\mu=1$.
\end{comment}

% remove argument of knowledge
\begin{comment}
\begin{definition}[Argument of Knowledge]\label{def:argofknowledge}
Let $(\setup,\prover,\verifier)$ be an public coin interactive protocol. We say that
$(\setup,\prover,\verifier)$ is an {\em argument of knowledge} for the language
$\mc{L}$ if for every $\ppt$
prover $P^\ast$, there exists an expected polynomial time extractor $\extr$ such that for all $x$:
%\begin{comment}
{\small
\begin{align*}
\condprob{(x,w)\in \mc{L}}{
\begin{array}{c}
\sigma\gets\setup(\secparam) \\
w\gets \extr^{\mc{O}}(x,\sigma)
\end{array}
}\\
\geq \condprob{\langle
P^\ast(x,\sigma),\verifier(x,\sigma)\rangle=1}{\sigma\gets \setup(\secparam)} -
\negl(\lambda)
\end{align*}}
%\end{comment}
%\begin{align}
%&\prob{(x,w)\in\mc{L}| \sigma \sample \setup(\secparam); w\sample \extr^{\mc{O}}(x,\sigma)}\\
%\geq & \prob{\innp{P^*(x,\sigma)}{\verifier(x, \sigma)} = 1|\sigma\sample \setup(\secparam)} - \kappa(\lambda)
%\end{align}
for some negligible function $\negl$. In the above $\mc{O}$ denotes the transcript oracle  $\langle P^\ast(x,\sigma)$ ,$\verifier(x,\sigma)\rangle$which can be rewound to any previous state, and resumed with fresh randomness for the
verifier $\verifier$.
\end{definition}

Our argument of knowledge proofs rely on the following result from \cite{bulletproofs}.
While the result is originally stated and proved for showing witness-extended
emulation, we restate it for the case of argument of knowledge. The proof in
\cite{InnerProductDLS} also applies to this restricted case.
\end{comment}


\begin{lemma}[Forking Lemma,\cite{bulletproofs}]\label{lem:forkinglemma}
Let $(\setup,\prover,\verifier)$ be a $(2\mu+1)$-round public coin interactive
protocol. Let $\chi$ be a $\ppt$ witness extraction algorithm that outputs a
witness with probability $1-\negl(\secpar)$ from an $(n_1,\ldots,n_\mu)$-tree
of accepting transcripts. Assuming that $\prod_{i=1}^\mu n_i$ is bounded by a
polynomial in security parameter $\secpar$, the protocol
$(\setup,\prover,\verifier)$ is an argument of knowledge.
\end{lemma}
\commentA{change 'move' to 'round' in all the above text.}


\section{Proofs}
\begin{proof}[Proximity 2D]
The extractor $\extr$ starts by emulating the verifier in the protocol
$\proximityTwoD$. At the step 5, it uses the extractor $\extr_{ip}$ for the
inner product protocol to extract the witness $z$ for the $\innerproduct$
subprotocol. If the initial transcript rejects, $\extr$ fails with output
$\bot$. Otherwise, it rewinds transcript to step 2, till it finds $\ell+1$
additional accepting transcripts. Let $\tau^i,\delta^i$ denote the random
vectors in step 2, in the $i^{th}$ accepting transcript (we let $i=0$ denote the
initial transcript). Similarly, let $x^i$, $\mathsf{cm}^i$ denote the values of
vectors $x$ and $\mathsf{cm}$ in the $i^{th}$ accepting transcript, while $(z^i,o^i)$
denotes the witness extracted by the inner product extractor $\extr_{ip}$
corresponding to $x^i$ and $\mathsf{cm}^i$. If the inner product extractor
$\extr_{ip}$ fails with $z^i=\bot$ for any $i\in [\ell+1]$, the extractor $\extr$
fails with output $\bot$. From the inner product extractions we have the
following:
\begin{align*}
\begin{array}{rlll}
\comm(z^1,o^1)=& c_0 + \mu^1_1c_1+ & \hdots & +\mu^1_{\ell}c_{\ell} \\
\vdots & \vdots & \ddots & \vdots \\
\comm(z^{\ell+1},o^{\ell+1})=& c_0 + \mu^{\ell+1}_1c_1+ & \hdots & +\mu^{\ell+1}_{\ell}c_{\ell} 
\end{array}
\end{align*}
Let $\Lambda$ denote the $(\ell+1)\times (\ell+1)$ matrix $[1^{\ell+1}|\mathbf{M}]$ with $\mathbf{M}[i,j]=\mu^i_j$. For convenience of notation, we will consider row and column indices of $(\ell+1)\times (\ell+1)$ matrices in this proof to be over the set $\{0,\ldots,\ell\}$. 
Since $\mc{T}$ is a rank $\ell$ matrix, we see that $\Lambda$ is an invertible
matrix for random choices of vectors $\tau^i$, except with probability
$1/|\FF|$. Let $\Omega$ be the $(\ell+1)\times (\ell+1)$ matrix such that
$\Omega\Lambda=I_{\ell+1}$. Define $U_k = \sum_{i=0}^{\ell}\Omega[k,i]z^i$ and $\omega_k=\sum_{i=0}^{\ell}\Omega[k,i]o^i$. 
$k\in [\ell]$. Then for $k\geq 1$ we have,
\begin{align*}
\comm(U_k,\omega_k) &= \sum_{i=0}^{\ell}\Omega[k,i]\comm(z^i,o^i)  \\
	&= \sum_{i=0}^{\ell}\Omega[k,i]\sum_{j=0}^{\ell}\Lambda[i,j]c_j \\
	&= \sum_{j=0}^{\ell}\big(\sum_{i=0}^{\ell}\Omega[k,i]\Lambda[i,j]\big)c_j
\\
	&= c_k \text{ (using $\Omega\Lambda=I_{\ell+1})$ }
\end{align*}
Thus $\overline{U}=(U_1,\ldots,U_{\ell})$ and $\bm{\omega}=(\omega_1,\ldots,\omega_{\ell})$ satisfies $(\overline{U},\bm{\omega})=\open(\bm{c})$. Now we
consider the probability that $\overline{U}\mc{T}\in \mc{C}_2$. Let $A$ denote the event
that the first transcript accepts. Let $B\subseteq A$ be the event that
$(\overline{U}\neq\bot)\cap (\overline{U}\mc{T}\in \mc{C}_2)$. We have,
\begin{align*}
&\prob{B}\\ &= \prob{A} - \prob{A\cap \neg B} \\
	&\geq \prob{A} - \prob{A\cap (\overline{U}=\bot)} - \prob{A\cap
(\overline{U}\mc{T}\not\in \mc{C}_2)} \\
	&= \prob{A} - \prob{A\cap (\overline{U}=\bot)} - \prob{A\cap
(\mc{T}^T\overline{U}^T\mc{H}_2\neq 0)} \\
	&\geq \prob{A} - \prob{A\cap (\overline{U}=\bot)} -\\
& \text{     }\condprob{\tau^T\mc{T}^T\overline{U}^T\mc{H}_2\delta=
0}{\mc{T}^T\overline{U}^T\mc{H}_2\neq 0} \\
	&\geq \prob{\langle P^\ast(\bm{c},\sigma),\verifier(\bm{c},\sigma)\rangle=1}
- \big(\ell.\kappa_{ip}(\secpar)+1/|\FF|\big) - 2/|\FF|
\end{align*} 
In the above, we bound the probability $\condprob{\tau^T\mc{T}^T\overline{U}^T\mc{H}_2\delta=
0}{\mc{T}^T\overline{U}^T\mc{H}_2\neq 0}$ by $2/|\FF|$ as $\tau$ and $\delta$ are
distributed uniformly and independently of the extracted witness $\overline{U}$. The
probability $\prob{A\cap (\overline{U}=\bot)}$ is bound in terms of the probability
of inner product extractor returning an invalid witness or $\Lambda$ being
singular which is at most $\ell.\kappa_{ip}(\secpar)+1/|\FF|$. The
statement of the lemma now holds by setting
$\kappa_{2d}(\secpar):=\ell.\kappa_{ip}(\secpar) + 3/|\FF|$.
\end{proof}

%%% proximity 3d proof

\begin{proof}[Proximity 3D]
As before the extractor $\extr$ emulates the verifier in the protocol
$\proximityThreeD$. In case the first transcript is accepting, the extractor
rewinds the prover sufficiently many times to obtain a tree of accepting transcripts with $p$ distinct
choices of the randomness $r$ in step (1) and $\theta := tn$ different values of $Q$
on step (5). Let $r^{1},\ldots,r^{p}$ denote the values of $r$ in step (1)
and let $\tilde{\mathbf{c}}^{1},\ldots,\tilde{\mathbf{c}}^{p}$ denote the corresponding
commitment vectors in step (3). Similarly, let $Q^{uv}$ denote the $v^{th}$
choice of $Q$ corresponding to $u^{th}$ value of $r$ for $u\in [p]$ and $v\in
[\theta]$. We do not consider the initial transcript as part of the tree of
transcripts. We notice that we have $\cup_{v\in [\theta]}Q^{uv}=[n]$ for all
$u\in [p]$, except with probability at most $pn(1-1/n)^{nt}\leq pne^{-t}$. At
step (4), $\extr$ uses extractor $\extr_{2d}$ for the protocol $\proximityTwoD$
to extract $h\times \ell$ matrix $W^{u}$ for $u\in [p]$. The extractor $\extr$ fails with output $\bot$
if $\extr_{2d}$ fails in any of its invocations. By the property of $\extr_{2d}$
we have that $\open(\tilde{\bm{c}}^u)=W^u$ and $W^u\in L_1\oplus L_2$ for all
$u\in [p]$. We construct the witness $\ewit$ by solving for each $h$-length
vector $\ewit[i,\cdot,k]$ for $i\in [p], k\in [n]$ according to
\eqref{eq:extraction3d}. Let $T_1,\ldots,T_n$ denote the columns of matrix
$\mc{T}$. 
\begin{align}\label{eq:extraction3d}
\text{For each $k\in [n]$ we have:}  \nonumber \\
\begin{array}{cccc}
r^1_1\ewit[1,\cdot,k]+ &\hdots & +r^1_p\ewit[p,\cdot,k] &= W^1T_k \\
\vdots & \ddots & \vdots & \vdots \\
r^p_1\ewit[1,\cdot,k]+ &\hdots & +r^p_p\ewit[p,\cdot,k] &= W^pT_k 
\end{array}
\end{align}
Let $\Lambda$ denote the $p\times p$ matrix with $\Lambda[i,j]=r^i_j$. We may
assume that $\Lambda$ is invertible, except with probability $1/|\FF|$. Let
$\Omega$ be such that $\Omega\Lambda = I_p$. It can be seen that the system of
equations \eqref{eq:extraction3d} is satisfied by setting $\ewit[i,\cdot,k]=\sum_{j\in
[p]}\Omega[i,j]W^jT_k$ for $i\in [p],k\in [n]$. First we prove that
$\comm(\ewit[i,\cdot,k])=\pi[i,k]$ for all $i$ and $k$. Indeed,
\begin{align*}
\comm(\ewit[i,\cdot,k]) & =\sum_{j\in [p]}\Omega[i,j]\comm(W^jT_k) \\
	& = \sum_{j\in [p]}\Omega[i,j]\sum_{a\in [\ell]}\mc{T}[a,k]\comm(W^j[.,a]) \\
	& = \sum_{j\in [p]}\Omega[i,j]\sum_{a\in
[\ell]}\mc{T}[a,k]\tilde{\bm{c}}^j_a \\
	& = \sum_{j\in [p]}\Omega[i,j]\sum_{a\in [p]}r^j_a\pi[a,k] \text{ (using check
(7)) } \\
	& = \sum_{a\in [p]} \big(\sum_{j\in [p]}\Omega[i,j]\Lambda[j,a]\big)\pi[a,k]
\\
	& = \pi[i,k]
\end{align*}

Note that the columns
$\ewit[i,\cdot,k]$ are linear combinations of the columns of extracted matrices
$W^u, u\in [p]$ which are codewords in the code $L_2$. Thus, each column
$\ewit[i,\cdot,k]$ in the extracted witness $\ewit$ is a codeword in $L_2$. Thus
 $\ewit\in \mc{W}_2$. Next, we bound the probability that
$d_1(\ewit,\mc{W}_1)\leq e$. To do so, we define the
following events of interest:
\begin{itemize}
\item $A$: denotes the event that the first transcript is accepting. By
definition, $\prob{A}$ is the accepting probability $\prob{\langle
P^\ast(\sigma),\verifier(\sigma)\rangle=1}$.
\item $B$: denotes the sub-event of $A$ when the extracted opening $\ewit$ to the
commitment oracle $\pi$ is valid, i.e $d_1(\ewit,\mc{W}_1)\leq e$. 
\item $F$: denotes the event that $\extr$ fails with output $\bot$. This
happens when one of the invocations subprotocol extractor $\extr_{2d}$ fails, or
unlikely events that the matrix $\Lambda$ is singular, or that $\theta=tn$
repetitions of the random query locations $Q$ fail to cover the set of indices
$[n]$. As discussed in the proof, we can bound $\prob{F}$ by
$p\kappa_{2d}(\secpar) + pne^{-t} + 1/|\FF|$. 
\end{itemize}
Using an analysis similar to the one in the proof of Lemma
\ref{lem:proximity2d_sound}, we have:
\begin{equation}\label{eq:probeq1}
\prob{B}\geq \prob{A} - \prob{F} - \condprob{A}{d_1(\ewit,\mc{W}_1)>e}   
\end{equation}
Let $\tilde{\bm{c}}=(\tilde{c}_1,\ldots,\tilde{c}_\ell)$ denote the vector sent
by the prover in step (3). We consider the $n$-length vector $C_{adv}$
defined by $C_{adv}[k]=\sum_{i\in [\ell]}\mc{T}[i,k]\tilde{c}_i$ for $k\in [n]$. Let
$C_{hon}$ denote the honestly computed vector defined by $C_{hon}[k]=\sum_{i\in
[p]}r_i\pi[i,k]$. Note that for honest oracle $\pi\in\mc{W}$ and honest prover
we would have $C_{adv}=C_{hon}$. Let $\varepsilon := \Delta(C_{adv},C_{hon})$ denote the
hamming distance between the two vectors. Now we can write
$\condprob{A}{d_1(\ewit,\mc{W}_1)>e}$ as:
\begin{align}\label{eq:probeq3}
\condprob{A}{d_1(\ewit,\mc{W}_1)>e} &\leq \condprob{A}{d_1(\ewit,\mc{W}_1)>e,
\varepsilon \leq e}\nonumber \\ 
&\quad +\condprob{A}{d_1(\ewit,\mc{W}_1)>e, \varepsilon > e}
\end{align}
The above follows from the identity $\condprob{A}{X}\leq
\condprob{A}{X_1}+\condprob{A}{X_2}$ for $X=X_1\uplus X_2$. Next we note
that for $\varepsilon > e$, the check in step 7, succeeds with probability at most
$(n-e)^t/n^t=(1-e/n)^t$. Thus $\condprob{A}{d_1(\ewit,\mc{W}_1)>e,\varepsilon > e}\leq
(1-e/n)^t$. Next we bound $\condprob{A}{d_1(\ewit,\mc{W}_1)>e,\varepsilon\leq e}$.
With probability at least $1-\kappa_{2d}(\secpar)$, the extractor $\extr_{2d}$
produces a witness $\overline{U}$ such that $\overline{U}\mc{T}\in L_1\oplus L_2$, and
$\overline{U}=\open(\tilde{\bm{c}})$.  Let
$U_{adv} = \overline{U}\mc{T}$ denote the codeword in $L_1\oplus L_2$. We note that
$U_{adv}=\open(C_{adv})$. Define $U_{hon}=\sum_{i\in [p]}r_i\ewit[i,.,.]$. Note
that $U_{hon}=\open(C_{hon})$ by homomorphism of the commitment scheme. Since
$\varepsilon=d(C_{adv},C_{hon})\leq e$, we must have:
\begin{enumerate}[{\rm (i)}]
\item $\Delta_1(U_{adv},U_{hon})\leq e$ or,
\item There exists an index $j\in [n]$ such that $C_{adv}[j]=C_{hon}[j]$, but
their corresponding openings $U_{adv}[j]$ and $U_{hon}[j]$ are different. This
constitutes breakage to the binding property of the commitment scheme $\comm$. 
\end{enumerate}
Assuming $\Delta_1(U_{adv},U_{hon})\leq e$, we have $d_1(U_{hon},\mc{C}_1)\leq
e$. From the above, we can now write:
\begin{align*}\label{eq:probeq4}
&\condprob{A}{d_1(\ewit,\mc{W}_1)>e,\varepsilon\leq e}\\
\leq&
\condprob{d_1(U_{hon},\mc{C}_1)\leq e}{d_1(\ewit,\mc{W}_1)>e} + \mc{B}(\secpar)
\end{align*}
Since $U_{hon}=\sum_{i\in [p]}r_i\ewit[i,\cdot,\cdot]$, from Lemma
\ref{lem:3dcompression}, we have $\condprob{d_1(U_{hon},\mc{C}_1)\leq
e}{d_1(\ewit,\mc{W}_1)>e}<d_1/|\FF|$ where $d_1$ is the minimum distance of the
code $L_1$. Thus, Equation \eqref{eq:probeq3} gives us
\begin{equation}\label{eq:probeq5}
\condprob{A}{d_1(\ewit,\mc{W}_1)>e}\leq \left(1-\frac{e}{n}\right)^t,
\frac{d_1}{|\FF|}+\kappa_{2d}(\secpar) + \mc{B}(\secpar)
\end{equation}
From Equations \eqref{eq:probeq1} and \eqref{eq:probeq5} we see that the
statement of the Lemma holds for 
\[\kappa_{3d}(\secpar) := (p+1)\kappa_{2d}(\secpar) + pne^{-t} +
\left(1-\frac{e}{n}\right)^t + \frac{d_1+1}{|\FF|} + \mc{B}(\secpar) \]
\end{proof}


\begin{proof}[Linear Check]
We describe an extractor $\extr$ which outputs $\ewit$ such that $\open(\pi)=\ewit$ and $A\wit = b$ 
for $\wit=\dec(\ewit)$. The extractor $\extr$ uses the extractor $\extr_{3d}$ for
the protocol $\proximityThreeD$ to extract the witness $\ewit$ which opens to
the oracle $\pi$. Note that, with probability at least ($1-\kappa_{3d}(\secpar)$),
the extracted witness $\ewit$ satisfies $d_1(\ewit,\mc{W}_1)<e$ and $\ewit\in
\mc{W}_2$. It follows that for each slice $\ewit^i:=\ewit[i,\cdot,\cdot]$ for $i\in [p]$,
we have $\ewit^i\in \mc{C}_2$ and $d_1(\ewit^i,\mc{C}_1)<e$. Let $E\subseteq
[n]$ denote the indices of the {\em planes} where $\ewit$ differs from its closest
neighbor in $\mc{W}_1$. Since $e<d_1/2$, $E$ also denotes the set of columns
where slices $\ewit^i$ differ from their closest neighbors in $\mc{C}_1$. We
decode each slice seperately. Applying Lemma \ref{lem:bicdecoding} for each
slice, we have codewords $W^i\in L_1\oplus L_2$ for each $i\in [p]$ such that
$W^i[\cdot,k]=\ewit^i[\cdot,k]$ for all $k\not\in E$. Let $Q^i$, $i\in [p]$ be
the unique polynomials with $deg_x(Q^i)<m$ and $deg_y(Q^i)<\ell$ such that
$Q^i(\alpha_j,\eta_k)=W^i[j,k]$ for $(j,k)\in [h]\times [n]$. We define the
witness $\wit$ by $\wit[i,j,k] := Q^i(\alpha_j,\zeta_k)$ for $i\in [p],j\in
[m],k\in [s]$. Note that $\wit=\dec(\ewit)$. In case, the subprotocol extractor
$\extr_{3d}$ outputs $\ewit$ such that $d_1(\ewit,\mc{W}_1)>e$, the extractor
$\extr$ fails with output $\wit := \bot$. Let $\ewit_{\rm hon}$ denote the
encoding of $\wit$ given by $\ewit_{\rm hon}[i,j,k]=Q^i(\alpha_j,\eta_k)$ for
$i\in [p],j\in [h], k\in [n]$. Observe that we have $d_1(\ewit,\ewit_{\rm
hon})<e$. Let $\mc{S}$ denote the event that the
first transcript succeeds. As before, the key step to ensure soundness is to
upper bound the probability $\prob{\mc{S}\cap (A\wit\neq b)}\leq
\condprob{\mc{S}}{A\wit\neq b}\leq \condprob{\mc{S}}{r^TA\wit\neq r^Tb}+1/|\FF|$
where $r\sample \FF^N$ denotes the message sent by $\verifier$ in Step 1. Let
$P_{\rm hon}$ denote the matrix $P$ which is correctly computed from $\ewit_{\rm
hon}$ in
Step 3. Let $c_1,\ldots,c_{\ell}$ be the commitments sent (possibly,
adverserial) by the prover to the verifier in Step 4. Let $\overline{P}_{\rm adv}$ be the
witness extracted using the extractor $\extr_{2d}$ for the subprotocol in Step
10. Let $P_{\rm adv}$ denote the matrix $\overline{P}_{\rm adv}T$. Then with overwhelming
probability ($1-\kappa_{2d}(\secpar)$), $P_{\rm adv}\in \dashL_1\oplus \dashL_2$. Observe
that if $P_{\rm hon}=P_{\rm adv}$, then $r^TA\neq r^Tb$ implies that
$\overline{P}=\overline{P}_{\rm adv}$ does not satisfy Equation
\eqref{eq:necessarycondlin} and thus the subprotocol in Step 11 succeeds with
probability at most $\kappa_{ip}(\secpar)$, or $\extr$ succeeds in finding
distinct openings to the commitment in subprotocol in Step 11. Assume then that $P_{\rm hon}\neq
P_{\rm adv}$. Again, from subprotocol in Step 14, we conclude that $X_u =
\ewit[\cdot,j_u,k_u]$, except with probability $\kappa_{agg}(\secpar)$, or
$\extr$ outputs two openings to one of the commitments $\pi[j_u,k_u]$ for $u\in
[t]$. Similarly, assuming $\extr$ does not break binding of the commitment
scheme, the witnesses extracted for inner product protocols in Step 13, are the
columns $P_{\rm adv}[\cdot,k_u]$ for $u\in [t]$. Thus, with overwhelming probability, the inner product check in Step 13 is
equivalent to checking the following for all $u\in [t]$:
\begin{equation}\label{eq:check1}
P_{\rm adv}[j_u,k_u]=\sum_{i\in
[p]}R^i(\alpha_{j_u},\eta_{k_u})\ewit[i,j_u,k_u]
\end{equation}
For $u\in [t]$, let $\mc{E}_u$ denote the event that the above equation holds
for $u$. For $k_u\not\in E$, we have $\ewit[\cdot,\cdot,k_u]=\ewit_{\rm
hon}[\cdot,\cdot,k_u]$ and thus the right hand side in \eqref{eq:check1} equates
to $P_{\rm hon}[j_u,k_u]$. Now, $P_{\rm adv}$ and $P_{\rm hon}$ differ in at
least $\dashD_1$ columns, where $\dashD_1$ denotes the minimum distance of the
code $\dashL_1$. Let $E'$ denote the column indices where $P_{\rm adv}$ and
$P_{\rm hon}$ differ. For $k_u\in E'\backslash E$, the check succeeds if the
distinct codewords $P_{\rm adv}[\cdot,k_u]$ and $P_{\rm hon}[\cdot,k_u]$ agree
at position $j_u$. Since $(j_u,k_u)$ are sampled uniformly and independently
(also independent of extracted witness $\wit$), we have:
{\small
\begin{align}\label{eq:probeq6}
\prob{\mc{E}_u} &\leq \frac{n-\dashD_1+e}{n} +
\frac{\dashD_1-e}{n}.\frac{h-\dashD_2}{h}
\nonumber \\
	&\leq \frac{s+\ell+e}{n} + \frac{n-s-\ell-e}{n}.\frac{2m}{h} \nonumber \\
	& = \frac{2m}{h} +
\left(1-\frac{2m}{h}\right)\left(\frac{s+\ell+e}{n}\right)
\end{align}
}
For initial transcript to accept, all the $t$ checks should succeed, and thus we
have:
{\small
\begin{align}\label{eq:probeq7}
&\prob{\mc{S}\cap (A\wit\neq b)}\\
\leq&
\left(\frac{2m}{h}+\left(1-\frac{2m}{h}\right)\left(\frac{s+\ell+e}{n}\right)\right)^t
+ \kappa_{3d}(\secpar) + \mc{B}(\secpar) +
\frac{O(|C|)}{|\FF|} \nonumber \\
\leq& \left(1-\frac{e}{n}\right)^t +
\left(\frac{2m}{h}+\left(1-\frac{2m}{h}\right)\left(\frac{s+\ell+e}{n}\right)\right)^t
\\
& \qquad \, \, \,+ \frac{O(|C|)}{|\FF|} + \mc{B}(\secpar)
\end{align}
}
Thus the statement of the lemma holds for $\kappa_{lc}(\secpar) := (1-e/n)^t +
(2m/h + (1-2m/h)(2\ell/n + e/n))^t+O(|C|)/|\FF|+\mc{B}(\secpar)$.
\end{proof}


\begin{proof}[Simulation]
We consider the simulator as discussed. Let $\bm{r}=(r,\{j_u,k_u\}_{u\in
[t]},\beta,\tau,\delta,\rho,\tilde{\tau},\tilde{\delta})$ denote the vector
consisting of verifier's randomness, which is chosen exactly in the honest
protocol execution. Distributions $p(z|\bm{r})$ and $p(\tilde{z}|\bm{r})$ are
uniform distributions on $L_2$ due to the blinding vector $u_0\in L_2$ chosen by
the prover in the $\proximityTwoD$ protocol. Similarly, due to blinding vector
$P_0\in \dashL_2$ chosen in Step 3(d), the distribution
$p(z'|\bm{r})$ is also uniform on the vectors $z'$ in $\dashL_2$ satisfying
$\sum_{j\in [m]}z'[j]=0$. Since $t\leq \bi$, from Lemma
\ref{lem:boundedindependence}, we conclude that the planes
$\ewit[\cdot,\cdot,k_u]$ are distributed uniformly independent of $\bm{r}$. In
the real execution of the protocol, the prover chooses
$\omega_0,\omega_1,\ldots,\omega_{s+\ell}$ and $\nu_0$ independently at random
and computes:
\begin{align}\label{eq:simeq}
\chi_u &= \sum_{a\in [s+\ell]}T[a,k_u]\omega_a \forall u\in [t] \nonumber \\
\omega &= \omega_0 + \sum_{a\in [s+\ell]}\mu_a\omega_a \nonumber \\
\nu &= \nu_0 + \sum_{a\in [s+\ell]}\phi_a\omega_a
\end{align}    
Since any $t$ columns of $T$ are linearly independent, we see that the matrix of
coefficients in \eqref{eq:simeq} has full row rank, and thus the vector
$(\chi_1,\ldots,\chi_t,\omega,\nu)$ is disrtributed uniformly in $\FF^{t+2}$.
Similarly, $\{O[\cdot,k_u]\}_{u\in [t]}$ also consists of uniformly sampled
entries in $\FF$ as in the real protocol. We now consider computation of
$\tilde{\nu}$ in the honest protocol execution. We have:
\begin{align}\label{eq:simeq2}
\tilde{\nu} &= \tilde{\nu}_0 + \sum_{a\in [\ell]}\tilde{\mu}_a\tilde{\omega}_a
 \text{ where } \nonumber \\
\tilde{\omega}_a &= \sum_{i\in [p]}\rho_iO[i,a] \quad \forall a\in [\ell]
\end{align} 
Since $\tilde{\nu}_0$ is choesn randomly by the prover, the distribution of
$\tilde{\nu}$ is uniform, independent of other variables in the view. Finally,
the commitments in the view satisfy relations in Equation \ref{eq:commiteq} in
the real protocol, and are picked uniformly subject to those constraints by the
simulator. Thus the view output by the simulator perfectly simulates the
extended view of the verifier as defined. 
\begin{align}\label{eq:commiteq}
d_0 + \sum_{a=1}^{s+\ell}\mu_ac_a &= \comm(z,\nu) \nonumber \\
c_0 + \sum_{a=1}^{s+\ell}\varphi_ac_a &= \comm(z',\omega) \nonumber \\
\sum_{a=1}^{s+\ell}T[a,k_u]c_a &= \comm(P[\cdot,k_u],\chi_u) \text{ for } u\in [t] \nonumber \\
\sum_{a=1}^{\ell}\mc{T}[a,k_u]\tilde{c}_a &= \comm\big(\sum_{i\in
[p]}
[\cdot,k_u],\tilde{O}[\cdot, k_u]\big) \text{ for } u\in [t] \nonumber \\
\beta\tilde{d}_0 + \sum_{a=1}^{\ell}\tilde{\mu}_a\tilde{c}_a &=
\comm(\tilde{z},\tilde{\nu})
\end{align} 

\end{proof}

%%%%%%%%%%%%%%%%%%%%%%%%%%%%%%%%%%%%%%%%%%%%%%%%%%%%
\begin{comment}
\subsection{Semi-honest to Malicious Security: }\label{app:semi-honesttomalicious} 
To achieve security against malicious provers, every prover $\distprover$ sends Zero-Knowledge Argument of Knowledge (ZKAoK) along with every committed value sent to $\Ag$, which says $\distprover$ knows what value is committed. A malicious secure MPC executes multiplication. Furthermore, all messages are sent via broadcast.
%In our protocol, when a prover $\distprover$ sends any commitment output to the aggregator, in the maliciously secure protocol $\distprover$ with that commitment output, he additionally sends a Zero-Knowledge Argument of Knowledge (ZKAoK) which says $\distprover$ knows which value is committed. This additional step is required in the simulation so that the simulator can extract the committed value and maliciously secure MPC to compute $\mathsf{Mult}$. Furthermore, all the messages to $\Ag$ are via broadcast. 
%\pnote{Give the overview of the proof. and explain why the does not go through when the aggregator is corrupt.}

\noindent{\bf Overview of the proof:} 
With the above modification, our protocol can withstand malicious provers. We can design a simulator for the protocol, which uses $\extrac$, the extractor of ZKAoK, Simulator $\Sim_{Mult}$ for the secure MPC for multiplication, and zero-knowledge simulator $\Sim_{ZK}$ of Graphene. 
$\Sim$ receives the messages from the corrupt parties to $\Ag$ due to broadcast. Then, $\Sim$ calls $\extrac$ and obtains opening of $\shr{\comoracle}$ for all corrupt $\xi$. Using $\Sim_{ZK}$, $\Sim$ prepares the messages of the honest parties, and if there is any interaction required between the provers, that interaction is generated via $\Sim_{Mult}$. $\Sim$ checks if all the messages are consistent or not with all the previous messages if not $\Sim$ sends consistent but random values on behalf of honest parties, otherwise, just before sending $\shr{z_{lc}}$ and $\shr{z_{qd}}$, $\Sim$ invokes the $\Func_{\DPZK}$ if it outputs 1, then $\Sim$ generates messages for honest parties using $\Sim_{ZK}$ otherwise it picks messages uniformly at random.
%\pnote{I think a little detailing is required here.}
%We will describe, in high-level, how to construct a simulator $\Sim$ which can generate a view which is indistinguishable from a real execution, where $\Sim$ has access to the extractor $\extrac$ for the ZKAoK and the simulator $\Sim_{Mult}$ for the secure multiplication and the zero-knowledge simulator $\Sim_{ZK}$ of \name. $\Sim$ receives the messages from the corrupt parties to $\Ag$ due to broadcast. Then, $\Sim$ calls $\extrac$ and obtains opening of $\shr{\comoracle}$ for all corrupt $\xi$. Using $\Sim_{ZK}$, $\Sim$ prepares the messages of the honest parties, and if there is any interaction required between the provers, that interaction is generated via $\Sim_{Mult}$. $\Sim$ checks if all the messages are consistent or not with all the previous messages if not $\Sim$ sends consistent but random values on behalf of honest parties, otherwise, just before sending $\shr{z_{lc}}$ and $\shr{z_{qd}}$, $\Sim$ invokes the $\Func_{\DPZK}$ if it outputs 1, then $\Sim$ generates messages for honest parties using $\Sim_{ZK}$ otherwise it picks messages uniformly at random.

\noindent{\bf Difficulty in simulation while aggregator is malicious:} The above proof works against malicious provers but semi-honest aggregator. To generate the messages on behalf of honest parties, $\Sim$ needs the output of $\Func_{DPZK}$. If $\Ag$ deviates from the protocol at the last round, then the simulator cannot generate an appropriate message, which gives the same output as real-execution. Though we don't have a simulator when $\Ag$ is malicious, we do not have an attack either. Also, it is possible to identify the malicious behavoiur of $\Ag$ since messages to $\Ag$ are via broadcast, and $\Ag$ does not require any internal randomness to generate messages to the $\verifier$. 
%In the above construction, it considered that provers might deviate while interacting among themselves. However, we do not consider the scenario if $\Ag$ deviates. We do not have a simulator that can simulate a view when all the provers (corrupt and honest) behaving correctly, but $\Ag$ sends a wrong combined value to $\verifier$. To resolve, one can try to invoke the $\Func_{\DPZK}$ after $\Ag$ sending the final value to $\verifier$. However, $\Ag$ needs honest parties messages to generate the message for $\verifier$, and the simulator needs the output of the $\Func_{\DPZK}$. So, it is difficult to get a simulator, which can generate a view indistinguishable from the real world execution. Though we can not construct a simulator for our protocol, we do not have an attack in this corruption model. Also, it is possible to identify the malicious behavoiur of $\Ag$ since messages to $\Ag$ are via broadcast, and $\Ag$ does not require any internal randomness to generate messages to the $\verifier$. 
\end{comment}

\begin{comment}
\begin{lemma}\label{lem:privacy}
For $\xi\in [K]$, let $\View^{\xi}_{\adv}$\pnote{fix notation} denotes the view of the adversary $\adv$ corrupting parties in $I$, where $I$ is a proper subset of the sets of the provers. Then there is a $\ppt$ simulator $\Sim$ which can generate a view of $\adv$ which is indistinguishable from the real execution of the protocol given there is a secure MPC for multiplication which withstand against the corrupt parties in $I$. \pnote{(Maybe a broadcast required)}
\end{lemma}

\begin{proof}
To prove the above theorem we will design a simulator $\Sim$ which has access to the ideal functionality $\Func_{\DPZK}$ (~\ref{func:DPZK}). Provers in $I$ encode and commit to their inputs and $\Sim$ gets $\shr{\comoracle}, \shr{\comoracle}_a$ $\forall a\in\{x,y,z\}$ and $\forall \xi \in I$. Since the provers are giving ZKAoK in addition to the commitment, by the extractor of argument of knowledge, $\Sim$ extracts and gets shares of $\wit, \wit_x,\wit_y,\wit_z$ of the parties in $I$. 
%Now $\Sim$ checks if all the steps of Linear and Quadratic checks are done using the same inputs extracted by $\Sim$. If yes, then $\Sim$ calls the functionality $\Func_{\DPZK}$ if it outputs 1, then $\Sim$ uses the same approach of  
%$\Sim$ picks random challenges for both linear and quadratic checks on behalf of the verifier (picking random challenges on behalf of the verifier is fine since the verifier is semi-honest). 
Then $\Sim$ uses the simulator for \name and gets an accepting transcript. Using that computes $\shr{\comoracle}, \shr{\comoracle}_a$ $\forall \xi\notin I$, using homomorphic property of the commitment scheme.
$\Sim$ gets $\shr{c_0}, \shr{c_1}, \ldots, \shr{c_{s+\ell}}$ and $\shr{Z}$ for all $\xi\in I$ in the linear check, as all the provers are sending this values to $\Ag$, which is done using broadcast. Using extraction of commitment $\Sim$ gets the decommitted values. If there is some inconsistency in the values extracted from $\shr{\comoracle}, \shr{\comoracle}_a$ $\forall a\in\{x,y,z\}$ and $\forall \xi \in I$ and these decommitted values then
%output $\abort$, else continue.
sets $\mathsf{state}$ as fail, otherwise pass.
For quadratic check, $\Sim$ calls the simulator, $\Sim_M$, of the secure multiplication protocol and extracts the input of the MPC, and does the consistency check, as in the linear case. If there is some inconsistency then 
%output $\abort$, else continue.
sets $\mathsf{state}$ as fail, otherwise, pass.
Similarly, $\mathsf{state}$ is set for the remaining part of the protocol.
(This part is informally stated here, details will be provided in the full version.)
Now for both the protocols if $\mathsf{state}$ remains pass till the provers send the final messages to $\Ag$, which are the witnesses of the inner product arguments, 
\pnote{Should we add the details such as what are the consistency checks? From the protocol that is quite evident.}
then $\Sim$ calls the functionality $\Func_{\DPZK}$ with the input $\wit, x,y,z$ of the parties in $I$. If it outputs 1, then $\Sim$ generates the final messages of the honest parties using the accepting transcript produced by the zero-knowledge simulator, else $\Sim$ picks any random message which is consistent with the transcript so far.

If there is some inconsistency in some intermediate step, i.e., the $\mathsf{state}$ is fail, then $\Sim$ calls the functionality $\Func_{\DPZK}$ on some random value and proceeds accordingly with the simulation.

This proves that the view of the corrupt provers can be simulated, which ensures privacy of the honest provers.
\end{proof}

%\end{enumerate}
\end{comment}
%%%%%%%%%%%%%%%%%%%%%%%%%%%%%%%%%%%%%%%%%%%%%%%%%
\subsection{$\DPZK$ of Bulletproof for R1CS}\label{app:BulletproofsDPZK}
In \cite{InnerProductDLS}, \cite{bulletproofs}, authors present a proof for a Hadamard-product relation. Suppose $\C$ is a circuit of size $N$. In their formulation,  $\bm{a}_L$, $\bm{a}_R$ and $\bm{a}_O$ denote the vectors corresponding to the left wire, right wire and output wire respectively. Then, $\bm{a}_L \circ \bm{a}_R = \bm{a}_O$ holds and satisfy the following linear constraints: 
%$\innp{\bm{w}_{L,q}}{\bm{a}_L} + \innp{\bm{w}_{R,q}}{\bm{a}_R} + \innp{\bm{w}_{O,q}}{\bm{a}_O} = c_q$ where $1\leq q \leq Q \leq 2n$ with .
%for a multiplication $g$ gate in a circuit $\C$ has $\bm{a}_L$, $\bm{a}_R$ and $\bm{a}_O$ as left input, right input and output wire values respectively, then $\bm{a}_L \circ \bm{a}_R = \bm{a}_O$ holds. Other than this, these vectors are supposed to satisfy additional $Q\leq 2n$ linear constraints, which is of the following form:
$$\innp{\bm{w}_{L,q}}{\bm{a}_L} + \innp{\bm{w}_{R,q}}{\bm{a}_R} + \innp{\bm{w}_{O,q}}{\bm{a}_O} = c_q \, \forall q\in [Q]$$
with $\bm{w}_{L,q}, \bm{w}_{R,q}, \bm{w}_{O,q} \in \ZZ^N_p$ and $c_q \in \ZZ_p$.
Consider
$W_A = [\bm{w}_{A,1}, \ldots, \bm{w}_{A,Q}]^T$, for $A\in \{L,R,O\}$ %\vspace{10pt} 
%$W_R = [\bm{w}_{R,1}, \ldots, \bm{w}_{R,Q}]^T$, %\vspace{10pt} 
%$W_O = [\bm{w}_{O,1}, \ldots, \bm{w}_{O,Q}]^T$,%\vspace{10pt} 
and $\bm{c} = [c_1, \ldots, c_q]^T$
%\pnote{what is $W_v$ and $\bm{v}$ is not clear.}

\noindent In \cite{bulletproofs}, these above checks are reduced to a single inner product argument. $\prover$ commits to the input vectors $\bm{a}_L, \bm{a}_R$, output vector $\bm{a}_O$ and sends the commitment values $A_I$ and $A_O$ to $\verifier$. $\prover$ picks $s_L$ and $s_R$ uniformly at random and computes the commitment $S$ and sends to $\verifier$.
$\verifier$ gives random challenges $y, z$ from $\ZZ^\ast_p$ to the prover.
Then $\prover$ computes vector polynomials $l(X)$ and $r(X)$ using $\bm{a}_L, \bm{a}_R, \bm{a}_O, s_L, s_R, y, z$ and other public vectors, and computes a polynomial $t(X) = \innp{l(X)}{r(X)}$. The construction of $l(X)$ and $r(X)$ is such that the co-efficient of $X^2$ in $t(X)$ is independent of $\bm{a}_L, \bm{a}_R, \bm{a}_O, s_L, s_R$, which can be computed by $\verifier$, if $\bm{a}_L, \bm{a}_R, \bm{a}_O$ satisfy the Hadamard-product relation and linear constraints. To prove this, $\prover$ commits to all the co-efficients of $t(X)$ other than the co-efficient of $X^2$, and verifier gives a random point $x$ to evaluate $\bm{l} = l(x)$ and $\bm{r} = r(x)$. Which reduces to an inner product check whose witness is $\bm{l}$ and $\bm{r}$, where $\bm{l}$ and $\bm{r}$ are vectors of length $N$. 
\smallskip

 We give a $\DPZK$ version of Bulletproofs, where $\distprover$ starts the protocol with $\shr{\bm{a}_L}, \shr{\bm{a}_R}, \shr{\bm{a}_O}$ $\forall \xi \in [\Num]$ such that $\sum_{\xi\in [\Num]} \shr{\bm{a}_A} = \bm{a}_A$ $\forall A\in \{L,R,O\} $. $\Ag$ is an aggregator.
$\distprover$ computes the commitment $\shr{A_I}, \shr{A_O}, \shr{S}$ and sends these values to $\Ag$. $\Ag$ combines and sends $A_I$, $A_O$ and $S$ to $\verifier$.
The verifier broadcasts the random challenge $y,z \sample \ZZ^\ast_p$.
Then each prover computes $\shr{l}(X)$ and $\shr{r}(X)$.
All the provers interact securely to compute shares of $t(x) = \innp{l(X)}{r(X)}$, where $l(X)= \sum_{\xi\in [\Num]} \shr{l}(X)$ and $r(X) = \sum_{\xi\in [\Num]} \shr{r}(X)$, i.e., output for $\distprover$ is $\shr{t}(X)$ such that $\sum_{\xi\in [\Num]} \shr{t}(X) = t(X)$.
$\distprover$ commits to all the coefficients of $\shr{t}(X)$ other than the coefficient of $X^2$. Sends these committed values to $\Ag$. $\Ag$ combines and sends them to $\verifier$.
$\verifier$ sends a random $x$. $\distprover$ evaluates $\shr{l}(x) = \shr{\bm{l}}$ and $\shr{r}(x) = \shr{\bm{r}}$ and sends these values to $\Ag$. $\Ag$ computes $\bm{l} = \sum_{\xi\in [\Num]} \shr{\bm{l}}$ and $\bm{r} = \sum_{\xi\in [\Num]} \shr{\bm{r}}$.
Finally, $\Ag$ and $\verifier$ run the inner product protocol, same as the single prover protocol, where the witness is $\bm{l}$ and $\bm{r}$.
Since, $\bm{l}$ and $\bm{r}$ afre vectors of length $N$, MPC is required for $N$ multiplications or in other words, a MPC for a depth 1 circuit of size $N$.

%%%%%%%%%%%%%%%%%%%%%%%%%%%%%%%%%%%%%%%
\subsection{$\DPZK$ of Spartan for R1CS}\label{app:SpartanDPZK}
We will describe the possible distributed prover version of Spartan, given in \cite{spartan} as an Interactive Proof system. Zero knowledge and privacy among the provers may achieve in the similar way but we do not have a proof for that. 

Let $C$ be an R1CS circuit. In their formulation, an instance of an R1CS circuit was represented by a tuple $(\FF, A, B, C, io, m, n)$, where the circuit is defined by the matrices $A, B, C\in \FF^{m\times m}$, $io$ denotes the public input and output of the instance, and $m\geq |io|+1$ and there are at most $n$ non-zero entries in each matrix, $n=O(m)$. The size of the instance is $O(n)$.
An R1CS instance $(\FF, A, B, C, io, m, n)$ is said to be satisfiable if $\exists$ a witness $\wit\in \FF^{m-|io|-1}$ such that $(A\cdot z)\circ (B\cdot z) = (C \cdot z)$ where $z=(io, 1, \wit)$, $\cdot$ is the matrix vector product and $\circ$ is the Hadamard product of two vectors.
\paragraph{Low degree polynomial:} A multivariate polynomial $\calG$ over a finite field $\FF$ is called low degree polynomial if the degree of $\calG$ in each variable is exponentially smaller than $|\FF|$.
\paragraph{Low degree extension:} Suppose $g:\{0,1\}^m\rightarrow\FF$ is a function that maps $m$-bits elements into an element of $\FF$. A polynomial extension of $g$ is a low degree $m$-variate polynomial $\tilde{g}(\cdot)$ such that $\tilde{g}(x)=g(x)\; \forall x\in\{0,1\}^m$.
\paragraph{Sum-check:} Let $\calG$ be a low degree $\mu$-variate polynomial, $\calG:\FF^{\mu}\rightarrow\FF$ and degree of $\calG$ for each variable is at most $\ell$. Let $\prover_{SC}$ be a prover that claims 
$$\sum_{x_1\in\{0,1\}}\sum_{x_2\in\{0,1\}}\ldots\sum_{x_{\mu}\in\{0,1\}}\calG(x_1,\ldots,x_{\mu}) =T$$
i.e. the prover convinces the verifier that the polynomial sums up to $T$, if the sum is taken over all bit strings of length $\mu$.
There is a probabilistic reduction given in Spartan \cite{spartan}, where the verification is much more efficient, instead of evaluating at all the bit strings of length $\mu$, it can be done by evaluating in $O(\mu)$ many points.

Here we will discuss the idea given in Spartan \cite{spartan} which does not have zero-knowledge property. Let $PC$ is a polynomial commitment scheme. The verifier $\verifier$ needs to check that $(A\cdot Z)\circ (B\cdot Z) = C\cdot Z$ holds, where $Z = (io, 1, \wit)$.

The prover $\prover$ proceeds in the following way:
$\prover$ obtains a low degree extension polynomial $\tilde{\wit}$ of the witness $\wit$. Then prover commits to $\tilde{\wit}$ using $PC$. $(C,S)\leftarrow PC\cdot Commit(\pubp,\tilde{\wit})$ where $C$ is the commitment and $S$ is the opening hint, and sends $C$ to $\verifier$.
Define: $F_{io} = (A\cdot Z)\circ (B\cdot Z) - (C\cdot Z)$. Then $\verifier$ needs to check that whether $F_{io}$ is a zero vector or not. This can be viewed as a function over the bit strings by representing indices as strings, i.e. 
\begin{align*}
F_{io}(x) &= \left(\sum_{y\in\{0,1\}^s}A(x,y)Z(y)\right)\cdot \left(\sum_{y\in\{0,1\}^s}B(x,y)Z(y)\right)\\ &- \left(\sum_{y\in\{0,1\}^s}C(x,y)Z(y)\right) \; \ldots \; (1) 
\end{align*}

Then $\verifier$ needs to check $F_{io}(x)=0\; \forall x\in\{0,1\}^s$. But to use the sum-check protocol $F_{io}$ should be polynomial, and $F_{io}$ in equation (1) is not a polynomial. Consider the polynomial extension $\tilde{F}_{io}:\FF^s \rightarrow \FF$ defined in the following way:
\begin{align*}
\tilde{F}_{io}(x) &= \left(\sum_{y\in\{0,1\}^s}\tilde{A}(x,y)\tilde{Z}(y)\right)\cdot \left(\sum_{y\in\{0,1\}^s}\tilde{B}(x,y)\tilde{Z}(y)\right)\\ &- \left(\sum_{y\in\{0,1\}^s}\tilde{C}(x,y)\tilde{Z}(y)\right) \; \ldots \; (2) 
\end{align*}
The instance is satisfiable if and only if $\tilde{F}_{io}(x) = 0 \; \forall x\in\{0,1\}^s$. 
Instead of checking at all the bit strings of length $s$, this can be reduced further to check only at a random point on a different polynomial $Q_{io}(t)$, defined in the following way:

$Q_{io}(t)=\sum_{x\in\{0,1\}^s}\tilde{F}_{io}(x)\prod_{i\in[s]}eq(t,x)$ where $eq(t,x) = (t_ix_i+(1-t_i)(1-x_i))$
Note that, if $\tilde{F}_{io}(x)=0 \; \forall x\in\{0,1\}^s$ if and only if $Q_{io}$ is a zero polynomial. Then $Q_{io}$ evaluated at any random point should give 0. To check this $\verifier$ picks random $\tau$ and sends this to $\prover$.

Define: $Q_{io}(\tau)=\sum_{s\in\{0,1\}^s}\calG_{io,\tau}(x)$, where\\ 
$\calG_{io,\tau}(x)=\tilde{F}_{io}(x)\cdot eq(\tau,x)$.
Prover proves $\sum_{x\in\{0,1\}^s}\calG_{io,\tau}(x)=0$ using sum-check protocol. In the sum-check, it reduces to checking the evaluation at a random point, $r_x$, $\calG_{io,\tau}(r_x)=e_x$ or not, where $e_x$ is known to $\verifier$ and $\verifier$ picks $r_x$ uniformly at random. To do that $\verifier$ needs evaluation of $\tilde{F}_{io}(r_x)$ and $eq(\tau,r_x)$.
$\verifier$ can evaluate of $eq(\tau,r_x)$ locally and for $\tilde{F}_{io}(r_x)$, $\prover$ computes 
\begin{align*}
\overline{A}(x) = \sum_{y\in\{0,1\}^s}\tilde{A}(x,y)\tilde{Z}(y), \; \; \; &\nu_A = \overline{A}(r_x)\\
\overline{B}(x) = \sum_{y\in\{0,1\}^s}\tilde{B}(x,y)\tilde{Z}(y),\; \; \; &\nu_B = \overline{B}(r_x)\\
\overline{C}(x) = \sum_{y\in\{0,1\}^s}\tilde{C}(x,y)\tilde{Z}(y),\; \; \; &\nu_C = \overline{C}(r_x)
\end{align*}
and sends $\nu_A,\; \nu_B,\; \nu_C$ to $\verifier$. Using these values $\verifier$ can check $\calG_{io,\tau}(r_x) = e_x$ or not. Now $\verifier$ needs to check 
\begin{align*}
	\nu_A = \overline{A}(r_x)\\
	\nu_B = \overline{B}(r_x)\\
	\nu_C = \overline{C}_(r_x)
\end{align*}
These three checks can be combined together in the following way: $\verifier$ picks random $r_A,\; r_B,\; r_C$ and computes $c=r_A\nu_A+r_B\nu_B+r_C\nu_C$ and $\verifier$ uses sum-check protocol with $\prover$ to verify $\sum_{y\in\{0,1\}^s}M_{r_x}(y) = c$, where $M_{r_x}(y) = r_A\tilde{A}(r_x,y)\tilde{Z}(y)+r_B\tilde{B}(r_x,y)\tilde{Z}(y)+r_C\tilde{C}(r_x,y)\tilde{Z}(y)$.

 In the $\DPZK$ version of Spartan \cite{spartan} each prover holds a share of the witness $\wit$, say $\distprover$ has $\shr{\wit}$. In other words, $Z$ is distributedly shared among the provers. So, in the above protocol described in Spartan \cite{spartan} only $\calG_{io,\tau}$ generation is required interaction among the provers. Remaining all the messages can be generated by each prover locally and an aggregator can combine the messages to obtain corresponding message. Note that, $\calG_{io,\tau}(x)$ computation requires $O(n^2)$ multiplications which are shared across the provers.  

%%%%%%%%%%%%%%%%%%%%%%%%%%%%%%%%%%%%
\begin{comment}
\begin{figure}[h!]
\centering
\begin{framed}
\small
\begin{itemize}
\item {$\linearcheck(\FF,\GG,L_1,L_2,A\in \FF^{M\times N},b\in \FF^M,[\pi];\ewit)$}:
\pnote{Why $A$ is not a square matrix?}
\item {\bf Relation}: $\exists \ewit$ s.t. $\ewit=\open(\pi)$ and $A\wit = b$
for $\wit=\dec(\ewit)$.
\item {\bf Oracle Setup}: The prover $\prover$ computes
$\comoracle = \comm(\ewit)$ as in Section \ref{sec:construct_oracle}. 
The prover sets $\pi := \comoracle$ as the oracle.
\begin{enumerate}[{\rm 1.}]
\item $\verifier\rightarrow\prover$: $\verifier$ samples $r\sample \FF^N$ and
sends it to $\prover$.
\pnote{ dimension of $r$ is not correct}
\item $\prover\longleftrightarrow\verifier$: Both $\prover$ and $\verifier$
compute polynomials $R^i$, $i\in [p]$ such that
$R^i(\alpha_j,\zeta_k)=R[i,j,k]$, where $R=r^TA$. 
%\pnote{Polynomial $R^i(\cdot, \cdot)$ should not be constructed from $r$, it should be from $r^TA$}
\item $\prover$ computes:
	\begin{enumerate}
	\item Polynomials $p_j(\cdot) := \sum_{i\in
[p]}R^i(\alpha_j,\cdot).Q^i(\alpha,\cdot)$ for $j\in [h]$.
	\item An $h\times n$ matrix $P$ such that $P[j,k]=p_j(\eta_k)$.
	\item Commitments $c_k=\comm(P[\cdot,k],\omega_k)$ for $k\in [s+\ell]$
where $\omega_k\sample \FF$.
	\item Sample a codeword $P_0\in \dashL_2$ s.t $\sum_{j\in [m]}P_0[j]=0$.
Compute $c_0=\comm(P_0,\omega_0)$ for $\omega_0\sample \FF$.
	\end{enumerate}
\item $\prover\rightarrow\verifier$: The prover sends $c_0,c_1,\ldots,c_{s+\ell}$ to
the verifier.
\item $\verifier\rightarrow\prover$: $\verifier$ samples $(j_u,k_u)\sample [h]\times
[n]$ for $u\in [t]$. It also samples $\beta\sample\FF$. The verifier sends
$Q=\{(j_u,k_u):u\in [t]\}$ and $\beta$ to $\prover$.
\item $\prover\rightarrow\verifier$: The prover sends vectors
$X_u=\ewit[\cdot,j_u,k_u]$ for $u\in [t]$ to $\verifier$.
\item Oracle Queries: $\verifier$ queries the oracle $\pi$ with $\{k_u:u\in
[t]\}$. 
\item Oracle Answers: The oracle replies with columns $\pi[\cdot,k_u]$, $u\in
[t]$.
\item $\prover\longleftrightarrow\verifier$: Both $\prover$ and $\verifier$
compute $\varphi := \Phi^T[1^s]$ and $\mathsf{cm} := \beta c_0 + \sum_{k\in
[s+\ell]}\varphi_kc_k$.
\item $\prover$ and $\verifier$ run the subprotocol:
	\begin{itemize}
	\item $b_{\rm 2d} = \proximityTwoD(\FF,\GG,\dashL_1,\dashL_2,\bm{c};\overline{P})$
where $\bm{c}=(c_1,\ldots,c_{s+\ell})$ and $\overline{P}$ is the submatrix of $P$
consisting of the first $s+\ell$ columns. Here $\dashL_1=\rsc{\eta}{s+\ell}$ and
$\dashL_2=\rsc{\alpha}{2m}$.
	\end{itemize}
\item $\prover$ and $\verifier$ run the subprotocol:
	\begin{itemize}
	\item $b = \innerproduct(\GG,\bm{g},x,\mathsf{cm},v;z)$
with $x=(1^m,0^{h-m})$, $v=r^Tb$ and $z=\beta P_0+\overline{P}\varphi$.
	\end{itemize}
\item $\prover$ and $\verifier$ compute: For all $u\in [t]$ $c_{k_u} =
\sum_{a\in 2\ell}T[a,k_u]c_a$, where $T$ is the matrix such that $P=\overline{P}T$.
\item $\prover$ and $\verifier$ run inner product subprotocols for each $u\in
[t]$:
	\begin{itemize}
	\item $s_u=\innerproduct(\GG,\bm{g},x,c_{k_u},v;P[\cdot,k_u])$ with
$x=e_{j_u}$, $v=\sum_{i\in [p]}R^i(\alpha_{j_u},\eta_{k_u})X_u$.
	\end{itemize}
\item $\prover$ and $\verifier$ run aggregate inner product protocols for each
$u\in [t]$:
	\begin{itemize}
	\item
$a_u=\agginnerproduct(\GG,\bm{g},e_{j_u},\pi[\cdot,k_u],X_u;\ewit[\cdot,\cdot,k_u])$
	\end{itemize}
\item $\prover$ and $\verifier$ run the subprotocol: 
	\begin{itemize}
	\item $b_{\rm prox} = \proximityThreeD(\FF,\GG,L_1,L_2,[\pi];U)$
	\end{itemize}
\item The verifier accepts if all the subprotocols accept.
\end{enumerate}
\end{itemize}

\begin{itemize}
\item $\agginnerproduct(\FF,\GG,\bm{g},x,\bm{c}\in \GG^n,\bm{v}\in \FF^n;W\in
\FF^{m\times n})$:
\item {\bf Relation}: $\forall i\in [n]$: $W[i,\cdot]=\open(\bm{c}[i])$,
$\innp{x}{W[i,\cdot]}=v[i]$. 
\begin{enumerate}
\item $\verifier\rightarrow\prover$: Verifier samples $\delta\in \FF^n$ and
sends it to the prover.
\item $\prover\leftrightarrow\verifier$ compute: $\mathsf{cm}=\sum_{i\in
[n]}\delta_i\bm{c}[i]$, $V=\sum_{i\in [n]}\delta_i\bm{v}[i]$
\item $\prover$ and $\verifier$ run the subprotocol:
	\begin{itemize}
	\item $b=\innerproduct(\FF,\GG,\bm{g},\delta,\mathsf{cm},V;\bm{v}^TW)$
	\end{itemize}
\item $\verifier$ accepts if $b=1$.
\end{enumerate}
\end{itemize}

\end{framed}
\caption{Linear Check Protocol}
%\label{fig:linearcheck}
\end{figure}


\begin{figure}[h!]
\centering
\begin{framed}
\small
\begin{itemize}
\item $\quadcheck(\FF,\GG,L_1,L_2,[\pi];\wit_x,\wit_y,\wit_z)$:
\item {\bf Relation}: $\exists (\ewit_x,\ewit_y,\ewit_z)$
s.t. $[\ewit_x||\ewit_y||\ewit_z]=\open(\pi)$, $\wit_{a}=\dec(\ewit_a)$ for $a\in
\{x,y,z\}$ and $\wit_x\circ \wit_y = \wit_z$.
\item {\bf Oracle Setup}: The prover $\prover$ computes $\comoracle_a=\comm(\ewit_a)$ 
for $a\in \{x,y,z\}$. It sets $\pi :=
[\comoracle_x||\comoracle_y||\comoracle_z]$ where the notation denotes vertical
stacking of the matrices. 
\begin{enumerate}[{\rm 1.}]
\item $\verifier\rightarrow\prover$: Verifier samples $r\sample\FF^p$ and sends
it to $\prover$.
\item $\prover$ computes:
	\begin{itemize}
	\item polynomials $p_j(\cdot):=\sum_{i\in
[p]}r_i\big(Q^i_x(\alpha_j,\cdot)Q^i_y(\alpha_j,\cdot)-Q^i_z(\alpha_j,\cdot)\big)$
for $j\in [h]$.
	\item $h\times n$ matrix $P$ such that $P[j,k]=p_j(\eta_k)$.
	\item commitments $c_1,\ldots,c_{2\ell}$ to the first $2\ell$ columns of
$P$.
	\end{itemize}
\item $\prover\rightarrow\verifier$: The prover sends $c_1,\ldots,c_{2\ell}$ to
the verifier.
\item $\verifier\rightarrow\prover$: $\verifier$ samples $(j_u,k_u)\sample
[h]\times [n]$ for $u\in [t]$. The verifier also samples $\tau\sample
\FF^{\ell}$. It sends $Q=\{(j_u,k_u):u\in [t]\}$ and $\tau$ to $\prover$.
\item $\prover\rightarrow\verifier$: The prover sends vectors
$X_u=\ewit_x[\cdot,j_u,k_u],Y_u=\ewit_y[\cdot,j_u,k_u],Z_u=\ewit_z[\cdot,j_u,k_u]$
to the verifier, for all $u\in [t]$.
\item Oracle Queries: $\verifier$ queries the oracle $\pi$ with $\{k_u: u\in
[t]\}$.
\item Oracle Response: The oracle responds with columns $\pi[\cdot,k_u]$ for
$u\in [t]$.
\item $\prover\leftrightarrow\verifier$: Both $\prover$ and $\verifier$ compute
$\varphi := \Phi^T\tau$ and $\mathsf{cm}:= \sum_{k\in [2\ell]}\phi_kc_k$.
\item $\prover$ and $\verifier$ run the subprotocol:
	\begin{itemize}
	\item $b_{\rm
2d}=\proximityTwoD(\FF,\GG,\dashL_1,\dashL_2,\bm{c};\overline{P})$ where
$\bm{c}=(c_1,\ldots,c_{2\ell})$ and $\overline{P}$ is the submatrix of $P$ consisting
of the first $2\ell$ columns. Here $\dashL_1=\rsc{\eta}{2\ell}$ and
$\dashL_2=\rsc{\alpha}{2m}$.
	\end{itemize}
\item $\prover$ and $\verifier$ run the subprotocol:
	\begin{itemize}
	\item $b=\innerproduct(\FF,\GG,\bm{g},x,\mathsf{cm},v;z)$
with $x=(\gamma,0^{h-m})$, $v=0$ and $z=\overline{P}\varphi$.
	\end{itemize}
\item $\prover$ and $\verifier$ compute: For all $u\in [t]$ $c_{k_u} =
\sum_{a\in 2\ell}T[a,k_u]c_a$, where $T$ is the matrix such that $P=\overline{P}T$.
\item $\prover$ and $\verifier$ run inner product subprotocols for each $u\in
[t]$:
	\begin{itemize}
	\item $s_u=\innerproduct(\GG,\bm{g},x,c_{k_u},v;P[\cdot,k_u])$ with
$x=e_{j_u}$, $v=\sum_{i\in [p]}r_i(X_u[i]\cdot Y_u[i] - Z_u[i])$.
	\end{itemize}
\item $\prover$ and $\verifier$ run aggregate inner product protocols for each
$u\in [t]$:
	\begin{itemize}
	\item
$a_{1u}=\agginnerproduct(\GG,\bm{g},e_{j_u},\pi_x[\cdot,k_u],X_u;\ewit_x[\cdot,\cdot,k_u])$
	\item
$a_{2u}=\agginnerproduct(\GG,\bm{g},e_{j_u},\pi_y[\cdot,k_u],Y_u;\ewit_y[\cdot,\cdot,k_u])$
	\item
$a_{3u}=\agginnerproduct(\GG,\bm{g},e_{j_u},\pi_z[\cdot,k_u],Z_u;\ewit_z[\cdot,\cdot,k_u])$
	\end{itemize}
\item $\prover$ and $\verifier$ run the subprotocol: 
	\begin{itemize}
	\item $b_{\rm prox} =
\proximityThreeD(\FF,L_1,L_2,e,[\pi];[\ewit_x||\ewit_y||\ewit_z])$
	\end{itemize}
\item The verifier accepts if all the subprotocols accept.
\end{enumerate}
\end{itemize}
\end{framed}
\caption{Quadratic Check Protocol}
\label{fig:quadcheck}
\end{figure}
\end{comment} 

\begin{comment}
\begin{figure}[h!]
\centering
\begin{framed}
\footnotesize
\begin{itemize}
\item {$\distlinearcheck(\FF,\GG,L_1,L_2,A\in \FF^{M\times N},b\in
\FF^M,[\pi];\shr{\ewit},\shr{0^h})$}
\item {\bf Relation}: $\exists \ewit$ s.t. $\ewit=\open(\pi)$ and $A\wit = b$
for $\wit=\dec(\ewit)$.
\item {\bf Oracle Setup}: 
	\begin{itemize}
	\item $\distprover\rightarrow\Ag$: The prover $\distprover$ computes
$\shr{\comoracle} = \comm(\shr{\ewit})$ as in Section
\ref{sec:construct_oracle}. It sends $\shr{\comoracle}$ to $\Ag$. 
	\item {\color{red} $\Ag$ computes: $\comoracle := \combine(\shr{\comoracle})$ and
sets $\pi := \comoracle$}.
	\end{itemize}
\begin{enumerate}[{\rm 1.}]
\item $\verifier\rightarrow\distprover$: $\verifier$ samples $r\sample \FF^N$ and
sends it to $\distprover$.
\item $\distprover\leftrightarrow\verifier$: Provers ($\distprover$) and $\verifier$
compute polynomials $R^i$, $i\in [p]$ such that
$R^i(\alpha_j,\zeta_k)=R[i,j,k]$ where $R=r^TA$. 
\item $\distprover$ computes:
	\begin{enumerate}
	\item Polynomials $\shr{p_j}(\cdot) := \sum_{i\in
[p]}R^i(\alpha_j,\cdot).\shr{Q^i}(\alpha,\cdot)$ for $j\in [h]$. Here the
polynomial $\shr{Q^i}$ interpolates the witness share
$\shr{\wit}[i,\cdot,\cdot]$ on $G$.
	\item An $h\times n$ matrix $\shr{P}$ such that $\shr{P}[j,k]=\shr{p_j}(\eta_k)$.
	\item Commitments $\shr{c_k}=\comm(\shr{P}[\cdot,k],\shr{\omega}_k)$ for $k\in [s+\ell]$
where $\shr{\omega}_k\sample \FF$.
	\item Sample a codeword $\shr{P_0}\in \dashL_2$ s.t $\sum_{j\in
[m]}\shr{P_0}[j]=0$.
Compute $\shr{c_0}=\comm(\shr{P_0},\shr{\omega_0})$ for $\shr{\omega_0}\sample \FF$.
	\end{enumerate}
\item $\distprover\rightarrow\Ag$: Provers send
$\shr{c_0},\shr{c_1},\ldots,\shr{c_{s+\ell}}$ and $\shr{Z}$ to $\Ag$.

% Aggregation step
\item {\color{red} $\Ag$ computes: $(c_0,c_1,\ldots,c_{s+\ell}) :=
\combine(\shr{c_0},\shr{c_1},\ldots,\shr{c_{s+\ell}})$ and $Z :=
\combine(\shr{Z})$}.

\item $\verifier\rightarrow\distprover$: $\verifier$ samples $(j_u,k_u)\sample [h]\times
[n]$ for $u\in [t]$. It also samples $\beta\sample\FF$. The verifier sends
$Q=\{(j_u,k_u):u\in [t]\}$ and $\beta$ to provers $\distprover$.
\item $\distprover$ computes:
	\begin{itemize}
	\item $\shr{X_u}=\shr{\ewit}[\cdot,j_u,k_u]$ for $u\in [t]$.
	\item $\shr{Y_u}=\shr{P}[\cdot,k_u]$ for $u\in [t]$.
	\item $\shr{Z}=\beta\shr{P_0} + \shr{\overline{P}}\varphi + \shr{0^h}$ for $\varphi=\Phi^T[1^s]$.
	\item $\shr{\omega}=\beta\shr{\omega_0} +
\sum_{a=1}^{s+\ell}\varphi_a\shr{\omega_a}$.
	\item $\shr{\omega_{k_u}}=\sum_{a\in [s+\ell]}T[a,k_u]\shr{\omega_a}$
for $u\in [t]$.
	\end{itemize}
\item $\distprover\rightarrow\Ag$: Prover $\distprover$ sends $\shr{X_u}$,
$\shr{Y_u}$, $\shr{\omega_{k_u}}$ for $u\in [t]$, $\shr{Z}$ and $\shr{\omega}$ to $\Ag$.


% Aggregation step
\item {\color{red} $\Ag$ computes: $X_u=\combine(\shr{X_u})$,
$P[\cdot,k_u]=\combine(\shr{Y_u})$ for $u\in [t]$. It computes
$Z=\combine(\shr{Z})$ and $\omega=\combine(\shr{\omega})$ and 
$\omega_{k_u}=\combine(\shr{\omega_{k_u}})$ for $u\in [t]$}.
  
\item Oracle Queries: $\verifier$ queries the oracle $\pi$ with $\{k_u:u\in
[t]\}$. 
\item Oracle Answers: The oracle replies with columns $\pi[\cdot,k_u]$, $u\in
[t]$.
\item $\Ag\leftrightarrow\verifier$: Both $\Ag$ and $\verifier$
compute $\varphi := \Phi^T[1^s]$ and $\mathsf{cm} := \beta c_0 + \sum_{k\in
[s+\ell]}\varphi_kc_k$.
\item $\Ag$ and $\verifier$ run the subprotocol:
	\begin{itemize}
	\item $b_{\rm 2d} =
\distproxTwoD(\FF,\GG,s+\ell,\dashL_1,\dashL_2,\bm{c};[[\overline{P}]],[[\bm{\omega}]])$
where $\bm{c}=(c_1,\ldots,c_{s+\ell})$ and $\overline{P}$ is the submatrix of $P$
consisting of the first $s+\ell$ columns. Here $\dashL_1=\rsc{\eta}{s+\ell}$ and
$\dashL_2=\rsc{\alpha}{2m}$.
	\end{itemize}
\item $\Ag$ and $\verifier$ run the subprotocol:
	\begin{itemize}
	\item $b = \innerproduct(\GG,\bm{g},x,\mathsf{cm},v;z,\omega)$
with $x=(1^m,0^{h-m})$, $v=r^Tb$ and $z=Z$.
	\end{itemize}
\item $\Ag$ and $\verifier$ compute: For all $u\in [t]$ $c_{k_u} =
\sum_{a\in 2\ell}T[a,k_u]c_a$, where $T$ is the matrix such that $P=\overline{P}T$.
\item $\Ag$ and $\verifier$ run inner product subprotocols for each $u\in
[t]$:
	\begin{itemize}
	\item
$s_u=\innerproduct(\GG,\bm{g},x,c_{k_u},v;P[\cdot,k_u],\omega_{k_u})$ with
$x=e_{j_u}$, $v=\sum_{i\in [p]}R^i(\alpha_{j_u},\eta_{k_u})X_u$.
	\end{itemize}
\item $\Ag$ and $\verifier$ run aggregate inner product protocols for each
$u\in [t]$:
	\begin{itemize}
	\item
$a_u=\agginnerproduct(\GG,\bm{g},e_{j_u},\pi[\cdot,k_u],X_u;\ewit[\cdot,\cdot,k_u])$
	\end{itemize}
\item $\prover$ and $\verifier$ run the subprotocol: 
	\begin{itemize}
	\item $b_{\rm prox} =
\distproxThreeD(\FF,\GG,L_1,L_2,[\pi];[[\ewit]])$
	\end{itemize}
\item The verifier accepts if all the subprotocols accept.
\end{enumerate}
\end{itemize}
\end{framed}
\caption{Distributed Linear Check Protocol}
\label{fig:distlincheck}
\end{figure}


\begin{figure}[h!]
\centering
\begin{framed}
\small
\begin{itemize}
\item $\distquadcheck(\FF,\GG,L_1,L_2,[\pi];[[\wit_x]],[[\wit_y]],[[\wit_z]])$:
\item {\bf Relation}: $\exists (\ewit_x,\ewit_y,\ewit_z)$
s.t. $[\ewit_x||\ewit_y||\ewit_z]=\open(\pi)$, $\wit_{a}=\dec(\ewit_a)$ for $a\in
\{x,y,z\}$ and $\wit_x\circ \wit_y = \wit_z$.
\item {\bf Oracle Setup}: 
\begin{itemize}
\item $\distprover\rightarrow\Ag$: $\distprover$ computes $\shr{\comoracle_a}=\comm(\shr{\ewit_a})$ 
for $a\in \{x,y,z\}$. It sends $\shr{\comoracle_a}$, $a\in \{x,y,z\}$ to $\Ag$.
\item $\Ag$ computes: $\comoracle_a=\combine(\shr{\comoracle_a})$ for $a\in
\{x,y,z\}$ and sets $\pi := [\comoracle_x||\comoracle_y||\comoracle_z]$.
\end{itemize}
\begin{enumerate}[{\rm 1.}]
\item $\verifier\rightarrow\distprover$: Verifier samples $r\sample\FF^p$ and sends
it to $\distprover$.
\item Provers $\distprover$ run the MPC: $\shr{\ewit_x.\ewit_y}\leftarrow
\mathsf{Mult}(\shr{\ewit_x},\shr{\ewit_y})$ to obtain shares of the hadamard
product of the encodings.
\item $\prover$ computes:
	\begin{itemize}
	\item $h\times n$ matrix $\shr{P}$ such that 
	$\shr{P}[j,k]=\sum_{i\in
[p]}r_i(\shr{\ewit_x[i,j,k]\ewit_y[i,j,k]}-\shr{\ewit_z[i,j,k]})$.
	\item commitments $\shr{c_k}=\comm(\shr{P}[\cdot,k],\shr{\omega_k})$ for
$k\in [2\ell]$ where $\shr{\omega_k}$ are sampled randomly for all $k\in
[2\ell]$.
	\end{itemize}
\item $\distprover\rightarrow\Ag$: The prover $\distprover$ sends
$\shr{c_1},\ldots,\shr{c_{2\ell}}$ to $\Ag$.
\item $\Ag\rightarrow \verifier$: $\Ag$ computes $\bm{c}=(c_1,\ldots,c_{2\ell})$ where
$c_k=\combine(\shr{c_k})$ for $k\in [2\ell]$. It sends $\bm{c}$ to $\verifier$.

\item $\verifier\rightarrow\distprover$: $\verifier$ samples $(j_u,k_u)\sample
[h]\times [n]$ for $u\in [t]$. The verifier also samples $\tau\sample
\FF^{\ell}$. It sends $Q=\{(j_u,k_u):u\in [t]\}$ and $\tau$ to $\distprover$.
\item $\distprover$ computes:
	\begin{itemize}
	\item $\shr{X_u}=\shr{\ewit_x}[\cdot,j_u,k_u]$, $u\in [t]$.
	\item $\shr{Y_u}=\shr{\ewit_y}[\cdot,j_u,k_u]$, $u\in [t]$.
	\item $\shr{Z_u}=\shr{\ewit_z}[\cdot,j_u,k_u]$, $u\in [t]$.
	\item $\shr{\omega}=\sum_{a\in 2\ell}\varphi_a\shr{\omega_a}$.
	\item $\shr{\omega_{k_u}}=\sum_{a\in 2\ell}T[a,k_u]\shr{\omega_a}$,
$u\in [t]$.
	\end{itemize}
\item $\distprover\rightarrow\Ag$: The prover $\distprover$ sends
$\shr{X_u},\shr{Y_u},\shr{Z_u}$, $\shr{\omega_{k_u}}$ for $u\in [t]$ and
$\shr{\omega}$ to $\Ag$.
\item $\Ag$ computes: $\Ag$ obtains $X_u,Y_u,Z_u,\omega_{k_u}$ for $u\in [t]$
from the respective shares. It also obtains $\omega$ from its shares. 
\item Oracle Queries: $\verifier$ queries the oracle $\pi$ with $\{k_u: u\in
[t]\}$.
\item Oracle Response: The oracle responds with columns $\pi[\cdot,k_u]$ for
$u\in [t]$.
\item $\Ag\leftrightarrow\verifier$: Both $\Ag$ and $\verifier$ compute
$\varphi := \Phi^T\tau$ and $\mathsf{cm}:= \sum_{k\in [2\ell]}\phi_kc_k$.
\item $\Ag$ and $\verifier$ run the subprotocol:
	\begin{itemize}
	\item $b_{\rm
2d}=\distproxTwoD(\FF,\GG,\dashL_1,\dashL_2,\bm{c};[[\overline{P}]],[[\bm{\omega}]])$ where
$\bm{c}=(c_1,\ldots,c_{2\ell})$ and $\overline{P}$ is the submatrix of $P$ consisting
of the first $2\ell$ columns. Here $\dashL_1=\rsc{\eta}{2\ell}$ and
$\dashL_2=\rsc{\alpha}{2m}$.
	\end{itemize}
\item $\Ag$ and $\verifier$ run the subprotocol:
	\begin{itemize}
	\item $b=\innerproduct(\FF,\GG,\bm{g},x,\mathsf{cm},v;z,\omega)$
with $x=(\gamma,0^{h-m})$, $v=0$ and $z=\overline{P}\varphi$. Note that $z$ is
the unique codeword in $\dashL_2$ with $\sum_{j\in [m]}z[j]=0$ so $\Ag$ knows
this without receiving shares from the provers.
	\end{itemize}
\item $\Ag$ and $\verifier$ compute: For all $u\in [t]$ $c_{k_u} =
\sum_{a\in 2\ell}T[a,k_u]c_a$, where $T$ is the matrix such that $P=\overline{P}T$.
\item $\Ag$ and $\verifier$ run inner product subprotocols for each $u\in
[t]$:
	\begin{itemize}
	\item
$s_u=\innerproduct(\GG,\bm{g},x,c_{k_u},v;P[\cdot,k_u],\omega_{k_u})$ with
$x=e_{j_u}$, $v=\sum_{i\in [p]}r_i(X_u[i]\cdot Y_u[i] - Z_u[i])$.
	\end{itemize}
\item $\Ag$ and $\verifier$ run aggregate inner product protocols for each
$u\in [t]$:
	\begin{itemize}
	\item
$a_{1u}=\agginnerproduct(\GG,\bm{g},e_{j_u},\pi_x[\cdot,k_u],X_u;\ewit_x[\cdot,\cdot,k_u])$
	\item
$a_{2u}=\agginnerproduct(\GG,\bm{g},e_{j_u},\pi_y[\cdot,k_u],Y_u;\ewit_y[\cdot,\cdot,k_u])$
	\item
$a_{3u}=\agginnerproduct(\GG,\bm{g},e_{j_u},\pi_z[\cdot,k_u],Z_u;\ewit_z[\cdot,\cdot,k_u])$
	\end{itemize}
\item $\Ag$ and $\verifier$ run the subprotocol: 
	\begin{itemize}
	\item $b_{\rm prox} =
\distproxThreeD(\FF,L_1,L_2,e,[\pi];[\shr{\ewit_x||\ewit_y||\ewit_z}])$
	\end{itemize}
\item The verifier accepts if all the subprotocols accept.
\end{enumerate}
\end{itemize}
\end{framed}
\caption{Distributed Quadratic Check Protocol}
\label{fig:distquadcheck}
\end{figure}
\end{comment}
 
\section{Graphene for R1CS}\label{sec:graphener1cs}
We use the linear and quadratic check protocols from Section \ref{sec:graphene}
to describe an efficient protocol for rank one constraint system (R1CS). Let 
$A,B$ and $C$ be $M\times N$ matrices. We prove existence of $\wit\in \FF^N$
satisfying $A\wit\circ B\wit=C\wit$ by showing existence of
$\wit_x,\wit_y,\wit_z$ and $\wit\in \FF^N$ satisfying the linear relations $A\wit=\wit_x$,
$B\wit=\wit_y$ and $C\wit=\wit_z$ and a quadractic relation
$\wit_x\circ\wit_y=\wit_z$. We probabilistically reduce the three linear
relations to the linear relation:
\begin{align}\label{eq:comblincheck}
\begin{bmatrix}
\gamma_xI\,|\gamma_yI\,|\gamma_zI\,|-(\gamma_xA+\gamma_yB+\gamma_zC)
\end{bmatrix}\begin{bmatrix}
\wit_x\\
\wit_y\\
\wit_z\\
\wit
\end{bmatrix}=\bm{0}
\end{align}

for $\gamma_x,\gamma_y,\gamma_z\sample \FF$. As before, we
$\wit_x,\wit_y,\wit_z$ and $\wit$ as $p\times m\times s$ matrices. Let $\bwit$
denote the $4p\times m\times s$ matrix formed by stacking $\wit_x,\wit_y,\wit_z$
and $\wit$ along ``slices''. The encoding $\ewit\gets\enc(\bwit)$ and commitment to
the encoding $\comoracle\gets \comm(\ewit)$ are computed as in Sections
~\ref{sec:witencoding} and ~\ref{sec:construct_oracle}. Note that
$\comoracle$ is a $4p\times n$ matrix. 
We now present the high level protocol for R1CS instance $(A,B,C)$.\smallskip

\begin{figure}[h]
	{\footnotesize
		\begin{framed}
\noindent{$\grapheneRCS(\mathsf{pp},A,B,C,[\pi];\wit_x,\wit_y,\wit_z,\wit)$}:\\
\noindent{\bf Relation}: $A\wit\circ B\wit=C\wit$. \\
\noindent{\bf Oracle Setup}: Compute $\comoracle$ as described above. Set $\pi
:= \comoracle$.
\begin{enumerate}[{\rm 1.}]
\item $\verifier\rightarrow\prover$: $\gamma_x,\gamma_y,\gamma_z\sample \FF$,
$r_{lc}\sample \FF^M$, $r_{qd}\sample \FF^{p}$, $\rho\sample \FF^{4p}$.
\item $\prover\rightarrow\verifier$: $\prover$ computes $\tilde{\ewit}=\sum_{i\in
[4p]}\rho_i\ewit[i,\cdot,\cdot]$ and sends commitments
$\tilde{c}_1,\ldots,\tilde{c}_\ell$ to $\tilde{\ewit}$.
\item $\prover\leftrightarrow\verifier$ compute: $R=r_{lc}^TW$ for
$W=[\gamma_xI\,||\gamma_yI\,||\gamma_zI\,||-(\gamma_xA+\gamma_yB+\gamma_zC)]$,
 polynomials $R^i$,$i\in [4p]$ interpolating the slices of $R$ viewed as a
$4p\times m\times n$ matrix.
\item $\prover$ computes: 
\begin{itemize}
\item Polynomials $Q^i_x,Q^i_y,Q^i_z,Q^i$ for $i\in [p]$,
where polynomials $Q^i_a$, $i\in [p]$ correspond to $\wit_a$ for $a\in \{x,y,z\}$ and
polynomials $Q^i$,$i\in [p]$ correspond to $\wit$. 
\item $h\times n$ matrices $P_{lc}$ and $P_{qd}$ as ``P'' matrices for the linear check and
quadratic check respectively. Note that $p_j$ polynomial for $P_{lc}$ is given
by $p_j(\cdot)=\sum_{i=1}^p(R^i(\alpha_j,\cdot).Q_x^i(\alpha_j,\cdot)+
 R^{p+i}(\alpha_j,\cdot)Q^i_y(\alpha_j,\cdot)+R^{2p+i}(\alpha_j,\cdot)Q^i_z(\alpha_j,\cdot)+
R^{3p+i}(\alpha_j,\cdot)Q^i(\alpha_j,\cdot))$. The $p_j$ polynomials for the
matrix $P_{qd}$ are given by
$p_j(\cdot)=\sum_{i=1}^{p}r_{qd}[i](Q^i_x(\alpha_j,\cdot)Q^i_y(\alpha_j,\cdot)-Q^i_z(\alpha_j,\cdot))$.
\item Blinding vectors $U_{lc},U_{qd}\in \FF^{2m-1}$ for linear and quadratic check protocols
respectively, and commitments $c_0,d_0$ to vectors $U_{lc}$
and $U_{qd}$.
\end{itemize}
\item $\prover\rightarrow\verifier$: Commitments $c_0,c_1,\ldots,c_{s+\ell-1}$ for
the matrix $P_{lc}$ and commitments $d_0,d_1,\ldots,d_{2\ell-1}$ for matrix
$P_{qd}$. 
\item $\verifier\rightarrow\prover$: $Q=\{(j_u,k_u):u\in [t]\}$.
\item $\verifier\rightarrow\pi$: $\{k_u:u\in [t]\}$.
\item $\prover\rightarrow\verifier$: $S_u=\ewit[\cdot,j_u,k_u]$ for $u\in [t]$.
\item $\pi\rightarrow\verifier$: $\pi[\cdot,k_u]$, $u\in [t]$.
\item $\prover$ and $\verifier$ run the linear and quadratic check protocols in
parallel, parsing the vectors $S_u$ into $X_u,Y_u,Z_u,W_u$ as needed.
\item Check proximity as: $\prod_{a=1}^\ell
(\tilde{c}_a)^{\Lambda_{n,\ell}^T[a,k_u]}=\prod_{i=1}^{4p}(\pi[i,k_u])^{\rho_i}$ for $u\in [t]$.
\item $\verifier$ accepts if all the subprotocols accept.
\end{enumerate}
\end{framed}
}
\caption{$\grapheneRCS$ Protocol}
\label{fig:graphene}
\end{figure}
%%%%%%%%%%%%%%%%%%%%%%%%%%%%%%%%%%%%%%%%%%%%%%%%%%%%%
\subsection{Proof of knowledge of $\name$}
\begin{lemma}\label{app:POKgraphene}
	For all polynomially bounded provers $P^\ast$ and all $\pi\in\GG^{4p\times n}$, there exists an expected polynomial time extractor $\extrac$ with rewinding access to the transcript oracle $\tr = \innp{P^{\ast}}{\verifier^{\pi}}$ such that either $\extrac$ breaks the commitment binding, or it outputs a witness with overwhelming probability whenever $P^\ast$ succeeds, i.e.,
	
	{\footnotesize
		\begin{align*}
		\condprob{
			\begin{array}{c}
			{[}\ewit_x||\ewit_y||\ewit_z||\ewit{]}=\open(\pi)\wedge \\
			\wit_z=\wit_x\circ\wit_y\wedge \wit_x=A\wit\\
			\wit_y=B\wit \wedge \wit_z=C\wit
			\end{array}
		}{
			\begin{array}{c}
			\sigma %\sample \gen(\secparam) \\
			\leftarrow \gen(\secparam) \\
			{[}\ewit_x||\ewit_y||\ewit_z||\ewit{]}%\sample \extr^{\mc{O}}(\sigma)\\
			\leftarrow \extr^{\mathsf{tr}}(\sigma) \\ 
			\wit_a=\dec(\ewit_a), a\in \{x,y,z\}\\
			\wit=\dec(\ewit)
			\end{array}
		}\\
		\geq \epsilon(P^\ast) - \kappa_{\rm qd}(\secpar) -\kappa_{\rm lc}(\secpar)
		\end{align*}
	}
Where $\kappa_{\rm lc}(\secpar)$ and $\kappa_{\rm qd}(\secpar)$ are the negligible soundness error of $\linearcheck$ and $\quadcheck$ respectively. And $\epsilon(P^\ast)$ is the success probability of the prover.
\end{lemma}
\begin{proof}
	We will design an extractor $\extrac$ using the extractors $\extrac_{lc}$ and $\extrac_{qd}$ of $\linearcheck$ and $\quadcheck$ described in ~\ref{app:linear_soundness} and ~\ref{app:quad_soundness} respectively. $\extrac$ plays the role of the verifier and $\extrac_{lc}$, $\extrac_{qd}$ use the queries generated by $\extrac$.
\end{proof}
\begin{figure}[h]
	{\footnotesize
		\begin{framed}
	\noindent{$DP-\grapheneRCS(\mathsf{pp},A,B,C,[\pi];\wit_x,\wit_y,\wit_z,\wit)$}:\\
	\noindent{\bf Relation}: $A\wit\circ B\wit=C\wit$. \\
	\noindent{\bf Oracle Setup}: Compute $\comoracle$ as described above. Set $\pi
	:= \comoracle$.
	\begin{enumerate}[{\rm 1.}]
		\item $\verifier\rightarrow\distprover$: $\gamma_x,\gamma_y,\gamma_z\sample \FF$,
		$r_{lc}\sample \FF^M$, $r_{qd}\sample \FF^{p}$, $\rho\sample \FF^{4p}$.
		\item $\distprover\rightarrow\Ag$: $\distprover$ computes $\shr{\tilde{\ewit}}=\sum_{i\in
			[4p]}\rho_i\shr{\ewit}[i,\cdot,\cdot]$ and sends commitments
		$\shr{\tilde{c}_1},\ldots,\shr{\tilde{c}_\ell}$ to $\shr{\tilde{\ewit}}$.
		\item $\Ag \rightarrow \verifier$: $\Ag$ computes $\tilde{c}_k=\prod_{\xi \in [\Num]} \shr{\tilde{c}_k}$ and sends $\tilde{c}_1,\ldots, \tilde{c}_{\ell}$.
		\item $\distprover\leftrightarrow\verifier$ compute: $R=r_{lc}^TW$ for
		$W=[\gamma_xI\,||\gamma_yI\,||\gamma_zI\,||-(\gamma_xA+\gamma_yB+\gamma_zC)]$,
		polynomials $R^i$,$i\in [4p]$ interpolating the slices of $R$ viewed as a
		$4p\times m\times n$ matrix.
		\item $\distprover$ computes: 
		\begin{itemize}
			\item Polynomials $\shr{Q^i_x},\shr{Q^i_y},\shr{Q^i_z},\shr{Q^i}$ for $i\in [p]$,
			where polynomials $\shr{Q^i_a}$, $i\in [p]$ correspond to $\shr{\wit_a}$ for $a\in \{x,y,z\}$ and
			polynomials $\shr{Q^i}$,$i\in [p]$ correspond to $\shr{\wit}$. 
			\item \textbf{MPC}: provers interact to compute $\mathsf{Mult}(\shr{Q^i_x},\shr{Q^i_y})$ and $\distprover$ gets $\shr{Q^i_{xy}}$, share of the polynomial $Q^i_{xy}$.
			\item $h\times n$ matrices $\shr{P_{lc}}$ and $\shr{P_{qd}}$ as ``P'' matrices for the linear check and
			quadratic check respectively. Note that $\shr{p_j}$ polynomial for $\shr{P_{lc}}$ is given
			by $\shr{p_j}(\cdot)=\sum_{i=1}^p(R^i(\alpha_j,\cdot).\shr{Q_x^i}(\alpha_j,\cdot)+
			R^{p+i}(\alpha_j,\cdot)\shr{Q^i_y}(\alpha_j,\cdot)+R^{2p+i}(\alpha_j,\cdot)\shr{Q^i_z}(\alpha_j,\cdot)+
			R^{3p+i}(\alpha_j,\cdot)\shr{Q^i}(\alpha_j,\cdot))$. The $\shr{p_j}$ polynomials for the
			matrix $\shr{P_{qd}}$ are given by
			$\shr{p_j}(\cdot)=\sum_{i=1}^{p}r_{qd}[i](\shr{Q^i_{xy}}(\alpha_j,\cdot)-\shr{Q^i_z}(\alpha_j,\cdot))$.
			\item Blinding vectors $\shr{U_{lc}},\shr{U_{qd}}\in \FF^{2m-1}$ for linear and quadratic check protocols
			respectively, and commitments $\shr{c_0},\shr{d_0}$ to vectors $\shr{U_{lc}}$
			and $\shr{U_{qd}}$.
		\end{itemize}
		\item $\distprover\rightarrow\Ag$: Commitments $\shr{c_0},\shr{c_1},\ldots,\shr{c_{s+\ell-1}}$ for
		the matrix $\shr{P_{lc}}$ and commitments $\shr{d_0},\shr{d_1},\ldots,\shr{d_{2\ell-1}}$ for matrix
		$\shr{P_{qd}}$. 
		\item $\Ag \rightarrow \verifier$: $\Ag$ computes $c_k = \prod_{\xi \in [\Num]} \shr{c_k}$ $\forall k\in \{0,\ldots, s+\ell-1\}$ and $d_k = \prod_{\xi \in [\Num]} \shr{d_k}$ $\forall k\in \{0,\ldots,2\ell-1\}$ and sends $c_0,c_1,\ldots, c_{s+\ell-1}, d_0,d_1,\ldots,d_{2\ell-1}$.
		\item $\verifier\rightarrow\distprover$: $Q=\{(j_u,k_u):u\in [t]\}$.
		\item $\verifier\rightarrow\pi$: $\{k_u:u\in [t]\}$.
		\item $\distprover\rightarrow\Ag$: $\shr{S_u}=\shr{\ewit}[\cdot,j_u,k_u], \shr{Plc_{u}}=\shr{P_{lc}}[\cdot,k_u], \shr{Pqd_{u}}=\shr{P_{qd}}[\cdot,k_u]$ for $u\in [t]$.
		\item $\Ag$ computes $S_u=\sum_{\xi\in [\Num]} \shr{S_u}$, $Plc_{u}=\sum_{\xi\in [\Num]}\shr{Plc_{u}}$ and $Pqd_{u}=\sum_{\xi\in [\Num]}\shr{Pqd_{u}}$ for $u\in [t]$.
		\item $\Ag \rightarrow \verifier$: $S_u$ for $u\in[t]$
		\item $\pi\rightarrow\verifier$: $\pi[\cdot,k_u]$, $u\in [t]$.
		\item $\verifier \rightarrow \distprover$: $\delta_{lc}\sample \FF^{4p}, \delta_{qd}\sample \FF^p, \beta_{lc}, \beta_{qd} \sample \FF^\ast, \beta_x, \beta_y, \beta_z \sample \FF$
		\item $\distprover \rightarrow \Ag$: $\shr{Vlc_u} = \innp{\delta_{lc}}{\shr{U[\cdot,j_u,k_u]}}$,\\ $\shr{Vqd_u} = \innp{\delta_{qd}}{(\beta_x U_x[\cdot,j_u,k_u] + \beta_y U_y[\cdot,j_u,k_u]  + \beta_z U_z[\cdot,j_u,k_u])}$
		\item $\Ag$ computes $Vlc_{u} = \sum_{\xi\in [\Num]} Vlc_u$ and $Vqd_{u} = \sum_{\xi\in [\Num]} Vqd_u$
		\item $\Ag$ and $\verifier$ run the linear and quadratic check protocols in
		parallel, parsing the vectors $V_u$ into $X_u,Y_u,Z_u,W_u$ as needed. And compute $A_u = \beta_xX_u+\beta_yY_u+\beta_zZ_u$ for $u\in[t]$ and corresponding commitment.
		
		\item Check proximity as: $\prod_{a=1}^\ell
		(\tilde{c}_a)^{\Lambda_{n,\ell}^T[a,k_u]}=\prod_{i=1}^{4p}(\pi[i,k_u])^{\rho_i}$ for $u\in [t]$.
		\item $\verifier$ accepts if all the subprotocols accept.
	\end{enumerate}
	\end{framed}
}
\caption{Distributed $\grapheneRCS$ Protocol}
\label{fig:dpgraphene}
\end{figure}
%%%%%%%%%%%%%%%%%%%%%%%%%%%%%%%%%%%%%%%%%%%%%%%%%%%%%

\subsection{Reducing MPC Overhead for Small Shared Circuits}\label{sec:sharedcircuitopti}
Let $C$ be a circuit with $N$ wires (we assume a unique incoming wire for each
input gate), let $\wit=(w_1,\ldots,w_N)$ denote the vector of wire values in a
satisfying assignment according to some ordering on the wires. We consider the
case, when each input wire is ``assigned'' to a unique prover. Let $M$ be the
number of multiplication gates in $C$, and let $A$, $B$, $C$ 
 be $M\times N$ matrices such that $\wit_x=A\wit$, $\wit_y=B\wit$
and $\wit_z=C\wit$ are vectors of {\em left}, {\em right} and {\em out} values
of multiplication gates. For $i\in [m]$, we say that multiplication gate $i$ is
{\em isolated} if $\wit_x[i],\wit_y[i]$ and $\wit_z[i]$ depend on inputs of only
one prover. We call the remaining multiplication gates as {\em shared}. Let
$N_s$ be the number of shared gates. Without loss of generality, we can assume
that the gates $1,\ldots,N_s$ are shared (rows of  $A$, $B$, $C$ can be permuted
suitably). Now an additive sharing of the extended witness may be obtained by
(i) each prover locally computing the wires for isolated gates from their
inputs, while setting the shares of other parties for these wires to be $0$ and
(ii) running a secret sharing MPC on the subcircuit containing shared gates.
Let $p_s=\lceil N_s/ms \rceil$. Then the multiplication MPC $\mathsf{Mult}$ in
distributed quadratic check can be restricted to obtaining shares
$\shr{U_x[i,j,k].U_y[i,j,k]}$ for $i\in [p_s]$, as each prover can canonically 
the shares of other slices of $\ewit_x$ and $\ewit_y$. For slices, which depend
on the provers inputs, the prover computes randomized encoding of the slice, for
other slices (where its share is $0$) it computes a deterministic encoding by
setting the buffer columns to $0$. This gives an MPC of depth 1 on circuit of
size $O(\max(N_s, mn))$.

\subsection{Performance Parameters for Ligero and Bulletproofs}\label{sec:performance}
%We summarize the performance parameters for Ligero ~\cite{ligero}, and 
%Bulletproofs ~\cite{bulletproofs} that we use to derive our estimates in 
%Figure ~\ref{fig:standalonecompare}. Let $N$ denote the size of extended
%witness.

\noindent{\bf Ligero}: Let $m,s$ be such that $N=ms$. Let $t$ be a parameter,
and let $\ell=s+t$, $n=O(\ell)$ and $e\leq (n-\ell)/4$
Then the performance parameters for Ligero are then given by:
$\zkcomm=n+6\ell+4s+4mt+5t-5$,
 $t_\prover=O(N)\cfop+4m(\cfft(s)+\cfft(n))$,
$t_\verifier=O(N)+4m\cfft(s)$,
$\kappa_{lg}=(1-e/n)^t+5((2\ell+e)/n)^t$.

\noindent{\bf Bulletproofs}:
$\zkcomm=2\log(N)+13$, $t_\prover=9N\cexp +
2\cmexp(2N)+3\cmexp(N)$, $t_\verifier=N\cexp + \cmexp(2N)$, $\kappa_{bp}=N/|\FF|$.

For computing prover communication in distributed setting we use the expressions:
$\prcomm=MPC(\Num,\max(N_s,4m\ell),1)+\Num\times
[(4pn+5\ell+s+2)\bitsG+(4pt+4m+h)\bitsF]$, for $\dpname$ and
$\prcomm=MPC(\Num,N_s,1)+\Num\times [8\bitsG + N\bitsF]$ for Bulletproofs. Here
$MPC(\Num,N_s,1)$ denotes communication in an $\Num$ party MPC with circuit size
$N_s$ and depth as 1.
 
%%\section{Setting up \name{}} 
\section{ZK arguments with $O(\circsize^{1/c})$ proof size and sublinear public key operations for verifier}
We will start with the general framework that have been used to construct zero-knowledge arguments for arithmetic circuits and along with this we provide relevant information on the prior art, before we proceed to detail our two protocols.
The starting point of our protocols is the Interactive PCP (IPCP) argument presented in Ligero~\cite{ligero}.
Ligero uses elementary techniques, is concretely
efficient being devoid of public key operations, and achieves an argument
size which is square-root in size $\circsize$ of the arithmetic circuit. Ligero follows the IPCP paradigm
to first commit to the witness as an oracle. The prover messages of size $O(\sqrt{\circsize})$ are then checked by the verifier by making $O(\sqrt{\circsize})$ queries to the witness oracle.

The core technical contribution of our first protocol \name2D{} is to use homomorphic commitments and inner product arguments to 
further reduce the size of the queries to the witness oracle. The
reduced access to the witness oracle comes with another surprising benefit: the
verifier restricts its computations to those directly relevant to the parts of
witness being revealed. %This allows us to achieve sublinear verification on average. By carefully restricting the sub-protocols using expensive public key operations to asymptotically smaller circuits (much smaller for practically relevant instantiations),
 We keep the number of public key operations for the
verifier to be strictly sublinear.

Given that we require homomorphic commitments to support distributed proof generation,
the benefits of this single prover protocol are pronounced in the DPZK version.
    
\subsection{Arguments for Arithmetic Circuits}
Our protocol natively supports showing the satisfiability of the $\bbF$-arithmetic circuit for a finite field $\bbF$ which
is an %$\npol$
NP-complete language. For ease of presentation, we produce an argument for the statement ``$\exists \wit, \text{ s.t } \C(\wit)=1$'' for an $\bbF$-arithmetic circuit $\C$. This can be easily modified to yield arguments for the more typical %$\npol$ 
NP language $\calL_C = \{\stmt :\exists \wit \text{ s.t. } \C(\stmt,\wit)=1\}$, which we will briefly discuss later. The satisfiability of an arithmetic circuit $\C$ reduces to proving existence of vectors $x,y,z,\extwit$, where $\extwit$ is the extended witness generated from $\C(\stmt,\wit)$ such that:
$z = x \circ y$,
$x = A \extwit$,
$y = B \extwit$,
$z = C \extwit$ and
$P_{add} \extwit = 0$ for public matrices $A,B,C$ and $P_{add}$ (depending on $\C$). Thus, broadly in the IPCP setting we have the following protocol:
\begin{enumerate}
\item {\bf Oracle Setup}: The prover sets up witness oracle for $x,y,z,\extwit$ and provides query access to the verifier. 
\item {\bf Linear Checks}: The verifier runs a subprotocol with the prover to verify the linear constraints $x=A \extwit$, $y=B \extwit$, $z=C \extwit$, $P_{add} \extwit = 0$.
\item {\bf Quadratic Check}: The verifier runs a subprotocol with the prover to verify $z=x\circ y$.
\end{enumerate}


We will now describe \name2D{}. In brief, \name2D{} will make use of homomorphic commitments over the RS-encoded witness encoding.
%We will commit to the vector $\{\hat{f}^x_i(\eta)\}_{i \in [m]}$ using the homomorphic vector commitments. The vectors corresponding to each $\eta$ will be committed separately. Inner product arguments will then be used to remove the linear dependence of the proof size on the parameter $m$. 
And, instead of ``opening'' vectors from these oracles to the verifier, the prover in \name2D{} will provide an inner product argument on the vectors.
Using the inner-product arguments of \cite{InnerProductDLS, bulletproofs}, \name2D{} achieves the proof size of $O(\circsize^{1/c})$ for any $c \geq 2$ with $O(\circsize^{1-1/c})$ public key operations and $O(\circsize)$ overall complexity for the verifier. 
%-----------------------------------------------------------------------------------------------------
\subsection{Encoding}\label{sec:encode2D} 

%\subsection{Interactive Protocols for RS-code Oracles}
%\subsubsection{Reed Solomon Witness Encoding}
The linear and quadratic constraints discussed above do not naively admit a sublinear query, i.e, the verifier needs to access complete vectors to be convinced with high probability. Error correcting codes have been used to encode the witness in PCP constructions to enable verification with sublinear query. We discuss two such encodings based on Reed-Solomon codes which have been used in recent constructions \cite{ligero, aurora, STARK2019}.
\begin{comment}
To encode a vector $x\in \bbF^\circsize$, one specifies two domains $G,H\subseteq \bbF$. We will call $G$ as {\em interpolation} domain and $H$ as {\em evaluation} domain. The encoding in \cite{aurora} encodes the vector $x$ as a single Reed-Solomon codeword. This is done by first constructing a polynomial $\hat{f}^x$ which interpolates the vector $x$ on $G$, and then computing its evaluations $\langle \hat{f}^x(\eta) \rangle_{\alpha\in H}$ on points in $H$. The sizes of domains $G$ and $H$ need to be $\Omega(\circsize)$ in the above encoding. For a vector $x$, we will use the notation $\hat{f}^x$ to denote the polynomial interpolating $x$ on $G$, and $f^x$ to denote the vector of evaluations of $\hat{f}^x$ on $H$. Thus in the above scheme, $f^x$ is an encoding of the vector $x$.
\dnote{should we replace the notation $\langle \hat{f}^x(\eta) \rangle_{\alpha\in H}$ with $\langle \hat{f}^x(\eta) \rangle_{\eta \in H}$? Isn't $\eta$ the variable?}

An alternative encoding used in \cite{ligero} encodes parts of a vector separately, and thus the encoded vector corresponds to a set of Reed-Solomon codewords, or a single codeword of an Interleaved Reed-Solomon code (see Definition \ref{defn:interleavedcode}). More specifically, one chooses integers $m$ and $\ell$ such that $m\ell \geq \circsize$ and domains $G$ and $H$ of size $\Omega(\ell)$. The vector $x\in \bbF^\circsize$ is written as
$x=(x_1|\cdots|x_m)$ where $x_i\in \bbF^{\ell}$ for all $i\in [m]$. The vector $x$ is encoded as $(f^x_1,\ldots,f^x_m)$ where $f^x_i$ encodes $x_i$ as described before, i.e $f^x_i=\langle \hat{f}^x_i(\eta)\rangle_{\alpha\in H}$ where the polynomial $\hat{f}^x_i$ interpolates the vector $x_i$ on $G$. 
\end{comment}
%----------------------------------------------------------------------------------------------------
%\Encoding
%-----------------------------------
Let $\extwit \in \bbF^{|\C|}$ be the witness vector. Let $m$ and $\ell$ be integers such that $m\ell\geq \circsize$. we choose ordered domains $G=\{\zeta_1,\ldots,\zeta_\ell\}$ and $H=\{\eta_1,\ldots,\eta_n\}$, We will call $G$ as {\em interpolation} domain and $H$ as {\em evaluation} domain. Then write the vectors $\extwit$ as $\extwit = (\extwit_{1},\ldots,\extwit_{m})$ where each $\extwit_{i}\in \bbF^\ell$ for $i \in [m]$. Construct $Q_i(\cdot)$, a polynomial of degree $<k$, by interpolating the vector $\extwit_{i}$ on $G$ and $Q_{i}(\cdot)$ denotes the corresponding evaluation of $Q_{i}$ on $H$. We define the RS-encoded witness $\ewit\in \bbF^{m\times n}$ as $\ewit[i,j]=Q_{i}(\eta_j)$ for $ i\in [m]$ and $j\in [n]$. We now construct a commitment oracle $\comoracle$ from $\ewit$. Additionally, the prover needs to add rows to blind the oracle, for linear check, the prover picks a random polynomial $p_{blind}(\cdot)$ such that $\sum_{j\in[s]} p_{blind}(\zeta_j) = 0$ and for quadratic check, the prover picks random polynomial $p_{blind}(\cdot)$ such that $p_{blind}(\zeta_j) =0$ $\forall j\in[s]$ and for proximity check, the prover includes another random codeword from $\rsc{\eta}{n,\ell}$ with $\ewit$. 
%---------------------------------------------------------------------------------------------------
% Oracle from commitment
%-----------------------------------
\subsection{Oracle Construction}\label{sec:commit2D}
Throughout, we assume $\bbF$ is a prime field. Let $\com$ denote the Pedersen vector commitment scheme over $\bbF^m$ with randomness space as $\bbF$ and commitment space as group $\bbG$ with independent generators $g_1,\ldots,g_m, h$. Define $c_{j} = \com(\ewit[\cdot,j],\delta_{j})$, $j\in [n]$ where the notation $X[\cdot,j]$ denotes the $m$-length vector $(X[1,j],\ldots,X[m,j])$ and $\delta_{j}$ denotes the randomness for computing the commitment $c_{j}$. We define the oracle $\comoracle$ as $\comoracle[j]=c_{j}$. The oracle $\comoracle$ answers queries of the type $Q\subseteq [n]$, responding with elements $\comoracle[j]$ for $j\in Q$.\footnote{Defining the commitment vector as $\comoracle$ might seem superfluous here. But, the usage of $\comoracle$ will play a prominent role in our 3D protocol. We just define the notation here for the uniformity in our descriptions of the 2D and the 3D versions.} 
%---------------------------------------------------------------------------------------------------
\begin{comment} 
\subsubsection{Linear Check}
	Without loss of generality, we will discuss an argument for proving linear constraints of the form $Ax=0$ where $A$ is a public matrix and $x$ is a (secret) vector. The case $Ax=y$ is easily transformed to the required form by defining $A'=[A|-I]$ and $x'=(x,y)$, and running the protocol on $A'$ and $x'$. Assume that $x\in \bbF^\circsize$ and $A\in \bbF^{\circsize\times \circsize}$. To achieve sublinear query complexity, we do two things: (i) we make a high probability reduction to the problem of proving $\innp{r^TA}{x}=0$ where $r\sample \bbF^\circsize$ is sent by the verifier (ii) encode the witness $x$ using Reed-Solomon code as we described earlier.
	
	We first consider the encoding used in \cite{aurora} which encodes $x$ as a single codeword $f^x$. The prover then provides oracle access to $f^x$. Let $\hat{r}\in\bbF[x]$ be the polynomial that interpolates the vector $r^TA$ on $G$. Then the linear check reduces to checking $\sum_{\zeta\in G}\hat{r}(\zeta) \cdot \hat{f}^x(\zeta) = 0$. In \cite{aurora}, the aforementioned identity is checked with query size of $O(\log|H|)$ using specially developed sumcheck protocol for univariate polynomials. The verifier still incurs $O(\circsize)$ work to compute the encoding $\hat{r}$ for the size $\circsize$ vector $r^TA$. 
	
	We now consider the interleaved encoding of the witness. Looking ahead, the interleaved encoding will have a direct impact on the circuit share complexity of the distributed proof generation in \name{}. Here, we write the vector $x\in \bbF^\circsize$ as $x=(x_1|\cdots|x_m)$ where each $x_i\in \bbF^\ell$ for some $m\ell \geq \circsize$, and $x_m$ padded as necessary. For suitable domains $G$ and $H$, we interpolate each chunk of the vector on $G$ 
	separately via polynomials $\hat{f}^x_1,\ldots,\hat{f}^x_m$. Similarly, we write $r^TA=(r_1,\ldots,r_m)$ with $r_i\in \bbF^n$ and construct polynomials $\hat{r}_i$	for $i\in [m]$. The inner product check $\innp{r^TA}{x}=0$ is then equivalent to checking $\sum_{i\in [m]}\sum_{\zeta\in G}\hat{r}_i(\zeta)\hat{f}^x_i(\zeta)=0$. The	latter is checked by the prover sending the polynomial $\hat{p}=\sum_{i\in
	[m]}\hat{r}_i.\hat{f}^x_i$ to the verifier, and verifier checking $\sum_{\zeta\in G}\hat{p}(\zeta)=0$. Choosing $m,\ell\approx O(\circsize^\frac{1}{2})$ this incurs square-root communication from the prover. How can the verifier be sure that $\hat{p}$ was indeed computed correctly from the witness polynomials $\hat{f}^i_x$? As we formally prove in the later sections, this can be accomplished by the verifier querying the polynomial evaluations (oracles) at a constant number of locations, say $\eta_1,\ldots,\eta_q$. The verifier then checks
	that $\hat{p}$ is consistent at the queried locations by verifying $\hat{p}(\eta_j)=\sum_{i\in [m]}\hat{r}_i(\eta_j)\hat{f}^x_i(\eta_j)$ for
	all $j\in [q]$. This incurs a total query complexity of $q.m=O(\circsize^\frac{1}{2})$.	
	From the perspective of verifier's efficiency, it only needs to compute evaluations of polynomials $\hat{r}_i$, $i\in [m]$ for $\eta\in \{\eta_1,\ldots,\eta_q\}$. %%We will show	that this can be done in expected sublinear time, after a one time $O(||A||)$  pre-processing. We will also discuss why a similar pre-processing does not work for the earlier encoding scheme.
\end{comment}
%------------------------------------------------------------------
%Linear Check
\subsection{Linear Check}\label{sec:lincheck2D}
In this section, we describe an IPCP that allows a prover to prove knowledge of
witness $\wit\in \FF^N$ satisfying a linear constraint of the form $A\wit = b$
for some {\em public} $A\in \FF^{M\times N}$ and $b\in \FF^M$. As before we veiw $\wit$ as
$m\times s$ matrix where $N=ms$. As desribed previously in Sections
\ref{sec:encode2D} and \ref{sec:commit2D}, the prover obtains $\ewit\gets 
\enc(\wit)$ and $\pi=\comoracle \gets \ocom(\ewit)$. %The prover then sets the oracle  $\pi := \comoracle$, which can then be queried by the verifier. 
The broad  outline of the protocol is--(a) reduce the problem to checking an inner-product argument via a codeword commitment; (b) check consistency of the codeword commitment and $\pi$ (c) check if $\pi$ committed to a well-formed codeword. The protocol appears in 
Figure ~\ref{fig:linearcheck2D}. In Figure ~\ref{fig:linearcheck2D}, we use
$\mathsf{pp}$ to denote the public parameters, consisting of $(\FF, \GG, \rsc{\eta}{n,s},\bm{g},h)$.

\smallskip

To check $A\wit = b$, the verifier samples random $r\sample \FF^M$ and asks the prover to prove $r^TA\wit = r^Tb$. Both the prover and the verifier view the vector $r^TA\in \FF^N$ as $m\times s$ matrix $R$ and interpolate polynomials $R^i(x)$ for $i\in[m]$ with $\deg(R) < s$ satisfying $R^i(\zeta_j) = R[i,j]\text{ } \forall i\in[m], j\in[s]$. Let $Q^i(x)$, $i\in[m]$ denote the polynomials used in interpolating (and encoding) witness $\wit$. Then $\wit[i,j] = Q^i(\zeta_j)$. The check $\innp{R}{\wit} = r^Tb$ reduces to $\sum_{i,j} R^i(\zeta_j)Q^i(\zeta_j) = r^Tb$ where $i,j$ run over indices in $[m], [s]$ respectively. Therefore, for the polynomial $p(\cdot) = \sum_{i\in[m]}R^i(\cdot)Q^i(\cdot)$, we can rewrite the expression as $\sum_{j\in[s]} p(\zeta_j) = r^Tb$. The prover sends the polynomial $p(\cdot)$ to the verifier and the verifier checks if the above identity holds. Additionally, the polynomial $p(\cdot)$ should also satisfy the following identity:
\begin{align*}
	p(\eta_j) &= \sum_{i\in[m]} R^i(\eta_j)\ewit[i,j] \text{ } \forall j\in[n]\\
	p(\eta_j) &= \innp{R_j}{U[\cdot,j]} 
\end{align*}
where $R_j = [R^1(\eta_j), \ldots, R^m(\eta_j)]$
To check this identity, the verifier sends $t$ many random queries $\{j_u:u\in[t]\}$ to the oracle and obtains $\pi[j_u] = c_{j_u}$. Then the prover and the verifier run inner product arguments to prove the above identity in the randomly chosen points by the verifier. 
For zero knowledge, $p(\cdot)$ is required to be blinded. For that puropose the prover adds already chosen $p_{blind}(\cdot)$ to $p(\cdot)$. $p_{lin}(\cdot)=p(\cdot)+p_{blind}(\cdot)$, which satisies the identity $\sum_{j\in[s]}p_{lin}(\zeta_j) = 0$ and 
$p_{lin}(\eta_j) = \innp{R_j||1}{U[\cdot,j]}$
%%%%%%%%%%%%%%%%%%%%%%%%%%%%%%%%%%%%%%%%%%%%%%%%%%%%%%%%
\vspace{-0.4cm}
\begin{figure}[h!]
	{\footnotesize
		%\centering
		\begin{framed}
			\noindent{$\linearcheckTwoD(\mathsf{pp},A\in \mc{M}_{M,N},b\in \FF^M,[\pi];\ewit)$}: 			%\pnote{why $\wit$ is not part of the witness of linear check protocol?}
			
			\noindent{\bf Relation}: $\ewit=\open(\pi)\wedge A\wit=b$ for $\wit=\dec(\ewit)$.
			
			\begin{enumerate}[{\rm 1.}]
				\item $\verifier\rightarrow\prover$: $r\sample \FF^M$.
				\item $\prover$ and $\verifier$ compute: Polynomials $R^i$, $i\in [p]$ interpolating $R=r^TA$ as in Section ~\ref{sec:lincheck2D}. 
				%\item $\prover$ (a) computes matrix $P$ from $R$ and $\ewit$ as described in Section ~\ref{sec:lincheck}, (b) samples $P_0\sample \FF^{2m-1}$, $\omega_0\sample \FF$ and computes $c_0\gets \comm(P_0,\omega_0)$, (c) computes $(c_1,\ldots,c_{s+\ell-1}) \gets \pccom(P)$.
				\item $\prover$ computes the polynomial $p(\cdot)$ described in Section ~\ref{sec:lincheck2D}. then sends  to the verifier 
				\item $\prover\rightarrow\verifier$: $p_{lin}(\cdot)$.
				\item $\verifier\rightarrow\prover$: $Q=\{j_u:u\in [t]\}$ for $j_u\sample [n]$ for $u\in [t]$.
				\item $\verifier\rightarrow\pi$: $\{j_u:u\in [t]\}$.
				%\item $\prover\rightarrow\verifier$: $\ewit[\cdot,j_u,k_u]$ for $u\in [t]$.
				\item $\pi\rightarrow\verifier$: $\pi[j_u]$ for $u\in [t]$.
				%\item $\verifier\rightarrow\prover$: $\delta\sample \FF^p$, $\beta\sample \FF\backslash \{0\}$. 
				\item $\prover$ and $\verifier$ run inner product arguments to check:
				\begin{enumerate}
					\item $\innerproduct(\mathsf{pp}, R_{j_u}||1, 
					\pi[j_u], p_{lin(\eta_{j_u})}; U[\cdot,j_u])$
				\end{enumerate}
				\item $\verifier$ checks:
				 $\sum_{j\in[s]} p_{lin}(\zeta_j) = r^Tb$
				\item $\verifier$ accepts if all the checks succeed.
			\end{enumerate}
		\end{framed}
		\vspace{-0.4cm}
		\caption{Linear Check Protocol}
		\label{fig:linearcheck2D}
	}
\end{figure}

The correctness of the above protocol follows from the correctness of \cite{ligero} and the inner product argument. We now prove the proof of knowledge of the protocol through the following lemma.
\begin{lemma}
	For a $P^*$ which makes a verifier accept the above linear check protocol, there is an expected $\ppt$ extractor $\extr$ with rewinding access to $P^*$ which outputs a valid witness or breaks the binding of the commitment scheme with overwhelming probability.
\end{lemma}
.\dnote{We should probably include a defn of proof of knowledge in the prelims. Even if we extend this lemma to witness extended emulation, we do not have a single prover definition for WEE in the paper.}
\begin{proof}
	Let $\extr_{\ip}$ be the $\ppt$ extractor for the inner product argument used. For a transcript, the prover's messages include the polynomial $p(\cdot)$ and its messages during the inner-product argument of $\innp{R_{j_u}}{\ewit[\cdot,j_u]} = p(\eta_{j_u})$.
	$\extr$ would use $\extr_\ip$ with the commitment $\comoracle[j_u]$ to obtain $\ewit[\cdot,j_u]$ in expected polynomial time. 
	
	To obtain a valid witness for the linear check protocol, $\extr$ rewinds the prover to Step 4 after obtaining an accepting transcript. $\extr$ adds the set of $j_u$ indices obtained to a set $S$ and repeats the rewinding process till $|S| = n$. 
	There are two possibilities here for the obtained matrix $U$:
	\begin{itemize}
		\item if $d(U, L^m) > e$, $\extr$ outputs $U$. (The proximity check would have verified $d(U,L^m) \leq e$, and hence a $U$ otherwise would mean $\extr$ outputs a collision to the commitment scheme used).
		\dnote{should we define a valid witness as one with distance less than $e$? Else, it seems like we can't have an independent proof and we have to bring in proximity check verifying $d<e$.}
		\item if $d(U, L^m) \leq e$, the soundness analysis for the linear check in \cite{ligero} ensures that $U$ decodes to $x$ such that $A\wit=b$ with overwhelming probability. %and hence $\com(\calU^*)=\cm$.
	\end{itemize}
	An analysis similar to the one in the proof of Lemma~\ref{lem:proximity} proves that $\extr$ can attain $|S| = n$ in an expected polynomial time.
\end{proof}
\begin{comment}
%Linear check
%---------------------------
\subsection{Linear Check with Commitment Oracle}\label{subsec:lincheck2D}
The linear check $A\wit=b$ can be reduced to checking $\innp{r^TA}{\wit}=r^Tb$, where the verifier samples a random $r\sample \bbF^{m\ell}$ and sends it to the prover. As in \cite{ligero}, the prover and verifer first compute $R=r^TA$, then write $R$ as $(R_{1}|\cdots|R_{m})$ where each $R_i\in \bbF^\ell$. Both the prover and the verifier also compute degree $<\ell$ polynomials $R_{i}$ interpolating the vector $R_{i}$ on $G$. The required check in terms of polynomials can be expressed as:
\begin{equation}\label{eq:lincheck2D}
\sum_{\zeta\in G}\sum_{j\in [m]}
R_{i}(\zeta) \cdot Q_{i}(\zeta) = r^Tb.
\end{equation}
The prover computes the polynomial $p(\cdot)=\sum_{i\in[m]}R_i(\cdot) \cdot Q_i(\cdot)$. This polynomial $p(\cdot)$ of degree $< k+\ell-1$ is sent to the verifier who checks $\sum_{\zeta\in G}p(\zeta)=r^Tb$. The verifier needs to check if the polynomial $p(\cdot)$ is correctly computed from the witness oracle $\comoracle$ to guard against dishonest provers. Following \cite{ligero}, it is enough for the verifier to query the polynomial evaluations at a constant number of locations, say $\eta_{j_1},\ldots,\eta_{j_u}$ and then check that $p(\cdot)$ is consistent at the queried locations. We observe that the verifier only uses these queried values to prove some inner-product relations with other (publicly-known) vectors. Hence, our idea is to make the prover use inner-product arguments to prove the consistency of $p(\cdot)$ to the verifier.

We will now formally describe the linear check for our protocol \name2D{}. It will check that a purported commitment oracle $\comoracle$ encodes witness $\wit$ satisfying the constraint $A\wit=b$ for a public matrix $A$ and $b$ is a public vector. As before, we assume $\wit \in \bbF^{\circsize}$ and $A\in \bbF^{\circsize\times \circsize}$ and
$\circsize = m \ell$ for some positive integers $m$ and $\ell$. We further assume that the prover has RS-encoded oracle $\ewit$ which opens to the commitment $\comoracle$ and is $e$-close to the interleaved code $L^{m}$. The prover and the verifier interact as follows: 

%$\prover$ sets $\cm_x$ as the oracle.
\begin{figure}[h!]
	\centering
	\begin{framed}
		\begin{itemize}
			\item {$\linearcheckTwoD (\FF, \GG, L[n,k,d], m, t, A \in \FF^{m\ell \times m\ell}, b\in \FF^{m\ell}, \bm{g}, [\pi]; \wit)$}:
			
			\item {\bf Relation}: $\exists \wit, \ewit$ s.t. $\ewit = \open(\pi), \ewit = \enc(\wit)$ and $A\wit = b$. 
			
			\item {\bf Oracle Setup}: The prover $\prover$ computes $\ewit = \enc(\wit)$ and $\comoracle = \com(\ewit)$ as in Sections ~\ref{subsec:encode2D} and ~\ref{subsec: commit2D}. The prover sets $\pi := \comoracle$ as the oracle.
		\end{itemize}
		\begin{enumerate}[{\rm 1.}]
			\item $\verifier \rightarrow \prover: $ $\verifier$ picks a random $r\in \bbF^{ml}$ and sends that $r$ to $\prover$.
			
			\item Both $\prover$ and $\verifier$ compute $R=r^TA$, which is a vector of size $m\ell$. Read $R$ in a matrix form where first $l$ elements of $R$ form the first row, next $l$ elements form the second row, similarly $R$ will have $m$ rows. Then they construct polynomials $R_i(\cdot)$ of degree $<l$ such that $R_i(\zeta_j)=R_{ij}$ $\forall i\in [m], j\in [\ell]$. 
			
			\item $\prover \rightarrow \verifier: $  $\prover$ computes a polynomial $p(\cdot)=\sum_{i\in[m]} ( R_i(\cdot)\cdot \hat{f}^x_i(\cdot))$ and sends $p(\cdot)$ to $\verifier$.
			
			\item $\verifier \rightarrow \prover: $ $\verifier$ samples $t$ distinct  indices $j_1,\ldots,j_t$ from the set $[n]$ independently at random and sends the indices to $\prover$.
			
			\item Oracle Queries: $\verifier$ queries the oracle $\pi$ with $\{j_u : u\in [t]\}$.
			
			\item Oracle Answers: The oracle replies with $\pi[j_u], u\in [t]$.
			
			\item $\prover \leftrightarrow \verifier: $ $\prover$ and $\verifier$ run a inner product subprotocol for each $u\in[t]$:
			\begin{itemize}
				\item $\innerproduct(\GG,\bm{g}, R_{j_u}, \pi[j_u], p(\eta_{j_u}); \ewit[\cdot,j_u])$ $\forall u\in [t]$,
				
				Where $R_{j_u} = (R_1( \eta_{j_u}) , \ldots , R_m( \eta_{j_u}))$ and $\ewit[\cdot,j_u]$ denotes the $m$-length vector $(\ewit[1,j_u], \ldots, \ewit[m,j_u])$ and $\pi[j_u] = \com(\ewit[\cdot, j_u])$. $\verifier$ proceeds if the arguments succeed for all $u \in [t]$.
			\end{itemize} 
			
			\item $\verifier$ also checks that $\sum_{j\in[l]} p(\zeta_j)=0$.
			
			\item $\verifier$ accepts if all the above checks are succeed.	  
		\end{enumerate}
	\end{framed}
	\caption{Linear Check for $\name$2D}
\end{figure}
The correctness of the above protocol follows from the correctness of \cite{ligero} and the inner product argument. We now prove the proof of knowledge of the protocol through the following lemma.
\begin{lemma}
	For a $P^*$ which makes a verifier accept the above linear check protocol, there is an expected $\ppt$ extractor $\extr$ with rewinding access to $P^*$ which outputs a valid witness or breaks the binding of the commitment scheme with overwhelming probability.
\end{lemma}
.\dnote{We should probably include a defn of proof of knowledge in the prelims. Even if we extend this lemma to witness extended emulation, we do not have a single prover definition for WEE in the paper.}
\begin{proof}
	Let $\extr_{\ip}$ be the $\ppt$ extractor for the inner product argument used. For a transcript, the prover's messages include the polynomial $p(\cdot)$ and its messages during the inner-product argument of $\innp{R_{j_u}}{\ewit[\cdot,j_u]} = p(\eta_{j_u})$.
	$\extr$ would use $\extr_\ip$ with the commitment $\comoracle[j_u]$ to obtain $\ewit[\cdot,j_u]$ in expected polynomial time. 
	
	To obtain a valid witness for the linear check protocol, $\extr$ rewinds the prover to Step 4 after obtaining an accepting transcript. $\extr$ adds the set of $j_u$ indices obtained to a set $S$ and repeats the rewinding process till $|S| = n$. 
	There are two possibilities here for the obtained matrix $U$:
	\begin{itemize}
		\item if $d(U, L^m) > e$, $\extr$ outputs $U$. (The proximity check would have verified $d(U,L^m) \leq e$, and hence a $U$ otherwise would mean $\extr$ outputs a collision to the commitment scheme used).
		\dnote{should we define a valid witness as one with distance less than $e$? Else, it seems like we can't have an independent proof and we have to bring in proximity check verifying $d<e$.}
		\item if $d(U, L^m) \leq e$, the soundness analysis for the linear check in \cite{ligero} ensures that $U$ decodes to $x$ such that $A\wit=b$ with overwhelming probability. %and hence $\com(\calU^*)=\cm$.
	\end{itemize}
	An analysis similar to the one in the proof of Lemma~\ref{lem:proximity} proves that $\extr$ can attain $|S| = n$ in an expected polynomial time.
\end{proof}
%-------------------------------------------------------------------------------------------------
\begin{comment}
\subsubsection{Quadratic Check}
The quadratic check involves the prover convincing the verifier that $x\circ y=z$ by providing oracle access to the vectors $x,y$ and $z$. Again, we
consider the encoding by parts we discussed for the linear check. We write $x=(x_1|\cdots|x_m)$, $y=(y_1|\cdots|y_m)$ and $z=(z_1|\cdots|z_m)$ and
construct polynomials $\hat{f}^x_i$, $\hat{f}^y_i$ and $\hat{f}^z_i$ for $i\in [m]$ as before. The quadratic check then reduces to showing that
$\hat{f}^x_i(\zeta) \cdot \hat{f}^y_i(\zeta)-\hat{f}^z_i(\zeta)=0$ for all $i\in [m]$ and $\zeta\in G$. With high probability, the above can be checked by
verifier sending a random vector $r\sample \bbF^m$ to the prover, and prover sending the polynomial $\hat{p}=\sum_{i\in [m] } r_i (\hat{f}^x_i \cdot \hat{f}^y_i - \hat{f}^z_i)$ to the verifier. The verifier checks that $\hat{p}(\zeta)=0$ for all $\zeta\in G$. The verifier also checks that $\hat{p}$ is correctly computed from the oracles by querying the oracles at small number of points $\eta_1,\ldots,\eta_q$ and checking that $\hat{p}(\eta_j)=\sum_{i\in [m]}r_i(\hat{f}^x_i(\eta_j)\cdot\hat{f}^y_i(\eta_j)-\hat{f}^z_i(\eta_j))$ for $j\in [q]$.
\end{comment}
%------------------------------------------------------------------
\subsection{Quadratic Check Protocol}\label{sec:quadcheck2D}
We now describe the IPCP which allows a prover to prove knowledge of 
$\wit_x$, $\wit_y$ and $\wit_z$ in $\FF^N$, satisfying $\wit_x\circ \wit_y =
\wit_z$. As before we view $\wit_x, \wit_y$ and $\wit_z$ as $m \times s$ matrices, where $N=ms$. 
As described previously in Sections ~\ref{sec:encode2D} and ~\ref{sec:commit2D}, the prover obtains encodings $\ewit_a \leftarrow \enc(\wit_a)$ and corresponding 
commitments $\comoracle_a \leftarrow \ocom(\ewit_a)$ $\forall a\in \{x,y,z\}$. Here, each of the commitment requires different set of generators. Let $\bm{g}_x, \bm{g}_y, \bm{g}_z$ are the generators for $\ewit_x, \ewit_y, \ewit_z$ respectively and the row corresponding to $p_{blind}(\cdot)$ is attached with $\ewit_y$. The prover
sets up the oracles $\pi := \comoracle_x||\comoracle_y|| \comoracle_z$. For a query $Q$, the verifier is provided the $\pi[j], \pi[j+m], \pi[j+2m]$ i.e., $\comoracle_x[j], \comoracle_y[j], \comoracle_z[j]$ respectively, $k\in Q$. 
Here we consider oracle as consisting of three sub-oracles for simplicity of description. For
better efficiency, $\wit_x$, $\wit_y$ and $\wit_z$ can be encoded together, and the oracle
access can be provided to the combined encoding. We defer this optimization till the 
discussion on concrete efficiency. The protocol appears in Figure~\ref{fig:quadcheck2D}. 

\smallskip

Let $Q^i_x, Q^i_y$ and $Q^i_z$, $i\in[p]$ be the polynomials interpolating the $i$th row of $\wit_x$, $\wit_y$ and $\wit_z$ respectively. Then for vectors $\wit_x, \wit_y$ and $\wit_z$ satisfying $\wit_x \circ \wit_y = \wit_z$, the polynomials $Q^i(\cdot) = Q^i_x(\cdot)\cdot Q^i_y(\cdot)-Q^i_z(\cdot)$ interpolate $\bm{0}^{s}$ on the set $\{\zeta_j:j\in[s]\}$ for all $i\in[m]$. We derive a simple probabilistic check below compressing along the $m$ direction. The verifier sends a random challenge $r\sample \FF^m$ to the prover and then the prover computes $p(\cdot) = \sum_{i\in[m]} r_i\cdot Q^i(\cdot)$. So, the prover sends the polynomial $p(\cdot)$ to the verifier, then verifier can check $p(\zeta_j) = 0$ $\forall j\in[s]$. Additionally, $p(\cdot)$ should satisfy the following identity:
\begin{align*}
	p(\eta_j) &= \sum_{i\in[m]}r_i\cdot[\ewit_x[i,j]\cdot \ewit_y[i,j] - \ewit_z[i,j]] \text{ } \forall j\in[n]\\
	p(\eta_j) &= \innp{r}{\ewit_x[\cdot,j]\circ \ewit_y[\cdot,j] - \ewit_z[\cdot,j]} \text{ } \forall j\in[n]\\
	p(\eta_j) &= \innp{r\circ \ewit_x[\cdot,j] || r } { \ewit_y [\cdot , j ] ||-\ewit_z[\cdot,j]}\text{ } \forall j\in[n]
\end{align*}
To check the above identity, the verifier sends $t$ random queries $\{j_u:u\in[t]\}$ to the oracle and obtains $\comoracle_a[j_u]$, for all $a\in\{x,y,z\}$ and $u\in[t]$. Then the prover and the verifier run inner product arguments to prove the above identity in the chosen points by the verifier. For zero knowledge, $p(\cdot)$ is required to be blinded. For that purpose the prover adds already chosen $p_{blind}(\cdot)$ to $p(\cdot)$. $p_{quad}(\cdot) = p(\cdot) + p_{blind} (\cdot)$, which satisfies the identity $p_{quad}(\zeta_j) = 0 \text{ } j\in[s]$ and $p_{quad}(\eta_j) = \innp{r\circ \ewit_x[\cdot,j]||r|| 1 }{\ewit_y[\cdot,j]||-\ewit_z[\cdot,j]||p_{blind}(\eta_j)}$
%%%%%%%%%%%%%%%%%%%%%%%%%%%%%%%%%%%%%%%%%%%%%%%%%%%%%%%
\begin{figure}[h!]
	{\footnotesize
		%\centering
		\begin{framed}
			\noindent{$\quadcheckTwoD(\mathsf{pp},[\pi]];\ewit_x, \ewit_y, \ewit_z)$}:
			
			\noindent{\bf Relation}: $\ewit_a=\open(\pi_a)$ for $a\in \{x,y,z\}$,
			$\wit_x \circ \wit_y = \wit_z$ where  $\wit_a=\dec(\ewit_a)$ for $a\in \{x,y,z\}$.
			
			\begin{enumerate}[{\rm 1.}]
				%\item $\verifier\rightarrow\prover$: $\rho\sample \FF^{3p}$.
				%\item $\prover$ computes: (a) $\tilde{\ewit}=\sum_{i=1}^p[\rho_i\ewit_x[i,\cdot,\cdot]+
				%\rho_{p+i}\ewit_y[i,\cdot,\cdot]+\rho_{2p+i}\ewit_z[i,\cdot,\cdot]]$, (b)
				%commitments $\tilde{c}_1,\ldots,\tilde{c}_\ell$ as $\tilde{c}_k = \prod_{i=1}^{p} (\pi_x[i,k])^{\rho_i}\cdot (\pi_y[i,k])^{\rho_{p+i}}\cdot(\pi_z[i,k])^{\rho_{2p+i}}$ $\forall k\in[\ell]$.
				%\item $\prover\rightarrow\verifier$: $\tilde{\bm{c}}=(\tilde{c}_1,\ldots,\tilde{c}_\ell)$.
				\item $\verifier\rightarrow\prover$: $r\sample \FF^p$.
				%\item $\prover\leftrightarrow\verifier$ compute: Polynomials $R^i$, $i\in [p]$ interpolating $R=r^TA$ as in Section ~\ref{sec:quadcheck}. 
				\item $\prover$ computes $p_{quad}(\cdot)$ described in Section ~\ref{sec:quadcheck2D}
				%(b) matrix $P$ such that $P[j,k] = p_j(\eta_k)$ as described in Section ~\ref{sec:quadcheck}, (c)
				%computes commitments $c_1,\ldots,c_{2\ell}$ from $P$. $\prover$ also
				%samples $P_0\sample \FF^{2m-1}$ with $P_0[j]=0^m$ for $j\in[m]$ and computes commitment $c_0$
				%to $P_0$. 
				\item $\prover\rightarrow\verifier$: $p_{quad}$
				\item $\verifier\rightarrow\prover$: $Q=\{j_u:u\in [t]\}$ for 	$Q\sample [n]$, $u\in [t]$
				\item $\verifier\rightarrow\pi$: $\{j_u:u\in [t]\}$.
				%\item $\prover\rightarrow\verifier$: $X_u=\ewit_x[\cdot,j_u,k_u]$ , $Y_u=\ewit_y[\cdot,j_u,k_u]$ and $Z_u=\ewit_z[\cdot,j_u,k_u]$ for $u\in [t]$.
				
				\item $\pi\rightarrow\verifier$: $\comoracle_x[j_u],\comoracle_y[j_u], \comoracle_z[j_u]$ for $u\in [t]$.
				%\item $\verifier\rightarrow\prover$: $\delta\sample
				%\FF^p$, $\beta_x\sample \FF$, $\beta_y\sample \FF$, $\beta_z\sample \FF$,
				%$\beta\sample \FF\backslash \{0\}$.
				\item $\prover$ and $\verifier$ compute:
				commitment of $r\circ \ewit_x[\cdot,j_u]||r||1, \ewit_y[\cdot,j_u]||-\ewit_z[\cdot,j_u]||p_{blind}(\eta_{j_u})$ by appropriately adjusting the generators.
				%\pnote{Mention that committing to x,y,z need different set of generators}
				Let $\bm{g}_1$ and $\bm{g}_2$ be the corresponding generators and $\bm{c}_1, \bm{c}_2$ are the commitment values.
				Then update $\mathsf{pp}$ by including $\bm{g}_1, \bm{g}_2$
				%$W_u=\sum_{i=1}^p\delta_i\big(\beta_x\ewit_x[i,\cdot,k_u]+\beta_y\ewit_y[i,\cdot,k_u]+\beta_z\ewit[i,\cdot,k_u]\big)$.
				%$\beta \FF\backslash \{0\}$. 
				%\item $\prover \text{ and }\verifier$ compute:
				%$V_u=\beta_xX_u+\beta_yY_u+\beta_zZ_u$ for $u\in [t]$.
				%$T_u=(C_u)^{\beta_x}\cdot(D_u)^{\beta_y}\cdot(E_u)^{\beta_z}$, for $u\in [t]$ where
				%$C_u=\prod_{i=1}^{p}(\pi_x[i,k_u])^{\delta_i}$, $D_u=\prod_{i=1}^{p}(\pi_y[i,k_u])^{\delta_i}$
				%and $E_u=\prod_{i=1}^{p}(\pi_y[i,k_u])^{\delta_i}$.
				\item $\prover$ and $\verifier$ run inner-product arguments to check:
				\begin{enumerate}
					\item
					$\innerproduct(\mathsf{pp},\bm{c}_1,\bm{c}_2, p(\eta_{j_u}); r\circ \ewit_x[\cdot,j_u]||r||1, \ewit_y[\cdot,j_u]||-\ewit_z[\cdot,j_u]||p_{blind}(\eta_{j_u}))$
					% $\innerproduct(\mathsf{pp},\bm{1}_{j_u}^T\Lambda_{h,2m-1},\mathsf{cm}_{k_u},v_u;\overline{P}[\cdot,k_u])$ for $u\in [t]$ where $\mathsf{cm}_{k_u}=\prod_{a=1}^{2\ell-1}(c_a)^{\Lambda_{n,2\ell-1}^T[a,k_u]}$, 
					%v_u=\sum_{i=1}^p r_i[X_u[i]\cdot Y_u[i] - Z_u[i]]$ (check consistency of $P$ with $\pi$).
					%\item $\innerproduct(\mathsf{pp},\gamma||0^{m-1},\mathsf{cm},0;z)$
					%where $z=\beta P_0 +\overline{P}\varphi$, $\varphi = \Phi^T\tau$ and
					%\mathsf{cm} =  (c_0)^{\beta}\cdot\prod_{a=1}^{2\ell-1} (c_a)^{\varphi_a}$ %(check the condition $r^TAw = r^Tb$).
					%\item
					%\innerproduct(\mathsf{pp},\bm{1}_{j_u}^T\Lambda_{h,m},T_u,\innp{\delta}{T_u};\overline{W}_u])$, where $\overline{W}_u$ stands for first $m$ entries of $W_u$ (consistency of $X_u, Y_u, Z_u$ with $\pi$). 
				\end{enumerate}
				\item $\verifier$ checks if $p_{quad}(\zeta_{j})=0$ $\forall j\in[s]$
				%\item $\verifier$ checks proximity of $\ewit_x,\ewit_y$
				%and $\ewit_z$ according to Eqn \eqref{eq:combinedproximity}.
				\item $\verifier$ accepts if all the checks succeed.
				\pnote{2 different types of inner products we are using.}
			\end{enumerate}
		\end{framed}
		\caption{Quadratic Check Protocol}
		\label{fig:quadcheck2D}
	}
\end{figure}
%%%%%%%%%%%%%%%%%%%%%%%%%%%%%%%%%%%%%%%%%%%%%%%%%%%%%%%
\begin{comment}
\begin{figure}[h!]
	\centering
	\begin{framed}
		\begin{itemize}
			\item {$\quadcheckTwoD (\mathsf{pp},[\pi]; \ewit_x, \ewit_y, \ewit_z)$}:
			\item {\bf Relation}: $\exists (\wit_x, \wit_y, \wit_z, \ewit_x, \ewit_y, \ewit_z)$ such that $(\ewit_x, \ewit_y, \ewit_z) = \open(\pi)$ and $\ewit_a = \enc(\wit_a)$ for $a\in \{x,y,z\}$ and $\wit_x \circ \wit_y = \wit_z$.
			\item {\bf Oracle Setup}: The prover $\prover$ computes $\ewit_a = \enc(\wit_a)$ and $\comoracle_a = \com(\ewit_a)$ for $a\in \{x,y,z\}$. It sets $\pi := [\comoracle_{x} || \comoracle_{y} || \comoracle_{z}]$ where the notation denotes vertical stacking of the vectors. Note that the generators used to construct commitment vectors for $\wit_x, \wit_y$ and $\wit_z$ should be independent. So, $\prover$ and $\verifier$ together generates $\bm{g}_x$, $\bm{g}_y$ and $\bm{g}_z$.
		\end{itemize}
		\begin{enumerate}
			\item $\verifier \rightarrow \prover: $ $\verifier$ picks $r$ uniformly at random from $\FF^{m}$ and sends it to $\prover$.
			
			\item $\prover \rightarrow \verifier: $ $\prover$ computes the polynomial $p(\cdot)= \sum_{i\in [m]} [r_i\cdot (Q^x_i(\cdot)\cdot Q^y_i(\cdot) - Q^z_i(\cdot))] $ and sends $p(\cdot)$ to $\verifier$. 
			
			\item $\verifier \rightarrow \prover: $ $\verifier$ sends $t$ randomly sampled indices $Q=\{j_u\}_{u\in[t]}$ from $[n]$.
			
			\item Oracle Queries: $\verifier$ queries the oracle $\pi$ with $\{j_u : u\in [t]\}$.
			\item Oracle Answers: The oracle responds with $\pi[j_u], u\in[t]$.
			\item $\prover \leftrightarrow \verifier: $ $\prover$ and $\verifier$ together generates $\bm{g}$ independent of all the generators used in committing $x,y,z$. Both run a inner product subprotocol for each $u\in[t]$:
			\begin{itemize}
				\item $\innerproduct(\GG,\bm{g}'||\bm{g}_r, \bm{g}_y||(\bm{g}_z)^{-1}, \comoracle_x[j_u]\cdot (\bm{g}_r)^r, p(\eta_{j_u}), \comoracle_y\cdot (\comoracle_z)^{-1}; r\circ\ewit_x[\cdot,j_u]||r, \ewit_y[\cdot,j_u]||-\ewit_z[\cdot,j_u])$ $\forall u\in [t]$,
				
				Where $\bm{g}'= (g'_1, \ldots, g'_m)$ is such that $g'_i = g_x^{r^{-1}}$ 
				$\verifier$ proceeds if the arguments succeed for all $u \in [t]$.
			\end{itemize} 
			%.\pnote{run a subprotocol to prove the innerproduct arguement for the following statement $\innp{r\circ \ewit_x[\cdot,j_u]}{\ewit_y[\cdot,j_u]} - \innp{r}{\ewit_z[\cdot,j_u]} = p(\eta_{j_u})$ $\forall u\in[t]$.} 
			
			\item $\verifier$ checks if $p(\zeta_j)=0$ $\forall j\in[l]$. If yes then accepts, else rejects.
		\end{enumerate}
	\end{framed}
	\caption{Quadratic Check for $\name$2D}
\end{figure}
\begin{comment}
%Quadratic check
%---------------------------------
\subsection{Quadratic Check}\label{subsec:quadcheck2D}
We now formally describe the interactive oracle protocol for checking the relation $x\circ y = z$ for vectors $x,y,z\in \bbF^\circsize$. Let $\ewit_x, \ewit_y$ and $\ewit_z$ denote the encodings of vectors $x$, $y$ and $z$ respectively via the RS code mentioned in Section ~\ref{subsec:encode2D}. Let $\comoracle_x,\comoracle_y$ and $\comoracle_z$ denote the respective commitment oracles. In brief, as in the linear check, it follows \cite{ligero} except to check whether $p(\cdot)$ is correctly computed from the oracles. When the verifier queries the oracles at small number of points $\eta_{j_1} , \ldots , \eta_{j_t}$, the prover and the verifier would involve in an inner-product argument for the verifier to verify that $p(\eta_{j_u})=\sum_{i\in [m]}r_i[Q^x_i(\eta_{j_u}) \cdot Q^y_i(\eta_{j_u})-Q^z_i(\eta_{j_u}))$ for $u\in [t]$.
Which can be viewed as:
\begin{align*}
&\innp{r}{(\ewit_x[\cdot,j_u]\cdot \ewit_y[\cdot,j_u] - \ewit_z[\cdot,j_u])} = p(\eta_{j_u}) \\
\Rightarrow &\innp{r}{(\ewit_x[\cdot,j_u]\cdot \ewit_y[\cdot,j_u])} + \innp{r}{-\ewit_z[\cdot,j_u]} = p(\eta_{j_u})
\end{align*}
%------------------------------------------------------------------------
There are two inner products in the above statement and the prover does not want to reveal the values of the individual inner products. Hence, \name2D combine them into a single inner product relation.

For each $u\in[t]$, $\prover$ runs the following inner product arguement with $\verifier$:
$$\innp{(r\circ \ewit_x[\cdot,j_u]||r)}{(\ewit_y[\cdot,j_u]||-\ewit_z[\cdot,j_u])} = p(\eta_{j_u})$$
To facilitate this, $\ewit_x, \ewit_y$ and $\ewit_z$ should have been committed with different independently chosen sets of generators. And, the set of generators used for committing $r$ (to be concatenated with $r \circ \ewit_x [\cdot, j_u])$ should also be independent of the above three sets of generators. $\verifier$ proceeds if the arguements succeed for all $u\in[t]$.
%-----------------------------------------------------------------------

%	$\prover$ sets $\cm_x, \cm_y$ and $\cm_z$ as the oracles.


%----------------------------------------------------------------------------------------------------------------------------
\begin{comment}
\begin{figure}[h!]
\centering
\begin{framed}
\begin{itemize}
\item {$\quadcheckTwoD (\FF, \GG, L, [\pi]; \wit_x, \wit_y, \wit_z)$}:
\item {\bf Relation}: $\exists (\wit_x, \wit_y, \wit_z, \ewit_x, \ewit_y, \ewit_z)$ such that $(\ewit_x, \ewit_y, \ewit_z) = \open(\pi)$ and $\ewit_a = \enc(\wit_a)$ for $a\in \{x,y,z\}$ and $\wit_x \circ \wit_y = \wit_z$.
\item {\bf Oracle Setup}: The prover $\prover$ computes $\ewit_a = \enc(\wit_a)$ and $\comoracle_{xy} = \com(\ewit_x \circ \ewit_y)$, $\comoracle_z = \com(\ewit_z)$. It sets $\pi :=[\comoracle_{xy} || \comoracle_{z}]$ where the notation denotes vertical stacking of the vectors that means $\pi$ has $n$ columns and 2 rows such that $\pi[1,j] = \comoracle_{xy}[j]$ and $\pi[2,j] = \comoracle_{z}[j]$ $\forall j\in[n]$.
\end{itemize}
\begin{enumerate}
\item $\verifier \rightarrow \prover: $ $\verifier$ picks $r$ uniformly at random from $\FF^{m}$ and sends it to $\prover$.

\item $\prover \rightarrow \verifier: $ $\prover$ computes the polynomial $p(\cdot)= \sum_{i\in [m]} [r_i\cdot (\hat{f}^x_i(\cdot)\cdot \hat{f}^y_i(\cdot) - \hat{f}^z_i(\cdot))] $ and sends $p(\cdot)$ to $\verifier$. 

\item $\verifier \rightarrow \prover: $ $\verifier$ sends $t$ randomly sampled indices $Q=\{j_u\}_{u\in[t]}$ from $[n]$.

\item Oracle Queries: $\verifier$ queries the oracle $\pi$ with $\{j_u : u\in [t]\}$.
\item Oracle Answers: The oracle responds with $\pi[j_u], u\in[t]$.
\item $\prover \leftrightarrow \verifier: $ $\prover$ and $\verifier$ run a inner product subprotocol for each $u\in[t]$:
\begin{itemize}
\item $\innerproduct(\GG,\bm{g}, r, \pi[1,j_u]\cdot \pi[2,j_u]^{-1}, p(\eta_{j_u}); \ewit_x[\cdot,j_u] \circ \ewit_y[\cdot, j_u], \ewit_z[\cdot,j_u])$ $\forall u\in [t]$,
%Where $R_{j_u} = (R_1( \eta_{j_u}) , \ldots , R_m( \eta_{j_u}))$ and $\ewit[\cdot,j_u]$ denotes the $m$-length vector $(\ewit[1,j_u], \ldots, \ewit[m,j_u])$ and $\pi[j_u] = \com(\ewit[\cdot, j_u])$.

$\verifier$ proceeds if the arguments succeed for all $u \in [t]$.
\end{itemize} 
\begin{comment}
.\pnote{run a subprotocol to prove the innerproduct arguement for the following statement $\innp{r\circ \ewit_x[\cdot,j_u]}{\ewit_y[\cdot,j_u]} - \innp{r}{\ewit_z[\cdot,j_u]} = p(\eta_{j_u})$ $\forall u\in[t]$.} 

There are two inner products in the above statement and the prover does not want to reveal the values of the individual inner products. Hence, \name2D combine them into a single inner product relation.

For each $u\in[t]$, $\prover$ runs the following inner product arguement with $\verifier$:
$$\innp{(r\circ \ewit_x[\cdot,j_u]||r)}{(\ewit_y[\cdot,j_u]||-\ewit_z[\cdot,j_u])} = p(\eta_{j_u})$$
To facilitate this, $\ewit_x, \ewit_y$ and $\ewit_z$ should have been committed with different independently chosen sets of generators. And, the set of generators used for committing $r$ (to be concatenated with $r \circ \ewit_x [\cdot, j_u])$ should also be independent of the above three sets of generators. $\verifier$ proceeds if the arguements succeed for all $u\in[t]$.

\item $\verifier$ checks if $p(\zeta_j)=0$ $\forall j\in[l]$. If yes then accepts, else rejects.
\end{enumerate}
\end{framed}
\caption{Quadratic Check for $\name$2D}
\end{figure}
\end{comment}
%----------------------------------------------------------------------------------------------------------------------------

The correctness of the above protocol again follows from the correctness of \cite{ligero} and the inner product argument. Its proof of knowledge property is proved through the following lemma.
\begin{lemma}
	For a $P^*$ which makes a verifier accept the quadratic check protocol, there is an expected $\ppt$ extractor $\extr$ with rewinding access to $P^*$ which outputs valid witnesses $\ewit_x$, $\ewit_y$ and $\ewit_z$ with overwhelming probability.
\end{lemma}
\begin{proof}
	Let $\extr_{\ip}$ be the $\ppt$ extractor for the inner product argument used. For a transcript, the prover's messages include the polynomial $p(\cdot)$ and its messages during the inner-product argument.
	$\extr$ would use $\extr_\ip$ to obtain $\left( (r \circ \ewit_x[\cdot,j_u] \, || \, r \right)$ and $\left( \ewit_y[\cdot,j_u] \, || \, -\ewit_z[\cdot,j_u] \right)$ in expected polynomial time. $\extr_\ip$ would use the commitments derived from $\comoracle_x[j_u]$, $\comoracle_y[j_u]$ and $\comoracle_z[j_u]$. From this output of $\extr_{\ip}$, $\extr$ can obtain $\ewit_x[\cdot,j_u]$, $\ewit_y[\cdot,j_u]$ and $\ewit_z[\cdot,j_u]$.
	
	To obtain a set of valid witnesses for the quadratic check protocol, $\extr$ rewinds the prover to Step 3 after obtaining an accepting transcript. $\extr$ adds the set of $j_u$ indices obtained to a set $S$ and repeats the rewinding process till $|S| = n$. At this point, $\extr$ has the complete $\ewit_x$, $\ewit_y$ and $\ewit_z$. There are two possibilities here:
	\begin{itemize}
		\item if $d(\ewit_\delta,L^m) > e$ for each $\delta = \{x, y, z\}$, $\extr$ outputs $\ewit_\delta$. (The proximity check would have verified $d(\ewit,L^m) \leq e$, and hence a $\ewit$ otherwise would mean $\extr$ outputs a collision to the commitment scheme used).
		\item if $d(\ewit_\delta,L^m) \leq e$ for each $\delta = \{x, y, z\}$, the soundness analysis for the quadratic check in \cite{ligero} ensures that $\ewit_\delta$ decodes to $\wit_{\delta}$ such that $\wit_x\circ \wit_y = \wit_z$ with overwhelming probability.
	\end{itemize}
	An analysis similar to the one in the proof of Lemma~\ref{lem:proximity} proves that $\extr$ can attain $|S| = n$ in an expected polynomial time.
\end{proof}
%---------------------------------------------------------------------------------------------------
\begin{comment}
\subsubsection{Proximity Test}
The correctness of the previous two checks, namely the linear check and the
quadratic check rely on the fact that the witness oracles are evaluations of
``low'' degree polynomials. There are several known low degree tests for polynomials
from PCP literature. The protocol in \cite{aurora} uses a recent test for
proximity by Ben-Sasson et al.\cite{IOPP_FRI2018} with particularly efficient
prover and $O(\log d)$ query complexity for polynomials of degree at most $d$. 
We use a variant of proximity test from \cite{ligero}, adapting it to work
with homomorphic commitments of the RS-encoded oracles and reducing the query
complexity.
\end{comment}
%---------------------------------------------------------------------------------------------------
%Proximity Check
%-----------------------------
\subsection{Proximity Protocol}\label{subsec:proximity2D}
The correctness of the previous two checks, namely the linear check and the
quadratic check rely on the fact that the witness oracles are evaluations of
``low'' degree polynomials. There are several known low degree tests for polynomials
from PCP literature. The protocol in \cite{aurora} uses a recent test for
proximity by Ben-Sasson et al.\cite{IOPP_FRI2018} with particularly efficient
prover and $O(\log d)$ query complexity for polynomials of degree at most $d$. 
We use a variant of proximity test from \cite{ligero}, adapting it to work
with homomorphic commitments of the RS-encoded oracles and reducing the query
complexity.

We finally describe the IPCP which allows to prove that if $\ewit$ is committed to $\pi$, then the $\ewit$ is an interleaved codeword. Our protocol for ``proximity'' of a purported codeword to the interleaved code. Let $\ewit\in \bbF^{m\times n}$ denote the purported codeword and let $e< d/3$ denote the proximity parameter. The commitment oracle $\comoracle$ corresponds to $\ewit$.
\dnote{To Nitin: why did you not use the notation $\rsoracle$ for $U$? Is it that only those $\ewit$s which satisfy the proximity test become $\rsoracle$?}
The prover and the verifier interact as follows:
%$\prover$ sets $\cm$ as the oracle.
verifier needs to check that each row of $\ewit$ is in $\rsc{\eta}{n,\ell}$. If $\ewit \notin \rsc{\eta}{n,\ell}$, then a random linear combination of the rows of $\ewit$ is not $\rsc{\eta}{n,\ell}$ with very high probability. So, the verifier sends a random challenge $\gamma\sample \FF^m$ to the prover. The prover sends $w=\gamma^T\ewit$. The verifier checks if $w\in \rsc{\eta}{n,\ell}$. The verifier sends $t$ queries $Q=\{j_u:u\in[t]\}$ to the oracle and gets the commitment of the $j_u$th column of $\ewit$, for $u\in[t]$. Additionally, $w$ should satisfy the following identity:
\begin{align*}
	w_j=\innp{\gamma}{\ewit[j]} \text{ } \forall j\in[n]
\end{align*}
The prover and the verifier run an inner product to prove that the above identity holds for $\{j_u:u\in[t]\}$. For zero knowledge, the prover inludes another row containing random codeword. Then $w=\gamma^T\ewit + U[m+1,\cdot]$ and the inner product identity becomes:
\begin{align*}
w_j=\innp{\gamma||1}{\ewit[j]} \text{ } \forall j\in[n]
\end{align*}
%%%%%%%%%%%%%%%%%%%%%%%%%%%%%%%%%%%%
%%%%%%%%%%%%%%%%%%%%%%%%%%%%%%%%%%%%%%%%%%%%%%%%%%%%%%%%
\vspace{-0.4cm}
\begin{figure}[h!]
	{\footnotesize
		%\centering
		\begin{framed}
			\noindent{$\proxcheckTwoD(\mathsf{pp},[\pi];\ewit)$}: 			%\pnote{why $\wit$ is not part of the witness of linear check protocol?}
			
			\noindent{\bf Relation}: $\ewit=\open(\pi)\wedge \ewit\in (\rsc{\eta}{n,s})^m$.
			
			\begin{enumerate}[{\rm 1.}]
				\item $\verifier\rightarrow\prover$: $\gamma \sample \FF^m$.
				\item $\prover \rightarrow \verifier$ $\prover$ computes $w=\gamma^T\ewit$ and sends $w$ to $\verifier$ in Section ~\ref{sec:lincheck2D}. 
				%\item $\prover$ (a) computes matrix $P$ from $R$ and $\ewit$ as described in Section ~\ref{sec:lincheck}, (b) samples $P_0\sample \FF^{2m-1}$, $\omega_0\sample \FF$ and computes $c_0\gets \comm(P_0,\omega_0)$, (c) computes $(c_1,\ldots,c_{s+\ell-1}) \gets \pccom(P)$.
				%\item $\prover$ computes the polynomial $p(\cdot)$ described in Section ~\ref{sec:lincheck2D}. then sends  to the verifier 
				%\item $\prover\rightarrow\verifier$: $p_{lin}(\cdot)$.
				\item $\verifier\rightarrow\prover$: $Q=\{j_u:u\in [t]\}$ for $j_u\sample [n]$ for $u\in [t]$.
				\item $\verifier\rightarrow\pi$: $\{j_u:u\in [t]\}$.
				%\item $\prover\rightarrow\verifier$: $\ewit[\cdot,j_u,k_u]$ for $u\in [t]$.
				\item $\pi\rightarrow\verifier$: $\pi[j_u]$ for $u\in [t]$.
				%\item $\verifier\rightarrow\prover$: $\delta\sample \FF^p$, $\beta\sample \FF\backslash \{0\}$. 
				\item $\prover$ and $\verifier$ run inner product argument to check:
				\begin{enumerate}
					\item $\innerproduct(\mathsf{pp}, \gamma||1, \pi[j_u], w_{j_u}); U[\cdot,j_u])$
				\end{enumerate}
				\item $\verifier$ checks:
				$w\in \rsc{\eta}{n,s}$
				\item $\verifier$ accepts if all the checks succeed.
			\end{enumerate}
		\end{framed}
		\vspace{-0.4cm}
		\caption{Proximity Check Protocol}
		\label{fig:proximitycheck2D}
	}
\end{figure}
%%%%%%%%%%%%%%%%%%%%%%%%%%%%%%%%%%%%
\begin{comment}
\begin{figure}[h!]
	\centering
	\begin{framed}
		\begin{itemize}
			\item {$\proxcheckTwoD(\FF, \GG, L[n,k,d], m, t, \bm{g}, [\pi]; \ewit)$}:
			\item {\bf Relation}: $\ewit = \open(\pi)$, $\ewit \in L^m$.
			\item {\bf Oracle Setup}: Prover computes $\comoracle$ from $\ewit$ as in Section ~\ref{subsec: commit2D} and sets $\pi := \comoracle$ as the oracle.
		\end{itemize}
		\begin{enumerate}
			\item $\verifier \rightarrow \prover :$ $\verifier$ as a challenge picks $\gamma \in \bbF^m$ uniformly at random and sends it to $\prover$.
			
			\item $\prover \rightarrow \verifier :$ $\prover$ computes $\w=\gamma^T\ewit$ and sends $\w$ to $\verifier$.
			
			\item $\verifier \rightarrow \prover :$ $\verifier$ picks a random subset $Q\subseteq [n]$ such that $|Q|=t$ and sends $Q$ to $\prover$.
			
			\item Oracle Queries: $\verifier$ queries the oracle $\pi$ with $\{j_u:u\in[t]\}$.
			
			\item Oracle Answers: The oracle responds with $\pi[j_u], u\in[t]$.
			
			\item $\prover \leftrightarrow \verifier: $ $\prover$ and $\verifier$ run a inner product subprotocol for each $u\in[t]$:
			\begin{itemize}
				\item $\innerproduct(\GG, \bm{g}, \gamma, \pi[j_u], \w_{j_u}; \ewit[\cdot,j_u])$ $\forall u\in [t]$.
			\end{itemize}
			%run a subprotocol to prove the innerproduct arguement for the following statement $\innp{\gamma}{\ewit_x[\cdot,j_u]}=u_{j_u}$ $\forall j_u\in Q$
			
			\item If $\verifier$ accepts the innerproduct arguement in the previous step, then checks if $\w\in L$. If yes then $\verifier$ outputs accept else outputs reject.
		\end{enumerate}
	\end{framed}
	\caption{Proximity Check for $\name$2D}
\end{figure}
\end{comment}


The correctness of the above protocol follows from \cite{ligero} and the inner product argument. The following lemma captures its proof of knowledge property.
\begin{lemma}\label{lem:proximity}
	For a $P^*$ which makes a verifier accept the proximity protocol, there is an expected $\ppt$ extractor $\extr$ with rewinding access to $P^*$ which outputs a valid witness or breaks the binding of the commitment scheme with overwhelming probability.
\end{lemma}
\begin{proof}
	Let $\extr_{\ip}$ be the $\ppt$ extractor for the inner product argument used. For a transcript, the prover's messages include the polynomial $p$ and its messages during the inner-product argument of $\innp{\gamma}{U[\cdot, j_u]} = w_{j_u}$.
	$\extr$ would use $\extr_\ip$ with the commitment $\comoracle[j_u]$ to obtain $U[\cdot,j_u]$ in expected polynomial time. 
	
	To obtain a valid witness for the proximity protocol, $\extr$ rewinds the prover to Step 3 after obtaining an accepting transcript. $\extr$ adds the set of $j_u$ indices obtained to a set $S$ and repeats the rewinding process till $|S| = n$. Let the obtained matrix be $U$.
	The soundness analysis for the proximity check in \cite{ligero} ensures that  $d(U, L^m) \leq e$ with overwhelming probability. 
	
	We will now estimate the number of rewindings required for $\extr$ to reach $|S| = n$. We will provide a crude upper bound on this to show that $\extr$ runs in expected polynomial time. To start with, we consider $t=1$. A larger $t$ will only reduce the number of rewindings required. Let $X_i$ be the discrete random variable that represents the number of rewindings to improve from $|S| = i-1$ to $|S| = i$ i.e., to pick a column not in $S$ when $|S| = i-1$. The base case $X_1=1$ because the first column selected will always be distinct. When $|S| = i-1$, there are $n-i+1$ columns remaining, the probability of selecting one of them is $(n-i+1)/n$. Since $X_i$ follows the geometric distribution, 
	\[
	E[X_i] = 1/ [(n-i+1)/n] = n/ (n-i+1)
	\]
	Let $X$ be the random variable for the number of rewindings to reach $|S| = n$.
	Following linearity of expectations, 
	\[
	E[X] = \sum_{i \in [n]} E[X_i] = \sum_{i \in [n]} n/ (n-i+1) = n \sum_{i \in [n]} 1/i = \theta (n \log n) 
	\]
	Thus, $\extr$ extracts the witness in an expected polynomial time.
\end{proof}
%--------------------------------------------------------------------------------------------------
\begin{comment} 
\section{\name2D{} - ZK arguments with $O(\circsize^{1/c})$ proof size and sublinear public key operations for verifier}
We will now describe \name2D{}. In brief, \name2D{} will make use of homomorphic commitments over the RS-encoded witness oracles.
%We will commit to the vector $\{\hat{f}^x_i(\eta)\}_{i \in [m]}$ using the homomorphic vector commitments. The vectors corresponding to each $\eta$ will be committed separately. Inner product arguments will then be used to remove the linear dependence of the proof size on the parameter $m$. 
And, instead of ``opening'' vectors from these oracles to the verifier, the prover in \name2D{} will provide an inner product argument on the vectors.
Using the inner-product arguments of \cite{InnerProductDLS, bulletproofs}, \name2D{} achieves the proof size of $O(\circsize^{1/c})$ for any $c \geq 2$ with $O(\circsize^{1-1/c})$ public key operations and $O(\circsize)$ overall complexity for the verifier. 

\subsection{Encoding}\label{subsec:encode2D} 
Let $\extwit \in \bbF^{|\C|}$ be the witness vector. Let $m$ and $\ell$ be integers such that $m\ell\geq \circsize$. We choose ordered domains $G=\{\zeta_1,\ldots,\zeta_\ell\}$ and $H=\{\eta_1,\ldots,\eta_n\}$. We then write the vectors $\extwit$ as $\extwit = (\extwit_{1},\ldots,\extwit_{m})$ where each $\extwit_{i}\in \bbF^\ell$ for $i \in [m]$. Construct $Q_i(\cdot)$, a polynomial of degree $<k$, by interpolating the vector $\extwit_{i}$ on $G$ and $Q_{i}$ denotes the corresponding evaluation of $Q_{i}$ on $H$. We define the RS-encoded witness $\ewit\in \bbF^{m\times n}$ as $\ewit[i,j]=Q_{i}(\eta_j)$ for $ i\in [m]$ and $j\in [n]$. We now construct a commitment oracle $\comoracle$ from $\ewit$.
\dnote{do we use the notation $f^x_i$?}
%If $|x|=ml$ then read $x$ as 
%$$x=
%\begin{bmatrix}
%x_{11} & x_{12} & \ldots & x_{1l}\\
%x_{21} & x_{22} & \ldots & x_{2l}\\
%& \vdots\\
%x_{m1} & x_{m2} & \ldots & x_{ml}
%\end{bmatrix}
%$$	
%Construct polynomials $\hat{f}^x_i(\cdot)$ of deg $k$ such that $\hat{f}^x_i(\zeta_j)=x_{ij}$ $\forall i\in [m], j\in [l]$ where $k>l$ and $l+t=k$.
%
%Define 
%$$ \ewit =
%\begin{bmatrix}
%u_{11} & u_{12} & \ldots & u_{1n}\\
%u_{21} & u_{22} & \ldots & u_{2n}\\
%& \vdots\\
%u_{m1} & u_{m2} & \ldots & u_{mn}
%\end{bmatrix}
%$$
%where $u_{ij}= \hat{f}^x_i(\eta_j)$ $\forall i\in[m], j\in[n]$ $n>k$. $\bm{\zeta}=\{\zeta_1,\ldots,\zeta_l\}$ we will call it interpolation domain and $\bm{\eta} = \{\eta_1,\ldots,\eta_n\}$ we will call it evaluation domain. 
%
%Let $L$ be the set of codewords and the above linear code has distance $d$. Then a correctly computed $\ewit_x$ is in $L^m$.
 
\subsection{Oracle Construction}\label{subsec: commit2D}
Throughout, we assume $\bbF$ is a prime field. Let $\com$ denote the Pedersen vector commitment scheme over $\bbF^m$ with randomness space as $\bbF$ and commitment space as group $\bbG$ with independent generators $g_1,\ldots,g_m, h$. Define $c_{j} = \com(\ewit[\cdot,j],\delta_{j})$, $j\in [n]$ where the notation $X[\cdot,j]$ denotes the $m$-length vector $(X[1,j],\ldots,X[m,j])$ and $\delta_{j}$ denotes the randomness for computing the commitment $c_{j}$. We define the oracle $\comoracle$ as $\comoracle[j]=c_{j}$. The oracle $\comoracle$ answers queries of the type $Q\subseteq [n]$, responding with elements $\comoracle[j]$ for $j\in Q$.\footnote{Defining the commitment vector as $\comoracle$ might seem superfluous here. But, the usage of $\comoracle$ will play a prominent role in our 3D protocol. We just define the notation here for the uniformity in our descriptions of the 2D and the 3D versions.} 
%We will use $\comoracle$ as the witness
%oracle, and adapt the subprotocols for checking linear constraints, quadratic
%constraints and proximity to this oracle.

%Let $\cm_x=(c_1,\ldots c_n)$ where 
%$c_j= \com( \begin{bmatrix}
%u_{1j} & u_{2j} & \ldots & u_{mj}
%\end{bmatrix}^T)$ $\forall j\in [n]=\comoracle_x$

\begin{comment}
\subsection{Linear Check with Commitment Oracle}\label{subsec:lincheck2D}
The linear check $A\wit=b$ can be reduced to checking $\innp{r^TA}{x}=r^Tb$, where the verifier samples a random $r\sample \bbF^{ml}$ and sends it to the prover. As in \cite{ligero}, the prover and verifer first compute $R=r^TA$, then write $R$ as $(R_{1}|\cdots|R_{m})$ where each $R_i\in \bbF^\ell$. Both the prover and the verifier also compute degree $<\ell$ polynomials $R_{i}$ interpolating the vector $R_{i}$ on $G$. The required check in terms of polynomials can be expressed as:
\begin{equation}\label{eq:lincheck2D}
\sum_{\zeta\in G}\sum_{j\in [m]}
R_{i}(\zeta) \cdot Q_{i}(\zeta) = r^Tb.
\end{equation}
The prover computes the polynomial $p(\cdot)=\sum_{i\in[m]}R_i(\cdot) \cdot Q_i(\cdot)$. This polynomial $p(\cdot)$ of degree $< k+\ell-1$ is sent to the verifier who checks $\sum_{\zeta\in G}p(\zeta)=0$. The verifier needs to check if the polynomial $p$ is correctly computed from the witness oracle $\comoracle$ to guard against dishonest provers. Following \cite{ligero}, it is enough for the verifier to query the polynomial evaluations at a constant number of locations, say $\eta_{j_1},\ldots,\eta_{j_u}$ and then check that $p(\cdot)$ is consistent at the queried locations. We observe that the verifier only uses these queried values to prove some inner-product relations with other (publicly-known) vectors. Hence, our idea is to make the prover use inner-product arguments to prove the consistency of $p(\cdot)$ to the verifier.

We will now formally describe the linear check for our protocol \name2D{}. It will check that a purported commitment oracle $\comoracle$ encodes witness $\wit$ satisfying the constraint $A\wit=b$ for a public matrix $A$. As before, we assume $\wit \in \bbF^{\circsize}$ and $A\in \bbF^{\circsize\times \circsize}$ and
$\circsize = m \ell$ for some positive integers $m$ and $\ell$. We further assume that the prover has RS-encoded oracle $\ewit$ which opens to the commitment $\comoracle$ and is $e$-close to the interleaved code $L^{m}$. The prover and the verifier interact as follows: 

%$\prover$ sets $\cm_x$ as the oracle.
\begin{figure}[h!]
\centering
\begin{framed}
	\begin{itemize}
		\item {$\linearcheckTwoD (\FF, \GG, L[n,k,d], m, t, A \in \FF^{m\ell \times m\ell}, b\in \FF^{m\ell}, \bm{g}, [\pi]; \wit)$}:
		\item {\bf Relation}: $\exists \wit, \ewit$ s.t. $\ewit = \open(\pi), \ewit = \enc(\wit)$ and $A\wit = b$. 
		\item {\bf Oracle Setup}: The prover $\prover$ computes $\ewit = \enc(\wit)$ and $\comoracle = \com(\ewit)$ as in Sections ~\ref{subsec:encode2D} and ~\ref{subsec: commit2D}. The prover sets $\pi := \comoracle$ as the oracle.
	\end{itemize}
\begin{enumerate}[{\rm 1.}]
	\item $\verifier \rightarrow \prover: $ $\verifier$ picks a random $r\in \bbF^{ml}$ and sends that $r$ to $\prover$.
	
	\item Both $\prover$ and $\verifier$ compute $R=r^TA$, which is a vector of size $ml$. Read $R$ in a matrix form where first $l$ elements of $R$ form the first row, next $l$ elements form the second row, similarly $R$ will have $m$ rows. Then they construct polynomials $R_i(\cdot)$ of degree $<l$ such that $R_i(\zeta_j)=R_{ij}$ $\forall i\in [m], j\in [l]$. 
	
	\item $\prover \rightarrow \verifier: $  $\prover$ computes a polynomial $p(\cdot)=\sum_{i\in[m]} ( R_i(\cdot)\cdot \hat{f}^x_i(\cdot))$ and sends $p(\cdot)$ to $\verifier$.
	
	\item $\verifier \rightarrow \prover: $ $\verifier$ samples $t$ distinct  indices $j_1,\ldots,j_t$ from the set $[n]$ independently at random and sends the indices to $\prover$.
	
	\item Oracle Queries: $\verifier$ queries the oracle $\pi$ with $\{j_u : u\in [t]\}$.
	
	\item Oracle Answers: The oracle replies with $\pi[j_u], u\in [t]$.
	
	\item $\prover \leftrightarrow \verifier: $ $\prover$ and $\verifier$ run a inner product subprotocol for each $u\in[t]$:
	\begin{itemize}
		\item $\innerproduct(\GG,\bm{g}, R_{j_u}, \pi[j_u], p(\eta_{j_u}); \ewit[\cdot,j_u])$ $\forall u\in [t]$,
		
	Where $R_{j_u} = (R_1( \eta_{j_u}) , \ldots , R_m( \eta_{j_u}))$ and $\ewit[\cdot,j_u]$ denotes the $m$-length vector $(\ewit[1,j_u], \ldots, \ewit[m,j_u])$ and $\pi[j_u] = \com(\ewit[\cdot, j_u])$. $\verifier$ proceeds if the arguments succeed for all $u \in [t]$.
	\end{itemize} 

	\item $\verifier$ also checks that $\sum_{j\in[l]} p(\zeta_j)=0$.
	
	\item $\verifier$ accepts if all the above checks are succeed.	  
\end{enumerate}
\end{framed}
\caption{Linear Check for $\name$2D}
\end{figure}
The correctness of the above protocol follows from the correctness of \cite{ligero} and the inner product argument. We now prove the proof of knowledge of the protocol through the following lemma.
\begin{lemma}
For a $P^*$ which makes a verifier accept the above linear check protocol, there is an expected $\ppt$ extractor $\extr$ with rewinding access to $P^*$ which outputs a valid witness or breaks the binding of the commitment scheme with overwhelming probability.
\end{lemma}
.\dnote{We should probably include a defn of proof of knowledge in the prelims. Even if we extend this lemma to witness extended emulation, we do not have a single prover definition for WEE in the paper.}
\begin{proof}
Let $\extr_{\ip}$ be the $\ppt$ extractor for the inner product argument used. For a transcript, the prover's messages include the polynomial $p(\cdot)$ and its messages during the inner-product argument of $\innp{R_{j_u}}{\ewit[\cdot,j_u]} = p(\eta_{j_u})$.
$\extr$ would use $\extr_\ip$ with the commitment $\comoracle[j_u]$ to obtain $\ewit[\cdot,j_u]$ in expected polynomial time. 

To obtain a valid witness for the linear check protocol, $\extr$ rewinds the prover to Step 4 after obtaining an accepting transcript. $\extr$ adds the set of $j_u$ indices obtained to a set $S$ and repeats the rewinding process till $|S| = n$. 
There are two possibilities here for the obtained matrix $U$:
\begin{itemize}
\item if $d(U, L^m) > e$, $\extr$ outputs $U$. (The proximity check would have verified $d(U,L^m) \leq e$, and hence a $U$ otherwise would mean $\extr$ outputs a collision to the commitment scheme used).
\dnote{should we define a valid witness as one with distance less than $e$? Else, it seems like we can't have an independent proof and we have to bring in proximity check verifying $d<e$.}
\item if $d(U, L^m) \leq e$, the soundness analysis for the linear check in \cite{ligero} ensures that $U$ decodes to $x$ such that $A\wit=b$ with overwhelming probability. %and hence $\com(\calU^*)=\cm$.
\end{itemize}
An analysis similar to the one in the proof of Lemma~\ref{lem:proximity} proves that $\extr$ can attain $|S| = n$ in an expected polynomial time.

%For $e < {d}/{3} $, $\innp{\prover^*(\cm, \calU^*, A, b)}{\verifier(\cm, A, b)} \rightarrow 1 $ then there is an expected $\ppt$ $\extrac^{\prover^*}(\cm) \rightarrow \calU^*$ such that with overwhelming probability $\calU^*$ satisfies one of the following events:
%	\begin{itemize}
%		\item $\com(\calU^*)=\cm$ and $d(\ewit_x,L^m) > e $
%		\item $\com(\calU^*)=\cm$ and $d(\ewit_x, l^m)\leq e \text{ and } x = \dec(\ewit_x)$ satisfies $ Ax = b$
%		\item $\extrac$ breaks the commitment scheme.
%	\end{itemize}
%
%	We have a $\ppt$ extractor, $\extrac_{innp}$, for the inner product argument for the statement $\innp{\bm{a}}{\bm{b}}=c$ where $\bm{a},\bm{b}$ are private, which can either extract $\bm{a}, \bm{b}$ or breaks the binding property of the commitment scheme with overwhelming probability. Now we will use $\extrac_{innp}$ to design a $\ppt$ extractor $\extrac$ which can extract the witness($\calU^*$) for which the argument in the above protocol is accepted.
%	
%	$\extrac$ emulates $\verifier$'s role in the protocol till step 5, then calls $\extrac_{innp}$ to get $\calU^*[\cdot,j_u]$ or collision for the commitment. $\extrac$ stores the indices in a set say $S$ and rewind the prover to step 5 and picks $t$ indices again uniformly at random again and follows the above procedure. $\extrac$ keeps rewinding till $|S|=n$.
%	
%	If $|S|=n$, then $\extrac$ has the whole $\calU^*$.
%	
%	Then $\extrac$ computes $d(\calU^*, L^m)$:
%	\begin{itemize}
%		\item if $d(\calU^*,L^m) > e$ then $\extrac$ outputs $\calU^*$, which satisfies $\com(\calU^*)=\cm$ otherwise gets a collision for the binding property.
%		
%		\item if $d(\calU^*,L^m) \leq e$ then by soundness analysis in \cite{Ligero2017}, the nearest codeword of $\calU^*$ decodes to $x$ such that $Ax\neq b$ has $\negl(\lambda)$ probabillity. That means nearest codeword of $\calU^*$ decodes to $x$ which satisfies $Ax=b$ with very high probability and $\com(\calU^*)=\cm$.
%	\end{itemize}
%	
%	Similar analysis of Theorem:~\ref{lem:proximity} proves that $\extrac$ requires polynomially many rewinding to extract the witness $x$.
%\end{proof}

\subsection{Quadratic Check}\label{subsec:quadcheck2D}
We now formally describe the interactive oracle protocol for checking the relation $x\circ y = z$ for vectors $x,y,z\in \bbF^\circsize$. Let $\ewit_x, \ewit_y$ and $\ewit_z$ denote the encodings of vectors $x$, $y$ and $z$ respectively via the RS code mentioned in Section ~\ref{subsec:encode2D}. Let $\comoracle_x,\comoracle_y$ and $\comoracle_z$ denote the respective commitment oracles. In brief, as in the linear check, it follows \cite{ligero} except to check whether $p(\cdot)$ is correctly computed from the oracles. When the verifier queries the oracles at small number of points $\eta_{j_1} , \ldots , \eta_{j_t}$, the prover and the verifier would involve in an inner-product argument for the verifier to verify that $p(\eta_{j_u})=\sum_{i\in [m]}r_i[Q^x_i(\eta_{j_u}) \cdot Q^y_i(\eta_{j_u})-Q^z_i(\eta_{j_u}))$ for $u\in [t]$.
Which can be viewed as:
\begin{align*}
&\innp{r}{(\ewit_x[\cdot,j_u]\cdot \ewit_y[\cdot,j_u] - \ewit_z[\cdot,j_u])} = p(\eta_{j_u}) \\
\Rightarrow &\innp{r}{(\ewit_x[\cdot,j_u]\cdot \ewit_y[\cdot,j_u])} + \innp{r}{-\ewit_z[\cdot,j_u]} = p(\eta_{j_u})
\end{align*}
%------------------------------------------------------------------------
	There are two inner products in the above statement and the prover does not want to reveal the values of the individual inner products. Hence, \name2D combine them into a single inner product relation.

For each $u\in[t]$, $\prover$ runs the following inner product arguement with $\verifier$:
$$\innp{(r\circ \ewit_x[\cdot,j_u]||r)}{(\ewit_y[\cdot,j_u]||-\ewit_z[\cdot,j_u])} = p(\eta_{j_u})$$
To facilitate this, $\ewit_x, \ewit_y$ and $\ewit_z$ should have been committed with different independently chosen sets of generators. And, the set of generators used for committing $r$ (to be concatenated with $r \circ \ewit_x [\cdot, j_u])$ should also be independent of the above three sets of generators. $\verifier$ proceeds if the arguements succeed for all $u\in[t]$.
%-----------------------------------------------------------------------

%	$\prover$ sets $\cm_x, \cm_y$ and $\cm_z$ as the oracles.
\begin{figure}[h!]
\centering
\begin{framed}
	\begin{itemize}
		\item {$\quadcheckTwoD (\FF, \GG, L[n,k,d], m, t, \bm{g}_x, \bm{g}_y, \bm{g}_z, [\pi]; \wit_x, \wit_y, \wit_z)$}:
		\item {\bf Relation}: $\exists (\wit_x, \wit_y, \wit_z, \ewit_x, \ewit_y, \ewit_z)$ such that $(\ewit_x, \ewit_y, \ewit_z) = \open(\pi)$ and $\ewit_a = \enc(\wit_a)$ for $a\in \{x,y,z\}$ and $\wit_x \circ \wit_y = \wit_z$.
		\item {\bf Oracle Setup}: The prover $\prover$ computes $\ewit_a = \enc(\wit_a)$ and $\comoracle_a = \com(\ewit_a)$ for $a\in \{x,y,z\}$. It sets $\pi := [\comoracle_{x} || \comoracle_{y} || \comoracle_{z}]$ where the notation denotes vertical stacking of the vectors. Note that the generators used to construct commitment vectors for $\wit_x, \wit_y$ and $\wit_z$ should be independent. So, $\prover$ and $\verifier$ together generates $\bm{g}_x$, $\bm{g}_y$ and $\bm{g}_z$.
	\end{itemize}
\begin{enumerate}
	\item $\verifier \rightarrow \prover: $ $\verifier$ picks $r$ uniformly at random from $\FF^{m}$ and sends it to $\prover$.
	
	\item $\prover \rightarrow \verifier: $ $\prover$ computes the polynomial $p(\cdot)= \sum_{i\in [m]} [r_i\cdot (Q^x_i(\cdot)\cdot Q^y_i(\cdot) - Q^z_i(\cdot))] $ and sends $p(\cdot)$ to $\verifier$. 
	
	\item $\verifier \rightarrow \prover: $ $\verifier$ sends $t$ randomly sampled indices $Q=\{j_u\}_{u\in[t]}$ from $[n]$.
	
	\item Oracle Queries: $\verifier$ queries the oracle $\pi$ with $\{j_u : u\in [t]\}$.
	\item Oracle Answers: The oracle responds with $\pi[j_u], u\in[t]$.
	\item $\prover \leftrightarrow \verifier: $ $\prover$ and $\verifier$ together generates $\bm{g}$ independent of all the generators used in committing $x,y,z$. Both run a inner product subprotocol for each $u\in[t]$:
	\begin{itemize}
		\item $\innerproduct(\GG,\bm{g}'||\bm{g}_r, \bm{g}_y||(\bm{g}_z)^{-1}, \comoracle_x[j_u]\cdot (\bm{g}_r)^r, p(\eta_{j_u}), \comoracle_y\cdot (\comoracle_z)^{-1}; r\circ\ewit_x[\cdot,j_u]||r, \ewit_y[\cdot,j_u]||-\ewit_z[\cdot,j_u])$ $\forall u\in [t]$,
		
		Where $\bm{g}'= (g'_1, \ldots, g'_m)$ is such that $g'_i = g_x^{r^{-1}}$ 
		$\verifier$ proceeds if the arguments succeed for all $u \in [t]$.
	\end{itemize} 
	%.\pnote{run a subprotocol to prove the innerproduct arguement for the following statement $\innp{r\circ \ewit_x[\cdot,j_u]}{\ewit_y[\cdot,j_u]} - \innp{r}{\ewit_z[\cdot,j_u]} = p(\eta_{j_u})$ $\forall u\in[t]$.} 
		
	\item $\verifier$ checks if $p(\zeta_j)=0$ $\forall j\in[l]$. If yes then accepts, else rejects.
\end{enumerate}
\end{framed}
\caption{Quadratic Check for $\name$2D}
\end{figure}

%----------------------------------------------------------------------------------------------------------------------------
\begin{comment}
\begin{figure}[h!]
	\centering
	\begin{framed}
		\begin{itemize}
			\item {$\quadcheckTwoD (\FF, \GG, L, [\pi]; \wit_x, \wit_y, \wit_z)$}:
			\item {\bf Relation}: $\exists (\wit_x, \wit_y, \wit_z, \ewit_x, \ewit_y, \ewit_z)$ such that $(\ewit_x, \ewit_y, \ewit_z) = \open(\pi)$ and $\ewit_a = \enc(\wit_a)$ for $a\in \{x,y,z\}$ and $\wit_x \circ \wit_y = \wit_z$.
			\item {\bf Oracle Setup}: The prover $\prover$ computes $\ewit_a = \enc(\wit_a)$ and $\comoracle_{xy} = \com(\ewit_x \circ \ewit_y)$, $\comoracle_z = \com(\ewit_z)$. It sets $\pi :=[\comoracle_{xy} || \comoracle_{z}]$ where the notation denotes vertical stacking of the vectors that means $\pi$ has $n$ columns and 2 rows such that $\pi[1,j] = \comoracle_{xy}[j]$ and $\pi[2,j] = \comoracle_{z}[j]$ $\forall j\in[n]$.
		\end{itemize}
		\begin{enumerate}
			\item $\verifier \rightarrow \prover: $ $\verifier$ picks $r$ uniformly at random from $\FF^{m}$ and sends it to $\prover$.
			
			\item $\prover \rightarrow \verifier: $ $\prover$ computes the polynomial $p(\cdot)= \sum_{i\in [m]} [r_i\cdot (\hat{f}^x_i(\cdot)\cdot \hat{f}^y_i(\cdot) - \hat{f}^z_i(\cdot))] $ and sends $p(\cdot)$ to $\verifier$. 
			
			\item $\verifier \rightarrow \prover: $ $\verifier$ sends $t$ randomly sampled indices $Q=\{j_u\}_{u\in[t]}$ from $[n]$.
			
			\item Oracle Queries: $\verifier$ queries the oracle $\pi$ with $\{j_u : u\in [t]\}$.
			\item Oracle Answers: The oracle responds with $\pi[j_u], u\in[t]$.
			\item $\prover \leftrightarrow \verifier: $ $\prover$ and $\verifier$ run a inner product subprotocol for each $u\in[t]$:
			\begin{itemize}
				\item $\innerproduct(\GG,\bm{g}, r, \pi[1,j_u]\cdot \pi[2,j_u]^{-1}, p(\eta_{j_u}); \ewit_x[\cdot,j_u] \circ \ewit_y[\cdot, j_u], \ewit_z[\cdot,j_u])$ $\forall u\in [t]$,
				%Where $R_{j_u} = (R_1( \eta_{j_u}) , \ldots , R_m( \eta_{j_u}))$ and $\ewit[\cdot,j_u]$ denotes the $m$-length vector $(\ewit[1,j_u], \ldots, \ewit[m,j_u])$ and $\pi[j_u] = \com(\ewit[\cdot, j_u])$.
			 	
			 	$\verifier$ proceeds if the arguments succeed for all $u \in [t]$.
			\end{itemize} 
			\begin{comment}
			.\pnote{run a subprotocol to prove the innerproduct arguement for the following statement $\innp{r\circ \ewit_x[\cdot,j_u]}{\ewit_y[\cdot,j_u]} - \innp{r}{\ewit_z[\cdot,j_u]} = p(\eta_{j_u})$ $\forall u\in[t]$.} 
			
			There are two inner products in the above statement and the prover does not want to reveal the values of the individual inner products. Hence, \name2D combine them into a single inner product relation.
			
			For each $u\in[t]$, $\prover$ runs the following inner product arguement with $\verifier$:
			$$\innp{(r\circ \ewit_x[\cdot,j_u]||r)}{(\ewit_y[\cdot,j_u]||-\ewit_z[\cdot,j_u])} = p(\eta_{j_u})$$
			To facilitate this, $\ewit_x, \ewit_y$ and $\ewit_z$ should have been committed with different independently chosen sets of generators. And, the set of generators used for committing $r$ (to be concatenated with $r \circ \ewit_x [\cdot, j_u])$ should also be independent of the above three sets of generators. $\verifier$ proceeds if the arguements succeed for all $u\in[t]$.
			
			\item $\verifier$ checks if $p(\zeta_j)=0$ $\forall j\in[l]$. If yes then accepts, else rejects.
		\end{enumerate}
	\end{framed}
	\caption{Quadratic Check for $\name$2D}
\end{figure}

%----------------------------------------------------------------------------------------------------------------------------

The correctness of the above protocol again follows from the correctness of \cite{ligero} and the inner product argument. Its proof of knowledge property is proved through the following lemma.
\begin{lemma}
For a $P^*$ which makes a verifier accept the quadratic check protocol, there is an expected $\ppt$ extractor $\extr$ with rewinding access to $P^*$ which outputs valid witnesses $\ewit_x$, $\ewit_y$ and $\ewit_z$ with overwhelming probability.
\end{lemma}
\begin{proof}
Let $\extr_{\ip}$ be the $\ppt$ extractor for the inner product argument used. For a transcript, the prover's messages include the polynomial $p(\cdot)$ and its messages during the inner-product argument.
$\extr$ would use $\extr_\ip$ to obtain $\left( (r \circ \ewit_x[\cdot,j_u] \, || \, r \right)$ and $\left( \ewit_y[\cdot,j_u] \, || \, -\ewit_z[\cdot,j_u] \right)$ in expected polynomial time. $\extr_\ip$ would use the commitments derived from $\comoracle_x[j_u]$, $\comoracle_y[j_u]$ and $\comoracle_z[j_u]$. From this output of $\extr_{\ip}$, $\extr$ can obtain $\ewit_x[\cdot,j_u]$, $\ewit_y[\cdot,j_u]$ and $\ewit_z[\cdot,j_u]$.

To obtain a set of valid witnesses for the quadratic check protocol, $\extr$ rewinds the prover to Step 3 after obtaining an accepting transcript. $\extr$ adds the set of $j_u$ indices obtained to a set $S$ and repeats the rewinding process till $|S| = n$. At this point, $\extr$ has the complete $\ewit_x$, $\ewit_y$ and $\ewit_z$. There are two possibilities here:
\begin{itemize}
\item if $d(\ewit_\delta,L^m) > e$ for each $\delta = \{x, y, z\}$, $\extr$ outputs $\ewit_\delta$. (The proximity check would have verified $d(\ewit,L^m) \leq e$, and hence a $\ewit$ otherwise would mean $\extr$ outputs a collision to the commitment scheme used).
\item if $d(\ewit_\delta,L^m) \leq e$ for each $\delta = \{x, y, z\}$, the soundness analysis for the quadratic check in \cite{ligero} ensures that $\ewit_\delta$ decodes to $\delta$ such that $x\circ y = z$ with overwhelming probability.
\end{itemize}
An analysis similar to the one in the proof of Lemma~\ref{lem:proximity} proves that $\extr$ can attain $|S| = n$ in an expected polynomial time.
\end{proof}
%\begin{theorem}
%	For $ e < \frac{d}{3}, $ if $\innp{\prover^*(\cm_1, \cm_2, \cm_3 , \calU^*_1	, \calU^*_2, \calU^*_3)}{\verifier(\cm_1, \cm_2, \cm_3)} \rightarrow 1$ then there is an expected $\ppt$  $\extrac^{\prover^*}(\cm_i)\rightarrow \calU^*_i$ $\forall i\in [3]$ such that with overwhelming probability $\forall i \in [3]$, $\calU^*_i$ satisfies one of the following event: 
%	\begin{itemize}
%		\item $\com(\calU^*_i) = \cm_i$ and $(\vee_{i=1}^{3} d(\calU^*_i, L^m)> e)$ 
%		\item $\com(\calU^*_i) = \cm_i$ and $\wedge_{i=1}{3} d(\calU^*_i,L^m)\leq e$ and $ x_i = \dec(\calU_i)$, where $\calU_i$ is the nearest codeword to $\calU^*_i$, for all $i\in [3]$, such that $x_1 \circ x_2 = x_3$
%		\item $\extrac$ breaks the binding of the commitment scheme.
%	\end{itemize} 
%\end{theorem}
%\begin{proof}
%	We have a $\ppt$ extractor, $\extrac_{innp}$, for the inner product argument for the statement $\innp{\bm{a}}{\bm{b}}=c$ where $\bm{a},\bm{b}$ are private, which can extract either $\bm{a}, \bm{b}$ or breaks binding property of the commitment scheme with overwhelming probability. Now we will use $\extrac_{innp}$ to design an expected $\ppt$ extractor $\extrac$ which can extract $\calU^*_1, \cal^*_2, \cal^*_3$ for which the arguemnet is accepted in the above protocol.
%	
%	$\extrac$ emulates $\verifier$'s role in the protocol till step 4, then calls $\extrac_{innp}$ to get $(r\circ \calU^*_1[\cdot,j_u]||r)$ and $(\calU^*_2[\cdot,j_u]||-\calU^*_2[\cdot,j_u])$ forall $u\in [t]$ or breaks the binding property of the commitment scheme. $\extrac$ stores all the indices in a set say $S$. From the output of $\extrac_{innp}$, $\extrac$ computes $\calU^*_1[\cdot,j_u], \calU^*_2[\cdot,j_u],\calU^*_3[\cdot,j_u]$. $\extrac$ rewinds $\prover$ to step 4 and again picks a random $Q$. $\extrac$ keeps repeating the above process till $|S|=n$.
%	
%	If $|S|=n$, then $\extrac$ has the whole of $\calU^*_1, \calU^*_2, \calU^*_3$.
%	
%	Then $\extrac$ checks if $\com(\calU^*_i)=\cm_i$, if no, then outputs a collision for the binding of the commitment scheme, else computes $d(\calU^*_i, L^m)$ for all $i \in [3]$:
%	
%	\begin{itemize}
%		\item if $\vee_{i=1}^{3} d(\calU^*_i, L^m) > e$, then outputs $\calU^*_i \forall i\in [3]$.
%		
%		\item if $\wedge_{i=1}^3 d(\calU^*_i, L^m) \leq e$, then by soundness analysis in \cite{Ligero2017}, the nearest codewords of $\calU^*_1, \calU^*_2, \calU^*_3$ decode to $x, y, z$ repesctively such that probability that $x \circ y \neq z$ is $\negl(\lambda)$. That means the decoded values of the nearest codewords of $\calU^*_1, \calU^*_2, \calU^*_3$ satisfy $x\circ y = z$ with very high probability.
%	\end{itemize}
%	
%	Similar analysis of Theorem~\ref{lem:proximity} proves that $\extrac$ requires polynomially many rewinding to extract the witness $x, y, z$.
%\end{proof}


\subsection{Proximity Protocol}\label{subsec:proximity2D}
We finally describe our protocol for ``proximity'' of a purported codeword to the interleaved code. Let $U\in \bbF^{m\times n}$ denote the purported codeword and let $e< d/3$ denote the proximity parameter. The commitment oracle $\comoracle$ corresponds to $U$.
\dnote{To Nitin: why did you not use the notation $\rsoracle$ for $U$? Is it that only those $U$s which satisfy the proximity test become $\rsoracle$?}
The prover and the verifier interact as follows:
	%$\prover$ sets $\cm$ as the oracle.
	\begin{figure}[h!]
		\centering
		\begin{framed}
			\begin{itemize}
				\item {$\proxcheckTwoD(\FF, \GG, L[n,k,d], m, t, \bm{g}, [\pi]; \ewit)$}:
				\item {\bf Relation}: $\ewit = \open(\pi)$, $\ewit \in L^m$.
				\item {\bf Oracle Setup}: Prover computes $\comoracle$ from $\ewit$ as in Section ~\ref{subsec: commit2D} and sets $\pi := \comoracle$ as the oracle.
			\end{itemize}
			\begin{enumerate}
			\item $\verifier \rightarrow \prover :$ $\verifier$ as a challenge picks $\gamma \in \bbF^m$ uniformly at random and sends it to $\prover$.
	
			\item $\prover \rightarrow \verifier :$ $\prover$ computes $\w=\gamma^T\ewit$ and sends $\w$ to $\verifier$.
	
			\item $\verifier \rightarrow \prover :$ $\verifier$ picks a random subset $Q\subseteq [n]$ such that $|Q|=t$ and sends $Q$ to $\prover$.
	
			\item Oracle Queries: $\verifier$ queries the oracle $\pi$ with $\{j_u:u\in[t]\}$.
			
			\item Oracle Answers: The oracle responds with $\pi[j_u], u\in[t]$.
			
			\item $\prover \leftrightarrow \verifier: $ $\prover$ and $\verifier$ run a inner product subprotocol for each $u\in[t]$:
			\begin{itemize}
				\item $\innerproduct(\GG, \bm{g}, \gamma, \pi[j_u], \w_{j_u}; \ewit[\cdot,j_u])$ $\forall u\in [t]$.
			\end{itemize}
			%run a subprotocol to prove the innerproduct arguement for the following statement $\innp{\gamma}{\ewit_x[\cdot,j_u]}=u_{j_u}$ $\forall j_u\in Q$
	
			\item If $\verifier$ accepts the innerproduct arguement in the previous step, then checks if $\w\in L$. If yes then $\verifier$ outputs accept else outputs reject.
			\end{enumerate}
		\end{framed}
	\caption{Proximity Check for $\name$2D}
	\end{figure}
The correctness of the above protocol follows from \cite{ligero} and the inner product argument. The following lemma captures its proof of knowledge property.
\begin{lemma}\label{lem:proximity}
For a $P^*$ which makes a verifier accept the proximity protocol, there is an expected $\ppt$ extractor $\extr$ with rewinding access to $P^*$ which outputs a valid witness or breaks the binding of the commitment scheme with overwhelming probability.
\end{lemma}
\begin{proof}
Let $\extr_{\ip}$ be the $\ppt$ extractor for the inner product argument used. For a transcript, the prover's messages include the polynomial $p$ and its messages during the inner-product argument of $\innp{\gamma}{U[\cdot, j_u]} = w_{j_u}$.
$\extr$ would use $\extr_\ip$ with the commitment $\comoracle[j_u]$ to obtain $U[\cdot,j_u]$ in expected polynomial time. 

To obtain a valid witness for the proximity protocol, $\extr$ rewinds the prover to Step 3 after obtaining an accepting transcript. $\extr$ adds the set of $j_u$ indices obtained to a set $S$ and repeats the rewinding process till $|S| = n$. Let the obtained matrix be $U$.
The soundness analysis for the proximity check in \cite{ligero} ensures that  $d(U, L^m) \leq e$ with overwhelming probability. 

We will now estimate the number of rewindings required for $\extr$ to reach $|S| = n$. We will provide a crude upper bound on this to show that $\extr$ runs in expected polynomial time. To start with, we consider $t=1$. A larger $t$ will only reduce the number of rewindings required. Let $X_i$ be the discrete random variable that represents the number of rewindings to improve from $|S| = i-1$ to $|S| = i$ i.e., to pick a column not in $S$ when $|S| = i-1$. The base case $X_1=1$ because the first column selected will always be distinct. When $|S| = i-1$, there are $n-i+1$ columns remaining, the probability of selecting one of them is $(n-i+1)/n$. Since $X_i$ follows the geometric distribution, 
\[
E[X_i] = 1/ [(n-i+1)/n] = n/ (n-i+1)
\]
Let $X$ be the random variable for the number of rewindings to reach $|S| = n$.
Following linearity of expectations, 
\[
E[X] = \sum_{i \in [n]} E[X_i] = \sum_{i \in [n]} n/ (n-i+1) = n \sum_{i \in [n]} 1/i = \theta (n \log n) 
\]
Thus, $\extr$ extracts the witness in an expected polynomial time.
\end{proof}

%\begin{theorem}\label{lem:proximity}
%	For $e < \frac{d}{3} $, if $\innp{\prover^*(\cm, \calU^*)}{\verifier(\cm)} $ is true, then there is an expected $\ppt$ $\extrac^{\prover^*}(\cm) \rightarrow \calU^*$ such that with overwhelming probability $\calU^*$ satisfies one of the following events: 
%	\begin{itemize}
%		\item $\com(\calU^*) = \cm$ and $ d(\calU^*, L^m) < e$
%		\item $\extrac$ breaks the binding of the commitment. 
%	\end{itemize} 
%\end{theorem}
%\begin{proof}
%	Now we will design a $\ppt$ $\extrac$ for the above protocol.
%	In step 4 $\extrac$ emulates $\verifier$ and picks $Q$ uniformly at random, then if innerproduct argument is accepted then run $\extrac_{innp}$, in polytime with overwhelming probability. $\extrac_{innp}$ outputs either $\calU^*[\cdot,j_u]$ $\forall j_u\in Q$ or breaks the binding property of the commitment scheme, and stores the indices of $Q$ in a set $S$. then rewinds to the step 4 and then again picks $Q$ uniformly at random, $\extrac$ keeps extracting $\calU^*[\cdot,j_u]$ and updates $S$ by including only the new indices repeating until $|S|=n$. That is $\extrac$ extracts the whole $\calU^*$.
%	
%	Now $\extrac$ checks computes $d(\calU^*, L^m)$. Probability that $d(\calU^*,L^m) > e$ is $\negl(\lambda)$, by soundness analysis in \cite{Ligero2017}.
%	
%	Therefore $\extrac$ outputs $\calU^*$ whcih satisfies $\com(\calU^*) = \cm$ such that $d(\calU^*, L^m) < e$ with very high probability or it breaks the binding of the commitment scheme.
%	
%	Now we need to prove that the number of rewinding is polynomial.
%	
%	To get the bound on the expected number of rewindings consider $t=1$. 
%	
%	Let $X$ be the discrete random variable that represents the number of purchases until each of the $n$ column is picked at least once.
%	
%	Let $X_i$ be the discrete random variable that represents the number of rewindings after the $(i-1)^{th}$ distinct column to select the $i^{th}$ distinct column. As a base case, $X_1=1$, because the first column selected will always be distinct.
%	
%	By Linearity of expectation, $E[X]=\sum_{i=1}^{n}E[X_i]$
%	
%	After the $(i-1)^{th}$ distinct column is picked, there are $n-i+1$ columns remaining to be picked. Let $A_i$ be the event that one of those columns is picked in the next rewinding. Then $Pr(A_i)=\frac{n-i+1}{n}$.
%	
%	$X_i$ follows a geometric distribution (of trials). It's expected value is
%	$$E[X_i]= \frac{1}{Pr(A_i)} = \frac{n}{n-i+1}$$
%	From before, $E[X]$ is equal to the sum of all these expectations: i.e.
%	$$E[X] = \sum\limits_{i=1}^{n} \frac{n}{n-i+1} = n\sum\limits_{i=1}^{n} \frac{1}{i} \approx n \log n$$
%	Which is polynomial in $n$ and so polynomial in security parameter.
%	%\pnote{Remaining analysis is to check the expected number of rewinding required to obtain $\ewit$.}
%\end{proof}
\end{comment}

\subsection{\textbf{Complete protocol:}}\label{completeprotocol} Let $L$ be a language in $\NP$ and $\stmt$ is an instance of $L$. Let $\prover$ be a prover that claims that the instance $\stmt$ is true, i.e. $\prover$ has a witness $\wit$ such that there is deterministic circuit $\C$ such that $\C(\stmt,\wit)=1$ iff $\stmt \in L$. 

Now $\prover$ wants to convince a verifier $\verifier$ that $\stmt\in L$ without revealing any information about the witness $\wit$. 

To do that $\prover$ gives a proof that $\C$ on input $(\stmt, \wit)$ is correctly executed and output 1. In other words $\prover$ proves that gate by gate evaluation is correctly done on an input which has a public part known to both $\prover$ and $\verifier$, and a private part which is only known to $\prover$. 
Note that $\C$ and $x$ both known to $\prover$ and $\verifier$. So without loss of generality we can assume that $\stmt$ is hardcoded in $\C$.

$\prover$ constructs the extended witness $\extwit$ in the following way: \\
Let $\C:\bbF^{n_i}\rightarrow \bbF$ such that $\prover$ has private input $\wit = (\wit_1,\ldots, \wit_{n_i})$ such that $\C(\wit)=1$.

Define the extended witness $\extwit = (\wit_1,\ldots,\wit_{n_i}, \beta_1,\ldots, \beta_s) \in \bbF^{m\ell}$, where $\beta_i$ is the output of the $i^{th}$ gate evaluating $\C(\wit)$, and $s$ is the number of gates in $\C$ and $m\ell>n_i + s$. $\prover$ defines a system of constraints that contains the following constraint for every multiplication gate g in the circuit $\C$ $$\beta_{a}.\beta_{b}-\beta{c}=0$$
and for every addition gate, the constraint 
$$\beta_a + \beta_b - \beta_c = 0$$
Where $\beta_a$, $\beta_b$ are the input values to the gate g and $\beta_c$ is the output value in the extended witness. For the output gate include the constraint $\beta_a + \beta_b - 1 = 0$ if the final gate is an addition gate, and $\beta_a\cdot \beta_b - 1 = 0$ if the final gate is an multiplication gate. 

$\prover$ constructs vectors $x,y,z \in \bbF^{m\ell}$ where the $j^{th}$ entry of $x,y$ and $z$ contains the values $\beta_a, \beta_b$ and $\beta_c$ corresponding to the $j^{th}$ multiplication gate in $\extwit$.

$\prover$ and $\verifier$ construct matrices $A, B$ and $C \in \bbF^{m\ell \times m\ell}$ such that 
$$x = A \extwit, y= B \extwit, z= C \extwit$$

Finally it constructs $P_{add} \in \bbF^{m\ell \times m\ell}$ such that the $j^{th}$ position of $P_{add} \extwit$ equals $\beta_a + \beta_b - \beta_c$ where $a, b$ and $c$ correspond to the $j^{th}$ addition gate of the circuit in $\extwit$.

$\prover$ encodes $\extwit , x, y , z$ using the defined encoding in sec:~\ref{subsec:encode2D} and gets $\ewit_{\extwit}, \ewit_x, \ewit_y$ and $\ewit_z$ and commit each of them using Pedersen vector commitment described in ~\ref{subsec: commit2D} and gets $\comoracle_{\extwit}, \comoracle_x, \comoracle_y, \comoracle_z$. To commit, pick different set of generators for each $\ewit_{\extwit}, \ewit_x, \ewit_y, \ewit_z$, which will give the property that commitment of $\ewit_{\extwit||x}$ is $\comoracle_{\extwit} \circ \comoracle_x$. Where $\extwit||x$ means that a new matrix is formed by adjoining columns of $\extwit$ followed by the columns of $x$. 

In the following protocol, $\verifier$ picks a random index set $Q\subseteq [n]$ of size $t$ and uses this $Q$ in all the followimg subprotocols.
\begin{figure}[h]
%\centering
\begin{framed}
\begin{itemize}
	\item {$\name$2D($\FF, \GG, L[n,k,d], m, t, \bm{g}, \bm{g}_x, \bm{g}_y, \bm{g}_z, A,B,C, P_{add}, [\pi]; \extwit, x, y, z$)}
	\item {\bf Relation}: $(\exists \extwit, x, y, z, \ewit_{\extwit}, \ewit_x, \ewit_y, \ewit_z)$ s.t $(\ewit_{\extwit} , \ewit_x, \ewit_y, \ewit_z) = \open(\pi)$, $\ewit_a = \enc(a)$ for $a\in \{\extwit, x,y,z\}$ and 
	\begin{itemize}
		\item $\ewit_{\extwit}||\ewit_x||\ewit_y||\ewit_z \in L^{4m}$, where $||$ notation denotes vertical stacking.
		\item $P_{add} \extwit = \bm{0}$
		\item $A\extwit = x$
		\item $B\extwit = y$
		\item $C\extwit = z$
		\item $x\circ y = z$
	\end{itemize}
	\item {\bf Oracle Setup}: The prover $\prover$ computes $\ewit_a =\enc(a)$ for $a\in \{\extwit,x,y,z\}$. Then $\prover$ commits to $\ewit_{\extwit}, \ewit_x, \ewit_x, \ewit_z$ using $\bm{g}, \bm{g}_x, \bm{g}_y, \bm{g}_z$ and gets $\comoracle_{\extwit}, \comoracle_{x}, \comoracle_{y}, \comoracle_{z}$.
\end{itemize}
$\prover$ and $\verifier$ run the following protocols in parallel and for all these protocols $\verifier$ sends a common query to the oracle and $\prover$.
\begin{enumerate}
	\item $\prover \leftrightarrow \verifier: $ $\prover$ and $\verifier$ run the subprotocol ~\ref{subsec:proximity2D} for proximity check for the matrix $\extwit||x||y||z$. Which is encoded to the matrix $\ewit{\extwit||x||y||z}$, and $\cm_{\extwit||x||y||z}$ is the corresponding commitment.
	\begin{itemize}
		\item $\proxcheckTwoD(\FF, \GG, L[n,k,d], 4m, t, \bm{g}, \bm{g}_x, \bm{g}_y, \bm{g}_z, [\pi]; \ewit_{\extwit}||\ewit_x||\ewit_y||\ewit_z$)
	\end{itemize}
	\item $\prover \leftrightarrow \verifier: $ $\prover$ and $\verifier$ run the subprotocol ~\ref{subsec:lincheck2D} for the linearity check for the public matrix $P_{add}$ of dimension $ml\times ml$, and the public vector is the 0-vector of length $ml$. This check ensures that the addition gates are corretly evaluated.
	\begin{itemize}
		\item $\linearcheckTwoD(\FF, \GG, L[n,k,d], m, t, P_{add}, \bm{0}, \bm{g}, [\pi]; \extwit)$
	\end{itemize}
	\item $\prover \leftrightarrow \verifier:$ $\prover$ and $\verifier$ run the subprotocol ~\ref{subsec:lincheck2D} for linearity check for the public matrix $[A|-I]$ where $I$ is the identity matrix of dimension $ml \times ml$, and the public vector is the 0 vector of length $2ml$. This check ensures that $x$ is correctly computed from $\extwit$.
	\begin{itemize}
		\item $\linearcheckTwoD(\FF, \GG, L[n,k,d], 2m, t, [A|-I], \bm{0}, \bm{g}, \bm{g}_x, [\pi]; \extwit, x)$
	\end{itemize}
	\item $\prover \leftrightarrow \verifier:$ $\prover$ and $\verifier$ run the subprotocol ~\ref{subsec:lincheck2D} for linearity check for the public matrix $[B|-I]$ where $I$ is the identity matrix of dimension $ml \times ml$, and the public vector is the 0 vector of length $2ml$.This check ensures that $y$ is correctly computed from $\extwit$.
	\begin{itemize}
		\item $\linearcheckTwoD(\FF, \GG, L[n,k,d], 2m, t, [B|-I], \bm{0}, \bm{g}, \bm{g}_y, [\pi]; \extwit, y)$
	\end{itemize}
	\item $\prover \leftrightarrow \verifier:$ $\prover$ and $\verifier$ run the subprotocol ~\ref{subsec:lincheck2D} for linearity check for the public matrix $[C|-I]$ where $I$ is the identity matrix of dimension $ml \times ml$, and the public vector is the 0 vector of length $2ml$.This check ensures that $z$ is correctly computed from $\extwit$.
	\begin{itemize}
		\item $\linearcheckTwoD(\FF, \GG, L[n,k,d], 2m, t, [C|-I], \bm{0}, \bm{g}, \bm{g}_z, [\pi]; \extwit, z)$
	\end{itemize}
	\item $\prover \leftrightarrow \verifier:$ $\prover$ and $\verifier$ run the subprotocol ~\ref{subsec:quadcheck2D} for quadratic check to prove that $x \circ y = z$.
	\begin{itemize}
		\item $\quadcheckTwoD (\FF, \GG, L[n,k,d], m, t, \bm{g}_x, \bm{g}_y, \bm{g}_z, [\pi]; x, y, z)$
	\end{itemize}
	\item If $\prover$ passes all the above check, then $\verifier$ accepts the argument, else rejects.
\end{enumerate}
\end{framed}
\caption{$\name$2D for R1CS}
\end{figure}
Note that if $\prover$ executed all the steps correctly, then it passes all the checks and so $\verifier$ accepts the proof. So completeness holds for the protocol.

To prove the proof of knowledge we will design an extractor which will output a witness if the argument is accepted.

\begin{theorem}
	For $e < \frac{d}{3}$, if $\innp{\prover^*(\comoracle_{\extwit||x||y||z}, \calU^*)}{\verifier(\comoracle_{\extwit||x||y||z})} \rightarrow 1$, then there is an expected $\ppt$ $\extrac^{\prover^*}(\comoracle_{\extwit||x||y||z}) \rightarrow \calU^*$ such that with overwhelming probability $\calU^*$ satisfies one of the following events:
	\begin{itemize}
		\item $\com(\calU^*)=\comoracle_{\extwit||x||y||z}$ and $A\extwit = x\text{ }\wedge\text{ } B \extwit = y\text{ } \wedge \text{ }C \extwit = z\text{ } \wedge \text{ }x\circ y =z\text{ } \wedge\text{ } P_{add} \extwit = \bm{0}$
		\item $\extrac$ breaks the binding property of the commitment scheme.
		
	\end{itemize}
	Let $\cm$ is the commitment of $\calU^*$, used in above protocol by $\prover$. If $\verifier$ accepts the argument generated by $\prover$, then there is an expected $\ppt$ extrator $\extrac$ having rewinding access to $\prover$, with polynomially many rewindings either outputs a correct $\extwit$ or breaks the binding property of the commitment scheme.
\end{theorem}
\begin{proof}
	We have extractors for the proximity check, linearity check and quadratic check, say $\extrac_{prox}, \extrac_{lin}, \extrac_{quad}$ are the extractors respectively. Using these extractors we will design  $\extrac$ for the complete protocol. 
	
	$\extrac$ emulates the role of the verifier and starts the protocol. 
	
	In the first step it calls $\extrac_{prox}$ and $\calU_1^*$.
	
	After executing the first step it calls $\extrac_{lin}$ and gets $\calU_2^*$. If $\calU_1^* = \calU_2^*$, then outputs collision and terminates.
	
	After executing the second if it is not terminated, it calls $\extrac_{lin}$ in the step 3, 4, 5 and gets $\calU_3^*$. 
	
	In the above 2 steps it gets collision that breaks the binding of the commitment otherwise proceeds.
	
	In this step $\extrac$ calls $\extrac_{quad}$, by concatanating the output of $\extrac_{quad}$ construct final matrix which equates with $\calU_1^* (=\calU_2^*=\calU_3^*)$. If not then that gives break of the binding property of the commitment scheme. Otherwise $\calU^*=\calU_1^*$ should satisfy the following:
	\begin{itemize}
		\item $\com(\calU*)=\comoracle_{\extwit}\cdot\comoracle_x\cdot\comoracle_y\cdot\comoracle_z$.
		\item $d(\calU^*,L^{4m}) < e$ and let $\calU$ is the closest codeword of $\calU^*$. Define $\ewit_{\extwit}$ to be the first $m$ rows of $\calU$
		
		Define $\ewit_{x}$ to be the $(m+1)^{th}$ rows to $2m^{th}$ rows of $\calU$
		
		Define $\ewit_{y}$ to be the $(2m+1)^{th}$ rows to $3m^{th}$ rows of $\calU$
		
		Define $\ewit_{z}$ to be the $(3m+1)^{th}$ rows to $4m^{th}$ rows of $\calU$
		
		Let $w= \dec(\ewit_{\extwit}), x=\dec(\ewit_x), y=\dec(\ewit_y), z=\dec(\ewit_z)$ such that
		
		$A\extwit = x \wedge B \extwit = y \wedge C \extwit = z \wedge x\circ y =z \wedge P_{add} \extwit = 0$.
	\end{itemize}
	Therefore $\extwit$ is a correct witness. Since all the above extractors use polynomial number of rewindings, $\extrac$ uses polynomial number of rewindings.		
\end{proof}
%--------------------------------------------------------------------------------------------------------------
\subsection{Zero - Knowledge}\label{subsec:zeroknowledge}
Above defined three subprotocols are not zero-knowledge inherently. However, converting them into zero-knowledge is easy. 

Consider the proximity check: In this subprotocol, $\verifier$ is learning $\w = \gamma^T\ewit$, which he cannot compute on his own. To prevent that modify $\ewit$ in the following way: include a random codeword in $(m+1)^{th}$ row, which blinds $\w$ and makes $\w$ a random codeword. 
In the remaining part, oracle sends $\comoracle[j_u]$, hiding property of the commitment scheme ensures that it does not reveal any information.
Inner product proofs are given for $t$ columns. Instead of giving the proof, If $\prover$ opens $t$ columns of $\ewit$, still it does not reveal any information about $\dec(\ewit)$ since $t$ is smaller than the degrees of the polynomials used in $\enc$.

Consider the linear check: In the first step $\prover$ sets the oracle $\comoracle$. The hiding property of the commitment scheme ensures that $\verifier$ is getting no information about $\ewit$ or $w$.
Then $\prover$ sends $p(\cdot) = \sum_{i\in[m]} R_i (\cdot) \cdot Q_i(\cdot)$ to check that if $\sum_{j\in [\ell]} p(\zeta_j) = r^Tb$. But $\verifier$ instead of learning whether $\sum_{j\in[\ell]} p(\zeta_j) = 0$ or not, gets the complete polynomial $p(\cdot)$. To avoid leaking additional information we need to blind $p(\cdot)$ by adding a blinding polynomial $p_{blind}(\cdot)$ of degree $< k + \ell - 1$ such that $\sum_{j\in[\ell]} p_{blind}(\zeta_j) = 0$. Include a new row to $\ewit$ at the end where $j$th entry of the row is $p_{blind}(\eta_j)$ $\forall j\in [n]$.
Since the number of inner product proofs is less than $t$, no information is leaked.

Consider the quadratic check: In the first step, commitments do not leak any information about the witness, or the encoded values.
Then $\prover$ sends $p(\cdot) = \sum_{i\in[m]} r_i\cdot [Q^x_i(\cdot)\cdot Q^y_i(\cdot) - Q^z_i(\cdot)]$. $\verifier$ is allowed to learn only if $p(\zeta_j)=0$ or not for all $j\in[\ell]$. To avoid leaking more information about $p(\cdot)$ we need to blind it. To do that pick a random polynomial $p_{blind}(\cdot)$ such that $p_{blind}(\zeta_j) = 0$ $\forall j\in [\ell]$. accordingly update $\ewit_x, \ewit_y,\ewit_z$ in the following way: pick there random codewords which are encodings of zeros, and append one of them each at the last of $\ewit_{x}, \ewit_{y}, \ewit_{z}$.
Inner product argument is the same as Proximity and linear check; it does not require any changes. 

\subsubsection{\bf Deisgning the simulator for the complete protocol: } In an actual execution of the protocol generates a transcript of the form:
\begin{align*}
\tau = \{ 
& \comoracle_{\extwit||x||y||z},\\
& \gamma(\in_R \bbF^{m}), r_1(\in_R \bbF^{m\ell}), r_2(\in_R \bbF^m), \\ 
& (\w' = \w+\w_{blind}), (q^{lin}(\cdot) = p^{lin}(\cdot)+p^{lin}_{blind}(\cdot)),\\
&(q^{quad}(\cdot) = p^{quad}(\cdot) + p^{quad}_{blind}(\cdot)),
\text{ } Q (|Q|=t),\\
& \text{ inner product proof for } \\
& (\innp{\gamma}{\ewit_{\extwit||x||y||z}[\cdot, Q]} = u_Q, \innp{R_Q}{\ewit_{\extwit||x||y||z}[\cdot, Q]}=q^{lin}(\eta_Q),\\
& \innp{(r\circ \ewit_x[\cdot,Q]||r)}{(\ewit_y[\cdot,Q]||-\ewit_z[\cdot,Q])} = q^{quad}(\eta_{Q})
\}
\end{align*}
Consider a protocol, which is same as above protocol with the difference that $\prover$ instead of proving the inner product arguments opens the corresponding columns of $\ewit$. If this new protocol has zero knowledge property then our protocol also have zero knowledge property, since in our protocol whatever $\verifier$ can compute, $\verifier$ of the new protocol can also compute. It is easy to prove this by reduction.

Now we will prove that the new protocol is zero-knowledge. It will have the transcript of the following form:
\begin{align*}
\tau' = \{
& \comoracle_{\extwit||x||y||z},\\
& \gamma(\in_R \bbF^{m}), r_1(\in_R \bbF^{ml}), r_2(\in_R \bbF^m), \\ 
& (\w' = \w+\w_{blind}), (q^{lin}(\cdot) = p^{lin}(\cdot)+p^{lin}_{blind}(\cdot)),\\ 
&(q^{quad}(\cdot) = p^{quad}(\cdot) + p^{quad}_{blind}(\cdot)), \text{ }
Q (|Q|=t),\\
& \ewit_{\extwit}[\cdot, Q], \ewit_{x}[\cdot, Q], \ewit_{y}[\cdot, Q], \ewit_{z}[\cdot, Q]
\}
\end{align*}

Let $\Sim$ be the simulator. $\Sim$ does the following:
\begin{itemize}
	\item  picks a random subset $Q$ of size $t$.
	\item  uniformly at random chooses $\gamma \in \bbF^m$.
	\item  uniformly at random chooses $r_1 \in \bbF^{m\ell}$ and $r_2 \in \bbF^m$.
	\item  chooses $t$ columns for $\ewit_{\extwit||x||y||z}$ according to the indices of $Q$.
	\item  computes the commitment of columns indexed by $Q$ for $\ewit_{\extwit||x||y||z}$, and for remaining positions picks uniform values from the range of $\com$, that fixes $\comoracle_{\extwit||x||y||z}$.
	\item  computes components of $\w'$ indexed by $Q$ using $\ewit_{\extwit||x||y||z}$ and $\gamma$. Out of $n$ for remaining $n-t$ picks values for $\w'$ in such a way that $\w'$ is a valid codeword.
	\item  picks a random polynomial $q^{lin}(\cdot)$ such that degree is $<k+\ell-1$ and $\sum_{j\in [\ell]} q^{lin}(\zeta_j) = 0$.
	\item  picks a random polynomial $q^{quad}(\cdot)$ such that degree is $<2k-1$ and $q^{quad}(\zeta_j) = 0$ $\forall j\in [\ell]$.
\end{itemize} 
Then $\Sim$ outputs a transcript $\tau''$ which is computationally indistinguishable from $\tau'$. Therefore the new protocol has zero knowledge property, and hence \name2D has zero-knowledge property.

\subsection{$\DPZK$ construction from $\name$2D}

Consider $\prover_1, \ldots, \prover_{\Num}$ are the provers for the statement $\stmt$ where $\stmt$ is in the language and $\wit$ is the corresponding witness such that for the deterministic verification algorithm $\C$, $\C(\stmt, \wit)=1$. Consider the scenario when the witness is distributed among the provers i.e. no one prover possesses the complete witness who can generate the correct proof so that the verifier $\verifier$ accepts the proof.

Consider that prover $\prover_{\nu}$ possesses $\wit_{\nu}$ as his secret such that which $\wit_{\nu}$ is the part of witness that $\prover_{\nu}$ is going to use to generate the proof to convince $\verifier$ where $(\wit_1,\ldots, \wit_{\Num})=\wit$. Then the provers run an MPC for a function which outputs $\extwit_{\nu}$ to $\prover_{\nu}$ where $\sum_{\nu=1}^{\Num}\extwit_{\nu} = \extwit$.

Then each prover doess the following:
\begin{itemize}
	\item $\prover_{\nu}$ encodes $\extwit_{\nu}$ in the same way mentioned in Section:\ref{sec:encode2D} and constructs encoded matrix $\ewit_{\extwit_{\nu}}$.
	\item $\prover_{\nu}$ commits to $\ewit_{\extwit_{\nu}}$ in the same way mentioned in Section:\ref{sec:commit2D} and constructs commitment vector $\comoracle_{\nu}$.
\end{itemize}

Now we will describe how provers are responding to the challenges given by the verifier for different checks:
\paragraph{Proximity Check:}
Provers choose an aggregator $\Ag$, need not be trusted, such that $\Ag$ interacts with the verifier $\verifier$. 

%Set $\sum_{\nu \in [\Num]} \comoracle_{\nu} = \comoracle$ as the oracle.
\begin{figure}[h!]
	\centering
	\begin{framed}
		\begin{itemize}
			\item {$\DPZK \proxcheckTwoD(\FF, \GG, L[n,k,d], m, t, \bm{g}, [\pi]; \ewit)$}
			\item {\bf Relation}: $\sum_{\nu \in [\Num]} \ewit_{\nu} = \ewit = \open(\pi)$, $\ewit \in L^m$.
			\item {\bf Oracle Setup}: 
			\begin{itemize}
				\item $\nu$th prover computes $\comoracle_{\nu}$ from $\ewit_{\nu}$ as in Section ~\ref{subsec: commit2D} and sends $\comoracle_{\nu}$ to the aggregator $\Ag$.
				\item $\Ag$ sets $\pi := \comoracle = \prod_{\nu \in [\Num]} \comoracle_{\nu}$ as the oracle.
			\end{itemize}
		\end{itemize}
		\begin{enumerate}
			\item $\verifier \rightarrow \Ag : $ The verifier samples $\gamma \stackrel{\$} {\leftarrow} \bbF^m$ and sends to the $\Ag$.
		
			\item $\Ag \rightarrow \prover_{\nu} : $ The aggregator $\Ag$ forwards the challenge $\gamma$ to all the provers $\prover_{\nu}$ $\forall \nu\in [\Num]$.
	
			\item $\prover_{\nu} \rightarrow \Ag : $ $\prover_{\nu}$ computes $\w_{\nu} = \gamma^T\ewit_{\nu}$.
			
			\item $\Ag \rightarrow \verifier : $ The aggregator aggregates $\w_{\nu}$'s from all the provers and computes $\w=\sum_{\nu\in[\Num]} \w_{\nu}$ and sends $\w$ to the verifier.
	
			\item $\verifier \rightarrow \Ag : $ The verifier sends $t$ randomly picked sampled indexes $Q= \{j_u:u\in[t]\}$ from $[n]$.
			
			\item $\Ag \rightarrow \prover_{\nu} : $ $\Ag$ forwards $Q$ to the provers.
			\item Oracle Queries: $\verifier$ queries the oracle $\pi$ with $\{j_u:u\in[t]\}$.
			
			\item Oracle Answers: The oracle responds with $\pi[j_u], u\in[t]$.
			
			\item $\prover_{\nu} \rightarrow \Ag : $ $\prover_{\nu}$ sends the columns vectors 
			$\ewit_{{\nu}}[\cdot,j_u]$ to the aggregator $\Ag$.
	
			\item $\Ag \leftrightarrow \verifier : $ $\Ag$ computes $\ewit[\cdot,j_u] = \sum_{\nu\in[\Num]} \ewit_{{\nu}}[\cdot,j_u]$. Then $\Ag$ and $\verifier$ run a subprotocol to prove the innerproduct argument for the following statement $\w_{j_u} = \innp{\gamma}{\ewit[\cdot, j_u]}$ $\forall u\in[t]$.
			\begin{itemize}
				\item $\innerproduct(\GG, \bm{g}, \gamma, \pi[j_u], \w_{j_u}; U[\cdot, j_u] )$
			\end{itemize}
	
			\item If $\verifier$ accepts the inner product argument in the previous step, then checks if $\w\in L$. If yes then $\verifier$ outputs accept, else outputs reject.
		\end{enumerate}
	\end{framed}
\caption{Proximity Check for $\name$2D in $\DPZK$ setting}
\end{figure}

\paragraph{Linear Check: }
Provers choose an aggregator $\Ag$, need not be trusted, such that $\Ag$ interacts with the verifier $\verifier$ to prove that $A\wit=b$ where $\prover_{\nu}$ has $\wit_{\nu}$ such that $\sum_{\nu\in[\Num]} \wit_{\nu}=\wit$.
\begin{comment} 
Then each prover doess the following:
\begin{itemize}
	\item $\prover_{\nu}$ encodes $x_{\nu}$ in the same way mentioned in subsec:\ref{subsec:encode2D} and constructs encoded matrix $\ewit_{x_{\nu}}$.
	\item $\prover_{\nu}$ commits to $\ewit_{x_{\nu}}$ in the same way mentioned in subsec:\ref{subsec: commit2D} and constructs commitment vector $\comoracle_{\nu}$.
\end{itemize}
\end{comment}
%Set $\sum_{\nu \in [\Num]} \comoracle_{\nu} = \comoracle$ as the oracle.

\begin{figure}[h!]
\centering
\begin{framed}
	\begin{itemize}
		\item {$\DPZK \linearcheckTwoD (\FF, \GG, L[n,k,d], m, t, A \in \FF^{m\ell \times m\ell}, b\in \FF^{m\ell}, \bm{g}, [\pi]; \wit)$}:
		\item {\bf Relation}: $\exists \{\wit_{\nu}, \ewit_{\nu} : \nu\in [\Num]\}$ s.t. $\wit = \sum_{\nu \in [\Num]} \wit_{\nu}$ and $\sum_{\nu \in [\Num]} \ewit_{\nu}= \ewit = \open(\pi), \ewit_{\nu} = \enc(\wit_{\nu})$ and $A\wit = b$. 
		\item {\bf Oracle Setup}: 
		\begin{itemize}
			\item The prover $\prover_{\nu}$ computes $\ewit_{\nu} = \enc(\wit_{\nu})$ and $\comoracle_{\nu} = \com(\ewit_{\nu})$ as in Sections ~\ref{subsec:encode2D} and ~\ref{subsec: commit2D} for $\nu \in [\Num]$. Then $\prover_{\nu}$ sends $\comoracle_{\nu}$ to $\Ag$.
			\item $\Ag$ sets $\pi := \comoracle = \prod_{\nu \in [\Num]} \comoracle_{\nu}$ as the oracle.
		\end{itemize}
	\end{itemize}
\begin{enumerate}
	\item $\verifier \rightarrow \Ag : $ $\verifier$ picks a random $r\in \bbF^{m\ell}$ and sends that $r$ to $\Ag$.
	%\item $\Ag \rightarrow \prover_{\nu} : $ $\Ag$ forwards $r$ to all the provers $\prover_{\nu}$ $\forall \nu\in[\Num]$. 
	
	\item $\Ag$ and $\verifier$ compute $R=r^TA$, which is a vector of size $m\ell$. read $R$ in a matrix form where the first $\ell$ elements of $R$ forms the first row, next $\ell$ elements forms the second row, similarly $R$ will have $m$ rows. Then they construct polynomials $R_i(\cdot)$ of degree $<\ell$ such that $R_i(\zeta_j)=R_{ij}$ $\forall i\in[m], j\in[\ell]$.
	
	\item $\Ag \rightarrow \prover_{\nu} : $ $\Ag$ forwards $R_i(\cdot)$ polynomials for all $i\in[m]$ to all the provers $\prover_{\nu}$ $\forall \nu\in[\Num]$.
	
	\item $\prover_{\nu} \rightarrow \Ag : $ $\prover_{\nu}$ computes a $p_{\nu}(\cdot) = \sum_{i\in[m]} (R_i(\cdot)\cdot Q_{{\nu}_i}(\cdot))$ and sends $p_{\nu}(\cdot)$ to the aggregator $\Ag$.
	
	\item $\Ag \rightarrow \verifier : $ $\Ag$ aggregates the polynomials and computes $p(\cdot)=\sum_{\nu\in[\Num]} p_{\nu}(\cdot)$ and sends $p(\cdot)$ polynomial to the verifier $\verifier$.
	
	\item $\verifier \rightarrow \Ag : $ The veifier $\verifier$ sends $t$ randomly picked sampled indexes $Q= \{j_u: u\in[t]\}$ from $[n]$. 
	
	\item $\Ag \rightarrow \prover_{\nu} : $ $\Ag$ forwards $Q$ to the provers.
	
	\item Oracle Queries: $\verifier$ queries the oracle $\pi$ with $\{j_u:u\in[t]\}$.
	
	\item Oracle Answers: The oracle responds with $\pi[j_u], u\in[t]$.
	
	\item $\prover_{\nu} \rightarrow \Ag : $ $\prover_{\nu}$ sends the columns vectors $\ewit_{\nu}[\cdot,j_u]$ to the aggregator $\Ag$.
	
	\item $\Ag \leftrightarrow \verifier : $ $\Ag$ computes $\ewit[\cdot,j_u] = \sum_{\nu \in [\Num]} \ewit_{\nu}[\cdot, j_u]$. Then $\Ag$ and $\verifier$ run a subprotocol:
	\begin{itemize}
		\item $\innerproduct(\GG,\bm{g}, R_{j_u}, \pi[j_u], p(\eta_{j_u}); \ewit[\cdot,j_u])$ $\forall u\in [t]$,
		
		Where $R_{j_u} = (R_1( \eta_{j_u}) , \ldots , R_m( \eta_{j_u}))$ and $\ewit[\cdot,j_u]$ denotes the $m$-length vector $(\ewit[1,j_u], \ldots, \ewit[m,j_u])$ and $\pi[j_u] = \com(\ewit[\cdot, j_u])$. $\verifier$ proceeds if the arguments succeed for all $u \in [t]$.
	\end{itemize}
 
	\item If $\verifier$ accepts the inner product arguments, then checks if $\sum_{j\in[l]} p(\zeta_j) = r^Tb$. If yes then accepts, else outputs reject.
\end{enumerate}
\end{framed}
\caption{Linear Check for $\name$2D in $\DPZK$ setting}
\end{figure}
\paragraph{Quadratic Check: }
Provers choose an aggregator $\Ag$, need not be trusted, such that $\Ag$ interacts with the verifier $\verifier$ to prove that $x \circ y = z$ where $\prover_{\nu}$ has $x_{\nu}, y_{\nu}$ and $z_{\nu}$ such that $\sum_{\nu \in [\Num]}x_{\nu} = x, \sum_{\nu \in [\Num]}y_{\nu} = y$ and $\sum_{\nu \in [\Num]}z_{\nu} = z$
\begin{comment}
Then each prover doess the following:
\begin{itemize}
	\item $\prover_{\nu}$ encodes $x_{\nu}, y_{\nu}, z_{\nu}$ in the same way mentioned in subsec:\ref{subsec: encode} and constructs encoded matrix $\ewit_{x_{\nu}}, \ewit_{y_{\nu}}, \ewit_{z_{\nu}}$.
	\item $\prover_{\nu}$ commits to $\ewit_{x_{\nu}}, \ewit_{y_{\nu}}, \ewit_{z_{\nu}}$ in the same way mentioned in subsec:\ref{subsec: commit} and constructs commitment vector $\comoracle_{x_{\nu}}, \comoracle_{y_{\nu}}, \comoracle_{z_{\nu}}$.
\end{itemize}
\end{comment}
%Set $\sum_{\nu \in [\Num]} \comoracle_{x_{\nu}} = \comoracle_x$, $\sum_{\nu \in [\Num]} \comoracle_{y_{\nu}} = \comoracle_y$ and $\sum_{\nu \in [\Num]} \comoracle_{z_{\nu}} = \comoracle_z$ as the oracles.
\begin{figure}[h!]
\centering
	\begin{framed}
			\begin{itemize}
			\item {$\quadcheckTwoD (\FF, \GG, L[n,k,d], m, t, \bm{g}_x, \bm{g}_y, \bm{g}_z, [\pi]; \wit_x, \wit_y, \wit_z)$}:
			\item {\bf Relation}: $\exists \{\wit_{x_{\nu}}, \wit_{y_{\nu}}, \wit_{z_{\nu}}, \ewit_{x_{\nu}}, \ewit_{y_{\nu}}, \ewit_{z_{\nu}} : u\in [t]\}$ such that $\sum_{\nu \in [\Num]} \wit_{a_{\nu}} = \wit_a$ for $a\in \{x,y,z\}$ and $\sum_{\nu \in [\Num]} \ewit_{a_{\nu}} = \ewit_a$ for $a\in \{x,y,z\}$ $(\ewit_x, \ewit_y, \ewit_z) = \open(\pi)$ and $\ewit_a = \enc(\wit_a)$ for $a\in \{x,y,z\}$ and $\wit_x \circ \wit_y = \wit_z$.
			%\item {\bf Oracle Setup}: The prover $\prover$ computes $\ewit_a = \enc(\wit_a)$ and $\comoracle_a = \com(\ewit_a)$ for $a\in \{x,y,z\}$. It sets $\pi := [\comoracle_{x} || \comoracle_{y} || \comoracle_{z}]$ where the notation denotes vertical stacking of the vectors. Note that the generators used to construct commitment vectors for $x, y$ and $z$ should be independent. So, $\prover$ and $\verifier$ together generates $\bm{g}_x$, $\bm{g}_y$ and $\bm{g}_z$.
			\item {\bf Oracle Setup}: 
			\begin{itemize}
				\item The prover $\prover_{\nu}$ computes $\ewit_{a_{\nu}} = \enc(\wit_{a_{\nu}})$ and $\comoracle_{a_{\nu}} = \com(\ewit_{a_{\nu}})$ as in Sections ~\ref{subsec:encode2D} and ~\ref{subsec: commit2D} for $\nu \in [\Num]$ and sends $\comoracle_{a_{\nu}}$ to $\Ag$ $\forall a\in \{x,y,z\}$.
				\item $\Ag$ sets $\comoracle_a = \prod_{\nu \in [\Num]} \comoracle_{a_{\nu}}$ for $a\in \{x,y,z\}$. Then, sets $\pi := [\comoracle_x || \comoracle_y || \comoracle_z]$ as the oracle.
			\end{itemize}
		\end{itemize}
		\begin{enumerate}
			\item $\verifier \rightarrow \Ag : $ $\verifier$ picks a random $r\in \bbF^m$ and sends $r$ to $\Ag$.
	
			\item $\Ag \rightarrow \prover_{\nu} : $ $\Ag$ forwards $r$ to all the provers $\prover_{\nu}$.
			
			\item MPC\text{ among the provers}:  Provers interact to compute the polynomial $Q^{xy}_i(\cdot) = Q^{x}_i(\cdot)\cdot Q^{y}_i(\cdot)$ with the inputs $Q^{x_{\nu}}_i(\cdot)$ and $Q^{y_{\nu}}_i( \cdot )$. and $\prover_{\nu}$ gets the output $Q^{{xy}_{\nu}}_i(\cdot)$ such that $\sum_{\nu \in [\Num]} Q^{{xy}_{\nu}}_i(\cdot) = Q^{xy}_i(\cdot)$ $\forall i\in [m]$.
			
			\item $\prover_{\nu} \rightarrow \Ag : $ $\prover_{\nu}$ computes the polynomial $p_{\nu}(\cdot) = \sum_{i\in [m]} [r_i \cdot (Q^{{xy}_{\nu}}_i(\cdot)  - Q^{z_{\nu}}_i(\cdot))]$ and sends $p_{\nu}(\cdot)$ to $\Ag$.
	
			\item $\Ag \rightarrow \verifier : $ $\Ag$ aggregates the polynomials and computes $p(\cdot) = \sum_{\nu \in [\Num]} \prover_{\nu}(\cdot)$ and sends $p(\cdot)$ polynomial to the verifier $\verifier$.
			
			\item $\verifier \rightarrow \Ag : $ The veifier $\verifier$ sends $t$ randomly picked sampled indexes $Q= \{j_u: u\in[t]\}$ from $[n]$. 
			
			\item $\Ag \rightarrow \prover_{\nu} : $ $\Ag$ forwards $Q$ to the provers.
			
			\item Oracle Queries: $\verifier$ queries the oracle $\pi$ with $\{j_u:u\in[t]\}$.
			
			\item Oracle Answers: The oracle responds with $\pi[j_u], u\in[t]$.
	
			\item $\prover_{\nu} \rightarrow \Ag : $ $\prover_{\nu}$ sends $\ewit_{x_{\nu}}[\cdot, j_u], \ewit_{y_{\nu}}[\cdot, j_u]$ and $\ewit_{z_{\nu}}[\cdot,j_u]$ $\forall u\in[t]$.
	
			\item $\Ag \leftrightarrow \verifier : $ 
			\begin{itemize} 
				\item $\Ag$ computes $\ewit_{x}[\cdot, j_u] = \sum_{\nu \in [\Num]} \ewit_{x_{\nu}}[\cdot, j_u]$, $\ewit_{y}[\cdot, j_u] = \sum_{\nu \in [\Num]} \ewit_{y_{\nu}}[\cdot, j_u]$ and $\ewit_{z}[\cdot, j_u] = \sum_{\nu \in [\Num]} \ewit_{z_{\nu}}[\cdot, j_u]$.
				
				\item $\innerproduct(\GG,\bm{g}'||\bm{g}_r, \bm{g}_y||(\bm{g}_z)^{-1}, \comoracle_x[j_u]\cdot (\bm{g}_r)^r, p(\eta_{j_u}), \comoracle_y\cdot (\comoracle_z)^{-1}; r\circ\ewit_x[\cdot,j_u]||r, \ewit_y[\cdot,j_u]||-\ewit_z[\cdot,j_u])$ $\forall u\in [t]$,
				
				Where $\bm{g}'= (g'_1, \ldots, g'_m)$ is such that $g'_i = g_x^{r^{-1}}$ 
				$\verifier$ proceeds if the arguments succeed for all $u \in [t]$.
			\end{itemize} 
	
			\item $\verifier$ checks if $p(\zeta_j) = 0$ $\forall j \in [l]$. If yes then accepts, else rejects.
		\end{enumerate}
	\end{framed}
	\caption{Quadratic Check for $\name$2D in $\DPZK$ setting}
\end{figure}
Run the complete protocol mentioned ~\ref{completeprotocol} before, using these distributed version of the checks.

\paragraph{Correctness: } It is easy to see that correctness holds if all the provers together possess the extended witness $\extwit$. 

\paragraph{Soundness: } The following theorem proves that if the provers do not have a valid witness for the statement $\stmt$, then the verifier rejects the proof with very high probability.
\pnote{Replace $\cm_{\extwit||x||y||z}$ by the oracle $\pi$}
\begin{theorem}
	For $e < \frac{d}{3}$, if $\innp{\Pi^*(\pi,\calU^*) }{\verifier^{\pi}} \rightarrow 1$, then there is an expected $\ppt$ $\extrac^{\Pi^*}(\pi) \rightarrow \calU^*$ such that with overwhelming probability $\calU^*$ satisfies one of the following events:
	\begin{itemize}
		\item $\com(\calU^*)=\cm_{\extwit||x||y||z}$ and $A\extwit = x \wedge B \extwit = y \wedge C \extwit = z \wedge x\circ y =z \wedge P_{add} \extwit = 0$
		\item $\extrac$ breaks the binding property of the commitment scheme.
		
	\end{itemize}
	Let $\cm$ is the commitment of $\calU^*$, computed by the aggregator $\Ag$ in the above protocol. If $\verifier$ accepts the argument, then there is an expected $\ppt$ extractor $\extrac$ having rewinding access to $\Ag$ with polynomially many rewindings either outputs a correct $\extwit$ or breaks the binding property of the commitment scheme.
\end{theorem}

\begin{proof}
	The proof is the same as the proof of Theorem 2.
\end{proof}

\paragraph{Zero-knowledge: } The protocol has zero-knowledge property after blinding, which was used for single prover protocol as $\verifier$ is not learning anything more than what he was learning in the standard(single prover) zero-knowledge protocol described before. The same theorem follows here.

\paragraph{Privacy among the provers: }  In the oracle construction phase, provers will interact to get the shares of zero. Shares of zero are required to achieve the hiding of the polynomials send by the provers to the aggregator. In single prover case, the verifier is not learning anything about the polynomials constructed by the prover in the $\linearcheckTwoD$ and $\quadcheckTwoD$ because $q^{lin}(\cdot)$ is a random polynomial which adds up to zero when evaluated on $\bm{\zeta}$ and $q^{quad}(\cdot)$ is a random polynomial which evaluates to zero for all $\zeta \in \bm{\zeta}$. However, simply blinding the same way for each of the provers in multi prover setting is not enough. 
Since witness shares for each party separately need not satisfy the constraints (linear and quadratic).  So, evaluation on $\bm{\zeta}$ is additional information to $\Ag$. So, each prover needs to add shares of zero to their polynomial. To avoid that, for both $p_{\nu}^{lin}(\cdot)$ and $p_{\nu}^{quad}(\cdot)$ each prover, in addition to corresponding hiding polynomials add shares of zero to each coefficient, so that all the polynomials have the same distribution as random polynomial, and when the aggregator adds the polynomials, the aggregated polynomial is correct and random from the appropriate set of polynomials.
The following theorem proves that the above protocol has the $t$-privacy among the provers property if the MPC used in quadratic check is secure under $t$-corruption.

\begin{theorem}
	Let out of the $\Num$ provers, $t$ of them are corrupt, and $\Ag$ is one of the provers who may be corrupt. If the MPC used in the quadratic check for the function that outputs $Q^{xy}_i(\cdot)$ is secure under $t$ malicious adversary, then the distributed proof generation protocol described above has the privacy among the provers property. 
\end{theorem} 

\begin{proof}
	Let $T\subset [\Num]$ is the set of corrupt parties.
	We will discuss two cases: 
	
	Case I: When $\Ag$ is not a corrupt party.
	Then other than $MPC$ for the function that outputs $Q^{xy}_i(\cdot)$, there is no interaction among the provers. So the protocol is secure only if the above MPC is secure. By the hypothesis of the theorem, the protocol is secure.
	
	Case II: When $\Ag$ is corrupt.
	To prove the secrecy, we will design a simulator. 
	By the hypothesis of the theorem, the MPC used is secure; therefore, there is a simulator $\Sim_{M}$, which can generate a transcript of the interactions for the MPC.
	Moreover, As the protocol is zero-knowledge, a simulator designed in ~\ref{subsec:zeroknowledge} gives a simulator, say $\Sim_{Z}$.
	
	Now we will design a simulator $\Sim$ using $\Sim_{M}$ and $\Sim_{Z}$.
	
	$\Sim$ has input $\{\stmt, \{\extwit_{\nu}\}_{\nu \in T}\}$.
	
	$\Sim$ constructs $\ewit_{\extwit_{\nu}}, \comoracle_{\extwit_{\nu}}$, $\ewit_{x_{\nu}}, \comoracle_{y_{\nu}}$, $\ewit_{y_{\nu}}, \comoracle_{y_{\nu}}$ and $\ewit_{z_{\nu}}, \comoracle_{z_{\nu}}$ $\forall \nu\in T$.
	And computes:
	
	\noindent $\comoracle_{\extwit_{T}} = \sum_{\nu \in T} \comoracle_{\extwit_{\nu}}$,
	$\comoracle_{x_{T}} = \sum_{\nu \in T} \comoracle_{x_{\nu}}$,	
	$\comoracle_{y_{T}} = \sum_{\nu \in T} \comoracle_{y_{\nu}}$,	
	$\comoracle_{z_{T}} = \sum_{\nu \in T} \comoracle_{z_{\nu}}$
	
	\noindent $\Sim$ calls the simulator $\Sim_{Z}$ on input $\stmt$ and gets 
	\begin{align*}
	\tau' = \{
	& \comoracle_{\extwit||x||y||z},\\
	& \gamma(\in_R \bbF^{m}), r_1(\in_R \bbF^{ml}), r_2(\in_R \bbF^m), \\ 
	& (\w' = \w+\w_{blind}), (q^{lin}(\cdot) = p^{lin}(\cdot)+p^{lin}_{blind}(\cdot)),\\ 
	&(q^{quad}(\cdot) = p^{quad}(\cdot) + p^{quad}_{blind}(\cdot)), \text{ }Q (|Q|=t),\\
	& \ewit_{\extwit}[\cdot, Q], \ewit_{x}[\cdot, Q], \ewit_{y}[\cdot, Q], \ewit_{z}[\cdot, Q]
	\}
	\end{align*}
	$\Sim$ computes 
	
	$\comoracle_{\extwit_{\overline{T}}} = \comoracle_{\extwit}-\comoracle_{\extwit_{T}}$
	and picks $\Num - t$ random values such that these values add up to $\comoracle_{\extwit_{\overline{T}}}$
	
	$\comoracle_{x_{\overline{T}}} = \comoracle_{x}-\comoracle_{x_{T}}$
	and picks $\Num - t$ random values such that these values add up to $\comoracle_{x_{\overline{T}}}$
	
	$\comoracle_{y_{\overline{T}}} = \comoracle_{y}-\comoracle_{y_{T}}$
	and picks $\Num - t$ random values such that these values add up to $\comoracle_{y_{\overline{T}}}$
	
	$\comoracle_{z_{\overline{T}}} = \comoracle_{z}-\comoracle_{z_{T}}$
	and picks $\Num - t$ random values such that these values add up to $\comoracle_{z_{\overline{T}}}$
	
	Similarly $\Sim$ generates the responses for the honest provers for $\w', q^{lin}(\cdot)$. 
	
	For generating the shares of $q^{quad}(\cdot)$, $\Sim$ calls $\Sim_{M}$.
	
	Furthermore, the remaining shares $\Sim$ generates using the transcript generated by $\Sim_{Z}$, which gives a complete transcript of the interactions.
	
	This proves that the above protocol has the property of privacy among the provers. 
\end{proof}

\paragraph{Zero knowledge under collusion: } protocol described above has zero knowledge under collusion property. By theorem ~\ref{theo:equivalent}, as the protocol has zero knowledge and privacy among the provers property, which implies zero-knowledge under collusion.

\subsection{The pitfalls of 2D}
The need for 3D
