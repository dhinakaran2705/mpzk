\section{Construction of DPZK}
A simple way to construct a DPZK protocol is to start with a single-prover protocol and run MPC among the provers to generate each message to be sent to the verifier. The randomness in turn provided by the verifier can be broadcasted to the provers.
Hence, starting with a $\round$-round interactive zero-knowledge proof system $\innp{P = \{p_i\}_{i\in[\round]}}{V = \{v_i\}_{i \in [\round]}}$, we can obtain a DPZK protocol by having the provers run MPC for each round $i \in [\round]$ to compute the output of $p_i$. The recent literature suggests proof systems which additionally commit to oracles at each round (IOP \cite{aurora}) or only at the beginning (IPCP \cite{ligero}). The verifier can additionally query the oracle on his positions of interest. The DPZK protocol based on a single-prover IOP protocol should also support homomorphism in these oracles for the proof size to be independent of the number of parties involved.

%We will formalize this simple DPZK transformation based on the Polynomial IOP framework \cite{polynomialiop,ahp} proposed recently. This framework captures most of the protocols from the literature on practical zero-knowledge proofs [CHECK precision of this sentence]. Writing down this formalism simplifies our future sections on the construction of DPZK based on the single prover protocols.



%<change from here>

We will first discuss how the state-of-art single prover zero-knowledge proof protocols can be made to support distributed proof generation.
The literature on single prover zero-knowledge proofs with transparent setup has undergone tremendous progress over the last few years in terms of the concrete proof sizes. The literature has also introduced an ``updatable SRS'' setting \cite{sonic, libra, supersonic} to reduce the trust in the trusted setup while obtaining reasonable proof sizes (in between the trusted \cite{pinnochio_PHGR} and the untrusted setup \cite{aurora, bulletproofs}). Table \ref{tab:SPZK} summarizes the state-of-art protocols in each dimension: proof size, prover and verifier complexities.

Among these works, only Bulletproofs \cite{bulletproofs} discuss distributed proof generation. Even in \cite{bulletproofs}, this discussion was limited to range proofs. In this section, we will discuss how we adapt these protocols to support distributed proof generation. %Looking ahead, all these protocols will be expensive in terms of the complexity of the provers. This will lead us to next section, where we will present our DPZK protocol \name{} which performs better than .
%Note that any protocol can be made to support distributed proof generation by compiling the prover algorithm with an MPC protocol. In this section we will identify the steps in these ZK protocols which can be performed over the shares of the protocol and try to minimize the steps which need to be run using an MPC.

%\subsection{DPZK-Aurora}


\subsection{DPZK-Spartan}\label{subsec:spartan}
Spartan \cite{spartan} is a zkSNARK protocol supporting arbitrary circuits with a transparent setup, $O(|\C|^{1/c})$ proof size for $c \geq 2$, an amortized sublinear verifier complexity. Spartan has three high level steps:
\begin{enumerate}
\item Construct a low-degree multivariate polynomial $\tilde{G}$ such that circuit satisfiability can be checked by performing a sum-check protocol \cite{sumcheck} over this polynomial.
\item Use the polynomial commitment scheme from \cite{hyrax} to obtain a commitment of this polynomial from the prover. The traditional sum-check protocol involves the prover sending partial evaluations on this polynomial to the verifier, but this would incur a $O(|\C|)$ proof size due to the degree and the number of variables in the polynomial. Instead, the Spartan prover proves these evaluations to the verifier using the polynomial commitment scheme to obtain a succinct proof.
\item Further use polynomial commitments to achieve sub-linear verification time amortized over multiple verifications on the same circuit.
\end{enumerate}
Spartan encodes the extended witness vector $\extwit$ as a polynomial $Z(\cdot)$ such that $Z(i)$ provides the value at wire $i$. Let $\tilde{Z}$ be the polynomial extension of $Z$.
The first step, the construction of the polynomial $\tilde{G}$, involves a multiplication of the polynomial $\tilde{Z}$ on some set of $|\C|$ terms (from $\log |\C|$ variables $u_1$) with itself on $|\C|$ terms (from a different distinct set of $\log |\C|$ variables $u_2$). $\tilde{G}$ will be evaluated on random points in the latter steps. This multiplication is the (only) step which will require an MPC protocol to obtain DPZK-Spartan. 

The multilinear extension $\tilde{Z}(\cdot)$ can be obtained by an inner product of the extended witness vector $\extwit$ with a vector of polynomials on the input variables. Hence, to enable the computation of $\tilde{Z}(u_1) \cdot \tilde{Z}(u_2)$, the prover needs to compute the pairwise product of the elements of $\extwit$. In the $\DPZK$ setting, with $\extwit$ being additively shared among the provers, $O(|\C|^2)$ MPC multiplications have to be run by the provers. 
\dnote{is the above explanation too succinct?}
\dnote{check if the multiplication of two sets of $|\C|$ terms takes $|\C|^2$ MPC mults even if they are got from $\log |\C|$ variables..}
%------ Do we need Ligero in this section? since Ligero is subsumed by Aurora in all aspects of SPZK---
%\subsection{DPZK-Ligero}
%We discuss this because this serves as the background to our protocol.
%%Discuss the version with proof size depending on and growing with the number of provers. 

\subsection{DPZK-Bulletproofs}\label{subsec:bulletproofs}
.\dnote{TODO: circuit share complexity for the Bulletproofs protocol for arbitrary arithmetic circuits}
Consider the setting of the distributed prover, i.e., that witness is distributed among the provers, say $\Num$ provers together hold the witness. Here we will discuss the distributed prover version of the bulletproofs for R1CS circuits. We will start by describing the distributed prover version of zero-knowledge inner produc; then, we will use the protocol to design a protocol for R1CS.

We will start with reducing a R1CS circuit into multiple inner product. Then we will use the zero knowledge inner product to construct the proof of R1CS circuit. Consider that the circuit has $n$ multiplcation gates. Corresponding to the $j^{th}$ wire let $x_j, y_j$ and $z_j$ are the left input, right input and output wires respectively. Then the vectors $\bm{x}, \bm{y}, \bm{z}\in \bbF^n$ can be obtained from the extended witness $\extwit$, by multiplying public matrices $A, B, C$ which is depended only on the circuit. Therefore we have 
\begin{align}
A \extwit = \bm{x} \\
B \extwit = \bm{y} \\
C \extwit = \bm{z} 
\end{align}
and $\bm{x}, \bm{y}, \bm{z}$ will satisfy the following eqution:
\begin{equation}
	\bm{x} o \bm{y} =\bm{z}
\end{equation}
For any $\bm{r}\in \bbF^n$, from equation (1), (2), (3) and (4), the following conditions are true:
\begin{align*}
\langle \bm{r}^T \cdot A, \extwit \rangle = \langle \bm{r}, \bm{x} \rangle = v_1 \\
\langle \bm{r}^T \cdot B, \extwit \rangle = \langle \bm{y}, \bm{r} \rangle = v_2 \\
\langle \bm{r}^T \cdot C, \extwit \rangle = \langle \bm{r}, \bm{z} \rangle = v_3 \\
\langle \bm{r} o \bm{x}, \bm{y} \rangle = \langle \bm{r}, \bm{z} \rangle = v_4 
\end{align*} 
We have zero knowledge inner product argument protocol for the following instance:
\begin{equation} 
\{(P,V,\bm{g},\bm{h},g,h);(\bm{a},\bm{b},\sigma,\delta): P=h^{\sigma}\bm{g}^{\bm{a}}\bm{h}^{\bm{b}}, V=h^{\delta}g^{\langle \bm{a},\bm{b}\rangle}\} 
\end{equation}
where the public part is $\{(P,V,\bm{g},\bm{h},g,h)$ and secret to the prover is $(\bm{a},\bm{b},\sigma,\delta)$.\\
Define: $v=\langle \bm{a},\bm{b}\rangle$\\
%Finally proving the circuit satisfiability for R1CS can be viewed as 
%$$\{(P_1,P_2,P_3,P_4,P_5,P_6,P_7,P_8, V_1,V_2,V_3,V_4,V_5,V_6,V_7,V_8, P_1=)\}$$
So if we apply the above protocol on these 8 inner products that gives a proof of circuit satisfiability of R1CS, for example consider $\langle \bm{r}^T \cdot A, \extwit \rangle = v_1$, in this case consider $\bm{a}=\bm{r}^T\cdot A$, $\bm{b}=\extwit$ and $v=v_1$, and use the protocol (9) in these setting.
\subsubsection{Multiprover version of zero knowledge inner product argument}
Now we will discuss the multiprover version of the zero knowledge inner product proof: where there will an aggregator $\Ag$(need not be trusted) who will iinteract with the verifier.\\
Consider $\prover_1,\ldots, \prover_{\Num}$ are provers for the inner product statement i.e. $\{(P,V,\bm{g},\bm{h},g,h)\}$ and the corresponding witness is $(\bm{a},\bm{b}, \sigma, \delta)$, where $P= h^{\sigma} \bm{g}^{\bm{a}} \bm{h}^{\bm{b}}$, $V=h^{\delta}g^{v}$. Let party $\prover_i$ has $(\bm{a}_i,\bm{b}_i,\sigma_i,\delta_i)$ such that
\begin{align}
	\sum\limits_{i=1}^{N}\bm{a}_i=\bm{a},\text{ }
	\sum\limits_{i=1}^{N}\bm{b}_i=\bm{b},\text{ }
	\sum\limits_{i=1}^{N}\sigma_i=\sigma,\text{ }
	\sum\limits_{i=1}^{N}\delta_i=\delta \text{ }
\end{align}
\begin{enumerate}
	\item In this step each party chooses their blinding vectors and commits to that vector. $\prover_i$ samples $\bm{s}^i_L, \bm{s}^i_R \sample \bbZ_p^n$ and $\rho_i\sample \bbZ_p$, then computes $S^i=h^{\rho_i}\bm{g}^{\bm{s}^i_L}\bm{h}^{\bm{s}^i_R}$ and sends $S^i$ to $\Ag$.
	\item In this step $\Ag$ computes the commitment of the blinding vector, which is sum of all the provers blinding vector. $\Ag$ computes $S = \prod_{i=1}^{\Num}S^i = h^{\sum_{i=1}^{\Num}\rho_i}\bm{g}^{\sum_{i=1}^{\Num}\bm{s}_L^i}\bm{h}^{\sum_{i=1}^{\Num}\bm{s}^i_R} = h^{\rho}\bm{g}^{\bm{s}_L}\bm{h}^{\bm{s}_R}$, where $\rho = \sum_{i=1}^{\Num}\rho_i$, $\bm{s}_L =\sum_{i=1}^{\Num} \bm{s}^i_L,  \bm{s}_R=\sum_{i=1}^{\Num}\bm{s}^i_R$, and $\Ag$ sends $S$ to the verifier $\verifier$.
	\item Prover $\prover_i$ constructs a vector polynomial $l_i(X)=\bm{a}_i + X. \bm{s}^i_L$ and $r_i(X)=\bm{b}_i + X. \bm{s}^i_R$. All the provers involve in an MPC to obtain the shares of the polynomial $T(X)=\langle \sum_{i=1}^{\Num} l_i(X),\sum_{i=1}^{\Num} r_i(X)\rangle$. Let $\prover_i$ gets the share $T_i(X)$ of $T(X)$, where $T(X)=\sum_{i=1}^{\Num}T_i(X)$. Let $T_i(X)= t^i_0+t^i_1.X+t^i_2.X^2$, then $\sum_{i=1}^{\Num} t^i_0=\langle \bm{a}, \bm{b} \rangle$. Now $\prover_i$ commits to $t^i_j$ using randomness $\tau^i_j$ i.e. $T^i_j=h^{\tau^i_j}g^{t^i_j}$ and sends $T^i_j$ to $\Ag$ for $j\in \{1,2\}$.
	\item $\Ag$ computes $T_j=\prod_{i=1}^{\Num} T^i_j$ which gives the commitment of $t_j \forall j\in\{1,2\}$ and sends $T_1,T_2$ to the verifier $\verifier$.
	\item $\verifier$ samples $x\leftarrow_\$\bbZ^*_p$ and sends it to the aggregator $\Ag$.
	\item $\Ag$ sends the $x$ to the provers. 
	\item $\prover$ computes $\bm{l}_i=l_i(x), \bm{r}_i=r_i(x)$, $\mu_i= \sigma_i+\rho_i.x$, $\tau^i_x= \delta_i+\tau^i_1.x+\tau^i_2.x^2$ and sends $\bm{l}_i,\bm{r}_i, \mu_i, \tau^i_x$ to $\Ag$.
	\item $\Ag$ defines $\bm{l}=\sum_{i=1}^{\Num}\bm{l_i}$ and $\bm{r}=\sum_{i=1}^{\Num}\bm{r_i}$ and $\hat{t}=\langle \bm{l},\bm{r}\rangle =T(x)$. $\Ag$ computes $Q=\bm{g^lh^r}$, $\mu=\sum_{i=1}^{\Num} \mu_i$ and $\tau_x = \sum_{i=1}^{\Num}\tau_x^i$. And finally $\Ag$ sends $Q, \mu, \tau_x, \hat{t}$ to $\verifier$.
\end{enumerate}
$\verifier$ checks: 
\begin{enumerate}
	\item $h^{\tau_x}g^{\hat{t}} \stackrel{?}{=} V\cdot T_1^x\cdot T_2^{x^2}$ to check the correct evaluation of the committed polynomial $T$ at $x$.
	\item $P\cdot S^x \stackrel{?}{=} h^{\mu}\cdot Q$. 
	\item Runs the inner product proof from \cite{bulletproofs}, which need not be zero knowledge on the input $(Q, \hat{t}, \bm{g}, \bm{h}, g)$ with the aggregator $\Ag$.
\end{enumerate}
The verifier $\verifier$ accepts the proof if the checks succeed and the inner product argument accepts.

The completeness, soundness and zero knowledge hold and the proof is similar to zero knowledge inner product argument.
\pnote{are we adding Nitin's write up of zk-inner product?}
We can argue the privacy by saying that provers are interacting only to get the shares of $T(X)$ and that is being done using secure MPC, and remaining all the interactions are done with the aggregator $\Ag$ only. Who is learning nothing more than whatever $\verifier$ learns in zero knowledge inner produt argument. 
$\Ag$ has commitments of the shares of $s_L$ and $s_R$, and computes the commitments of $s_L$ and $s_R$, hiding property of the commitment ensures that $\Ag$ learns no information about the secrets $s_L, s_R$ as well as their shares.

$\Ag$ receives the commitments of the coefficients of the shares of the polynomial $T(X)$, and reconstructs the commitments of the coefficients of $T(X)$, again hiding property of the commitment ensures that $\Ag$ learns no information about the polynomial $T(X)$.

In the next step, $\Ag$ gets $\hat{l}, \hat{r}, \mu, \tau_x$, which verifier gets in single prover zero knowledge inner product argument, the zero knowledge property of the above ensures that $\hat{l}, \hat{r}, \mu, \tau_x$ is not leaking any information about the secrets.

Therefore $\Ag$ is not learning anything new. Which implies privacy of the provers are preserved.

\subsubsection{optimization}
We can combine the inner product $\langle \bm{r}^T \cdot A, \extwit \rangle = \langle \bm{r}, \bm{x} \rangle = v_1$ and $\langle \bm{r}^T \cdot C, \extwit \rangle = \langle \bm{r}, \bm{z} \rangle = v_3$ which can be represented as $\langle \bm{r}^T \cdot(\alpha.A+\gamma.C), \extwit \rangle = \langle \bm{r}, (\alpha.\bm{x}+\gamma.\bm{z}) \rangle = v_3$ where $\alpha, \gamma$ are randomly chosen by the verifier $\verifier$. In the commitment we are considering that if $\langle \bm{a},\bm{b}\rangle= c$ is in the statement of the inner product, then the commitment of $\bm{a}$ and $\bm{b}$ will be of the form $h^{\sigma} \bm{g}^{\bm{a}} \bm{h}^{\bm{b}}$. Note that in the commitment there should not be any known relation between the generators and for the equation $\langle \bm{r} o \bm{x}, \bm{y} \rangle = \langle \bm{r}, \bm{z} \rangle = v_4$ there should not be the known relation between the generator $\bm{g}$ which is used to commit to $\bm{x}$ with the generator which is used to commit to $\bm{y}$, for that reason the commitment for $\bm{y}$ is considered in left in $\langle \bm{y}, \bm{r} \rangle = v_2$. For that reason this inner product we could not batch with the prior inner products. In this case we need to give 6 inner product proofs instead of 8. 
\dnote{To Protik: Rewrite the section below}.

\subsection{DPZK of Bulletproof for R1CS from their direct construction:}
Following \cite{BCC+16}, \cite{bulletproofs} presented a proof for a Hadamard-product relation. Suppose for a multiplication $g$ gate in a circuit $\C$ has $\bm{a}_L$, $\bm{a}_R$ and $\bm{a}_O$ as left input, right input and output wire values respectively, then $\bm{a}_L \circ \bm{a}_R = \bm{a}_O$ holds. Other than this, these vectors are supposed to satisfy additional $Q\leq 2n$ linear constraints, which is of the following form:
$$\innp{\bm{w}_{L,q}}{\bm{a}_L} + \innp{\bm{w}_{R,q}}{\bm{a}_R} + \innp{\bm{w}_{O,q}}{\bm{a}_O} = c_q$$
for $1\leq q \leq Q$, with $\bm{w}_{L,q}, \bm{w}_{R,q}, \bm{w}_{O,q} \in \ZZ^n_p$ and $c_q \in \ZZ_p$.
Consider\\ $W_L = [\bm{w}_{L,1}, \ldots, \bm{w}_{L,Q}]^T$,\\ 
$W_R = [\bm{w}_{R,1}, \ldots, \bm{w}_{R,Q}]^T$, \\
$W_O = [\bm{w}_{O,1}, \ldots, \bm{w}_{O,Q}]^T$,\\
and $\bm{c} = [c_1, \ldots, c_q]^T$\\
\pnote{what is $W_v$ and $\bm{v}$ is not clear.}

Here we are following the same notations which are used in \cite{bulletproofs}.
For arithmetic circuit satisfiability zero knowledge proof, \cite{bulletproofs} proves the following statement:

\noindent{\bf Statement:} $\{g,h\in\GG, \bm{g},\bm{h}\in \GG^n, \V\in \GG^m, W_L, W_R, W_O \in \ZZ^{Q\times n}_p, \bm{c}\in \ZZ^Q_p\}$

\noindent{\bf Witness:} $\{\bm{a}_L, \bm{a}_R, \bm{a}_O \in \ZZ^n_p, \bm{v},\bm{\gamma}\in \ZZ^m_p \}$

\noindent{\bf Relation:} $V_j = g^{v_j}h^{\gamma_j} \forall j\in[m] \wedge \bm{a}_L \circ \bm{a}_R = \bm{a}_O \wedge W_L\cdot \bm{a}_L + W_R\cdot \bm{a}_R + W_O\cdot \bm{a}_O = W_v\cdot \bm{v} +\bm{c}$

\noindent In \cite[Protocol 3]{bulletproofs} is explained how to generate the proof for an arbitrary arithmetic circuit. Here we will describe how to convert the protocol given in \cite[Protocol 3]{bulletproofs}. Let $\Num$ parties are there such that $\prover_i$ has $\bm{a}_{L,i}, \bm{a}_{R,i}, \bm{a}_{O,i}$ and $\bm{v}_{i}, \bm{\gamma}_i$ as the shares of the witness such that 
$\sum_{i\in[\Num]} \bm{a}_{L,i} = \bm{a}_L$,
$\sum_{i\in[\Num]} \bm{a}_{R,i} = \bm{a}_R$,
$\sum_{i\in[\Num]} \bm{a}_{O,i} = \bm{a}_O$,
$\sum_{i\in[\Num]} \bm{v}_{i} = \bm{v}$,
$\sum_{i\in[\Num]} \bm{\gamma}_{i} = \bm{\gamma}$. 

Among the provers one prover is chosen arbitrarily as the aggregator $\Ag$.

\begin{enumerate}
	\item $i$th prover randomly picks $\alpha_i, \beta_i, \rho_i \sample \ZZ_p$ and $\bm{s}_{L,i}, \bm{s}_{R,i} \sample \ZZ_p^n$ to blind there inputs, and computes $A_I^i = h^{\alpha_i}\bm{g}^{\bm{a}_{L,i}}\bm{h}^{\bm{a}_{R,i}}$, $A^i_O = h^{\beta_i}\bm{g}^{\bm{a}_O}$ and $S^i = h^{\rho}\bm{g}^{\bm{s}_{L,i}}\bm{h}^{\bm{s}_{R,i}}$
	and sends $A^i_I, A^i_O, S^i$ to the aggregator $\Ag$, $\forall i\in [\Num]$.
	
	\item $\Ag$ computes $A_I = \prod_{i \in [\Num]} A_I^i$, $A_O = \prod_{i \in [\Num]} A_O^i$ and $S = \prod_{i \in [\Num]} S^i$ and sends to $\verifier$.
	
	\item $\verifier$ picks $y,z \sample \ZZ^*_p$ and sends to $\Ag$.
	
	\item $\Ag$ forwards $y,z$ to all the provers.
	
	\item All the provers and the verifier computes :\\
	$\bm{y}^n = (1, y, y^2, \ldots, y^{n-1}) \in \ZZ^n_p$\\
	$\bm{z}^{Q+1}_{[1:]} = (z, z^2, \ldots, z^Q) \in \ZZ^Q_p$\\
	$\delta(y,z) = \innp{\bm{y}^{-n}\circ (\bm{z}^{Q+1}_{[1:]}\cdot W_R)}{\bm{z}^{Q+1}_{[1:]}\cdot W_L}$\\
	
	\item $\prover_i$ computes:
	if 
	\begin{align*}
	i=1 : &l_i(X) = \bm{a}_{L,i}\cdot X + \bm{a}_{O,i} \cdot X^2 + \bm{s}_{L,i} \cdot X^3 + \bm{y}^{-n}\circ (\bm{z}_{[1:]}^{Q+1} \cdot W_R)\cdot X \\
	&r_i(X) = \bm{y}^n\circ \bm{a}_{R,i}\cdot X + \bm{y}^n \circ \bm{s}_{R,i}\cdot X^3 - \bm{y}^n + \bm{z}^{Q+1}_{[1:]} \cdot (W_L\cdot X + W_O)\\
	i \neq 1 : &l_i(X) = \bm{a}_{L,i}\cdot X + \bm{a}_{O,i} \cdot X^2 + \bm{s}_{L,i} \cdot X^3\\
	&r_i(X) = \bm{y}^n\circ \bm{a}_{R,i}\cdot X + \bm{y}^n \circ \bm{s}_{R,i}\cdot X^3
	\end{align*}
	
	\item MPC phase: Provers interact among with $\prover_i$'s input $l_i(X)$ and $r_i(X)$ to compute the polynomial $t(X) = \innp{\sum_{i \in [\Num]}l_i(X)}{\sum_{i \in [\Num]}r_i(X)}$ and $\prover_i$ gets the output $t_i(X)$ such that $\sum_{i\in[\Num]} t_i(X) = t(X)$.
	and $t(X)= \sum_{j=1}^6 t_j\cdot X^j \in \ZZ_p[X]$ and $t^i(X)= \sum_{j=1}^6 t^i_j\cdot X^j \in \ZZ_p[X]$ $\forall i\in [\Num]$
	\item $\prover_i$ comutes $\bm{w}_i = W_L\cdot \bm{a}_{L,i} + W_R\cdot \bm{a}_{R,i} + W_O\cdot \bm{a}_{O,i}$
	
	\item $\prover_i$ picks $\tau^i_j \sample \ZZ_p$ $\forall j\in\{1,3,4,5,6\}$\\
	and computes $T_j^i = g^{t_j^i}h^{\tau_j^i}$ and sends $T_j^i$ to $\Ag$ for all $i\in [\Num]$ and $j\in \{1,3,4,5,6\}$.
	
	\item $\Ag$ computes $T_j = \prod_{i=1}^{\Num} T_j^i$ for $j \in \{1,3,4,5,6\}$ and sends these values to the verifier $\verifier$.
	
	\item $\verifier$ picks $x\sample \ZZ^ast_p$ and sends to $\Ag$.
	
	\item $\Ag$ forwards $x$ to all the provers.
	
	\item $\prover_i$ computes:
	\begin{itemize}
		\item $\bm{l}_i = l_i(x) \in \ZZ^n_p$
		\item $\bm{r}_i = r_i(x) \in \ZZ^n_p$
		\item $\tau_x^j = \sum\limits_{j=1, j\neq 2}^{6} \tau_j^i\cdot x^i \in \ZZ_p$
		\item $\mu_i = \alpha_i\cdot x +\beta_i \cdot x^2 +\rho_i\cdot x^3 \in \ZZ_p$
	\end{itemize}  
	sends these values to $\Ag$.
	
	\item $\Ag$ combines and gets 
	\begin{itemize} 
		\item $\bm{l} = \sum_{i\in[\Num]} \bm{l}_i$
		\item $\bm{r} = \sum_{i\in[\Num]} \bm{r}_i$
		\item $\hat{t} = \innp{\bm{l}}{\bm{r}}$
		\item $\tau_x = \sum_{i\in [\Num]} \tau_x^i + x^2\cdot \innp{\bm{z}_{[1:]}^{Q+1}}{W_V\cdot \gamma} \in \ZZ_p$
		\item $\mu = \sum_{i\in [\Num]} \mu_i$
	\end{itemize}
	
	\item $\Ag$ sends $\tau_x, \mu, \hat{t}$ to $\verifier$. 
	
	\item $\Ag \leftrightarrow \verifier$ run a subprotocol for the following inner product argument:
	\begin{itemize}
		\item $\innerproduct(\bm{g}, \bm{h'}, P\cdot h^{-\mu}, \hat{t}; \bm{l},\bm{r})$, where $\bm{h'}$ is such that $h'_i = h_i^{y^{-i+1}}$ $\forall i\in[n]$
	\end{itemize} 
\end{enumerate}

\subsection{Summary of existing protocols}
Argue why we need a better a single prover protocol amenable to multiple provers. 
--- Augment Table \ref{tab:SPZK} with the columns needed for multiple provers and use it for the argument--- 
\begin{comment}
\begin{center}\label{tab:SPZK}
	\begin{tabular}{|>{\centering}p{1.8cm}|p{2cm}|p{1.8cm}|>{\centering}p{1.3cm}|p{1.8cm}|p{1.2cm}|p{1.2cm}| } 
		\hline
		Protocols & \multicolumn{3}{c|}{Prover's complexity} & Verifier's complexity & Proof size & rounds \\
		\hline
		&computation & communi- cation & MPC rounds& & & \\ 
		\hline
		Ligero & $O(n\log n)M$ & NA & NA &$O(n)M$ & $O(\sqrt{n})$ & 4\\
		\hline
		Bulletproofs & $O(nE+q(n+m)M)$ & $O(2N-1)$ & $O(\log n)$& $O(nE+$ $q(n+m)M)$ & $O(\log n)$ & $O(\log n)$  \\
		\hline
		Spartan & & & & & &\\
		\hline
		Aurora & & & & & &\\
		\hline
		\name2D & $O(n \log n)M$  $+nE)$ & $O(2N-1)$ & 1 &$O((nM +$ $ n^{1-1/c}E))$ & $O(n^{1/c})$ & $O(\log n)$ \\
		\hline
		\name3D & $O(nM+nE)$ & $O(2N-1)$& 1 & $O(nM+$ $n^{1-2/c}E)$ & $O(n^{1/c})$ & $O(\log n) $\\
		\hline
	\end{tabular}
\end{center}
\end{comment}
\begin{center}\label{tab:SPZK}
	\resizebox{.98\textwidth}{!}{
	%\footnotesize
	\begin{tabular}{|c|c|c|c|c|}
		\toprule
		\centering 
		%Protocols & \multicolumn{3}{c|}{Prover's complexity} & Verifier's complexity & Proof size & rounds \\
		Parameters & Spartan & Bulletproofs & \name2D & \name3D\\
		\midrule
		Argument size & $N^{1/c}$ & $\log (N)$& $N^{1/c}$  & $N^{1/c}$\\
		\hline
		Rounds(ZK) & $\log (N)$ & $\log (N)$ & $\log (N)$ & $\log (N)$\\
		\hline 
		Prover's complexity & $O(N)E$ + $O(N\log(N))M$ & $O(N)E$ & $O(\frac{N}{\log(N)})E + O(N\log(N))M$ & $O(\frac{N}{\log(N)})E + O(N\log(N))M$ \\
		\hline
		Verifier'c Complexity& $O(N^{1-1/c})E^\ast$ & $O(N)E$ &$O(N^{1-1/c})E + O(N)M$ & $O(N^{1-2/c})E + O(N)M$\\
		\hline
		Prover communication& MPC($N_s^2,1$) + $\Num N^{\ast\ast}$& MPC($N_s,1$) + $\Num N$& MPC($N_s,1$) + $\Num N^{1-1/c}$& MPC($N_s,1$) + $\Num N^{1-1/c}$\\
		\hline 
		Number of MPC & $\log(N)$& 1 & 1 & 1 \\
		\bottomrule
	\end{tabular}
}
%\caption{Comparison table}
\end{center}

Here $N$ is the size of the circuit and, $c$ is a positive integer of our choice. $\Num$ is the number provers in the $\DPZK$ setting. 
$M$ is to indicate the amount of computation required for a single multiplication on the field elements, and $E$ is to indicate the amount of computation required for a single exponentiation on the group elements, in which $\mathsf{Dlog}$ is assumed to be hard. MPC$(x,y)$ denotes the number of MPC, for $x$ input function with depth-$y$ circuit, required for a protocol and $N_s$ denotes the size of the shared circuit, discussed in ~\ref{}.

$\ast$ Spartan \cite{spartan} achieves the above mentioned verifier's comlexity in the amortization, whereas without amortization is the verifier's complexity is $O(N)$ many exponentiations.

$\ast\ast$ It is not clear that \cite{spartan} can even be converted into $\DPZK$ with all privacy conditions, in the ~\ref{subsec:spartan}, we discussed that Spartan \cite{spartan} it is not provably secure as provers may leak privacy to the aggregator.