\subsection{Quadratic Check Protocol}\label{sec:quadcheck}
We now describe the IPCP which allows a prover to prove knowledge of 
$\wit_x$, $\wit_y$ and $\wit_z$ in $\FF^N$, satisfying $\wit_x\circ \wit_y =
\wit_z$. As before we view $\wit_x, \wit_y$ and $\wit_z$ as $p \times m \times s$ matrices, where $N=pms$. 
As described previously in Sections ~\ref{sec:witencoding} and ~\ref{sec:construct_oracle}, the prover obtains encodings $\ewit_a \leftarrow \enc(\wit_a)$ and corresponding 
commitments $\comoracle_a \leftarrow \ocom(\ewit_a)$ $\forall a\in \{x,y,z\}$. The prover
sets up the oracles $\pi_a := \comoracle_a$ for $a\in \{x,y,z\}$. For a query $Q$,
the verifier is provided the columns $\pi_a[\cdot,k]$ for $a\in \{x,y,z\}$ and $k\in Q$. 
Here we consider oracle as consisting of three sub-oracles for simplicity of description. For
better efficiency, $\wit_x$, $\wit_y$ and $\wit_z$ can be encoded together, and the oracle
access can be provided to the combined encoding. We defer this optimization till the 
discussion on concrete efficiency. The protocol appears in Figure~\ref{fig:quadcheck}. 

%Once again, the protocol requires the prover to construct encodings
%$\ewit_x=\enc(\wit_x)$, $\ewit_y=\enc(\wit_y)$ and $\ewit_z=\enc(\wit_z)$ as
%described in Section \ref{sec:witencoding}. Thereafter, the prover uses
%commitment scheme $\comm$ to commit to these encodings as $\comoracle_x =
%\comm(\ewit_x)$, $\comoracle_y = \comm(\ewit_y)$ and $\comoracle_z = \comm(\ewit_z)$. 
%The prover forms the oracle $\pi\in \GG^{3p\times n}$ by vertically stacking the
%$p\times n$ matrices $\comoracle_x,\comoracle_y$ and $\comoracle_z$. As before,
%for a query $Q\subseteq [n]$, the oracle answers with columns $\pi[\cdot,k]$ for
%$k\in Q$. The columns returned by the oracle can be parsed into constituent columns 
%$\comoracle_x[\cdot,k]$, $\comoracle_y[\cdot,k]$ and $\comoracle_z[\cdot,k]$
%canonically. We again discuss the key ingredients of the protocol.

\noindent{\em Reduction to Inner-product}: Let $Q^i_x,Q^i_y$ and $Q^i_z, i\in [p]$ be the
polynomials interpolating the $i^{th}$ slices of $\wit_x$, $\wit_y$ and $\wit_z$ respectively.
Then for vectors $\wit_x,\wit_y,\wit_z$ satisfying
$\wit_x\circ \wit_y=\wit_z$, the polynomials $Q^i=Q^i_x\cdot Q^i_y - Q^z_i$ interpolate
$\bm{0}^{m\times s}$ on the set $\{(\alpha_j,\zeta_k):j\in [m],k\in [s]\}$ for all $i\in [p]$. In three simple steps, we derive a simple probabilistic check below compressing in all the three dimensions. First, the above check reduces to   checking  that the polynomial $F := \sum_{i\in [p]}r_iQ^i$ interpolates
$\bm{0}^{m\times s}$ on the same set for a randomly sampled $r\in \FF^p$. 
Second, this further reduces to checking
$p(\cdot) := \sum_{j\in [m]}\gamma_jF(\alpha_j,\cdot)$ interpolates $\bm{0}^s$ on
$\overline{\bm{\zeta}}$ for randomly sampled $\gamma=(\gamma_1,\ldots,\gamma_m)\in \FF^m$.
Third, this further reduces to checking $\innp{\tau}{p(\overline{\bm{\zeta}})}=0$ for a random $\tau\in \FF^s$. 

\noindent{\em Intermediate Commitment:} As in the linear check protocol, we
construct $h\times n$ matrix $P$ which serves as a bridge between the oracles
$\pi_a$ $\forall a \in \{x,y,z\}$ and the vector $p(\overline{\bm{\zeta}})$.
For $j\in [h]$, let $p_j(\cdot)$ denote the polynomial $\sum_{i\in[p]}
r_iQ^i(\alpha_j, \cdot)$. We define the matrix $P$ by $P[j,k]= p_j(\eta_k)$ for
$j\in[h], k\in[n]$, which  is in the product code $\rsc{\eta}{n,2\ell-1}\otimes \rsc{\alpha}{h,2m-1}$. As
before we commit to $P$ by committing to the sub-matrix $\overline{P}$
consisting of the first $2m-1$ rows and the first $2\ell-1$ columns of $P$. Let
$(c_1,\ldots,c_{2\ell-1}) \gets \pccom(P)$.


\noindent{\em Consistency of $P$ and $p(\overline{\bm{\zeta}})$:} This part is almost same as the linear check. The differences are: (a) $p(\cdot) = \sum_{j\in[m]}p_j(\cdot)$ is a polynomial of degree $<2\ell-1$, (b) $\Phi$ is a $s \times 2\ell -1$ matrix that takes $p(\etabar)$ to $p(\zetabar)$, where $p(\etabar) = [p(\eta_1),\ldots, p(\eta_{2\ell-1})]$ and $p(\zetabar) = [p(\zeta_1),\ldots, p(\zeta_{s})]$.  
Similar to the linear check, the inner-product finally reduces to
$\innp{[\gamma||0^{m-1}]^T}{z}$, where $z= \beta P_0 + \overline{P} 
\varphi$ and $P_0\in \FF^{2m-1}$ is a random vector such that $P_0[j]=0$ for
$j\in [m]$, and $\varphi = \Phi\times \tau$. Commitment $c_0$ of $P_0$ is sent to the verifier a-priori. Therefore, commitment to $z$ is $\cm = c_0^{\beta}\prod_{k=1}^{2\ell-1}(c_k)^{\varphi_k}$.

\begin{comment} 
Let$\overline{\bm{\eta}}$ denote the vector $(\eta_1, \ldots, \eta_{2\ell})$, and
let $\Phi$ denote the $s\times 2\ell$ matrix such that
$p(\overline{\bm{\zeta}}) = \Phi p(\overline{\bm{\eta}})$. Observe
that $p(\overline{\bm{\eta}}) = \overline{P}^T [\gamma||0^m]^T$ and thus
$p(\zetabar) = \Phi \overline{P}^T [\gamma||0^m]^T$. Now we have
$\innp{\tau}{p(\zetabar)} = p(\zetabar)^T \tau = [\gamma||0^m] \overline{P}
\Phi^T \tau = \innp{[\gamma||0^m]^T}{\overline{P}\varphi}$ where $\varphi =
\Phi^T \tau$. Given commitments $c_1, \ldots, c_{2\ell}$, the commitment to
$\overline{P}\varphi$ can be computed as $\cm = \prod_{k\in[2\ell]} (c_k)^{\varphi_k}$. Using an inner product argument the prover can show that the commitment
$\cm$ opens to vector $z$ such that $\innp{[\gamma||0^m]^T}{z} = 0$. Binding
property of the commitment ensures that $z = \overline{P}\varphi$ with
overwhelming probability.
As in the linear check, the prover uses a random vector $P_0\in \FF^{2m}$ such
that $P_0[j]=0$ for $j\in [m]$ to blind the vector $\overline{P}\varphi$. The
prover initially commits to $P_0$ by sending a commitment $c_0$ to it, and later
sends uses the vector $z=\beta P_0 + \overline{P}\varphi$ in the preceeding
inner product check.
\end{comment}

\noindent{\em Consistency of the oracles and  $P$:} 
Similar to linear check, the verifier checks
the consistency at randomly sampled positions $Q=\{(j_u,k_u) : u\in[t] \}
\subset [h]\times[n]$. It queries the oracles for the columns
$\pi_a[\cdot, k_u]$ for $a\in \{x,y,z\}$ and $u\in [t]$. 
Then it queries the prover for vectors $X_u = \ewit_x[\cdot,j_u,k_u], Y_u =
\ewit_y[\cdot, j_u,k_u], Z_u = \ewit_z[\cdot,j_u,k_u]$ for $u\in [t]$. Let
$\bm{1}_{j_u}$ denote the unit vector in $\FF^h$ with $1$ in the $j_u^{th}$ position. The
prover and verifier run inner-product arguments to establish: 

\begin{enumerate}[{\rm 1.}]
	\item $\innp{\bm{1}_{j_u}}{P[\cdot,k_u]} = \sum_{i\in[p]} r_i[X_u[i]\cdot Y_u[i]
- Z_u[i]]$ for $u\in[t]$ with $c_{k_u}$ which can be computed from $c_1,\ldots,c_{2\ell-1}$, as in Section~\ref{sec:matrixcommitment})  as the commitment of  $P[\cdot,k_u]$.
	
	\item $\innp{\bm{1}_{j_u}}{\ewit_x[i,\cdot,k_u]} = X_u[i]$, $\innp{\bm{1}_{j_u}}{\ewit_y[i,\cdot,k_u]}=Y_u[i]$ and $\innp{\bm{1}_{j_u}}{\ewit_z[i,\cdot,k_u]}=Z_u[i]$. These inner-products check the consistency of the vectors sent by the prover with the respective oracles. These can be verified in a similar manner to that in the linear check protocol. 
\end{enumerate}

\noindent{\em Proximity Check for Oracle:} We combine the proximity check for the three 
encodings $\ewit_x,\ewit_y$ and $\ewit_z$. The verifier sends a vector
$\rho\sample \FF^{3p}$ and asks the verifier to commit to the matrix
$\tilde{\ewit}=\sum_{i=1}^p[\rho_i\ewit_x[i,\cdot,\cdot]+
\rho_{p+i}\ewit_y[i,\cdot,\cdot]+\rho_{2p+i}\ewit_z[i,\cdot,\cdot]]$
by sending commitments $\tilde{\bm{c}}=(\tilde{c}_1,\ldots,\tilde{c}_\ell)$, which are computed as $\tilde{c}_k = \prod_{i=1}^{p} (\pi_x[i,k])^{\rho_i}\cdot (\pi_y[i,k])^{\rho_{p+i}}\cdot(\pi_z[i,k])^{\rho_{2p+i}}$ $\forall k\in[\ell]$. Thereafter, for
$u\in [t]$, the verifier checks: %\\~

{\small
\begin{align}\label{eq:combinedproximity}
\prod_{a\in[\ell]} \tilde{c}_a^{\Lambda_{n,\ell}^T[a,k_u]} = \prod_{i=1}^{p} (\pi_x[i,k_u])^{\rho_i}\cdot (\pi_y[i,k_u])^{\rho_{p+i}}\cdot(\pi_z[i,k_u])^{\rho_{2p+i}}
%\innph{\Lambda_1[k_u,\cdot]}{\tilde{\bm{c}}}=
%\innph{\rho}{\pi_x[\cdot,k_u]\,||\,\pi_y[\cdot,k_u]\,||\,\pi_z[\cdot,k_u]}.
\end{align}
} 

%As in the linear check protocol, using $p(\overline{\bm{\zeta}})=\Phi p(\overline{\bm{\eta}})$, we get the following inner product check
%$\innp{(\gamma,0^{h-m})}{\overline{P}\varphi}=0$ where $\varphi=\Phi^T\tau$. 
%Once again, we
%ask the prover to ``commit'' to $F$ using a tamper resistant structure, like a codeword,
%which enables the verifier to check the aforementioned condition, as well as to
%ensure that the commitment is consistent with oracle replies and prior
%messages.
%\noindent{\em Reduction to Inner Products}: 
%The prover computes 
%$h\times n$ matrix $P$ given by $P[j,k]=F(\alpha_j,\eta_k)$. 

%It commits to $P$ using commitments $(c_1,\ldots,c_{2\ell})$ to the first $2\ell$ columns of $P$.
%Note that each row of $P$ commits to univariate component polynomials
%$F(\alpha_j,\cdot)$ of $F$ via their evaluations of $\bm{\eta}$. 
%Again, the verifier checks $p(\overline{\bm{\zeta}})=\bm{0}^s$ via the inner product
%check $\innp{\tau}{p(\overline{\bm{\zeta}})}=0$ for a random $\tau\in \FF^s$. As in the
%linear check protocol, using $p(\overline{\bm{\zeta}})=\Phi p(\overline{\bm{\eta}})$, we
%get the following inner product check
%$\innp{(\gamma,0^{h-m})}{\overline{P}\varphi}=0$ where $\varphi=\Phi^T\tau$. 
%The commitment to the vector $\overline{P}\varphi$ can be homomorphically computed
%from $c_1,\ldots,c_{2\ell}$.

%\noindent{\em Checking consistency with Oracle}: As in the linear check, the
%verifier uniformly and independently samples $(j_u,k_u)\in [h]\times [n]$ for
%$u\in [t]$, and queries the oracle $\pi$ for columns $\pi[\cdot,k_u]$. Let
%$\pi_x[\cdot,k_u]$, $\pi_y[\cdot,k_u]$ and $\pi_z[\cdot,k_u]$ denote the parse
%of $\pi[\cdot,k_u]$ into commitments corresponding to $\ewit_x,\ewit_y$ and
%$\ewit_z$ respectively. Further, the verifier asks prover for vectors
%$\ewit_x[\cdot,j_u,k_u]$, $\ewit_y[\cdot,j_u,k_u]$ and $\ewit_z[\cdot,j_u,k_u]$
%for $u\in [t]$. The verifier then checks the following:
%\begin{enumerate}[{\rm (i)}]
%\item For all $u\in [t]$: $P[j_u,k_u]=\sum_{i\in
%[p]}r_i(\ewit_x[i,j_u,k_u]\cdot\ewit_y[i,j_u,k_u]-\ewit_z[i,j_u,k_u])$.
%\item Checks that vectors $\ewit_x[\cdot,j_u,k_u]$ are consistent with
%commitments $\pi_x[\cdot,k_u]$ as in linear check protocol. Similar checks are
%made for $\ewit_y[\cdot,j_u,k_u]$ and $\ewit_z[\cdot,j_u,k_u]$.
%\end{enumerate}
%We present the full protocol in Figure \ref{fig:quadcheck}. The completeness of
%the protocol can again be verified by direct calculation. We state the soundness
%of the protocol below:
