\begin{comment}
\subsection{Distributed-prover Zero-knowledge Protocols}
This discussion will formally define a zero-knowledge protocol which supports multiple proves and distributed proof generation.
\subsubsection{Definition of DPZK}
Consider a language $L \in \npol$ and the corresponding relation $R \in \pol$  \commentA{what is $\pol$} such that
\[
\stmt \in L \Leftrightarrow R(\stmt, \wit)=1 \text{ for some witness } \wit
\]
Let $\prover_1, \ldots, \prover_{\Num}$ be $\Num$ provers. In our setting, we let each party have an additive share of the witness $\wit$. For prover $i \in [\Num]$, let $\wit_i$ be its share.  The witness $\wit$ and its shares have length equal to the number of input wires of the circuit $\C$ representing the relation $R$. We will discuss later in this section why we decided to do an additive share of the witness. \commentA{the following didn't make sense to me}For now, note that a partition of the witness among the provers such that each prover owns a non-intersecting piece of the witness is a sub-class of additive sharing since the share of a prover can be its partition of the witness padded with zeros in the rest of positions.


A DPZK protocol consists of three probabilistic polynomial time algorithms: $(\setup, \Pi, \verifier)$. 
\begin{itemize}
\item $\setup$ takes as input the security parameter $1^\secp$ and optionally a trapdoor $\tau$ and outputs the public parameters $\sigma$ of the system.
\item The interactive proof system is an $\round$-round protocol $\langle \Pi = \{\pi_i\}_{i \in [\round]}, \verifier \rangle$ where in every round $i \in [\round]$, we have the output $m_i$ of $\pi_i(\st^i_1, \ldots, \st^i_{\Num})$ provided to $\verifier$. Here, $\st^i_j$ is the state of the prover $\prover_j$ at round $i$. $\st^1_j$ is set to $\prover_j$'s share of the witness $\wit_j$, randomness $r_j$, the public parameters $\sigma$ and updated later as the protocol proceeds. The output of $\verifier$ is either an $\tt accept$ (1) or a $\tt{reject}$ (0). Also, let $\tr$ be the transcript of the protocol.
\end{itemize}
A $\DPZK$ protocol for a language $L$ satisfies the following properties: 
correctness, soundness, zero-knowledge and privacy among provers.


\begin{comment}
\begin{myboxfig}{Ideal Functionality for Distributed ZK}{funcsabort}
	\begin{center}
		\textbf{ $\FDPZK$}
	\end{center}
	%Each honest party $P_i$ sends its input $x_i$ to the functionality. Corrupted parties may send the trusted party arbitrary inputs as instructed by the adversary. When sending the inputs to the trusted party, the adversary is allowed to send a special $\abort$ command as well.
	\begin{description}
	\item[--] $ $
	%	\item[\textbf{Input:}] On message $(\sid,\In, x_i$) from a party $P_i$ $(i \in [3])$, do the following: if $(\sid,\In, *)$ message was received from $P_i$, then ignore. Otherwise record $x_i' = x_i$ internally. 	If $x_i'$ is outside of the domain for $P_i$ $(i \in [3])$, consider $x_i' = \abort$. 
		
	%	\item[\textbf{Output to adversary:}] If there exists $i \in [3]$ such that $x_i' = \abort$, send $( \sid,\Output, \bot)$ to all the parties. Else, send $(\sid,\Output, y)$ to the adversary, where $y = f(x_1', x_2', x_3')$.
		
	%	\item[\textbf{Output to selected honest parties:}] 	Receive $(\select, \{I\})$ from adversary, where $\{I\}$ denotes a subset of the honest parties. %(where $\{I\}$ may denote empty set or ). If an honest party belongs to $I$, send $(\sid,\Output, y)$, else send $(\sid,\Output, \bot)$. 
		
		%		If there exists $i \in [3]$ such that $x_i' = \abort$, send $( \sid,\Output, \bot)$ to all the parties. Else, send $(\sid,\Output, y)$ to the corrupt party $P^*$, where $y = f(x_1', x_2', x_3')$.
		
	\end{description}
\end{myboxfig}

\commentA{I discussed with protik yesterday, it's better to change this to rael/ideal world setting.}
\dnote{self: give a name for Privacy among provers}
\paragraph{Completeness}: %We will define correctness assuming that the witness set is minimal.
Given $\sigma \gets \Setup(1^\secp, \tau)$ and a valid witness $\wit$ corresponding to $\stmt \in L$, when the initial states are set with $\sigma$ and the shares of $\wit$, $\langle \Pi, \verifier \rangle$ outputs 1 with probability 1.

\paragraph{Soundness}:  The basic definition of soundness ensures the existence of a valid witness for $\stmt \in L$. For every instance $\stmt \notin L$ and any $\ppt$ algorithm $\Pi^*$, $\verifier$ accepts with probability at most $\negl(\secp)$.

The notion of proof of knowledge of the witness by the prover is captured by the notion of knowledge extraction where an extractor extracts the witness whenever the verifier accepts using an oracle access to the prover. The stronger notion of witness extended emulation \cite{Lindell03} proposes the existence of an emulator which additionally produces a simulated transcript between the prover and the verifier irrespective of whether the verifier accepts or not. We will now define the notion of witness-extended emulation for the $\DPZK$ setting, adapted from \cite{Groth11}. 
%\commentA{I do not understand the below definition; two probabilities are indistinguishable? why is $\adv$ outputting one goven $\tr$???? same with remaining definitions}
\begin{definition}
The argument $(\Setup, \Pi, \verifier)$ has computational witness-extended emulation if for all deterministic polynomial time $\Pi^*$ there exists an expected polynomial time emulator $\chi$ such that for all non-uniform polynomial time adversaries $\adv$ and all $\tau$
\begin{align*}
& \pr \left[\sigma \leftarrow \Setup(1^\secp, \tau); (\stmt, \{\wit_i\}_{i \in [\Num]}) \leftarrow \adv(\sigma); \tr \leftarrow \langle \Pi^*, \verifier \rangle: \adv(\tr) = 1 \right] \\
\approx \; & \pr \left[\sigma \leftarrow \Setup(1^\secp, \tau); (\stmt, \{\wit_i\}_{i \in [\Num]}) \leftarrow \adv(\sigma); (\tr, \wit) \leftarrow \chi^{\langle \Pi^*, \verifier \rangle}(\sigma, x): \right. \\
&\;\;\; \adv(\tr) = 1 \left. \text{ and if } \tr \text{ accepts then } \stmt \in L \text{ with } \wit \text{ as witness} \right] 
\end{align*}

where $\chi$ has access to the oracle $\langle \Pi^*, \verifier \rangle$ which produces the transcript. $\chi$ can rewind $\langle \Pi^*, \verifier \rangle$ to any round and resume the protocol with fresh randomness from the verifier.
\end{definition}
%.\dnote{1. should the notation $\langle \Pi, \verifier \rangle$ be modified to also take their respective inputs?\\
%2. Proof of knowledge \textit{vs} argument of knowledge}
Note that this follows from the single prover definition since an adversarial prover in the single prover definition can be any $\ppt$ algorithm, and this is independent of whether the $\ppt$ algorithm is run by a single prover or a protocol between multiple provers.
Also, allowing for any adversarial $\Pi^*$ captures adversarial behaviour by all the parties.

\paragraph{Zero-knowledge}: 
The traditional notion of zero-knowledge ensures that the proof does not reveal any information beyond the fact that the instance is in the language.
This is a property with respect to the view of the verifier $\verifier$. The protocol $\Pi$ and the number of provers involved to produce these messages do not impact the zero-knowledge property. The following definition captures this formally.
\begin{definition}
The argument $(\Setup, \Pi, \verifier)$ is a special honest verifier zero-knowledge argument for $L$ if there exists a PPT simulator $S$ such that for all non-uniform polynomial time adversaries $A$ and all $\tau \in \secp^{O(1)}$ 

\begin{align*}
&\pr [\sigma \leftarrow \Setup(1^\secp, \tau); (\stmt ,\wit ,\rho) \leftarrow \adv(\sigma); \tr \leftarrow \langle \Pi, \verifier_\rho \rangle:\\
 &\stmt \in L \text{ with witness } \wit \text{ and } \adv(\tr)=1 ]\\
\approx \; &\pr [\sigma \leftarrow \Setup(1^\secp, \tau); (\stmt,\wit,\rho) \leftarrow \adv(\sigma); \tr \leftarrow S(\sigma,\stmt,\rho):\\ &\stmt \in L \text{ with witness } \wit \text{ and } \adv(\tr) = 1 ]
\end{align*}
\end{definition}
%.\dnote{Auxiliary input zk or just without it?}

\paragraph{Zero Knowledge under $t$-collusions}: 
The notion of zero-knowledge, as we discussed earlier, considers the knowledge gained by a verifier on seeing the proof. However, when there are multiple provers, this does not capture the setting where the verifier colludes with a subset of provers trying to learn information about the witness shares of the remaining provers. We define the notion of zero-knowledge under $t$-collusions ($t$-$\ZKUC$) to capture this.

\begin{definition}
An argument $\innp{\Pi}{\verifier}$ with $\Num$ provers has the property of computational zero-knowledge under $t$-collusions ($t$-$\ZKUC$), if for every $\ppt$ interactive machine $\verifier^*$ and any subset of colluding provers $T \subseteq [\Num]$ of size $\leq t$, there exists a $\ppt$ algorithm $\Sim$ such that the following holds for any instance $\stmt \in L$ with witness $\wit$ and for any $\rho$: 
\[
 \{ \innp { \prover_{\overline{T}}( \st^{i-1}_{\overline{T}} ) } { \prover_T ( \st^{i-1}_T ) } ( \stmt  , m_i): i\in [\round] \} \cup \innp{\Pi(\stmt, \wit)}{\verifier^*(\stmt, \rho, \wit_T)}  \stackrel{c}{\approx}  \Sim (\stmt, \rho)
\]
where in the malicious setting, $\Sim$ has access to the ideal functionality $\IF{DPZK}$ that outputs 1 or 0 according to the extracted input of the corrupt parties, and in the semi-honest setting, $\Sim$ has input, $\wit_T$, of the corrupt parties and output of the execution.
$\st^{i-1}_T$ consists of the states of the corrupt parties of $i$th round and $\st^{i-1}_{\overline{T}}$ consists of the states of the honest parties of $i$th round. 
\end{definition}

.\pnote{Define this notion for malicious adversary}
\end{comment}
%------------------------------------------------------------------------------------------------
% Functionality based definition:
\section{Definition for Distributed-prover Zero-knowledge Protocol}\label{sec:security model}
This discussion will formally define a zero-knowledge protocol which supports multiple provers and distributed proof generation. With the eventual goal of protocols that are non-interactive and publicly-verifiable,  we keep our focus on  public-coin (where the verifier only sends truly random messages) protocols in mind.  Our definition can be extended to private-coin protocols. Consider a language $L \in \npol$ and the corresponding relation $R$ such that
\[
\stmt \in L \Leftrightarrow R(\stmt, \wit)=1 \text{ for some witness } \wit
\]
Let $\prover_1, \ldots, \prover_{\Num}$ be $\Num$ provers. %In our setting, we let each party have an additive share of the witness $\wit$. For prover $i \in [\Num]$, let $\wit_i$ be its share.  The witness $\wit$ and its shares have length equal to the number of input wires of the circuit $\C$ representing the relation $R$. We will discuss later in this section why we decided to do an additive share of the witness. \commentA{the following didn't make sense to me}For now, note that a partition of the witness among the provers such that each prover owns a non-intersecting piece of the witness is a sub-class of additive sharing since the share of a prover can be its partition of the witness padded with zeros in the rest of positions.
A public- coin DPZK protocol consists of three probabilistic polynomial time algorithms: $(\setup, \Pi, \verifier)$ as defined below. 
\begin{itemize}
\item $\setup$ takes as input the security parameter $1^\secp$ and optionally a trapdoor $\tau$ and outputs the public parameters $\sigma$ of the system. The trapdoor input as well as the public parameter can possibly be empty.
\item The interactive proof system is an $\round$-step protocol $\langle \Pi = \{\pi_i\}_{i \in [\round]}, \verifier \rangle$ where in every step $i \in [\round]$,  $\verifier$ broadcasts a challenge and we have the output $m_i$ of an $\Num$-input function $\pi_i(\st^i_1, \ldots, \st^i_{\Num})$ provided to $\verifier$ in response to the challenge.   Here, $\st^i_j$ is the state of the prover $\prover_j$ at step $i$. $\st^1_j$ is set to $\prover_j$'s share of the witness $\wit_j$, randomness $r_j$, the public parameters $\sigma$ and updated later as the protocol proceeds and in particular include the public challenges of $\verifier$.  Computing $\pi_i$, without leaking the states to each other, requires interaction amongst the provers. The output of $\verifier$ is either an {\tt accept} (1) or a {\tt reject} (0).% Also, let $\tr$ be the transcript of the protocol.  and
\end{itemize}
Intuitively, a $\DPZK$ protocol for a language $L$ will satisfy the following properties: (a)  correctness:  when the provers and the verifier are good, and  the provers hold a valid witness, the verifier should accept; (b) soundness:  the corrupt provers cannot make the verifier accept without holding a valid (joint) witness; (c) zero-knowledge: a corrupt verifier does not learn any information, except  the assertion of the statement; (d)  witness-hiding: the witness of the honest provers remain hidden from a collusion of a corrupt verifier and a subset of provers. Below, we formally prove security of such protocols based on real/ideal world paradigm. 

\subsection{Real-Ideal world definition for DPZK}

We prove the security of our protocols based on the standard real/ideal world paradigm.  Essentially, the security of a protocol is analyzed by comparing what an adversary can do in the real execution of the protocol to what it can do in an ideal execution,  that is considered secure by definition (in the presence of an incorruptible trusted party). In an ideal execution, each party sends its input to the trusted party over a secure channel, the trusted party computes the function based on these inputs and sends to each party its respective output.  Informally, a protocol
is secure if whatever an adversary can do in the real protocol (where no trusted party exists) can be done in the above described ideal computation. We refer to \cite{Canetti00,Goldreich2001,Lindell17,CohenL14} for further details regarding the security model.  


\paragraph{The ideal execution.} The ``ideal" world execution of DPZK  involves parties in $\Partyset$ that includes $\Num$ provers $\{\prover_1, \ldots, \prover_{\Num}\}$ and a verifier $\verifier$,  an ideal adversary $\Sim$ who may corrupt various subset of parties in $\Partyset$, and a  functionality $\Func_{\DPZK}$.  The functionality is parametrized with an  NP language $L$ and  the corresponding relation/verification function $R$. Each prover $\prover_i$ sends $(\stmt,\wit_i)$ to $\Func_{\DPZK}$, where $\stmt$ is the statement and $\wit_i$ is the $i$th part of the witness. The verifier $\verifier$ sends $\stmt$.  $\Func_{\DPZK}$ computes $\wit = \wit_1 \oplus \wit_2,\ldots,\oplus \wit_\Num$, where $\oplus$ is the combining function of the parts of the witness held by the provers distributedly, and sends $R(\stmt,\wit)$ to $\verifier$ and $\bot$ to every $\prover_i$. $\Func_{\DPZK}$  is described below. 
\begin{figure}[H]
	\centering
	\begin{framed}
		\textbf{Functionality} (The Distributed Prover Zero Knowledge Functionality $\Func_{\DPZK}$)\\~
		The functionality is parametrized with an NP relation $R$ of an NP language $L$.
		\begin{itemize}
		\item[--] Upon receiving input $x_i = (\stmt, \wit_i)$ from $\prover_i$ $\forall i\in[\Num]$ and $\stmt$ from $\verifier$, do the following: if $x_i$ is an empty string or falls outside the range of the domain of $\prover_i$'s input,  reset $x_i = \abort$. 
			\item[--] If any of the $x_i$ is $\abort$, send $\abort$ to everyone in $\Partyset$. Otherwise,   compute $\wit = \wit_1 \oplus \wit_2,\ldots, \oplus \wit_\Num$, where $\oplus$ is the combining function of the parts of the witness held by the provers distributedly and send $R(\stmt, \wit)$ to $\verifier$ and $\bot$ to everyone else. 
		\end{itemize}
	\end{framed}
	\caption{Ideal Functionality $\Func_{\DPZK}$}
\end{figure} \label{func:DPZK}

Let the corrupt set be denoted as $I$. We let $\Ideal_{\Func_\DPZK, \Sim(z),I}(\vec{x})$ denote the random variable consisting of the output pair of the honest parties and the ideal-world adversary $\Sim$ controlling the corrupt parties in $I$ upon inputs $\vec{x} = (x_1  ,\ldots,x_{\Num},x_\verifier)$ for the parties and auxiliary input $z$ for $\Sim$.  


\paragraph{The real execution.} In the real model, the parties run $\Pi_\DPZK$ protocol.  We consider a synchronous network with private point-to-point channels amongst the provers, and an authenticated broadcast channel. This means that the computation proceeds in rounds, and in each round parties can send private messages to other parties and can broadcast a message to all other parties. We stress that the adversary cannot read or modify messages sent over the point-to-point channels, and that the broadcast channel is authenticated, meaning that all parties know who sent the message and the adversary cannot tamper with it in any way. Nevertheless, the adversary is assumed to be rushing, meaning that in every given round it can see the messages sent by the honest parties before it determines the messages sent by the corrupted parties. As per $\Pi_\DPZK$, the parties first produce the output of $\setup$. Next, in $i$th step, $\verifier$ broadcasts a challenge and the  parties run a distributed protocol to compute $\pi_i$ with respective states and sends the output $m_i$ of the computation to $\verifier$.  The protocol ends after $\round$ steps.

In summary,  the ``real" world execution involves the $\ppt$\ parties  in $\Partyset$,  and a real world adversary $\Adv$ who may corrupt  a set of parties in $I$ maliciously.  Let $\Real_{\Pi, \Adv(z),I}(\vec(x))$ denote the random variable consisting of the output pair of the honest parties and the adversary $\Adv$ controlling the corrupt parties in $I$ in the real execution, upon inputs $\vec{x}$ (defined in the same way as the ideal world) for the parties and auxiliary input $z$ for $\adv$. 


\paragraph{Security via Indistinguishability of Real and Ideal world.} The definition is given below
\begin{definition}
Let $\Func_{\DPZK}$ be a $\Num+1$-party  functionality and let $\Pi_{\DPZK}$ be a $(\Num+1)$-party protocol involving $\Partyset$ for DPZK. We say that  $\Pi_\DPZK$ {\em securely realizes} $\Func_{\DPZK}$~if for every $\ppt$ probabilistic  real-world adversary $\Adv$, there exists an $\ppt$  expected polynomial-time ideal-world adversary $\Sim$, such that for every $I \subset \Partyset$, every $\vec{x} \in (\{0,1\}^*)^n$ where $\abs{x_1} = \ldots= \abs{x_n}$, and every $z \in \bit^*$, it holds that:
	
	$$\Big\{\Ideal_{\Func_{\DPZK}, \Sim(z),I}(\vec{x}) \Big\} \equiv \Big\{ \Real_{\Pi_\DPZK, \Adv(z),I}(\vec{x}) \Big\}. $$
When $I$ is  $\{\prover_1,\ldots,\prover_\Num\}$, $\{\verifier\}$, a proper $t$-size subset of $\{\prover_1,\ldots,\prover_\Num\}$, a proper $t$-size subset of $\{\prover_1,\ldots,\prover_\Num\}$ plus   $\verifier$, the above indistinguishability  is referred as Soundness with Witness Extraction (SoWE), Zero-Knowledge (ZK), witness-hiding (WH), witness-hiding with collusion (WHwC).   	
\end{definition}

While we give a very strong definition as above, motivated by several practical aspects, we relax the adversarial power and will focus on a subset of the above properties for our constructions. We elaborate below on this.

\subsection{Our setting for DPZK}
Our final goal is to produce arguments/proofs that is non-interactive and publicly-verifiable. Most of the applications we foresee work best  with these features. With these features as end-goal, we  will design public-coin honest-verifier  protocols that via Fait-Shamir heuristic \cite{IOP2016} \commentA{Protik: add ref}  can be turned into non-interactive and publicly-verifiable proofs/arguments.  In the honest-verifier setting, the verifier is considered to be only semi-honestly corrupt and so the ZK property needs to be proven keeping such a weaker adversary in mind. On the other hand,   the prover can be maliciously corrupt.  As a result,  SoWE and  WH  need to be proven against this stronger adversarial setting. Interestingly,  non-interactiveness and publicly-verifiability make the two properties WH and WHwC equivalent, since the verifier has nothing more to add to the view of the corrupt provers in the case of collusion. In summary,  we will prove three properties for our protocols: (a) ZK assuming a semi-honest verifier, (b)  SoWE and WH tolerating malicious  provers. As for the form of witness partition across the prover's, we assume that the provers jointly hold an additive sharing of the witness $\wit$. This is general enough since \commentA{Dhinakaran: add the argument here}.


\subsection{Related Notions}
\commentA{someone who read marcel's and pedersen's need to rewrite this}
Closely related to our notion is the work of \cite{marcel} and \cite{pedersen} \commentA{protik fill in} which are referred as  threshold proofs. These works  distribute the prover?s side of interactive proofs of knowledge over multiple parties due to various reasons starting from  improving the security against theft of the user?s identity to improving robustness by ensuring that only a restricted size subset of the provers may be corrupted. The primary difference between the work of \cite{marcel} is that while we allow interaction amongst the prover, they do not. Also their work primarily captures sigma protocols. Our definition captures a broader class of protocols via allowing interaction amongst the provers.  On other words, threshold proofs are a sub-class of the protocols we cater too. Not just to increase security via decentralisation of the prover (or disytributing prover;s task), our goal is to capture scenarios where multiple provers would like to jointly prove a statement.  Our real-ideal world based definition is inline with MPC style definitions and is more powerful. \commentA{someone copare with pedersen; I don't have a copy of his work}


%We would like to finally point out that at many scenarios  
%In contrast to the definition in section ~\ref{sec:security model}, all the parties except for the verifier, in the execution of the protocol they may deviate arbitrarily, but the verifier is not allowed to deviate from the protocol. The reason behind considering a preposterously looking adversarial model is that we want to design a zkSNARK which is efficient, secure, publicly verifiable and distributed proof generation property. We know from \cite{}, any interactive honest (public coin) verifier can be transformed into Non-interactive zero-knowledge using \textit{Fiat–Shamir heuristic}. Here the verifier can be semi-honest, i.e., verifier will follow the protocol steps and sends all the challenges uniformly at random.
%-Start with corruption model and give a brief reasoning.
%.\pnote{Describe honest verifier}

\begin{comment}
If all the parties in $\Partyset$ are honest, then the protocol should return 1, this property is called \textbf{\textit{completeness}}.

In this setting, we have two types of corruption models, and that depends on which parties are corrupt:

-- Consider $I$, the set of all corrupt parties consists of all the provers in $\Partyset$, i.e., together they do not have a correct witness for $\stmt$, but they try to cheat in such a way so that the verifier outputs 1. What we want from the protocol is that such a scenario can happen with negligible probability. In other word, whenever in the protocol $\verifier$ outputs 1, then there must be a correct witness $\wit$ such that $\prover_i$ has $\wit_i$ for which $\wit = (\wit_1, \ldots, \wit_{\Num})$, with very high probability. This notion is called \textbf{\textit{proof of knowledge}}. 
	\begin{definition}
		Let $\Func_{\DPZK}$ is the functionality which is defined as ~\ref{func:DPZK} and let $\innp{\Pi}{V}$ is a $\DPZK$ $\Num$-party protocol involving $\Partyset$. We say that $\innp{\Pi}{V}$ realizes $\Func_{\DPZK}$ with computational proof of knowledge property if for every $\ppt$ real world adversary $\Adv$, corrupting at most all provers, there exists a $\ppt$ knowledge extractor $\extrac$ such that for every $\stmt \in L$, for which $\verifier$ outputs 1, $\extrac$ can extract a witness $(\wit_1, \ldots, \wit_{\Num})$ for which $\Func_{\DPZK}$ outputs 1.
	\end{definition}    
	
-- Consider $I$ consists of $\verifier$ and $t$ provers. That is, some of the provers may deviate from the protocol to gain additional information about the honest provers' secret; to achieve this, they may take additional help from $\verifier$'s view. We call a protocol $\innp{\Pi}{\verifier}$ has \textbf{\textit{zero knowledge under t-collusion}}($t$-ZKUC) property, if in this corruption model, corrupted provers and verifier whatever they can compute, the same thing can be computed from the functionality output of the functionality. 
	\begin{definition}
		Let $\Func_{\DPZK}$ is the functionality which is defined as ~\ref{func:DPZK} and let $\innp{\Pi}{V}$ is a $\DPZK$ $\Num$-party protocol involving $\Partyset$. We say that $\innp{\Pi}{V}$ realizes $\Func_{\DPZK}$ with computational zero knowledge under $t$-collusion property if for every $\ppt$ real world adversary $\Adv$, corrupting $I$, at most $t$ provers and verifier, then there exists a $\ppt$ simulator $\Sim$ such that for every $\stmt$, and $\{\wit_i : i\in [\Num]\}$, it holds that: 
		$$\Big\{\Ideal_{\Func_{\DPZK}, \Sim(\secparam),I}(\wit_1, \ldots, \wit_{\Num}) \Big\} \stackrel{c}{\approx} \Big\{ \Real_{\innp{\Pi}{\verifier}, \Adv(\secparam, \wit_T),I}(\wit_1, \ldots, \wit_{\Num}) \Big\}. $$
		Where $\wit_T$ is the input of the corrupt parties.
	\end{definition}

Note that if $I$ consists of the only verifier, then the view of the verifier should not reveal any information about neither the witness nor the inputs of the provers. This notion is called \textbf{\textit{zero knowledge}}, which is subsumed by the notion $t$-ZKUC.
\end{comment}
