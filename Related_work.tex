\subsection{Related Work}

In the era of cryptocurrecies, anonymity is one of the most desirable feature. To achieve this Zcash adapts zero-knowledge in their system. In past few years zero-knowledge gained immense pace in research to obtain better verification time, better proving time and most importantly better proof size. In the paper \cite{Ligero2017}, a zero-knowledge protocol was given, which attains sublinear proof size, without any trusted setup and assuming that finding the collision for hash function is difficult. Then in \cite{Bulletproofs} zero-knowledge arguement of logarithmic size without trusted setup, assuming disctrete log is hard. And then \cite{Aurora} gave another IOP based protocol which is better than \cite{Ligero2017} in terms of proof size, but not as good as \cite{Bulletproofs}, but hardness assumption is only collision resistant hash function. \cite{spartan} provides an arguement which attains sublinear proof as well as sublinear verification, on the hardness assumption of discrete log. Verification of \cite{Bulletproofs} requires exponentiation of linear order, which is very expensive in terms of practice. Whereas \cite{Aurora} does not require exponentiation but as all it's messages are consructed as oracles, it takes too much memory in practice. \cite{spartan} is better in terms of verification. In our work we provide a zero-knowledge protocol which has sublinear proof size, same as \cite{spartan} and in verification number of required field operations is linear in size of the circuit but number of exponentiations is very less, lesser than \cite{spartan}, in practice which can outperform all the above works.

In \cite{DDS} notion of distributed prover protocol was introduced. In that paper protocol was designed for digital signature. Another constribution of our paper is that to define the notion of distributed proof generation for zero-knowledge and providing an efficient protocol which supports this notion. In \cite{Bulletproofs}, it was given that the range proof supports the notion of distributed proof generation. In \cite{Trinocchio}, a zero-knowledge protocol given in \cite{pinnochio_PHGR} was converted into a distributed proof generation zero-kowledge protocol, but this protocols are based on trusted setup. Later in this paper, we will describe how state-of-the-art zero-knowledge protocols lack of efficiency while converting into distributed proof generation zero-knowledge protocol. \cite{spartan} requires an MPC of quadratic order in the size of the verification circuit. \cite{Ligero2017, Aurora} can not be transformed into $\DPZK$ directly, to convert this the oracle constructions are needed to be changed such that it must have homomorphic property. This makes $\DPZK$ from \cite{Aurora} very inefficient because every message in this protocol is constructed as an oracle. We describe a different approach to obtain $\DPZK$ from inner product adapted from \cite{Bulletproofs}.