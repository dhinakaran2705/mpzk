\subsection{Related Work}

In the era of cryptocurrencies, anonymity is one of the most desirable features. To achieve this, Zcash adapts zero-knowledge in their system. In the past few years, zero-knowledge gained immense pace in research to obtain better verification time, better proving time, and, most importantly, better proof size. In the paper \cite{ligero}, a zero-knowledge protocol was given, which attains sublinear proof size, without any trusted setup and assuming that finding the collision for a hash function is difficult. Then in \cite{Bulletproofs} zero-knowledge argument of logarithmic size without trusted setup, assuming the discrete log is hard. And then \cite{Aurora} gave an IOP based protocol which is better than \cite{ligero} in terms of proof size, but not as good as \cite{Bulletproofs}, but hardness assumption is a collision-resistant hash function. \cite{spartan} provides an argument which attains sublinear proof as well as sublinear verification, on the hardness assumption of the discrete log. Verification of \cite{Bulletproofs} requires the exponentiation operation of linear order, which is very expensive in terms of practice. Whereas \cite{Aurora} does not require exponentiation, but as all its messages are constructed as oracles, it takes too much memory in practice. \cite{spartan} is better in terms of verification. In our work, we provide a zero-knowledge protocol which has a sublinear proof size, same as \cite{spartan} and in verification number of required field operations is linear in size of the circuit but the number of exponentiations is very less, lesser than \cite{spartan}, in practice which can outperform all the above works.

In \cite{DDS} notion of distributed prover protocol was introduced, where provers jointly provide a signature. Generalization of this work came in 2012 by Marcel et al. \cite{EfficientTZ}, which gave a generic notion of the distributed proof generation system. In this paper, the protocols were restricted to $\Sigma$-protocol.
%Another contribution of our paper is to define the notion of distributed proof generation for zero-knowledge.
In our work, we are revisiting the definition and providing an efficient protocol for general arithmetic circuits, which supports this notion. In \cite{Bulletproofs}, it was given that the range proof supports the notion of distributed proof generation. In \cite{Trinocchio}, a zero-knowledge protocol provided in \cite{pinnochio_PHGR} was converted into a distributed proof generation zero-knowledge protocol, but these protocols are based on the trusted setup. Later in this paper, we will describe how state-of-the-art zero-knowledge protocols lack efficiency while converting into distributed proof generation zero-knowledge protocol. \cite{spartan} requires an MPC of quadratic order in the size of the verification circuit. \cite{ligero, Aurora} can not be transformed into $\DPZK$ directly. These works can be made $\DPZK$; in that case, the oracle constructions are needed to be changed such that it must have homomorphic property. This makes $\DPZK$ from \cite{Aurora} very inefficient because every message in this protocol is constructed as an oracle. We describe a different approach to obtain $\DPZK$ from the inner product adapted from \cite{Bulletproofs}.
\pnote{Include Marcel Keller's paper \cite{EfficientTZ}}