%\newcommand{\enc}{\mathsf{Enc}}
\subsection{Witness Encoding}\label{sec:witencoding}
We start by describing a randomized encoding of the prover's extended witness
$\wit\in \FF^N$ (henceforth referred as witness), where $N$ denotes the number
of wires in the arithmetic circuit representing the $\npol$ relation. Let $p,m$ and $s$ be
integers such that $N=pms$. We canonically view the
witness $\wit$ 
as $p\times m\times s$ matrix with entries $\wit[i,j,k]$ for $i\in [p]$,
$j\in [m]$ and $k\in [s]$. The encoding is specified by an independence 
parameter $\bi$, integers $\ell := s+\bi$, $h>2m$, $n>2\ell$, and sequences
$\bm{\zeta},\bm{\eta},\bm{\alpha}$ of distinct points in $\FF$ with cardinality 
$\ell,n,h$ respectively. We write $\bm{\zeta}=(\zeta_1,\ldots,\zeta_\ell)$,
$\bm{\eta}=(\eta_1,\ldots,\eta_n)$ and $\bm{\alpha}=(\alpha_1,\ldots,\alpha_h)$. 
Next we define the interpolation domain $G$ as $G=\{(\alpha_j,\zeta_k): j\in[m],
k\in [\ell]\}$ and evaluation domain $H$ as $H=\{(\alpha_j,\eta_k): j\in [h],
k\in [n]\}$. Finally, we encode $\wit$ as follows and denote the below randomized computation as $\ewit\gets \enc(\wit)$, where  $\enc(\wit)$ is the random variable denoting the encodings of $\wit$:
\begin{enumerate}[{\rm (i)}]
\item First we embed $\wit$ into a $p\times m\times \ell$ matrix $\hat{\wit}$
where $\hat{\wit}[i,j,k]=\wit[i,j,k]$ for $k\leq s$, while the entries
$\hat{\wit}[i,j,k]$ for $k>s$ are sampled from $\FF$ uniformly at random.
\item We construct bivariate polynomials $Q^i(x,y)$ with $deg_x(Q)<m$ and
$deg_y(Q) $ $<\ell$ such that $Q^i$ interpolates the slice
$\hat{\wit}[i,\cdot,\cdot]$ on $G$, i.e,
$Q^i(\alpha_j,\zeta_k)=\hat{\wit}[i,j,k]$. 
\item Let $\ewit$ denote the $p\times h\times n$ matrix, where the slice
$\ewit[i,\cdot,\cdot]$ consists of evaluations of $Q^i$ on $H$, i.e,
$\ewit[i,j,k]=Q^i(\alpha_j,\eta_k)$ for $i\in [p], j\in [h]$ and $k\in [n]$.
Then $\ewit$ is a randomized encoding of $\wit$.
\end{enumerate}
 It is easily seen
that $\ewit[i,\cdot,\cdot]\in \rsc{\eta}{n,\ell}\otimes \rsc{\alpha}{h,m}$. 
We remark that the above
encoding can be computed using $O(N\log N)$ field operations by applying $\fft$ along the rows and columns of the slices. %$m$ direction and $s$ direction.%(see Appendix for details \commentA{missing section no.}).
%%  Computing the encoding in O(Nlog N) time: Move to Appendix
%\noindent{\em Efficiently computing the encoding}: Although the previous
%description of the encoding involves the bivariate polynomials, 
%the prover does not explicitly need to compute the bivariate polynomials to
%construct the encoding. We describe an efficient method to construct the
%encoding. Given an $m\times \ell$
%matrix $\matx$, the prover first computes polynomials $p_j$ for $j\in
%[m]$ as $p_j(y) := \ifft(\matx[j,\cdot], \bm{\zeta})$. Then it constructs a $m\times n$
%matrix $\maty$, where the $j^{th}$ row of $\maty$ is evaluation of $p_j$ on the set
%$\bm{\eta}$, i.e, $\maty[j,\cdot] := \fft(p_j, \bm{\eta})$. Next, the prover constructs
%polynomials $q_k$ for $k\in [n]$ by interpolating the column $\maty[\cdot,k]$ on
%$\bm{\alpha}^0 = (\alpha_1,\ldots,\alpha_m)$, i.e, $q_k(x) := \ifft(\maty[\cdot,k],
%\bm{\alpha}^0)$. It obtains the encoding $\enc(\matx)$ as the $h\times n$ matrix
%$\matz$ whose columns are evaluations of polynomials $q_k$ on the set $\bm{\eta}$, i.e,
%$\matz[\cdot,k]=\fft(q_k,\bm{\alpha})$. The above computation involves
%$O(mn\log(mn))$ operations in $\FF$ for each $m\times \ell$ matrix. Thus, computing $\enc(\wit)$ takes
%$O(pmn\log(mn))$ which is $O(N\log{N})$. We also remark that $\maty$ is a submatrix
%of $\matz$, and thus
%$p_j=\ifft((\matz[j,1],\ldots,\matz[j,\ell]),(\eta_1,\ldots,\eta_{\ell}))$ for $j\in
%[m]$. This allows us to consistently define polynomials $p_j$ for $m<j\leq h$ by
%$p_j=\ifft((\matz[j,1],\ldots,\matz[j,\ell]),(\eta_1,\ldots,\eta_{\ell}))$.
%%
The encoding $\enc$ satisfies the following {\em bounded independence} property. The proof is elementary and it will be presented in the full version.% proved in Appendix \commentA{appendix number is missing}:
\begin{lemma}[Bounded Independence]\label{lem:boundedindependence}
Let $B\subseteq [n]$ be a set of size $\bi$. Let $\mc{U}(p,h,b)$ denote the
set of $p\times h\times b$ matrices $\matx$ such
that $\matx[i,\cdot,k]$ is a codeword in $\rsc{\alpha}{h,m}$ for all $i\in
[p],k\in [\bi]$. Then for any $p\times m\times s$ matrix $\wit$, the random
variable $\ewit_B := \{\ewit[\cdot,\cdot,B]: \ewit\sample \enc(\wit)\}$ is
distributed uniformly on $\mc{U}(p,h,b)$.
\end{lemma}


\subsection{Codes and Matrices}\label{sec:codesandmatrices}
For code $\rsc{\eta}{n,\ell}$, let $\Lambda_{n,\ell}$ denote the $n\times \ell$ matrix for the linear transformation that maps a vector $x\in \FF^\ell$ 
to the unique codeword $y$ in  $\rsc{\eta}{n,\ell}$ such that $y_i=x_i$ for
$i\in [\ell]$. For codes  $\rsc{\alpha}{h,m}$, $\rsc{\eta}{n,s+\ell-1}$, $
\rsc{\eta}{n,2\ell-1}$, and $\rsc{\alpha}{h,2m-1}$, let
$\Lambda_{h,m},\Lambda_{n,s+\ell-1},\Lambda_{n,2\ell-1}$ and $\Lambda_{h,2m-1}$ be similar matrices. We denote the corresponding
parity-check matrices as
$\mc{H}_{n,\ell},\mc{H}_{h,m},\mc{H}_{n,s+\ell-1},\mc{H}_{n,2\ell-1}$. 

We notate the set of three dimensional $p\times h\times n$ matrices as $\mc{M}_{p,h,n}$ and
the set of two dimensional $h\times n$ matrices as $\mc{M}_{h, n}$. We
assume standard distance metrics on the sets $\mc{M}_{p,h,n}$ and $\mc{M}_{h,n}$.
Let $\mc{W}_1$ denote the set of matrices
$U$ in $\mc{M}_{p,h,n}$ such that the $n$-length vector $U[i,j,\cdot]$ is a
codeword in $\rsc{\eta}{n,\ell}$ for all $i,j$. Similarly let $\mc{W}_2$ denote the set of
matrices $U$ such that the $h$-length vector $U[i,\cdot,k]$ is a codeword in
$\rsc{\alpha}{h,m}$ for all $i,k$.  Let $\mc{W}=\mc{W}_1\cap \mc{W}_2$. $\mc{W}$ denotes the set of {\em well-formed} encodings  and consists of $U$ such that each slice $U[i,\cdot,\cdot]\in \rsc{\eta}{n,\ell}\otimes \rsc{\alpha}{h,m}$. 

\begin{comment}
We now fix the notation for some of the codes that will be
frequently used throughout this section. We use $L_1$ and $L_2$ to denote codes
$\rsc{\eta}{n,\ell}$ and $\rsc{\alpha}{h,m}$ respectively. Let $\mc{C}_1 := \ric{L_1}{h}$ 
and $\mc{C}_2 :=\cic{L_2}{n}$ denote the interleaved codes of $L_1$ and $L_2$. In addition, 
we use codes $L_3 := \rsc{\eta}{n,s+\ell-1}$, $L_4 := \rsc{\eta}{n,2\ell-1}$ and 
$L_5=\rsc{\alpha}{h,2m-1}$ to encode some intermediate computations in our protocols.
Let $\Lambda_{n,\ell}$ denote the matrix for the linear transformation that maps a vector $x\in \FF^\ell$ 
to the unique codeword $y$ in $L_1$ such that $y_i=x_i$ for $i\in [\ell]$. Thus $\Lambda_{n,\ell}$ is 
an $n\times \ell$ matrix. Let
$\Lambda_{h,m},\Lambda_{n,s+\ell-1},\Lambda_{n,2\ell-1}$ and $\Lambda_{h,2m-1}$ be similar matrices
for the codes $L_2,L_3,L_4$ and $L_5$ respectively. We denote the
parity check matrices for $L_i$  by $\mc{H}_i$ for $i\in \{1,\ldots,5\}$. 
We notate the set of three dimensional $p\times h\times n$ matrices as $\mc{M}_{p,h,n}$ and
the set of two dimensional $h\times n$ matrices as $\mc{M}_{h, n}$. We
assume standard distance metrics on the sets $\mc{M}_{p,h,n}$ and $\mc{M}_{h,n}$.
\end{comment}

\subsection{Commitment of product codewords}\label{sec:matrixcommitment}

We now discuss the commitment scheme for matrices in $\rsc{\eta}{n,\ell} \otimes \rsc{\alpha}{h,m}$. Similar scheme works for other product codes also. 
A matrix $U$ in the above product code is
completely determined by the sub-matrix $\overline{U}=\{U[j,k]: j\in [m], k\in [\ell]\}$ and can be expressed as $U=\Lambda_{h,m}\overline{U}\Lambda_{n,\ell}^T$. 
A commitment to $U$ is defined as  a vector of commitments $\bm{c}=(c_1,\ldots,c_\ell)$
where $c_k=\comm(\overline{U}[\cdot,k])$ is the vector commitment for $k^{th}$
column of $\overline{U}$ for $k\in [\ell]$. We use the notation $\bm{c} = (c_1,\ldots,c_\ell) \gets \pccom(U)$. Given $\bm{c}$, commitment to $k$th column of $U$ for $k > \ell$ 
can be computed as  $c_k=\prod_{a\in [\ell]}(c_a)^{\Lambda_{n,\ell}^T[a,k]}$, thanks to the homomorphicity.

%The commitment scheme presents no difficulty in  proving inner-product relations for a column (or a linear combination of them) with a public vector. For example, to prove $\innp{x}{U[\cdot,k]}=v$ for any $k \in [n]$ and a public vector $x$, we first homomorphically compute commitment to $\overline{U}[\cdot,k]$ as $c_k=\prod_{a\in [\ell]}(c_a)^{\Lambda_1^T[a,k]}$.  Then we reformulate the inner-product as  $\innp{x}{U[\cdot,k]}=\innp{x}{\Lambda_2\overline{U}[\cdot,k]}=\innp{x^T\Lambda_2}{\overline{U}[\cdot,k]}$ where the last formulation involves a public vector $x^T\Lambda_2$ and private vector corresponding to commitment $\mathsf{cm}$. We make repeated use of such inner-product checks.   

\subsection{Oracle Construction}\label{sec:construct_oracle} 
Unlike prior IOP constructions such as \cite{ligero, aurora}, we additionally
obtain a homomorphic commitment on the encoded witness $\enc(\wit)$ and provide
oracle access to the commitment. 
For all $(i,k)\in [p]\times
[n]$, we  compute the commitment %sample randomness $\delta_{ik}\sample \FF$, and obtain
$c_{ik}=\comm(V_{ik})$ for the vector 
$V_{ik}=(\ewit[i,1,k],\ldots,\ewit[i,m,k])\in \FF^m$. 
Finally we define $p\times n$ matrix $\comoracle$ as
$\comoracle[i,k]=c_{ik}$. We write this as $\comoracle \gets \ocom(\ewit)$. We provide oracle access to $\comoracle$ where for a
query $Q\subseteq [n]$, the oracle responds with columns $\comoracle[\cdot,k]$ for
$k\in Q$. Note that $i^{th}$ row of the matrix $\comoracle$ commits to the
$i^{th}$ slice of $\ewit$.   This is different from  commitment of a product codeword 
(Section ~\ref{sec:matrixcommitment})  where only 
$\ell$ columns are committed. %This is because the verifier cannot access all $\ell$ commitments to the first $\ell$ columns as it only has oracle access to $\comoracle$. 
\begin{comment}
\subsection{Well-formed Encodings}\label{sec:wellformedenc}
Let $\mc{W}$ denote the subset of $\mc{M}_{p,h,n}$
consisting of matrices $U$ such that $U[i,\cdot,\cdot]\in L_1\otimes L_2$ for all $i\in [p]$. 
We call an encoding $\ewit$ to be {\em well-formed} if $\ewit\in \mc{W}$. Note
that an encoding $\ewit\gets \enc(\wit)$ for any $\wit\in \FF^N$ is well formed. 
Let $\mc{W}_1$ denote the set of matrices
$U$ in $\mc{M}_{p,h,n}$ such that the $n$-length vector $U[i,j,\cdot]$ is a
codeword in $L_1$ for all $i,j$. Similarly let $\mc{W}_2$ denote the set of
matrices $U$ such that the $h$-length vector $U[i,\cdot,k]$ is a codeword in
$L_2$ for all $i,k$. It can be seen that $\mc{W}=\mc{W}_1\cap \mc{W}_2$. For $U^\ast\in \mc{M}_{p,h,n}$ define
$d(U^\ast,\mc{W}_i)=\min\{\Delta_i(U^\ast,U):U\in \mc{W}_i\}$ for $i=1,2$.
\end{comment}
\subsection{Witness Decoding}\label{sec:witdecoding}
We describe a decoding procedure $\dec$ for obtaining a witness $\wit$ from an encoding
$\ewit$. Let $\ewit\in \mc{W}$ be a well-formed encoding. Such an encoding can
be decoded slice by slice, i.e, for each $i\in [p]$, we interpolate bivariate
polynomial $Q^i\in \FF[x,y]$ with $deg_x(Q^i)<m$ and $deg_y(Q^i)<\ell$ such that
$Q^i$ interpolates $\ewit[i,\cdot,\cdot]$ on evaluation domain $H=\{(\alpha_j,\eta_k): j\in [h],
k\in [n]\}$. This can be accomplished using
standard algorithms. The decoded witness $\wit$ is then given by
$\wit[i,j,k]=Q^i(\alpha_j,\zeta_k)$. We extend the above decoding procedure to
recover from slightly malformed encodings. Let $\ewit^*\in \mc{M}_{p,h,n}$ be such
that $d_1(\ewit^*,\mc{W}_1)<e_1 <(n-\ell)/2 $ and $d_2(\ewit^*,\mc{W}_2)<e_2
<(h-m)/2$. In this case, from the distance property of the codes $\rsc{\eta}{n,\ell}$
and $\rsc{\alpha}{h,m}$ it follows that there is at most one $\ewit\in \mc{W}$ such that
$\Delta_1(\ewit^*,\ewit)<e_1$ and $\Delta_2(\ewit^*,\ewit)<e_2$. Such a $\ewit$
may be efficiently ``recovered'' from $\ewit^*$ using algorithms for Reed-Solomon decoding
(c.f. ~\cite{CodingTheory}). We then define $\dec(\ewit^*)$ as $\dec(\ewit)$.

\subsection{Linear Check Protocol}\label{sec:lincheck}
In this section, we describe an IPCP that allows a prover to prove knowledge of
witness $\wit\in \FF^N$ satisfying a linear constraint of the form $A\wit = b$
for some $A\in \FF^{M\times N}$ and $b\in \FF^M$. As before we veiw $\wit$ as
$p\times m\times s$ matrix where $N=pms$. As desribed previously in Sections
\ref{sec:witencoding} and \ref{sec:construct_oracle}, the prover obtains $\ewit\gets 
\enc(\wit)$ and $\pi=\comoracle \gets \ocom(\ewit)$. %The prover then sets the oracle  $\pi := \comoracle$, which can then be queried by the verifier. 
The broad  outline of the protocol is--(a) reduce the problem to checking an inner-product argument via a product codeword commitment; (b) check consistency of the product codeword commitment and $\pi$ (c) check if $\pi$ committed to a well-formed codeword. The protocol appears in 
Figure \ref{fig:linearcheck}. In Figure ~\ref{fig:linearcheck}, we use
$\mathsf{pp}$ to denote the public parameters, consisting of $(\FF, \GG, \rsc{\eta}{n,\ell},
\rsc{\alpha}{h,m},\rsc{\eta}{n,s+\ell-1},\rsc{\eta}{n,2\ell-1},
\rsc{\alpha}{h,2m-1}, \bm{g},h)$.

\smallskip

\noindent {\em Reduction to Inner-product}: To check $A\wit=b$, the verifier 
samples random $r\sample \FF^M$ and asks the prover to prove $r^TA\wit=r^Tb$.
Both prover and verifier view the
vector $r^TA\in \FF^N$ as a $p\times m\times s$ matrix $R$ and interpolate
polynomials $R^i(x,y)$ for $i\in [p]$ with $deg_x(R^i)<m$ and $deg_y(R^i)<s$
satisfying $R^i(\alpha_j,\zeta_k)=R[i,j,k]$. Let $Q^i$, $i\in [p]$ denote the
polynomials used in interpolating (and encoding) witness $\wit$. Then, 
$\wit[i,j,k]=Q^i(\alpha_j,\zeta_k)$. The check $\innp{R}{\wit}=r^Tb$ reduces to
$\sum_{i,j,k}R^i(\alpha_j,\zeta_k).Q^i(\alpha_j,\zeta_k)=r^Tb$ where $i,j$ and
$k$ run over indices in $[p],[m]$ and $[s]$ respectively. 
%The preceding identity can be succinctly expressed as $\sum_{k\in [s]}p(\zeta_k)=r^Tb$, where $p(\cdot)$ denotes the polynomial $\sum_{j=1}^m\sum_{i=1}^p R^i(\alpha_j,\cdot)Q^i(\alpha_j,\cdot)$. Let $\zetabar$ denote the vector $(\zeta_1,\ldots,\zeta_s)$ and let $p(\zetabar)$ denote the vector $(p(\zeta_1),\ldots,p(\zeta_s))$. Then the previous polynomial identity reduces to the inner product check $\innp{1^s}{p(\zetabar)}=r^Tb$.  
For $j\in [h]$, denoting  the polynomial $\sum_{i\in
[p]}R^i(\alpha_j,\cdot)Q^i(\alpha_j,\cdot)$ as $p_j$ (which will be of degree
less than $s + \ell$), we can rewrite the expression as
$\sum_{k\in[s], j\in[m]}p_j(\zeta_k)= r^Tb$.  We define the matrix $P$ by
$P[j,k]=p_j(\eta_k)$ for $j\in [h],k\in [n]$ which is a product codeword from
$\rsc{\eta}{n,s+\ell-1} \otimes \rsc{\alpha}{h,2m-1}$ and fully determined by
sub-matrix $\overline{P}$ consisting of the first $2m-1$ rows and the first
$s+\ell-1$ columns of $P$. Now our idea is to commit  $P$ using  commitment for
product codeword i.e. $(c_1,\ldots,c_{s+\ell-1}) \gets \pccom(P)$ and write the
above expression as an inner-product argument involving $\overline{P}$ such as
$\innp{*}{\overline{P}\varphi} = r^Tb$, where $*$ and $\varphi$ are public vectors 
 and so the commitment for $\overline{P} \varphi$ can
be computed given the commitment of the columns of $\overline{P}$.

%Define $p(\cdot) = \sum_{j\in [m]} p_j(\cdot)$, $p(\zetabar) = [p(\zeta_1),\ldots,p(\zeta_s)]$, and $p(\etabar) = [p(\eta_1),\ldots,p(\eta_{s+\ell})]$. \newline
%\noindent {\em Consistency of $P$ and $p(\zetabar)$} 
Let $\Phi$ be a $s\times (s+\ell-1)$ matrix such that $[p_j(\zeta_1), \ldots,
p_j(\zeta_s)]^T = \Phi  [p_j(\eta_1), \ldots, p_j(\eta_{s+\ell-1})]^T$ for
$j\in[h]$. We have: %$p(\zetabar)^T = \Phi \times p(\etabar)^T$. \\
$\big[ \sum_{j\in[m]} p_j(\zeta_1), \ldots,  \sum_{j\in[m]}
p_j(\zeta_s)\big]^T 
    =\Phi  \left[\sum_{j\in[m]} p_j(\eta_1), \ldots, \sum_{j\in[m]}
p_j(\eta_{s+\ell-1})\right]^T
	= \Phi  \overline{P}^T  [1^m||0^{m-1}]^T
	%\bigg[\sum_{j\in[m]} p_j(\zeta_1), \ldots, \sum_{\j\in[m]} p_j(\zeta_s)\bigg]^T &= [1^m||0^m]\times \overline{P} \times \Phi^T\\
$.
Therefore, the check $\sum_{k\in[s], j\in[m]}p_j(\zeta_k)=r^Tb$ reduces to $\innp{[\sum_{j\in[m]} p_j(\zeta_1), \ldots, \sum_{\j\in[m]} p_j(\zeta_s)]^T}{[1^s]^T} =r^Tb$ 
$\Rightarrow \innp{[1^m||0^{m-1}]^T}{\overline{P}\times \Phi^T \times [1^s]^T} = r^Tb$.
Here, $[1^m||0^{m-1}]^T$ and $\varphi = \Phi^T \times [1^s]^T$ are public vectors. 
Given commitment of $P$,  $c_1,\ldots,c_{s+\ell-1}$, the
commitment to $\overline{P}\varphi$ can be computed as
$\mathsf{cm}=\prod_{k=1}^{s+\ell-1}(c_k)^{\varphi_k}$.
In the protocol, 
the prover initially commits to a random $P_0\in \FF^{2m-1}$ subject to $\innp{1^{m}|| 0^{m-1}}{P_0}=0$, and 
uses $\beta P_0 + \overline{P} \varphi$ as witness in the inner product protocol. Here 
$\beta\sample \FF\backslash \{0\}$ is randomly chosen by the verifier. This randomization
precludes the need for the inner-product argument to be zero knowledge, and as we will later
see, helps reduce interaction among provers in its distributed variant.
\smallskip  


\begin{comment} 
  %$[p_j(\zeta_1),\ldots,p_j(\zeta_s)]^T =\Phi \times [p_j(\eta_1),\ldots,p_j(\eta_{s+\ell})]^T$ $\forall j\in[h]$ . Define $p(\cdot) = \sum_{j\in [m]} p_j(\cdot)$. 
 %$\Rightarrow [\sum_{j\in[m]} p_j(\zeta_1),\ldots,\sum_{j\in[m]}p_j(\zeta_s)]^T =\Phi \times [\sum_{j\in[m]}p_j(\eta_1),\ldots,\sum_{j\in[m]} p_j(\eta_{s+\ell})]^T$. Define $p(\cdot) = \sum_{j\in [m]} p_j(\cdot)$.
 Observe that, $p(\etabar)^T = \overline{P}^T \times [1^m || 0^m]^T$\\
 $\Rightarrow p(\zetabar)^T =\Phi \times \overline{P}^T \times [1^m || 0^m]^T$\\
 $\Rightarrow p(\zetabar)^T =[1^m || 0^m] \times \overline{P}\times \Phi^T$\\
 Therefore, the check $\sum_{k\in[s]}\sum_{j\in[m]}p_j(\zeta_k)=r^Tb$ reduces to $\innp{p(\zetabar)}{[1^s]} =r^Tb$ that can be viewed as $\innp{[1^m || 0^m] \times \overline{P}\times \Phi^T}{[1^s]} = \innp{[1^m||0^m]^T}{\overline{P}\times \Phi^T \times [1^s]^T} = r^Tb$. 
 \end{comment} 
 
  %Using an inner-product argument
 %the prover can show that the commitment $\mathsf{cm}$ opens to vector $z$ such
 %that $\innp{1^m||0^m}{z}=r^Tb$. Binding property of the commitment ensures that
 %$z=\overline{P}\varphi$ with overwhelming probability.
 
\begin{comment}
\noindent {\em A product codeword Commitment}: 
We introduce an $h\times n$ matrix $P$
which serves as a bridge between the oracle $\pi$ and the vector
$p(\zetabar)$. For $j\in [h]$, let $p_j$ denote the polynomial $\sum_{i\in
[p]}R^i(\alpha_j,\cdot)Q^i(\alpha_j,\cdot)$ (which will be of degree less than $s + \ell$). We define the matrix $P$ by
$P[j,k]=p_j(\eta_k)$ for $j\in [h],k\in [n]$. Noting that the matrix $P$ is in the
product of codes $\rsc{\eta}{n,s+\ell-1}$ and $\rsc{\alpha}{h,2m-1}$ and is fully determined by sub-matrix $\overline{P}$ consisting of the first $2m$ rows and
the first $s+\ell$ columns of $P$, we let the
prover commit to product code-word $P$, via column commitments of  $\overline{P}$ as $c_1,\ldots,c_{s+\ell}$. Intuitively, the code structure on $P$
helps in achieving soundness, as a cheating prover would be forced to commit to 
a $P$ which is substantially different from an honest $P$, and thus increasing
the likelihood of getting ``caught'' in the consistency checks below.\smallskip 

\noindent {\em Consistency of matrix $P$ and $p(\zetabar)$}:
Let $\etabar$ denote the vector
$(\eta_1,\ldots,\eta_{s+\ell})$, and let $\Phi$ denote the $s\times (s+\ell)$
matrix such that $p(\zetabar)=\Phi \times p(\etabar)$. Further, observe that
$p(\etabar)=\overline{P}^T[1^m||0^m]^T$ and thus
$p(\zetabar)=\Phi\overline{P}^T[1^m||0^m]^T$. Now, since
$\innp{1^s}{p(\zetabar)}=p(\zetabar)^T[1^s]=[1^m||0^m]\overline{P}\Phi^T[1^s]=\innp{1^m||0^m}{\overline{P}\varphi}$
where $\varphi=\Phi^T[1^s]$, the inner-product check $\innp{1^s}{p(\zetabar)}=r^Tb$ reduces to $\innp{1^m||0^m}{\overline{P}\varphi} = r^Tb$. 
Given commitment of $P$,  $c_1,\ldots,c_{s+\ell}$, the
commitment to $\overline{P}\varphi$ can be computed as
$\mathsf{cm}=\prod_{k=1}^{s+\ell}(c_k)^{\varphi_k}$. %Using an inner-product argument
%the prover can show that the commitment $\mathsf{cm}$ opens to vector $z$ such
%that $\innp{1^m||0^m}{z}=r^Tb$. Binding property of the commitment ensures that
%$z=\overline{P}\varphi$ with overwhelming probability.
 In the full protocol, 
the prover initially commits to a random $P_0\in \FF^{2m}$ subject to $\innp{1^m}{P_0}=0$ and 
uses $\beta P_0 + \overline{P}\varphi$ as witness in the inner product protocol. Here 
$\beta\sample \FF\backslash \{0\}$ is randomly chosen by the verifier. This randomization
precludes the need for the inner-product argument to be zero knowledge, and as we will later
see, helps reduce interaction among provers in its distributed variant.\smallskip 
\end{comment}


\noindent{\em Consistency of oracle $\pi$ and $P$}: The verifier additionally
needs to determine if the committed $P$ and the oracle $\pi$ are consistent or not. The verifier proceeds to check the
consistency at randomly sampled $t$ positions given by $\{(j_u,k_u): u\in [t]\}$ from
$[h] \times [n]$ for a $t=O(\secpar)$. It queries the oracle for the columns $\pi[\cdot,k_u]$ and the prover for vectors $\ewit[\cdot,j_u,k_u]$ for
$u\in [t]$. Let $X_u$, $u\in [t]$ denote the $p$-length vectors sent by the prover. For an honest prover $X_u=U[\cdot, j_u, k_u]$. For $u\in [t]$, let %\commentA{changed this notation. change in other places}$
$\bm{1}_{j_u}$ denote the unit vector in $\FF^h$ with $1$ in the position
$j_u$. The prover and the verifier run inner-product arguments to establish the following:
\begin{enumerate}[{\rm 1.}]
\item $\innp{\bm{1}_{j_u}}{P[\cdot,k_u]}= X_u[i]$ for $u\in$
\begin{comment} 
$\bm{\alpha} \times \bm{\eta}$ for a $t=O(\secpar)$. It queries the oracle for the columns $\pi[\cdot,k_u]$ and the prover for vectors $\ewit[\cdot,j_u,k_u]$ for
$u\in [t]$. Let $X_u$, $u\in [t]$ denote the $p$-length vectors sent by the prover. For an honest prover $X_u=U[\cdot, j_u, k_u]$. For $u\in [t]$, let $\bm{1}_{j_u}$ denote the unit vector in $\FF^h$ with $1$ in the position
$j_u$. The prover and the verifier run inner-product arguments to establish the following:
\begin{enumerate}[{\rm 1.}]
\item $\innp{\bm{1}_{j_u}}{P[\cdot,k_u]}=\sum_{i\in [p]}R^i(\alpha_{j_u},\eta_{k_u})\cdot X_u[i]$ for $u\in
>>>>>>> 4030958531832f4c52b0d0c2ff8b08777c566d5a
\end{comment}
$[t]$. It is readily checked that for $P$ computed as per the protocol, 
and honest vectors $X_u$, the identity holds. The inner-product
can be checked as in Section ~\ref{sec:matrixcommitment}.

\item $\innp{\bm{1}_{j_u}}{\ewit[i,\cdot,k_u]}=X_u[i]$ for all $u\in [t],i\in
[p]$. %Here $W_{iu}$ denotes the vector $\ewit[i,\cdot,k_u]$. 
Again, 
the inner-products can be verified as in Section ~\ref{sec:matrixcommitment}.
Moreover, the checks for each
$u\in [t]$ can be aggregated, leading to one inner-product check for each $u\in
[t]$.
\end{enumerate}

\noindent{\em Proximity Check for Oracle}: This check forces a prover to commit
to an encoding which is ``close'' to well-formed encoding $\mc{W}$. To check proximity,  %(before sending messages $r,Q$)
the verifier initially  sends a vector $\rho\sample
\FF^p$ and asks the prover to send commitments
$(\tilde{c}_1,\ldots,\tilde{c}_\ell)$ to $\tilde{U}=\sum_{i\in
[p]}\rho_i\ewit[i,\cdot,\cdot]$. It then checks the relations:
\begin{equation}\label{eq:proxchecks}
 \prod_{a=1}^\ell(\tilde{c}_a)^{\Lambda^T_{n,\ell}[a,k_u]}=\prod_{i=1}^p(\pi[i,k_u])^{\rho_i} 
\text{ for } u\in [t].
\end{equation} 
It can be
seen that for an honest computation, both the commitments open to the vector
$\sum_{i\in [p]}\rho_iV_{iu}$ where $V_{iu}=(\ewit[i,1,k_u],\ldots,\ewit[i,m,k_u])$.
%The complete linear check protocol is described in Figure \ref{fig:linearcheck}.
%the protocol $\agginnerproduct$ denotes the protocol 
%for veryfying inner products of several commited vectors with a common vector. 
%The completeness of the linear check protocol can be easily verified. We sketch the proof for its
%soundness, leaving the detailed proof to the Appendix.
%%%%%%%%%%%%%%%%%%%%%%%%%%%%%%%%%%%%%%%%%%%%%%%%%%%%%%%%
\begin{comment} 
in Figure ~\ref{fig:linearcheck}, $\mathsf{pp}$ is the public parameter, consisted of the following entries: \\
{\small$\mathsf{pp} = (\FF, \GG, \rsc{\eta}{n,\ell}, \rsc{\alpha}{h,m},\rsc{\eta}{n,s+\ell-1},\rsc{\eta}{n,2\ell-1}, \rsc{\alpha}{h,2m-1}, \bm{g},h)$}
%Linear Check of Graphene
\begin{figure}[h!]
	{\footnotesize
		%\centering
		\begin{framed}
			\noindent{$\linearcheck(\mathsf{pp},A\in \mc{M}_{M,N},b\in \FF^M,[\pi];\ewit)$}: 			%\pnote{why $\wit$ is not part of the witness of linear check protocol?}
			
			\noindent{\bf Relation}: $\ewit=\open(\pi)\wedge A\wit=b$ for $\wit=\dec(\ewit)$.
			
			\begin{enumerate}[{\rm 1.}]
				\item $\verifier\rightarrow\prover$: $\rho\sample \FF^p$.
				\item $\prover$ computes: $\tilde{U}=\sum_{i\in [p]}\rho_i\ewit[i,\cdot,\cdot]$, 
				commitments $\tilde{c}_1,\ldots,\tilde{c}_\ell$ as in Section ~\ref{sec:matrixcommitment}.
				\item $\prover\rightarrow\verifier$: $\tilde{\bm{c}}=(\tilde{c}_1,\ldots,\tilde{c}_\ell)$.
				\item $\verifier\rightarrow\prover$: $r\sample \FF^M$.
				\item $\prover\leftrightarrow\verifier$ compute: Polynomials $R^i$, $i\in [p]$ interpolating $R=r^TA$
				as in Section ~\ref{sec:lincheck}. 
				\item $\prover$ computes: Matrix $P$ from $R$ and $\ewit$ as described in Section ~\ref{sec:lincheck}. Samples $P_0\sample \FF^{2m-1}$, $\omega_0\sample \FF$ and $c_0\gets \comm(P_0,\omega_0)$.
				Computes commitments $c_1,\ldots,c_{s+\ell-1}$ from $P$.
				\item $\prover\rightarrow\verifier$: $c_0,c_1,\ldots,c_{s+\ell-1}$.
				\item $\verifier\rightarrow\prover$: $Q=\{(j_u,k_u):u\in [t]\}$ for $Q\sample [h]\times [n]$ for $u\in [t]$.
				\item $\verifier\rightarrow\pi$: $\{k_u:u\in [t]\}$.
				\item $\prover\rightarrow\verifier$: $\ewit[\cdot,j_u,k_u]$ for $u\in [t]$.
				\item $\pi\rightarrow\verifier$: $\pi[\cdot,k_u]$ for $u\in [t]$.
				\item $\verifier\rightarrow\prover$: $\delta\sample \FF^p$, $\beta\sample \FF\backslash \{0\}$. 
				\item $\prover$ and $\verifier$ run inner product arguments to check:
				\begin{enumerate}
					\item $\innerproduct(\mathsf{pp},\bm{1}_{j_u}^T\Lambda_{h,2m-1},\mathsf{cm}_{k_u},v_u;\overline{P}[\cdot,k_u])$ 
					for $u\in [t]$ where $\mathsf{cm}_{k_u}=\prod_{a=1}^{s+\ell-1}c_a^{\Lambda^T_{n,s+\ell-1}[a,k_u]}$, 
					$v_u=\sum_{i=1}^pR^i(\alpha_{j_u},\eta_{k_u})U[i,j_u,k_u]$ (check consistency of $P$ with $\pi$).
					\item $\innerproduct(\mathsf{pp},1^m||0^{m-1},\mathsf{cm},r^Tb;z)$ where $z=\beta P_0+\overline{P}\varphi$ and $\mathsf{cm}=c_0^{\beta}\cdot \prod_{a=1}^{s+\ell-1}c_k^{\varphi_k}$ (check the condition $r^TAw=r^Tb$).
					\item $\innerproduct(\mathsf{pp},\bm{1}_{j_u}^T\Lambda_{h,m},C_u,\innp{\delta}{X_u})$ for $u\in [t]$ 
					where $C_u=\prod_{i=1}^p(\pi[i,k_u])^{\delta_i}$ (consistency of $X_u$ with $\pi$). 
				\end{enumerate}
				\item $\verifier$ checks: $\prod_{a=1}^\ell(\tilde{c}_a)^{\Lambda^T_{n,\ell}[a,k_u]}=\prod_{i=1}^p(\pi[i,k_u])^{\rho_i}$ for $u\in [t]$ (check proximity of $\ewit$ to $\mc{W}_1$).
				\item $\verifier$ accepts if all the checks succeed.
			\end{enumerate}
		\end{framed}
		\caption{Linear Check Protocol}
		\label{fig:linearcheck}
	}
\end{figure}
%%%%%%%%%%%%%%%%%%%%%%%%%%%%%%%%%%%%%%%%
\end{comment}

\begin{lemma}[Soundness]\label{lem:linearcheck_sound}
For all polynomially bounded provers $P^\ast$ and all $\pi\in \GG^{p\times n}$,
$A\in \FF^{M\times N}, b\in \FF^M$, there exists an expected polynomial time
extractor $\extr$ with rewinding access to transcript $\mathsf{tr}=\langle
P^\ast(\cdot),\verifier^\pi(\cdot)\rangle$ such that $\extr$ either breaks the 
commitment binding or outputs a witness with overwhelming probability whenever 
$P^\ast$ succeeds, i.e,
{\small
\begin{align*}
\condprob{\begin{array}{c}
\ewit=\open(\pi)\wedge \\
A\wit=b
\end{array}
}{
\begin{array}{c}
\sigma\gets \gen(\secparam) \\
\ewit\gets \extr^{\mathsf{tr}}(\bm{x},\sigma) \\
\wit\gets \dec(\ewit)
\end{array}}\geq
\epsilon(P^\ast)-\kappa_{lc}(\secpar)
\end{align*}
}
where $\epsilon(P^\ast):= \condprob{\langle P^\ast(\bm{x},\sigma),\verifier^\pi(\bm{x},\sigma)\rangle=1}{\sigma\gets \gen(\secparam)}$ denotes the success probability of $P^\ast$, $\kappa_{lc}$ denotes a negligible function, and $\bm{x}$ denotes the tuple $(A,b,M,N)$.
\end{lemma}

\noindent{\bf Remark}: When the simulator rewinds the transcript $\mathsf{tr}$,
and starts with fresh randomness for the verifier, it gets the prover message as
response if the randomness was used as the challenge to the prover, and it gets
oracle response if the randomness was used as query to the oracle.

We give a proof-sketch of Lemma ~\ref{lem:linearcheck_sound} in Appendix ~\ref{lem:linearcheck_sound}.
%\pnote{reference missing}

 
\begin{comment}
\subsection{Quadratic Check Protocol}
We now describe the IPCP which allows a prover to prove knowledge of vectors
$\wit_x$, $\wit_y$ and $\wit_z$ in $\FF^N$, satisfying $\wit_x\circ \wit_y =
\wit_z$. Once again, the protocol requires the prover to construct encodings
$\ewit_x=\enc(\wit_x)$, $\ewit_y=\enc(\wit_y)$ and $\ewit_z=\enc(\wit_z)$ as
described in Section \ref{sec:witencoding}. Thereafter, the prover uses
commitment scheme $\comm$ to commit to these encodings as $\comoracle_x =
\comm(\ewit_x)$, $\comoracle_y = \comm(\ewit_y)$ and $\comoracle_z = \comm(\ewit_z)$. 
The prover forms the oracle $\pi\in \GG^{3p\times n}$ by vertically stacking the
$p\times n$ matrices $\comoracle_x,\comoracle_y$ and $\comoracle_z$. As before,
for a query $Q\subseteq [n]$, the oracle answers with columns $\pi[\cdot,k]$ for
$k\in Q$. The columns returned by the oracle can be parsed into constituent columns 
$\comoracle_x[\cdot,k]$, $\comoracle_y[\cdot,k]$ and $\comoracle_z[\cdot,k]$
canonically. We again discuss the key ingredients of the protocol.

\noindent{\em Probabilistic Reduction}: Let $Q^i_x,Q^i_y$ and $Q^i_z, i\in [p]$ be the
polynomials interpolating the $i^{th}$ slices of $\wit_x$, $\wit_y$ and $\wit_z$
 as in Section \ref{sec:witencoding}. Then for vectors $\wit_x,\wit_y,\wit_z$ satisfying
$\wit_x\circ \wit_y=\wit_z$, the polynomials $Q^i=Q^i_x\cdot Q^i_y - Q^z_i$ interpolate
$\bm{0}^{m\times s}$ on the set $\{(\alpha_j,\zeta_k):j\in [m],k\in [s]\}$ for all $i\in [p]$. This can be probabilistically
checked by checking that the polynomial $F := \sum_{i\in [p]}r_iQ^i$ interpolates
$\bm{0}^{m\times s}$ on the above set for randomly sampled $r\in \FF^p$. Once again, we
ask the prover to ``commit'' to $F$ using a tamper resistant structure, like a codeword,
which enables the verifier to check the aforementioned condition, as well as to
ensure that the commitment is consistent with oracle replies and prior
messages.

\noindent{\em Reduction to Inner Products}: The prover computes 
$h\times n$ matrix $P$ given by $P[j,k]=F(\alpha_j,\eta_k)$. It commits to $P$
using commitments $(c_1,\ldots,c_{2\ell})$ to the first $2\ell$ columns of $P$.
Note that each row of $P$ commits to univariate component polynomials
$F(\alpha_j,\cdot)$ of $F$ via their evaluations of $\bm{\eta}$. To check that
$F$ interpolates $\bm{0}^{m\times s}$ on the points
$\{(\alpha_j,\zeta_k)\}_{j\in [m],k\in [s]}$, the verifier checks that
$p(\cdot) := \sum_{j\in [m]}\gamma_jF(\alpha_j,\cdot)$ interpolates $\bm{0}^s$ on
$\overline{\bm{\zeta}}$ for randomly sampled $\gamma=(\gamma_1,\ldots,\gamma_m)\in \FF^m$.
Again, the verifier checks $p(\overline{\bm{\zeta}})=\bm{0}^s$ via the inner product
check $\innp{\tau}{p(\overline{\bm{\zeta}})}=0$ for a random $\tau\in \FF^s$. As in the
linear check protocol, using $p(\overline{\bm{\zeta}})=\Phi p(\overline{\bm{\eta}})$, we
get the following inner product check
$\innp{(\gamma,0^{h-m})}{\overline{P}\varphi}=0$ where $\varphi=\Phi^T\tau$. 
The commitment to the vector $\overline{P}\varphi$ can be homomorphically computed
from $c_1,\ldots,c_{2\ell}$.

\noindent{\em Checking consistency with Oracle}: As in the linear check, the
verifier uniformly and independently samples $(j_u,k_u)\in [h]\times [n]$ for
$u\in [t]$, and queries the oracle $\pi$ for columns $\pi[\cdot,k_u]$. Let
$\pi_x[\cdot,k_u]$, $\pi_y[\cdot,k_u]$ and $\pi_z[\cdot,k_u]$ denote the parse
of $\pi[\cdot,k_u]$ into commitments corresponding to $\ewit_x,\ewit_y$ and
$\ewit_z$ respectively. Further, the verifier asks prover for vectors
$\ewit_x[\cdot,j_u,k_u]$, $\ewit_y[\cdot,j_u,k_u]$ and $\ewit_z[\cdot,j_u,k_u]$
for $u\in [t]$. The verifier then checks the following:
\begin{enumerate}[{\rm (i)}]
\item For all $u\in [t]$: $P[j_u,k_u]=\sum_{i\in
[p]}r_i(\ewit_x[i,j_u,k_u]\cdot\ewit_y[i,j_u,k_u]-\ewit_z[i,j_u,k_u])$.
\item Checks that vectors $\ewit_x[\cdot,j_u,k_u]$ are consistent with
commitments $\pi_x[\cdot,k_u]$ as in linear check protocol. Similar checks are
made for $\ewit_y[\cdot,j_u,k_u]$ and $\ewit_z[\cdot,j_u,k_u]$.
\end{enumerate}
We present the full protocol in Figure \ref{fig:quadcheck}. The completeness of
the protocol can again be verified by direct calculation. We state the soundness
of the protocol below:
\end{comment}

\subsection{Quadratic Check Protocol}\label{sec:quadcheck}
We now describe the IPCP which allows a prover to prove knowledge of vectors
$\wit_x$, $\wit_y$ and $\wit_z$ in $\FF^N$, satisfying $\wit_x\circ \wit_y =
\wit_z$. As before we view $\wit_x, \wit_y$ and $\wit_z$ as $p \times m \times s$ matrices, where $N=pms$. 
As described previously in Sections ~\ref{sec:witencoding} and ~\ref{sec:construct_oracle}, the prover obtains encodings $\ewit_a \leftarrow \enc(\wit_a)$ and corresponding 
commitments $\comoracle_a \leftarrow \comm(\ewit_a)$ $\forall a\in \{x,y,z\}$. The prover
sets up the oracles $\pi_a := \comoracle_a$ for $a\in \{x,y,z\}$. For a query $Q$,
the verifier is provided the columns $\pi_a[\cdot,k]$ for $a\in \{x,y,z\}$ and $k\in Q$. 
Here we consider oracle as consisting of three sub-oracles for simplicity of description. For
concrete efficiency, $\wit_x$, $\wit_y$ and $\wit_z$ can be encoded together, and the oracle
access can be provided to the combined encoding. We will defer this optimization till the 
discussion on concrete efficiency. The full protocol appears in Figure ~\ref{fig:quadcheck}. 

%Once again, the protocol requires the prover to construct encodings
%$\ewit_x=\enc(\wit_x)$, $\ewit_y=\enc(\wit_y)$ and $\ewit_z=\enc(\wit_z)$ as
%described in Section \ref{sec:witencoding}. Thereafter, the prover uses
%commitment scheme $\comm$ to commit to these encodings as $\comoracle_x =
%\comm(\ewit_x)$, $\comoracle_y = \comm(\ewit_y)$ and $\comoracle_z = \comm(\ewit_z)$. 
%The prover forms the oracle $\pi\in \GG^{3p\times n}$ by vertically stacking the
%$p\times n$ matrices $\comoracle_x,\comoracle_y$ and $\comoracle_z$. As before,
%for a query $Q\subseteq [n]$, the oracle answers with columns $\pi[\cdot,k]$ for
%$k\in Q$. The columns returned by the oracle can be parsed into constituent columns 
%$\comoracle_x[\cdot,k]$, $\comoracle_y[\cdot,k]$ and $\comoracle_z[\cdot,k]$
%canonically. We again discuss the key ingredients of the protocol.

\noindent{\em Reduction to Inner Product}: Let $Q^i_x,Q^i_y$ and $Q^i_z, i\in [p]$ be the
polynomials interpolating the $i^{th}$ slices of $\wit_x$, $\wit_y$ and $\wit_z$ respectively.
Then for vectors $\wit_x,\wit_y,\wit_z$ satisfying
$\wit_x\circ \wit_y=\wit_z$, the polynomials $Q^i=Q^i_x\cdot Q^i_y - Q^z_i$ interpolate
$\bm{0}^{m\times s}$ on the set $\{(\alpha_j,\zeta_k):j\in [m],k\in [s]\}$ for all $i\in [p]$. 
This can be probabilistically
checked by checking that the polynomial $F := \sum_{i\in [p]}r_iQ^i$ interpolates
$\bm{0}^{m\times s}$ on the same set for randomly sampled $r\in \FF^p$. 
To check that $F$ interpolates $\bm{0}^{m\times s}$ on the points
$\{(\alpha_j,\zeta_k)\}_{j\in [m],k\in [s]}$, the verifier checks that
$p(\cdot) := \sum_{j\in [m]}\gamma_jF(\alpha_j,\cdot)$ interpolates $\bm{0}^s$ on
$\overline{\bm{\zeta}}$ for randomly sampled $\gamma=(\gamma_1,\ldots,\gamma_m)\in \FF^m$.
Again, the verifier checks $p(\overline{\bm{\zeta}})=\bm{0}^s$ via the inner product
check $\innp{\tau}{p(\overline{\bm{\zeta}})}=0$ for a random $\tau\in \FF^s$. 

\noindent{\em Intermediate Commitment:} As in the linear check protocol, we
construct $h\times n$ matrix $P$ which serves as a bridge between the oracles
$\pi_a$ $\forall a \in \{x,y,z\}$ and the vector $p(\overline{\bm{\zeta}})$.
For $j\in [h]$, let $p_j(\cdot)$ denote the polynomial $\sum_{i\in[p]}
r_iQ^i(\alpha_j, \cdot)$. We define the matrix $P$ by $P[j,k]= p_j(\eta_k)$ for
$j\in[h], k\in[n]$. The matrix $P$ is in the product code $L_4\otimes L_5$. As
before we commit to $P$ by committing to the submatrix $\overline{P}$
consisting of the first $2m$ rows and the first $2\ell$ columns of $P$. Let
$c_1,\ldots,c_{2\ell}$ be the commitments to the columns of $\overline{P}$.

\noindent{\em Consistency of $P$ and $p(\overline{\bm{\zeta}})$:} Let
$\overline{\bm{\eta}}$ denote the vector $(\eta_1, \ldots, \eta_{2\ell})$, and
let $\Phi$ denote the $s\times 2\ell$ matrix such that
$p(\overline{\bm{\zeta}}) = \Phi p(\overline{\bm{\eta}})$. Observe
that $p(\overline{\bm{\eta}}) = \overline{P}^T [\gamma||0^m]^T$ and thus
$p(\zetabar) = \Phi \overline{P}^T [\gamma||0^m]^T$. Now we have
$\innp{\tau}{p(\zetabar)} = p(\zetabar)^T \tau = [\gamma||0^m] \overline{P}
\Phi^T \tau = \innp{[\gamma||0^m]^T}{\overline{P}\varphi}$ where $\varphi =
\Phi^T \tau$. Given commitments $c_1, \ldots, c_{2\ell}$, the commitment to
$\overline{P}\varphi$ can be computed as $\cm = \prod_{k\in[2\ell]} (c_k)^{\varphi_k}$. Using an inner product argument the prover can show that the commitment
$\cm$ opens to vector $z$ such that $\innp{[\gamma||0^m]^T}{z} = 0$. Binding
property of the commitment ensures that $z = \overline{P}\varphi$ with
overwhelming probability.
As in the linear check, the prover uses a random vector $P_0\in \FF^{2m}$ such
that $P_0[j]=0$ for $j\in [m]$ to blind the vector $\overline{P}\varphi$. The
prover initially commits to $P_0$ by sending a commitment $c_0$ to it, and later
sends uses the vector $z=\beta P_0 + \overline{P}\varphi$ in the preceeding
inner product check.

\noindent{\em Consistency of the oracles and  $P$:} 
As in the linear check, the verifier checks
the consistency at randomly sampled positions $Q=\{(j_u,k_u) : u\in[t] \}
\subset [h]\times[n]$. It queries the oracles for the columns
$\pi_a[\cdot, k_u]$ for $a\in \{x,y,z\}$ and $u\in [t]$. 
Then it queries the prover for vectors $X_u = \ewit_x[\cdot,j_u,k_u], Y_u =
\ewit_y[\cdot, j_u,k_u], Z_u = \ewit_z[\cdot,j_u,k_u]$ for $u\in [t]$. Let
$f_u$ denote the unit vector in $\FF^h$ with $1$ in the $j_u^{th}$ position. The
prover and the verifier run inner product arguments to establish the following: 

\begin{enumerate}[{\rm 1.}]
	\item $\innp{f_u}{P[\cdot,k_u]} = \sum_{i\in[p]} r_i[X_u[i]\cdot Y_u[i]
- Z_u[i]]$ for $u\in[t]$. It can be seen that the identity holds for honestly
computed $P$ and honest vectors $X_u,Y_u$ and $Z_u$. The inner product may be 
verified using an inner product protocol as described in Section ~\ref{sec:matrixcommitment}.
	
	\item $\innp{f_u}{\ewit_x[i,\cdot,k_u]} = X_u[i]$, $\innp{f_u}{\ewit_y[i,\cdot,k_u]}=Y_u[i]$ and $\innp{f_u}{\ewit_z[i,\cdot,k_u]}=Z_u[i]$. These inner products check the consistency of the vectors sent by the prover with the respective oracles. These can be verified in a similar manner to that in the linear check protocol. 
\end{enumerate}

\noindent{\em Proximity Check for Oracle:} We combine the proximity check for the three 
encodings $\ewit_x,\ewit_y$ and $\ewit_z$. The verifier sends a vector
$\rho\sample \FF^{3p}$ and asks the verifier to commit to the matrix
$\tilde{U}=\sum_{i=1}^p\rho_i\ewit_x[i,\cdot,\cdot]+
\sum_{i=p+1}^{2p}\rho_i\ewit_y[i,\cdot,\cdot]+\sum_{i=2p+1}^{3p}\rho_i\ewit_z[i,\cdot,\cdot]$
by sending commitments $\tilde{\bm{c}}=(\tilde{c}_1,\ldots,\tilde{c}_\ell)$. Thereafter, for
$u\in [t]$, the verifier checks: 
{\small
\begin{align}\label{eq:combinedproximity}
\prod_{a\in[\ell]} (\tilde{c}_a)^{\Lambda_1^T[a,k_u]} = \prod_{i=1}^{p} (\pi_x[i,k_u])^{\rho_i}\cdot (\pi_y[i,k_u])^{\rho_{p+i}}\cdot(\pi_z[i,k_u])^{\rho_{2p+i}}
%\innph{\Lambda_1[k_u,\cdot]}{\tilde{\bm{c}}}=
%\innph{\rho}{\pi_x[\cdot,k_u]\,||\,\pi_y[\cdot,k_u]\,||\,\pi_z[\cdot,k_u]}.
\end{align}
}
\begin{figure}[t!]
{\small
	%\centering
	\begin{framed}
		\noindent{$\quadcheck(\mathsf{pp},[\pi_x],[\pi_y],[\pi_z];\ewit_x, \ewit_y, \ewit_z)$}:
		
		\noindent{\bf Relation}: $\ewit_a=\open(\pi_a)$ for $a\in \{x,y,z\}$,
$\wit_x \circ \wit_y = \wit_z$ where  $\wit_a=\dec(\ewit_a)$ for $a\in \{x,y,z\}$.
		
		\begin{enumerate}[{\rm 1.}]
			\item $\verifier\rightarrow\prover$: $\rho\sample \FF^p$.
			\item $\prover$ computes: $\tilde{U}=\sum_{i\in [p]}\rho_i\ewit[i,\cdot,\cdot]$, 
			commitments $\tilde{c}_1,\ldots,\tilde{c}_\ell$ as in Section ~\ref{sec:matrixcommitment}.
			\item $\prover\rightarrow\verifier$: $\tilde{\bm{c}}=(\tilde{c}_1,\ldots,\tilde{c}_\ell)$.
			\item $\verifier\rightarrow\prover$: $r\sample \FF^p$.
			%\item $\prover\leftrightarrow\verifier$ compute: Polynomials $R^i$, $i\in [p]$ interpolating $R=r^TA$ as in Section ~\ref{sec:quadcheck}. 
			\item $\prover$ computes: $p_j(\cdot) = \sum_{i\in[p]} r_i[Q^i_x(\alpha_j,\cdot)Q^i_y(\alpha_j,\cdot) - Q^i_z(\alpha_j,\cdot)]$ $\forall j\in [h]$
			%\item 
			$\prover$ computes Matrix $P$ such that $P[j,k] = p_j(\eta_k)$ as described in Section ~\ref{sec:quadcheck}. %Samples $P_0\sample \FF^m$, $\omega_0\sample \FF$ and $c_0\gets \comm(P_0,\omega_0)$.
			Computes commitments $c_1,\ldots,c_{2\ell}$ from $P$. Prover also
samples $P_0\sample \FF^{2m}$ with $P_0[1:m]=0^m$ and computes commitment $c_0$
to $P_0$. 
			\item $\prover\rightarrow\verifier$: $c_0$, $c_1,\ldots,c_{2\ell}$.
			\item $\verifier\rightarrow\prover$: $Q=\{(j_u,k_u):u\in [t]\}$ for $(j_u,k_u)\sample [h]\times [n]$ for $u\in [t]$. And $\tau \sample \FF^s$, $\gamma \sample \FF^m$.
			\item $\verifier\rightarrow\pi$: $\{k_u:u\in [t]\}$.
			\item $\prover\rightarrow\verifier$: $X_u=\ewit_x[\cdot,j_u,k_u]$ , $Y_u=\ewit_y[\cdot,j_u,k_u]$ and $Z_u=\ewit_z[\cdot,j_u,k_u]$ for $u\in [t]$.
			
			\item $\pi\rightarrow\verifier$: $\pi[\cdot,k_u]$ for $u\in [t]$.
			\item $\verifier\rightarrow\prover$: $\delta\sample
\FF^p$, $\beta_x\sample \FF$, $\beta_y\sample \FF$, $\beta_z\sample \FF$,
$\beta\sample \FF\backslash \{0\}$.
			\item $\prover$ computes:
$W_u=\sum_{i=1}^p\delta_i\big(\beta_x\ewit_x[i,\cdot,k_u]+\beta_y\ewit_y[i,\cdot,k_u]+\beta_z\ewit[i,\cdot,k_u]\big)$.
%$\beta\sample \FF\backslash \{0\}$. 
			\item $\prover\leftrightarrow\verifier$ compute:
$V_u=\beta_xX_u+\beta_yY_u+\beta_zZ_u$ for $u\in [t]$.
$T_u=\beta_xC_u+\beta_yD_u+\beta_zE_u$, for $u\in [t]$ where
$C_u=\innph{\delta}{\pi_x[\cdot,k_u]}$, $D_u=\innph{\delta}{\pi_y[\cdot,k_u]}$
and $E_u=\innph{\delta}{\pi_z[\cdot,k_u]}$.
			\item $\prover$ and $\verifier$ run inner product arguments to check:
			\begin{enumerate}
				\item $\innerproduct(\mathsf{pp},f_u^T\Lambda_5,\mathsf{cm}_{k_u},v_u;\overline{P}[\cdot,k_u])$ for $u\in [t]$ where $\mathsf{cm}_{k_u}=\sum_{a=1}^{2\ell}\Lambda_4^T[a,k_u]c_a$, 
				$v_u=\sum_{i=1}^p r_i[X_u[i]\cdot Y_u[i] - Z_u[i]]$ (check consistency of $P$ with $\pi$).
				\item $\innerproduct(\mathsf{pp},\gamma||0^m,\mathsf{cm},0;z)$
where $z=\beta P_0 + \overline{P}\varphi$, $\varphi = \Phi^T\tau$ and
$\mathsf{cm} = \beta c_0+\sum_{a=1}^{2\ell} \varphi_ac_a$ %(check the condition $r^TAw = r^Tb$).
				\item
$\innerproduct(\mathsf{pp},f_u^T\Lambda_2,T_u,\innp{\delta}{T_u};W_u[1:m])$ (consistency of $X_u, Y_u, Z_u$ with $\pi$). 
			\end{enumerate}
			\item $\verifier$ checks proximity of $\ewit_x,\ewit_y$
and $\ewit_z$ according to the Equation \eqref{eq:combinedproximity}.
			\item $\verifier$ accepts if all the checks succeed.
		\end{enumerate}
	\end{framed}
	\caption{Quadratic Check Protocol}
	\label{fig:quadcheck}
}
\end{figure}

%As in the linear check protocol, using $p(\overline{\bm{\zeta}})=\Phi p(\overline{\bm{\eta}})$, we get the following inner product check
%$\innp{(\gamma,0^{h-m})}{\overline{P}\varphi}=0$ where $\varphi=\Phi^T\tau$. 
%Once again, we
%ask the prover to ``commit'' to $F$ using a tamper resistant structure, like a codeword,
%which enables the verifier to check the aforementioned condition, as well as to
%ensure that the commitment is consistent with oracle replies and prior
%messages.
%\noindent{\em Reduction to Inner Products}: 
%The prover computes 
%$h\times n$ matrix $P$ given by $P[j,k]=F(\alpha_j,\eta_k)$. 

%It commits to $P$ using commitments $(c_1,\ldots,c_{2\ell})$ to the first $2\ell$ columns of $P$.
%Note that each row of $P$ commits to univariate component polynomials
%$F(\alpha_j,\cdot)$ of $F$ via their evaluations of $\bm{\eta}$. 
%Again, the verifier checks $p(\overline{\bm{\zeta}})=\bm{0}^s$ via the inner product
%check $\innp{\tau}{p(\overline{\bm{\zeta}})}=0$ for a random $\tau\in \FF^s$. As in the
%linear check protocol, using $p(\overline{\bm{\zeta}})=\Phi p(\overline{\bm{\eta}})$, we
%get the following inner product check
%$\innp{(\gamma,0^{h-m})}{\overline{P}\varphi}=0$ where $\varphi=\Phi^T\tau$. 
%The commitment to the vector $\overline{P}\varphi$ can be homomorphically computed
%from $c_1,\ldots,c_{2\ell}$.

%\noindent{\em Checking consistency with Oracle}: As in the linear check, the
%verifier uniformly and independently samples $(j_u,k_u)\in [h]\times [n]$ for
%$u\in [t]$, and queries the oracle $\pi$ for columns $\pi[\cdot,k_u]$. Let
%$\pi_x[\cdot,k_u]$, $\pi_y[\cdot,k_u]$ and $\pi_z[\cdot,k_u]$ denote the parse
%of $\pi[\cdot,k_u]$ into commitments corresponding to $\ewit_x,\ewit_y$ and
%$\ewit_z$ respectively. Further, the verifier asks prover for vectors
%$\ewit_x[\cdot,j_u,k_u]$, $\ewit_y[\cdot,j_u,k_u]$ and $\ewit_z[\cdot,j_u,k_u]$
%for $u\in [t]$. The verifier then checks the following:
%\begin{enumerate}[{\rm (i)}]
%\item For all $u\in [t]$: $P[j_u,k_u]=\sum_{i\in
%[p]}r_i(\ewit_x[i,j_u,k_u]\cdot\ewit_y[i,j_u,k_u]-\ewit_z[i,j_u,k_u])$.
%\item Checks that vectors $\ewit_x[\cdot,j_u,k_u]$ are consistent with
%commitments $\pi_x[\cdot,k_u]$ as in linear check protocol. Similar checks are
%made for $\ewit_y[\cdot,j_u,k_u]$ and $\ewit_z[\cdot,j_u,k_u]$.
%\end{enumerate}
%We present the full protocol in Figure \ref{fig:quadcheck}. The completeness of
%the protocol can again be verified by direct calculation. We state the soundness
%of the protocol below:


\begin{lemma}[Soundness]\label{lem:quadcheck_sound}
For all polynomially bounded provers $P^\ast$ and all $\pi\in \GG^{3p\times n}$,
there exists an expected polynomial time extractor $\extr$ with rewinding access
to the transcript oracle $\mathsf{tr}=\innp{P^\ast(\cdot)}{\verifier^{\pi}(\cdot)}$
such that either $\extr$ breaks the commitment binding, or it outputs a witness
with overwhelming probability whenever $P^\ast$ succeeds, i.e,

{\footnotesize
\begin{align*}
\condprob{
\begin{array}{c}
{[}\ewit_x||\ewit_y||\ewit_z{]}=\open(\pi)\wedge \\
\wit_z=\wit_x\circ\wit_y
\end{array}
}{
\begin{array}{c}
\sigma %\sample \gen(\secparam) \\
\leftarrow \gen(\secparam) \\
{[}\ewit_x||\ewit_y||\ewit_z{]}%\sample \extr^{\mc{O}}(\sigma)\\
\leftarrow \extr^{\mathsf{tr}}(\sigma) \\ 
\wit_a=\dec(\ewit_a), a\in \{x,y,z\}
\end{array}
}\\
\geq \epsilon(P^\ast) - \kappa_{\rm qd}(\secpar)
\end{align*}
}
for some negligible function $\kappa_{qd}$. In the above, $\epsilon(P^\ast)$
denotes the success probability of the prover $P^\ast$ as before.
\end{lemma}
\begin{proof}
The proof is similar to the proof of the linear check protocol. Using similar
arguments, one can show that the above Lemma holds with:

{\footnotesize
\begin{equation*}
\kappa_{qd}(\secpar) := \left(1-\frac{e}{n}\right)^t +
\left(\frac{2m}{h}+\left(1-\frac{2m}{h}\right)\left(\frac{2\ell+e}{n}\right)\right)^t
+ \frac{O(|C|)}{|\FF|}
\end{equation*}
}
\end{proof}

\subsection{Zero Knowledge}
%\pnote{Change the simulation according to the new protocol}
We now prove protocols $\linearcheck$ and $\quadcheck$ to be honest verifier
zero knowledge by designing simulators for them. We will extend the verifier's
view with commitment openings (committed vector, randomness) for the inner
product protocols. This has two benefits: (i) the inner product argument need
not be zero knowledge, and (ii) in the distributed setting, the openings can be
shared with the aggregator which can complete the interaction with the verifer
without further involvement of the provers. The verifier's veiw for
the $\linearcheck$ protocol consists of:
\begin{itemize}[\leftmargin=0pt]
\item {\em Verifier Randomness}: The verifier's messages consist of $\rho, r, Q = \{(j_u,k_u)\}_{u\in [t]}, \beta, \delta$.
%Vector $\rho \in \FF^p$ for checking proximity, vector $r\in \FF^M$ as part of the reduction $r^TA\wit=r^Tb$, query position $Q= \{(j_u,k_u)\}_{u\in [t]}$ ,$\beta\in \FF^\ast$ used in randomizing the vector $\overline{P}\varphi$ and $\delta \in \FF^p$ to aggregate inner product checks.
%$\tau,\delta$ as part of $\proximityTwoD$ subprotocol in Step 10, $\rho\sample \FF^p$ as the random vector for compressing in subprotocol $\proximityThreeD$ in Step 15, $\tilde{\tau},\tilde{\delta}$ for the second invocation of $\proximityTwoD$ from within $\proximityThreeD$. We do not include the query positions as part of the subprotocol $\proximityThreeD$ as we assume that the same query positions sampled in Step 5 are used there. Summarizing, the verifier randomness consists of $\rho, r, Q, \beta, \delta$. %$\tau,\delta,\rho,\tilde{\tau},\tilde{\delta}$.

\item {\em Commitments}: The prover sends following commitments: $c_0, c_1,
\ldots, c_{s+\ell-1}$ and $\tilde{c}_1, \ldots, \tilde{c}_{\ell}$. The oracle
response consists of  $\pi[\cdot,k_u]$, for $u\in [t]$. We subsume the oracle
responses by including the whole oracle $\pi$ in the extended view.
%Commitment $c_0$ to the first $2m$ entries of the random codeword $P_0$ sampled in Step 6, commitments $c_1,\ldots,c_{s+\ell}$ to commit to matrix $P$, commitments $\tilde{c}_1,\ldots,\tilde{c}_{\ell}$ to the matrix $\tilde{U}$ used for proximity check. Additionally, the view contains commitments $\pi[\cdot,k_u]$ for $u\in [t]$ as part of oracle query response. Summarizing, the view consists of commitments $c_0,c_1,\ldots,c_{s+\ell}$, $\tilde{c}_1, \ldots, \tilde{c}_{\ell} , \pi$, as a part of the extended view of $\verifier$.%$\{\pi[\cdot,k_u]\}_{u\in [t]}$.

\item {\em Vectors}: The view includes vectors
$X_u=\ewit[\cdot,j_u,k_u]$ for $u\in [t]$, the vector $z = \beta P_0 +
\overline{P} \varphi$,  and vectors $P[\cdot, k_u]$ for $u\in [t]$.  We drop
$\{X_u\}_{u\in [t]}$ and $\{P[\cdot,k_u]\}_{u\in [t]}$ from the view as these
can be derived from $\ewit[\cdot,\cdot,k_u]$ and $r$. Thus, the vectors in the
view consist of: $z,\{\ewit[\cdot,\cdot,k_u]\}_{u\in [t]}$.

\item {\em Commitment Randomness}: We include randomness corresponding to
witness vectors in the inner product subprotocols as part of the extended view.
We include randomness $w_u$, $u\in
[t]$, corresponding to vectors $\overline{P}[\cdot,k_u]$, 
randomness $w$ for the vector $z=\beta P_0 +
\overline{P}\varphi$, randomness $O[\cdot,k_u]$, $u\in [t]$ 
corresponding to commitments $\pi[\cdot,k_u]$. These can be used to compute
randomness for the inner product protocols establishing consistency of vectors
$X_u$, $u\in [t]$ with the oracle.
 %$\nu=\nu_0+\sum_{a\in [s+\ell]}\mu_a\omega_a$ for the vector $z$ in the subprotocol $\proximityTwoD$ in Step 10 (here $\mu=\mc{T}\tau$), $\omega=\beta\omega_0+\sum_{a\in [s+\ell]}\varphi_ac_a$ for inner product in Step 11, $\chi_u=\sum_{a\in [s+\ell]}T[a,k_u]c_a$ for $u\in [t]$ for the inner products in Step 13, $\{O[\cdot,k_u]\}_{u\in [t]}$ for aggregate inner product arguments in Step 14, $\tilde{\nu}=\tilde{\nu}_0+\sum_{a\in [\ell]}\tilde{\mu}_a\tilde{c}_a$ for the vector $\tilde{z}$ in the $\proximityTwoD$ protocol called as part of $\proximityThreeD$ protocol in Step 15. Summarizing, the view includes $\nu,\omega,\{\chi_u\}_{u\in [t]},\{O[\cdot,k_u]\}_{u\in [t]},\tilde{\nu}$.
\end{itemize}
Next, we describe a simulator that outputs a view indistinguishable from the
above view.

\noindent{\bf Simulator}: 
%\begin{itemize}
 $\Sim$ picks $\{\rho, r, Q, \delta, \beta \}$ uniformly at random from their
respective domains. Then it picks $\ewit[\cdot, \cdot, k_u]$ uniformly such that
$\ewit[i,\cdot,k_u]\in \rsc{\alpha}{h,2m-1}$ for all $i\in [p]$ and $u\in [t]$.
$\Sim$ picks $z$
unifromly from $\FF^{2m-1}$ satisfying $\sum_{j\in[m]} z[j] = r^Tb$.
Subsequently,
$\Sim$ computes $c_0 \leftarrow \comm(P_0,w_0)$, $\cm \leftarrow \comm(z,w)$
where $P_0\sample \FF^{2m-1}$ is such that $\innp{1^m||0^{m-1}}{P_0}=0$ and $w_0,w\sample
\FF$. $\Sim$ picks $O\sample \FF^{p\times t}$, and computes $\tilde{U}[\cdot,
k_u] = \sum_{i\in[p]} \rho_i U[i,\cdot, k_u]$, $\tilde{O}[k_u]= \sum_{i\in[p]}
\rho_i O[i,u]$, and $\tilde{c}_{k_u}  \leftarrow \comm(\tilde{U}'[\cdot, k_u] ,
\tilde{O}[k_u] ) \, \forall u\in[t]$ and $\pi[i,k_u] = \comm(U'[i,\cdot,k_u],
O[i,u])$, where $U'[i,\cdot,k_u]$ consists of first $m$ entries of
$U[i,\cdot,k_u]$ and $\tilde{U}'[\cdot,k_u]$ consists of first $m$ entries of
$\tilde{U}[\cdot,k_u]$, where $i\in[p]$ and $u\in [t]$. Next, $\Sim$ picks
$\tilde{c}_1, \ldots, \tilde{c}_{\ell}$ in such a way that $\tilde{c}_{k_u} =
\prod_{a\in[\ell]} (\tilde{c}_a)^{\Lambda^T_{n,\ell}[a,k_u]} \, \forall u\in
[t]$. Picking such $\tilde{c}$ is efficient since the number of unknowns is more
than the number of constraints, and the coefficient matrix has full row rank.
Now $\Sim$ picks $\pi[\cdot, k]$ such that $\tilde{c}_k = \prod_{i\in[p]}(
\pi[i,k])^{\rho_i}$ for all $k\notin \{k_u:u\in [t]\}$. $\Sim$ picks $w_{k_u}$
uniformly at random for all $u\in [t]$ and computes $c_{k_u} \leftarrow
\comm(P'[\cdot,k_u], w_{k_u})$, where $P'[\cdot,k_u]$ is the vector consisting
of the first $2m-1$ entries
of $P[\cdot,k_u]$ for $ u\in[t]$. Finally $\Sim$ picks ${c}_1, \ldots, {c}_{s+\ell-1}$
in such a way that ${c}_{k_u} = \prod_{a\in[s+\ell-1]}
({c}_a)^{\Lambda^T_{n,s+\ell-1}[a,k_u]} \, \forall u\in [t]$
and $\cm =  c_0^{\beta} \cdot \prod_{a\in[s+\ell-1]} (c_a)^{\varphi_a}$. Picking such ${c}$ is efficient since the number of unknowns is more than the number of constraints, and the coefficient matrix has full row rank.
%\end{itemize}
\begin{comment}
The simulator outputs $r$, $\{j_u,k_u\}_{u\in [t]}$,
$\beta$, $\tau,\delta$, $\rho$, $\tilde{\tau},\tilde{\delta}$ by uniformly and
independently sampling them from their respective domains, as in the honest
execution of the protocol. Simulator also outputs $z,\tilde{z}$ uniformly from
$L_2$, and $z'$ uniformly from $\dashL_2$ satisfying $\sum_{j\in [m]}z'[j]=0$ . It outputs
$\ewit[\cdot,\cdot,k_u]$ uniformly such that each plane has columns as codewords
in $L_2$. 
Next, the simulator outputs $\omega,\nu,\tilde{\nu}$
and $\chi_1,\ldots,\chi_t$, $\{O[\cdot,k_u]\}_{u\in [t]}$ choosing them randomly and
independently from $\FF$. Finally, the simulator outputs
$c_0,d_0,\ldots,c_{s+\ell}$ and $\tilde{d}_0,\tilde{c}_1,\ldots,\tilde{c}_\ell$ choosing 
them uniformly from $\GG$ subject to the following constraints:
$d_0 + \sum_{a=1}^{s+\ell}\mu_ac_a = \comm(z,\nu)$,
$c_0 + \sum_{a=1}^{s+\ell}\varphi_ac_a = \comm(z',\omega)$,
$\sum_{a=1}^{s+\ell}T[a,k_u]c_a = \comm(P[\cdot,k_u],\chi_u)$ for $u\in [t]$,
$\sum_{a=1}^{\ell}\mc{T}[a,k_u]\tilde{c}_a = \comm\big(\sum_{i\in
[p]}\tilde{U}[\cdot,k_u],\tilde{O}[\cdot, k_u]\big)$ for $u\in [t]$,
$\beta\tilde{d}_0 + \sum_{a=1}^{\ell}\tilde{\mu}_a\tilde{c}_a =
\comm(\tilde{z},\tilde{\nu})$. 
\end{comment}
\begin{lemma}\label{lem:simlincheck}
The output of the above simulator is perfectly indistinguishable
from the extended view of the verifier in honest execution of the protocol
$\linearcheck$ for $t\leq \bi$.
\end{lemma}

Similarly, we can construct a simulator for the quadratic check.
%\begin{lemma}\label{lem:simquadcheck}
%There exists an efficient simulator $\Sim$ whose output is perfectly
%indistinguishable from the extended view of the verifier in the honest execution
%of the protocol $\quadcheck$ for $t\leq \bi$.
%\end{lemma}


\section{$\dpname$: Distributed Prover Variant}\label{sec:dpgraphen}
We now describe distributed protocol to produce a $\name$ proof for a
statement, when the witness is shared between several provers. We assume that
there are $\Num$ provers $\prover_1,\ldots,\prover_{\Num}$. For $\xi\in [\Num]$,
let $\shr{\wit}$ denote the prover $\distprover$'s share of the witness $\wit$.
We assume that the sharing is additive, i.e, $\sum_{\xi\in [\Num]} \shr{\wit} =
\wit$. Recall from Section ~\ref{sec:security model}, that there is an algorithm $\Ag$ 
which aggregates the messages received
from provers $\prover_1,\ldots,\prover_\Num$ and constructs the message to be
sent to the verifier $\verifier$. We assume one of the provers executes $\Ag$.
The verifier's  messages and messages used by the aggregator algorithm $\Ag$ 
are assumed to be available on an authenticated broadcast
channel. We specify the algorithm $\Ag$ implicity by describing the construction
of message to the verifier from the provers' messages for each round.
We first discuss a protocol which is secure when the provers are semi-honest, then 
briefly discuss how to ensure the privacy of the honest provers when the corrupt 
provers are malicious. 

\noindent{\bf Distributed Oracle Setup}: In distributed setting, each prover
$\distprover$ encodes his share $\shr{\wit}$ as $\shr{\ewit}=\enc(\shr{\wit})$
and computes the commitment $\shr{\comoracle}=\comm(\shr{\ewit})$. The provers
then share $\shr{\comoracle}$ with the aggregator $\Ag$ which sets the oracle
$\pi$ as $\pi := \combine(\shr{\comoracle})$.  

\begin{comment}
\noindent{\bf Distributed Proximity Test}: The provers jointly prove that 
the oracle is well formed as follows: On receiving the verifier’s challenge 
$r\in\FF^p$ on broadcast channel, the prover $\distprover$ locally computes
$\shr{\tilde{U}}=\sum_{i\in [p]}r_i\shr{\ewit}[i,\cdot,\cdot]$, commitments
$\shr{\tilde{c}_k}=\sum_{i\in [p]}\shr{\comoracle}[i,k]$ for $k\in [\ell]$. They
send the shares of the commitments to the aggregator, who computes
$(\tilde{c}_1,\ldots,\tilde{c}_\ell)=\combine(\shr{\tilde{c}_1},\ldots,\shr{\tilde{c}_\ell})$
and forwards these to the verifier. Next, the provers jointly prove that
$\tilde{c}_1,\ldots,\tilde{c}_\ell$ corresponds to a matrix $\overline{U}$ such
that $\overline{U}\mc{T}\in \mc{C}_2$. This is done via distributed variant of
membership protocol that we describe next.  

\noindent{\bf Distributed Membership Test}: 
This protocol reduces to each prover responding to verifier’s
challenge on their share, as in the single    prover setting. The prover
responses are aggregated to compute the response to    the verifier. The
complete protocol appears in Figure \ref{fig:distprox2d}. In Figure
\ref{fig:distprox2d}, we note that the aggregator obtains the witness to the
inner product protocol in step 7, and hence does not need to interact further
with the provers. 
\end{comment}

\begin{comment}
\noindent{\bf Distributed Linear Test}: We provide the complete distributed protocol in Figure \ref{fig:distlincheck}. Here we highlight key adaptations from the single prover variant. First we assume that the provers have a share of the vector $0^{2m}$ (as part of obtaining shares of the extended witness). In response to verifier’s challenge $r\in \FF^M$, each prover locally computes $R=r^TA$ and the associated polynomials $R^i$, $i\in [p]$. Since, the computation of the $P$ matrix is a linear operation, each prover obtains a share $\shr{P}$ of the $P$ matrix by locally computing on their share of the witness. The provers compute commitments to the columns of their share of the $P$ matrix, following the similar procedure as in the single prover linear check protocol, and send those to the aggregator. The aggregator combines the shares to obtain the commitment to the $P$ matrix. %Provers can jointly prove that $P$ matrix is a codeword in the product code using the distributed membership protocol as above. 
Similarly, the provers can send their shares of vectors $\shr{X_u}=\shr{\ewit}[\cdot,j_u,k_u]$ for $u\in [t]$, required for consitency checks. Due to the bounded independence property, for $t\leq \bi$, these shares do not leak. The only point of departure is the shares of $z=\beta P_0 + \overline{P}\varphi$. The randomization by a random codeword $P_0\in L_2$ such that $\sum_{j\in [m]}
P_0[j]=0$ was based on the distribution of the vector $\overline{P}\varphi$ in an honest execution of the protocol for a witness satisfying $r^TA\wit=r^Tb$. However, individual shares may not satisfy the preceeding condition, and hence the provers further add a share of $0^h$ to the share $\beta\shr{P}_0 + \shr{\overline{P}}\varphi$. This allows the aggregator $\Ag$ to obtain witnesses for all the inner product protocols that it runs with the verifier.  

\begin{figure}[t!]
	{\small
		%\centering
		\begin{framed}
			\noindent{$\distlinearcheck(\mathsf{pp}, \, A\in \mc{M}_{M,N}, \, b\in \FF^M, \, [\pi];\, \shr{\ewit}, \, \shr{0^{2m}})$}:
			%\pnote{why $\wit$ is not part of the witness of linear check protocol?}
			
			\noindent{\bf Relation}: $\ewit=\open(\pi)\wedge A\wit=b$ for $\shr{\wit}=\dec(\shr{\ewit})$ for all $\xi \in [\Num]$ and $\sum_{\xi\in[\Num]} \shr{\wit} = \wit$.
			
			\begin{enumerate}[{\rm 1.}]
				\item $\verifier\rightarrow\distprover$: $\rho\sample \FF^p$.
				\item $\distprover$ computes: $\shr{\tilde{U}} = \sum_{i\in [p]}\rho_i\shr{\ewit}[i,\cdot,\cdot]$, 
				commitments $\shr{\tilde{c}_1},\ldots,\shr{\tilde{c}_\ell}$ as in Section ~\ref{sec:matrixcommitment}.
				\item $\distprover\rightarrow\Ag$: $\shr{\tilde{\bm{c}}} = (\shr{\tilde{c}_1},\ldots,\shr{\tilde{c}_\ell})$.
				\item \textcolor{red}{$\Ag\rightarrow\verifier$: $\tilde{\bm{c}}=\combine(\shr{\tilde{\bm{c}}})$.} %(\tilde{c}_1,\ldots,\tilde{c}_\ell)$.
				\item $\verifier\rightarrow\distprover$: $r\sample \FF^M$.
				\item $\distprover\leftrightarrow\verifier$ compute: Polynomials $R^i$, $i\in [p]$ interpolating $R=r^TA$
				as in Section ~\ref{sec:lincheck}. 
				\item $\distprover$ computes: Matrix $\shr{P}$ from $R$ and $\shr{\ewit}$ as described in Section ~\ref{sec:lincheck}. Samples $\shr{P_0}\sample \FF^m$, $\shr{\omega_0} \sample \FF$ and $\shr{c_0}\gets \comm(\shr{P_0},\shr{\omega_0})$,  and $\shr{d_0}\gets \comm(\shr{0^2m},\shr{o})$ where $\shr{o} \sample \FF$.
				Computes commitments $\shr{c_1},\ldots,\shr{c_{s+\ell}}$ from $\shr{P}$.
				\item $\distprover\rightarrow\Ag$: $\shr{c_0} ,\shr{c_1} ,\ldots, \shr{c_{s+\ell}}, d_0$.
				\item \textcolor{red}{$\Ag\rightarrow\verifier$: $c_k = \combine(\shr{c_k}) \, \forall k\in [s+\ell]$ and sends $c_0,c_1,\ldots,c_{s+\ell}$.}
				\item $\verifier\rightarrow\distprover$: $Q=\{(j_u,k_u):u\in [t]\}$ for $(j_u,k_u)\sample [h]\times [n]$ for $u\in [t]$.
				\item $\verifier\rightarrow\pi$: $\{k_u:u\in [t]\}$.
				\item $\distprover\rightarrow\Ag$: $\shr{X_u}=\shr{\ewit}[\cdot,j_u,k_u],  \shr{P_u} = \shr{P}[\cdot,k_u]$ for $u\in [t]$.
				\item \textcolor{red}{$\Ag\rightarrow\verifier$: $X_u = \sum_{\xi\in[\Num]} \shr{X_u}, P_u= \sum_{\xi\in [\Num]} \shr{P_u}$ and sends ${X_u}$ for $u\in [t]$.}
				\item $\pi\rightarrow\verifier$: $\pi[\cdot,k_u]$ for $u\in [t]$.
				\item $\verifier\rightarrow\distprover$: $\delta\sample \FF^p$, $\beta \FF\backslash \{0\}$. 
				\item $\distprover\rightarrow\Ag$: $\shr{z} = \beta\shr{P_0} + \shr{\overline{P}}\varphi + \shr{0^{2m}}$ and sends $\shr{z}$.
				\item \textcolor{red}{$\Ag$ computes $z=\sum_{\xi\in[\Num]} \shr{z}$}.
				\item $\Ag$ and $\verifier$ run inner product arguments to check:
				\begin{enumerate}
					\item $\innerproduct(\mathsf{pp},f_u^T\Lambda_5,\mathsf{cm}_u,v_u;\overline{P}[\cdot,k_u])$ 
					\pnote{We can use $P[1:2m, k_u]$ to denote the first 2m entries of $P[\cdot,k_u]$ as $\overline{P}$ is the submatrix after cropping both the directions, $h$ and $n$.}
					for $u\in [t]$ where $\mathsf{cm}_u=\sum_{a=1}^{s+\ell}\Lambda_3[a,k_u]c_a$, 
					$v_u=\sum_{i=1}^pR^i(\alpha_{j_u},\eta_{k_u})X_u[i]$ (check consistency of $P$ with $\pi$).
					\item $\innerproduct(\mathsf{pp},1^m,\mathsf{cm},r^Tb;z)$ where $z=\beta P_0+\overline{P}\varphi$ and $\mathsf{cm}=\beta c_0+\sum_{a=1}^{s+\ell}\varphi_ac_a$ (check the condition $r^TAw=r^Tb$).
					\item $\innerproduct(\mathsf{pp},f_u^T\Lambda_2,C_u,\innp{\delta}{X_u})$ for $u\in [t]$ 
					where $C_u=\sum_{i=1}^p\delta_i\pi[i,k_u]$ (consistency of $X_u$ with $\pi$). 
				\end{enumerate}
				\item $\verifier$ checks: $\sum_{a=1}^\ell\Lambda_1[a,k_u]\tilde{c}_a=\sum_{i=1}^p\rho_i\pi[i,k_u]$ for $u\in [t]$ (check proximity of $\ewit$ to $\mc{W}_1$).
			\end{enumerate}
		\end{framed}
		\caption{Distributed Linear Check Protocol}
		\label{fig:distlincheck}
	}
\end{figure}
%=======
\end{comment}

\noindent{\bf Distributed Linear Check}: The messages sent by the prover to the
verifier in the linear check protocol include:
\begin{itemize}
\item Commitments $\tilde{c}_1,\ldots,\tilde{c}_\ell$ to the matrix $\tilde{U}=
\sum_{i\in [p]}\rho_i\ewit[i,\cdot,\cdot]$ for verifier's challenge $\rho\sample
\FF^p$.
\item Commitments $c_0,\ldots,c_{s+\ell-1}$ where $c_0$ is a commitment to random
vector $P_0\in \FF^{2m-1}$ satisfying $\innp{1^m||0^{m-1}}{P_0}=0$ and
$c_1,\ldots,c_{s+\ell-1}$ are commitments to the $h\times n$ matrix $P$.
\item The vectors $X_u=\ewit[\cdot,j_u,k_u]$ for $u\in [t]$, for verifier's
query $Q=\{(j_u,k_u):u\in [t]\}$.
\end{itemize}
We see that given verifier's challenges, each of the messages is linear function
of the encoding (which itself is a linear function of the witness). Hence, the
provers compute the respective messages on their shares, which can be trivially
combined by $\Ag$. In addition to above messages, we also want $\Ag$ to receive
witnesses to the inner product protocols namely, the vectors
$\overline{P}[\cdot,k_u]$, $W_u=\sum_{i\in [p]}\delta_i\ewit[i,\cdot,k_u]$ for
$u\in [t]$, $z=\beta P_0+\overline{P}\varphi$ and the randomness used to commit
the vectors. Each of these can again be obtained by combining the respective shares.
Note that the share $\shr{z}$ leaks $r^TA\shr{\wit}=\innp{1^m||0^{m-1}}{\shr{z}}$, which
is non-trivial knowledge about an individual witness share. Thus provers use a
random share $\shr{0^{2m-1}}$ to randomize their share of $z$, and send
$\shr{z}=\beta\shr{P_0}+\shr{\overline{P}}\varphi+\shr{0^{2m-1}}$. 
We provide the complete distributed linear protocol in Figure
\ref{fig:distlincheck} in Appendix.  

%Here we highlight key adaptations
%from the single prover variant. First we assume that the provers have a share of
%the vector $0^{2m}$ (as part of obtaining shares of the extended
%witness). In response
%to verifier’s challenge $r\in \FF^M$, each prover locally computes $R=r^TA$ and
%the associated polynomials $R^i$, $i\in [p]$. Since, the computation of the $P$
%matrix is a linear operation, each prover obtains a share $\shr{P}$ of the $P$
%matrix by locally computing on their share of the witness. The provers compute
%commitments to the columns of their share of the $P$ matrix, following the
%similar procedure as in the single prover linear check protocol, and broadcast
%their commitments. The aggregator combines the commitments to obtain the commitment
%to the $P$ matrix. %Provers can jointly prove that $P$ matrix is a codeword in
%Similarly,
%the provers can send their shares of vectors
%%$\shr{X_u}=\shr{\ewit}[\cdot,j_u,k_u]$ for $u\in [t]$, required for consitency
%checks. Due to the bounded independence property, for $t\leq \bi$, these shares
%do not leak. The only point of departure is the shares of $z=\beta P_0 +
%\overline{P}\varphi$. The randomization by a random codeword $P_0\in L_2$ such
%that $\sum_{j\in [m]}P_0[j]=0$ was based on the distribution of the vector
%$\overline{P}\varphi$ in an honest execution of the protocol for a witness
%satisfying $r^TA\wit=r^Tb$. However, individual shares may not satisfy the
%preceeding condition, and hence the provers further add a share of $0^{2m}$ to the
%share $\beta\shr{P}_0 + \shr{\overline{P}}\varphi$. This allows the aggregator
%algorithm
%$\Ag$ to obtain witnesses for all the inner product protocols that it runs with
%the verifier.  

\noindent{\bf Distributed Quadratic Check}: This is the only protocol whose distributed variant requires an additional interaction among the provers. Recall that in response to the verifier’s challenge $r\in \FF^p$, the provers need to compute the matrix $P$ given by:

{\footnotesize 
\begin{align*}
P[j,k] & =\sum_{i\in
[p]}r_i\big(Q_x^i(\alpha_j,\eta_k).Q_y^i(\alpha_j,\eta_k)-Q_z^i(\alpha_j,\eta_k)\big)\\    
& = \sum_{i\in [p]}r_i(\ewit_x[i,j,k].\ewit_y[i,j,k] - \ewit_z[i,j,k])
\end{align*}    
}
where $\ewit_x$, $\ewit_y$ and $\ewit_z$ are the encodings of the witness
vectors $\wit_x$, $\wit_y$ and $\wit_z$ respectively. 
Since the matrix $P$ above is completely determined by it's first $2m-1$ rows
and first $2\ell-1$ columns, the provers need to obtain the shares
$\shr{\ewit_x[i,j,k].\ewit_y[i,j,k]}$ for $i\in [p],j\in [2m-1],k\in [2\ell-1]$.
From the shares of witness $\shr{\wit_x},\shr{\wit_y}$ and $\shr{\wit_z}$ the
provers can locally compute shares $\shr{\ewit_x},\shr{\ewit_y},\shr{\ewit_z}$
of $\ewit_x,\ewit_y$ and $\ewit_z$. Now, the provers can perform an MPC with
$p.(2m-1).(2\ell-1)\approx 4N$ multiplication gates, and depth 1, to obtain the
shares $\shr{\ewit_x[i,j,k].\ewit_y[i,j,k]}$ for $i\in [p],j\in [2m-1],k\in [2\ell-1]$.
Concretely, we assume an MPC $\mathsf{Mult}$ with following input/output for
a prover $\distprover$:     
%\begin{align*}    
$\shr{\ewit_x[i,j,k].\ewit_y[i,j,k]}\leftarrow
\mathsf{Mult}(\shr{\ewit_x[i,j,k]},\shr{\ewit_y[i,j,k]})$    
%\end{align*}    
Thereafter, each prover obtains a share of matrix $P$, and the remaining protocol proceeds as
the distributed linear check protocol. 
The complete protocol for
distributed quadratic check appears in Figure \ref{fig:distquadcheck} in
Appendix \ref{app:protocolboxes}. In Appendix ~\ref{sec:sharedcircuitopti}, we
discuss how to optimize MPC overhead when the size of the shared circuit (see 
 Section ~\ref{sec:efficiencyparams}) is small.

\begin{figure}[!]
\centering
\resizebox{.5\textwidth}{!}{
\begin{tabular}{|c|c|c|c|c|c|c|c|c|c|}
\hline
$N$ & \multicolumn{3}{|c|}{Arg. Size($\zkcomm$) MB} &
        \multicolumn{3}{|c|}{Verifier Time($t_\verifier$) sec} &
        \multicolumn{3}{|c|}{Prover Time($t_\prover$) sec} \\
\hline
 & \textsf{G} & \textsf{L} & \textsf{B} &
   \textsf{G} & \textsf{L} & \textsf{B} &
   \textsf{G} & \textsf{L} & \textsf{B} \\
\hline
$2^{19}$ & 0.728 & 3.5 & 0.001 & 43.0 & 3.0 & 89.1 & 802 & 137 & 662 \\
$2^{20}$ & 0.759 & 4.9 & 0.001 & 45.2 & 6.5 & 178.2 & 1514 & 291 & 1324 \\
$2^{21}$ & 0.884 & 6.95 & 0.001 & 46.7 & 13.1 & 356.5 & 2877 & 582 & 2648 \\
$2^{22}$ & 1.28  & 9.8 & 0.001 & 88.5 & 28.3 & 713 & 5558 & 1258 & 5297 \\
$2^{23}$ & 1.31  & 13.8 & 0.001 & 97.4 & 56.6 & 1426 & 10757 & 2516 & 10595 \\
$2^{24}$ & 1.34  & 19.5 & 0.001 & 115 & 120 & 2852 & 20878 & 5044 & 21190 \\
\hline
\end{tabular}
}
\caption{Comparison of Graphene(G), Ligero(L) and Bulletproofs(B) in single
prover setting for 80 bits of security}
\label{fig:standalonecompare}
\end{figure}
\begin{comment}
\begin{figure}[h!]
\centering
\begin{framed}
\begin{itemize}
\item $\distproxTwoD(\FF,\GG,\ell,L_1,L_2,\bm{c};[[\overline{U}]],[[\bm{\omega}]])$:
\item {\bf Relation}: $(\overline{U},\bm{\omega})=\open(\bm{c})$ and
$\overline{U}\mc{T}\in \mc{C}_2$.
\begin{enumerate}[{\rm 1.}]
\item $\distprover\rightarrow \Ag$: Samples random codeword $u^\xi_0\in L_2$ and
computes $d^\xi_0=\comm(u^\xi_0,\nu^\xi_0)$ for randomly sampled $\nu^\xi_0$.
Sends $d^\xi_0$ to $\Ag$.
\item {\color{red} $\Ag\rightarrow \verifier$: $\Ag$ computes $d_0=\sum_{\xi\in
K}d^\xi_0$ and sends $d_0$ to $\verifier$ }.
\item $\verifier\rightarrow \distprover$: Verifier samples $\tau\sample \FF^m$,
$\delta\sample \FF^{h-m}$ and sends them to $\distprover$.
\item $\Ag\leftrightarrow \verifier$ compute: $\mu=\mc{T}\tau$,
$\mathsf{cm}=d_0+\sum_{i\in [\ell]}\mu_ic_i$, $x=\mc{H}_2\delta$.
\item $\distprover$ computes: $[[z]]^\xi=u^\xi_0+[[\overline{U}]]^\xi\mu$,
$[[\nu]]_\xi=\nu^\xi_0+\sum_{i\in [\ell]}\mu_i[[\omega_i]]^\xi$.
\item {\color{red} $\Ag$ computes: $z=\combine([[z]]^\xi)$,
$\nu=\combine([[\nu]]^\xi)$}.
\item $\Ag$ and $\verifier$ run the subprotocol:
	\begin{itemize}
	\item $b=\innerproduct(\FF,\GG,\bm{g},x,\mathsf{cm},0;z,\nu)$.
	\end{itemize}
\item $\verifier$ accepts if the subprotocol accepts.
\end{enumerate}
\end{itemize}
\end{framed}
\caption{Distributed Membership Test}
\label{fig:distprox2d}
\end{figure}


\begin{figure}[h!]
\centering
\begin{framed}
\begin{itemize}
\item {$\distproxThreeD(\FF,\GG,L_1,L_2,[\pi];[[\ewit]])$}:
\item {\bf Relation}: $\ewit=\open(\pi)$, $\ewit\in \mc{W}$.
\item {\bf Oracle Setup}: 
	\begin{itemize}
	\item $\distprover\rightarrow \Ag$: Each prover computes shares $[[\comoracle]]^\xi$ from $[[\ewit]]^\xi$ as $[[\comoracle]]^\xi=\comm([[\ewit]]^\xi)$ as in Section \ref{sec:construct_oracle}. 
	\item {\color{red} $\Ag$ computes: $\comoracle :=
\combine([[\comoracle]]^\xi)$ and sets $\pi := \comoracle$ as the oracle}.
	\end{itemize}
\begin{enumerate}[{\rm 1.}]
\item $\verifier\rightarrow\distprover$: Verifier samples $r\sample \FF^p$ and
sends $r$ to $\distprover$.
\item $\distprover$ computes:
	\begin{itemize}
	\item $\shr{\tilde{U}} := \sum_{i\in [p]}r_i\shr{\ewit}[i,\cdot,\cdot]$.
	\item $\shr{\tilde{c}_k} := \sum_{i\in [p]}\shr{\comoracle}[i,k]$ for
$k\in [\ell]$.
	\item $\shr{\tilde{\omega}_k} := \sum_{i\in [p]}O^\xi[i,k]$ for
$k\in [\ell]$.
	\end{itemize}
\item $\distprover\rightarrow\Ag$: The provers send $\shr{\tilde{\bm{c}}} :=
(\shr{\tilde{c}_1},\ldots,\shr{\tilde{c}_\ell})$ to $\Ag$.
\item {\color{red} $\Ag\rightarrow\verifier$: $\Ag$ computes $\tilde{\bm{c}} :=
\combine(\shr{\tilde{\bm{c}}})$ and sends
$\tilde{\bm{c}}=(\tilde{c}_1,\ldots,\tilde{c}_\ell)$ to $\verifier$}.
\item $\Ag$ and $\verifier$ run the subprotocol:
	\begin{itemize}
	\item
$b=\distproxTwoD(\FF,\GG,\ell,L_1,L_2,\tilde{\bm{c}};\shr{\tilde{U}})$.
	\end{itemize}
\item $\verifier$ queries: $\verifier$ samples $Q\subseteq [n]$ of size $t$ and
makes oracle queries for positions in $Q$.
\item Oracle Answers: The oracle responds with columns $\pi[\cdot,k]$ for $k\in
Q$.
\item $\verifier$ checks: The verifier checks $\sum_{i\in
[p]}r_i\pi[i,k]=\sum_{i\in [\ell]}\mc{T}[i,k]\tilde{c}_i$ for $k\in Q$.
\item $\verifier$ accepts if the above check succeeds and $b=1$. 
\end{enumerate}
\end{itemize}
\end{framed}
\caption{Distributed 3D Proximity Protocol}
\label{fig:distprox3d}
\end{figure}
\end{comment} 

We show that the messages received by the
aggregator in both the distributed protocols can be efficiently simulated, independent of
their views in the preceeding MPC protocols, provided the parties follow the
protocol and jointly possess a valid witness. We present a proof for the
distributed linear check protocol. The same for the distributed quadratic check
may be similarly argued.

\begin{lemma}\label{lem:distlincheckzk}
	For $\xi\in [\Num]$, let $\View^\xi_\Ag$ denote the view of the aggregator $\Ag$ in the distributed
	linear check protocol consisting of messages from the prover $\distprover$ with
	witness $\shr{\wit}$. Then there exists an efficient simulator $\simulator$
	which outputs a view indistinguishable from the joint view $\langle
	\View^1_\Ag,\ldots,\View^{\Num}_\Ag\rangle$ whenever
$\wit=\combine(\shr{\wit})$ is valid.
\end{lemma}
\begin{proof}
	
	Let  $\rho, \, r, \, \{(j_u,k_u)\}_{u\in[t]}, \, \delta, \, \beta$ denote the verifier
	randomness, which is shared by each view. Consider other messages in
	$\View^\xi_\Ag$. Let $\shr{\comoracle}, \shr{\bm{\tilde{c}}}, \shr{c_0}, \shr{c_1}, \ldots, \shr{c_{s+\ell-1}}, \shr{d_0}$ 
	%$c_0^\xi,c_1^\xi,\ldots,c_{s+\ell}^\xi,d_0^\xi,\tilde{c}_1^\xi,\ldots,\tilde{c}_\ell^\xi,\tilde{d}_0^\xi,\{\pi^\xi[\cdot,k_u]\}_{u\in[t]}$ 
	denote the commitments sent by $\distprover$. Also, $\Ag$ gets the openings
of commitments $\shr{\comoracle[\cdot,k_u]}, \shr{c_{k_u}}$.
	Finally, $\Ag$ gets the vector $\shr{z} = \beta\shr{P_0} + \shr{\overline{P}}\varphi + \shr{0^{2m-1}}$.
	%Similarly let $\nu^\xi,\omega^\xi,\{\chi^\xi_u\}_{u\in [t]},\{O^\xi[\cdot,k_u]\}_{u\in [t]})$ denote the commitment randomness sent by the $\distprover$. 
	%Finally, the (extended) view also includes $\ewit^\xi[\cdot,\cdot,k_u]$ for $u\in [t]$, $z^\xi=u_0^\xi+P^\xi\mu$, $\tilde{z}^\xi=\tilde{u}^\xi_0+\tilde{U}^\xi\tilde{\mu}$ and ${z'}^\xi=\beta P^\xi_0 + \overline{P}^\xi\varphi + \shr{0^h}$. 
	Note that other than $\shr{z}$, the other random variables in the views are
independent of the witness shares $\shr{\wit}$ and $\shr{z}$. Thus these
variables in each of the views are distributed independently and identically,
and are simulated as in the single prover case. By the properties of shares
$\shr{0^{2m}}$, the joint distribution of $\langle{z}\rangle^1,\ldots,\langle{z}\rangle^{\Num}$ 
can be seen to be the uniform distribution on the set
$\{(\langle{z}\rangle^1,\ldots,\langle{z}\rangle^{\Num})\in (\FF^{2m-1})^{\Num}:
z=\sum_{\xi\in[\Num]}\shr{z} \text{ and } \sum_{j\in[m]} z[j]=0 \}$, provided
the combined witness $\wit=\combine(\shr{\wit})$ satisfies $A\wit=b$. 
Thus, the view of the aggregator $\Ag$ can be perfectly simulated.
\end{proof}

