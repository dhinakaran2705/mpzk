%--------------------------------------------------------------
%User Defined Theorems
%--------------------------------------------------------------

\newtheorem{terminology}{Terminology}
%\newtheorem{theorem}{Theorem}[section]
%\newtheorem{corollary}[theorem]{Corollary}
%\newtheorem{lemma}[theorem]{Lemma}
%\newtheorem{conjecture}[theorem]{Conjecture}
%\newtheorem{proposition}[theorem]{Proposition}
%\theoremstyle{definition}
%\newtheorem{definition}[theorem]{Definition}
%\newtheorem{example}[theorem]{Example}
%\newtheorem{remark}[theorem]{Remark}
%\newtheorem{claim}[theorem]{Claim}
\newtheorem{notation}[theorem]{Notation}
\newtheorem{openproblem}[theorem]{Open Problem}


%--------------------------------------------------------------
% Macros for Operators
%--------------------------------------------------------------

\newcommand{\xor}{\oplus}
\newcommand{\Xor}{\bigoplus}
\newcommand{\band}{\odot}
\newcommand{\bAnd}{\bigodot}
\newcommand{\abs}[1]{| #1 |}
\newcommand{\Order}{\mathcal{O}}
\newcommand{\BigO}[1]{\ensuremath{\operatorname{O}\left(#1\right)}}
\newcommand{\cmark}{\ding{51}}
\newcommand{\xmark}{\ding{55}}
\newcommand{\defined}{\ensuremath{\stackrel{def}{=} }}
\newcommand{\iseq}{\ensuremath{\stackrel{?}{=} }}

%--------------------------------------------------------------
%Mathematical Objects
%--------------------------------------------------------------

\newcommand{\bitset}{\{0,1\}}
\newcommand{\Prob}{\ensuremath{\mathsf{Pr}}}
%\newcommand{\floor}[1]{\lfloor #1 \rfloor}

%--------------------------------------------------------------
%Fields and Rings
%--------------------------------------------------------------

\newcommand{\GF}[1]{\ensuremath{\mbox{GF}(2^ #1)}}
\newcommand{\Z}[1]{\ensuremath{\mathbb{Z}}_{2^{#1}}}
\newcommand{\AS}{\ensuremath{\mathbb{S}}}
\newcommand{\ZP}{\ensuremath{\mathbb{Z}}_{\mathsf{p}}}
\newcommand{\F}{\ensuremath{\mathbb{F}}}
\newcommand{\FP}{\ensuremath{\mathbb{F}}_{\mathsf{p}}}
\newcommand{\N}{\ensuremath{\mathbb{N}}}


%--------------------------------------------------------------
%MPC SubProtocols
%--------------------------------------------------------------

%\newcommand{\Adv}{\ensuremath{\mathcal{A}}\xspace}
\newcommand{\Sim}{\ensuremath{\mathcal{S}}}

%--------------------------------------------------------------
%MPC Variables
%--------------------------------------------------------------

\newcommand{\sparam}{\ensuremath{s}}
\newcommand{\abort}{\ensuremath{\mathtt{abort}}}
\newcommand{\flag}{\ensuremath{\mathsf{flag}}}
\newcommand{\MPC}{\ensuremath{\mbox{MPC}}}
\newcommand{\ctr}{\ensuremath{\mathsf{ctr}}}
\newcommand{\secparam}{\ensuremath{\kappa}\xspace}

\newcommand{\Com}{\ensuremath{\mathsf{Com}}} %Commitment

%--------------------------------------------------------------
%Circuit Variables
%--------------------------------------------------------------

\newcommand{\PS}{\ensuremath{\mathsf{A}}} 
\newcommand{\IS}{\ensuremath{\mathsf{I}}}
\newcommand{\OS}{\ensuremath{\mathsf{O}}}  
\newcommand{\MS}{\ensuremath{\mathsf{M}}} 
\newcommand{\DF}{\ensuremath{\mathsf{D}}}
\newcommand{\VL}{\ensuremath{\mathsf{l}}}

\newcommand{\ckt}{\ensuremath{\mathsf{ckt}}}

\newcommand{\plusleft}{\ensuremath{\mathsf{x}}}
\newcommand{\plusright}{\ensuremath{\mathsf{y}}}
\newcommand{\plusoutput}{\ensuremath{\mathsf{z}}}
\newcommand{\multleft}{\ensuremath{\mathbf{x}}}
\newcommand{\multright}{\ensuremath{\mathbf{y}}}
\newcommand{\multoutput}{\ensuremath{\mathbf{z}}}
\newcommand{\multr}{\ensuremath{\mathbf{r}}}
\newcommand{\multrt}{\ensuremath{\trunc{\multr}}}
\newcommand{\multu}{\ensuremath{\mathbf{u}}}
\newcommand{\multv}{\ensuremath{\mathbf{v}}}


%--------------------------------------------------------------
% ML - Macros for Vectors and Matrices
%--------------------------------------------------------------

\newcommand{\va}{\mathbf{a}}
\newcommand{\vb}{\mathbf{b}}
\newcommand{\vp}{\mathbf{p}}
\newcommand{\vq}{\mathbf{q}}
\newcommand{\vu}{\mathbf{u}}
%\newcommand{\vv}{\mathbf{v}}
\newcommand{\vw}{\mathbf{w}}
\newcommand{\vx}{\mathbf{x}}
\newcommand{\vy}{\mathbf{y}}
\newcommand{\vz}{\mathbf{z}}


\newcommand{\Va}{\mathsf{a}}
\newcommand{\Vb}{\mathsf{b}}
\newcommand{\Vh}{\mathsf{h}}
\newcommand{\Vp}{\mathsf{p}}
\newcommand{\Vq}{\mathsf{q}}
\newcommand{\Vr}{\mathsf{r}}
\newcommand{\Vs}{\mathsf{s}}
\newcommand{\Vu}{\mathsf{u}}
\newcommand{\Vv}{\mathsf{v}}
\newcommand{\Vw}{\mathsf{w}}
\newcommand{\Vz}{\mathsf{z}}
\newcommand{\Vrs}{\mathsf{rs}}







\setlist[description]{style=unboxed,leftmargin=0cm}

%--------------------------------------------------------------
%Itemize, Enumerate, Description with less space
%--------------------------------------------------------------
\newenvironment{compactlist}{
	\begin{list}{{$\bullet$}}{
			\setlength\partopsep{0pt}
			\setlength\parskip{0pt}
			\setlength\parsep{0pt}
			\setlength\topsep{0pt}
			\setlength\itemsep{0pt}
			\setlength{\itemindent}{0.4pt}
			\setlength{\leftmargin}{10pt}
		}
	}{
	\end{list}
}


\newenvironment{myitemize}{
	\begin{list}{{$\bullet$}}{
			\setlength\partopsep{0pt}
			\setlength\parskip{0pt}
			\setlength\parsep{0pt}
			\setlength\topsep{0pt}
			\setlength\itemsep{0pt}
			\setlength{\itemindent}{0.4pt}
			\setlength{\leftmargin}{10pt}
		}
	}{
	\end{list}
}


\newenvironment{myitemizeold}
{\begin{list}{$\bullet$}{ %\itemindent=-.3cm \listparindent=.6cm
			\itemindent=-0.1in
			\itemsep=0.0in
			\parsep=0.0in
			\topsep=0.0in
			\partopsep=0.0in}}{\end{list}}
\newcounter{itemcount}

\newenvironment{myenumerate}
{\setcounter{itemcount}{0}\begin{list}
{\arabic{itemcount}.}{\usecounter{itemcount} \itemindent=-0.2cm
\itemsep=0.0in
\parsep=0.0in
\topsep=5pt
\partopsep=0.0in}}{\end{list}}

\newenvironment{mydescription}
{\setcounter{itemcount}{0}\begin{list}
{\arabic{itemcount}.}{\usecounter{itemcount} \itemindent=-0.5cm
\itemsep=0.0in
\parsep=0.0in
\topsep=5pt
\partopsep=0.0in}}{\end{list}}

%--------------------------------------------------------------
%Commands for figures and tables and references
%--------------------------------------------------------------

\newcommand{\tabref}[1]{Table~\protect\ref{tab:#1}}
\newcommand{\secref}[1]{Section~\protect\ref{sec:#1}}
\newcommand{\lemref}[1]{Lemma~\protect\ref{lem:#1}}
\newcommand{\figref}[1]{Figure~\ref{fig:#1}}
\newcommand{\boxref}[1]{Figure~\ref{#1}}
\newcommand{\figlab}[1]{\label{fig:#1}}
\newcommand{\refeqn}[1]{Equation~\eqref{#1}}

\newenvironment{boxfig}[2]{% {#1}{#2} = {Caption}{label}
	\begin{figure}[ht!]
		\newcommand{\FigCaption}{#1}
		\newcommand{\FigLabel}{#2}
		\vspace{-.15cm}
		\begin{center}
			\begin{small}
				\begin{tabular}{@{}|@{~~}l@{~~}|@{}}
					\hline
					%\rule[-1ex]{0pt}{1ex}\begin{minipage}[!htb]{\textwidth}   
					\rule[-1.5ex]{0pt}{1ex}\begin{minipage}[b]{.97\linewidth}
						\vspace{1ex}
						\smallskip
					}{%
					\end{minipage}\\
					\hline
				\end{tabular}
			\end{small}
			\vspace{-0.3cm}
			\caption{\FigCaption}
			\figlab{\FigLabel}
		\end{center}
		\vspace{-0.4cm}
	\end{figure}
}


\newenvironment{boxfig*}[2]{% {#1}{#2} = {Caption}{label}
	\begin{figure*}[h!]		
		\fontsize{5}{5}\selectfont
		\newcommand{\FigCaption}{#1}
		\newcommand{\FigLabel}{#2}
		\vspace{-.05cm}
		\begin{center}
			\begin{small}			 
				\begin{adjustbox}{max width=\textwidth}
					\begin{tabular}{@{}|@{~~}l@{~~}|@{}}
						\hline
						%\rule[-1ex]{0pt}{1ex}\begin{minipage}[!htb]{\textwidth}   
						\rule[-1ex]{0pt}{1ex}\begin{minipage}[b]{.95\linewidth}
							\vspace{1ex}	
						}{%
						\end{minipage}\\
						\hline
					\end{tabular}	
				\end{adjustbox}		
			\end{small}
			\vspace{-0.1cm}
			\caption{\FigCaption}
			\figlab{\FigLabel}
		\end{center}
		\vspace{-.38cm}
	\end{figure*}
}

\newenvironment{myboxfig}[2]{% {#1}{#2} = {Caption}{label}
	\vspace{-0.5cm}
	\begin{figure}[htb!]		
		\fontsize{5}{5}\selectfont
		\newcommand{\FigCaption}{#1}
		\newcommand{\FigLabel}{#2}
		\vspace{-.15cm}
		\begin{center}
			\caption{\FigCaption}
			\begin{small}			 
				\begin{adjustbox}{max width=\textwidth}
					\begin{tabular}{@{}|@{~~}l@{~~}|@{}}
						\hline
						%\rule[-1ex]{0pt}{1ex}\begin{minipage}[!htb]{\textwidth}   
						\rule[-1ex]{0pt}{1ex}\begin{minipage}[b]{.95\linewidth}
							\vspace{1ex}	
						}{%
						\end{minipage}\\
						\hline
					\end{tabular}	
				\end{adjustbox}		
			\end{small}
			%	\vspace{-0.25cm}
			\figlab{\FigLabel}
		\end{center}
		\vspace{-.38cm}
	\end{figure}
}


\newenvironment{myboxfig*}[2]{% {#1}{#2} = {Caption}{label}
	\begin{figure*}[!htb]		
		\fontsize{5}{5}\selectfont
		\newcommand{\FigCaption}{#1}
		\newcommand{\FigLabel}{#2}
		\vspace{-.10cm}
		\begin{center}
			\caption{\FigCaption}
			\begin{small}			 
				\begin{adjustbox}{max width=\textwidth}
					\begin{tabular}{@{}|@{~~}l@{~~}|@{}}
						\hline
						%\rule[-1ex]{0pt}{1ex}\begin{minipage}[!htb]{\textwidth}   
						\rule[-1ex]{0pt}{1ex}\begin{minipage}[b]{.95\linewidth}
							\vspace{1ex}	
						}{%
						\end{minipage}\\
						\hline
					\end{tabular}	
				\end{adjustbox}		
			\end{small}
			\vspace{-0.25cm}
			\figlab{\FigLabel}
		\end{center}
		\vspace{-.38cm}
	\end{figure*}
}



%--------------------------------------------------------------
%New Style boxes
%--------------------------------------------------------------

%----- Box Environment ---------------------------------------------------------
\RequirePackage{mdframed}

%Basic box structure with title box
\newenvironment{titlebox}[5]
{\mdfsetup{
		style=#2,
		innertopmargin=1.1\baselineskip,
		skipabove={\dimexpr0.7\baselineskip+\topskip\relax},
		skipbelow={1em},needspace=3\baselineskip,
		singleextra={\node[#3,right=10pt,overlay] at (P-|O){~{\sffamily\bfseries #1 }};},%
		firstextra={\node[#3,right=10pt,overlay] at (P-|O) {~{\sffamily\bfseries #1 }};},
		frametitleaboveskip=9em,
		innerrightmargin=5pt
	}
	\newcommand{\TitleCaption}{#4}
	\newcommand{\TitleLabel}{#5}
	\begin{mdframed}[font=\small]
		\setlist[itemize]{leftmargin=13pt}\setlist[enumerate]{leftmargin=13pt}\raggedright% 
	}
	{\end{mdframed}
	\vspace{-2em}
	{\captionof{figure}{\normalfont \TitleCaption}\label{\TitleLabel}}
	\medskip
}

%title box style
\tikzstyle{normal} = [thick, fill=white, text=black, draw, rounded corners, rectangle, minimum height=.7cm, inner sep=3pt]
\tikzstyle{gray} = [thick, fill=gray!90, text=white, rounded corners, rectangle, minimum height=.7cm, inner sep=3pt]

%box style
\mdfdefinestyle{commonbox}{%
	align=center, middlelinewidth=1.1pt,userdefinedwidth=\linewidth
	innerrightmargin=0pt,innerleftmargin=5pt,innertopmargin=5pt,
	splittopskip=15pt,splitbottomskip=15pt
}
\mdfdefinestyle{roundbox}{style=commonbox,roundcorner=5pt,userdefinedwidth=\linewidth}

\newenvironment{systembox}[3]
{\vspace{\baselineskip}\begin{titlebox}{Functionality \normalfont #1}{roundbox}{normal}{#2}{#3}}
	{\end{titlebox}}

\newenvironment{gsystembox}[3]
{\vspace{\baselineskip}\begin{titlebox}{Global Functionality \normalfont #1}{roundbox}{normal}{#2}{#3}}
	{\end{titlebox}}

\newenvironment{protocolbox}[3]
{\begin{titlebox}{Protocol \normalfont #1}{commonbox}{normal}{#2}{#3}}
	{\end{titlebox}}

\newenvironment{algobox}[3]
{\begin{titlebox}{Algorithm \normalfont #1}{commonbox}{normal}{#2}{#3}}
	{\end{titlebox}}

\newenvironment{reductionbox}[3]
{\begin{titlebox}{Reduction \normalfont #1}{commonbox}{normal}{#2}{#3}}
	{\end{titlebox}}

\newenvironment{gamebox}[3]
{\begin{titlebox}{Game \normalfont #1}{commonbox}{gray}{#2}{#3}}
	{\end{titlebox}}

\newenvironment{simulatorbox}[3]
{\begin{titlebox}{Simulator \normalfont #1}{commonbox}{normal}{#2}{#3}}
	{\end{titlebox}}
%-------------------------------------------------------------------------------



%-------------------------------------------------------------------------------
% Macros used for the examples
%-------------------------------------------------------------------------------

%----- Algorithm Environment ---------------------------------------------------
%Header for Algorithms/Functionalities
\newcommand{\algoHead}[1]{\vspace{0.2em} \underline{\textbf{#1}} \vspace{0.3em}}
\newcommand{\algoHeadExt}[2]{\vspace{0.2em} \underline{\textbf{#1} #2} \vspace{0.3em}}

%Multiline Algo-States
\makeatletter
\algnewcommand{\ExtendedState}[1]{\State
	\parbox[t]{\dimexpr\linewidth-\ALG@thistlm}{\hangindent=\algorithmicindent\strut\hangafter=3#1\strut}}
\makeatother

%Algorithms States
\algnewcommand\algorithmicinput{\textbf{Input:}}
\algnewcommand\Input{\item[\algorithmicinput]}
\renewcommand{\algorithmicensure}{\textbf{Output:}}

%Algo Comments
\algrenewcommand{\algorithmiccomment}[1]{{\color{gray}// #1}}

%  Font and Notation
%-------------------------------------------------------------------------------
\newcommand{\xmath}[1]{\ensuremath{#1}\xspace}

\newcommand{\command}[1]{\xmath{\textsc{#1}}}
\newcommand{\algorithm}[1]{\xmath{\mathsf{#1}}}
\newcommand{\variable}[1]{\xmath{\mathtt{#1}}}
\newcommand{\parameter}[1]{\xmath{\mathtt{#1}}}

%Nice empty set
\let\oldemptyset\emptyset
\let\emptyset\varnothing

%Functionalities
\newcommand{\Func}[1][\relax]{\xmath{\mathcal{F}_{\textsc{#1}}}}
%-------------------------------------------------------------------------------

%--------------------------------------------------------------
%New Style boxes - Some Examples for reference
%--------------------------------------------------------------

\begin{comment}

%Basic box structure with title box
\newenvironment{titlebox}[3]
{\mdfsetup{
		style=#2,
		innertopmargin=1.1\baselineskip,
		skipabove={\dimexpr0.7\baselineskip+\topskip\relax},
		skipbelow={1em},needspace=3\baselineskip,
		singleextra={\node[#3,right=10pt,overlay] at (P-|O){~{\sffamily\bfseries #1 }};},%
		firstextra={\node[#3,right=10pt,overlay] at (P-|O) {~{\sffamily\bfseries #1 }};},
		frametitleaboveskip=9em,
		innerrightmargin=5pt
	}
	\begin{mdframed}[font=\small]\setlist[itemize]{leftmargin=13pt}\setlist[enumerate]{leftmargin=13pt}\raggedright% 
	}
	{\end{mdframed}}

\newenvironment{systembox}[1]
{\vspace{\baselineskip}\begin{titlebox}{Functionality \normalfont #1}{roundbox}{normal}}
	{\end{titlebox}}

\newenvironment{gsystembox}[1]
{\vspace{\baselineskip}\begin{titlebox}{Global Functionality \normalfont #1}{roundbox}{normal}}
	{\end{titlebox}}

\newenvironment{protocolbox}[1]
{\begin{titlebox}{Protocol \normalfont #1}{commonbox}{normal}}
	{\end{titlebox}}

\newenvironment{algobox}[1]
{\begin{titlebox}{Algorithm \normalfont #1}{commonbox}{normal}}
	{\end{titlebox}}

\newenvironment{reductionbox}[1]
{\begin{titlebox}{Reduction \normalfont #1}{commonbox}{normal}}
	{\end{titlebox}}

\newenvironment{gamebox}[1]
{\begin{titlebox}{Game \normalfont #1}{commonbox}{gray}}
	{\end{titlebox}}

\newenvironment{simulatorbox}[1]
{\begin{titlebox}{Simulator \normalfont #1}{commonbox}{normal}}
	{\end{titlebox}}


\begin{protocolbox}{$\algorithm{MyProtocol}$}
	\algoHead{Computation:}
	\begin{algorithmic}[1]
		\State Do things. 
		\State A long an complicated step which requires many words to describe. At some point we have a line break.
		\While{Condition}
		\State In the loop.
		\EndWhile
	\end{algorithmic}
\end{protocolbox}

\begin{systembox}{$\Func[example]$}
	Some text describing example functionality.\\[2ex]
	
	\algoHead{Initialization:}
	\begin{enumerate}
		\item In the first step do:
		\begin{enumerate}
			\item This
			\item and that.
		\end{enumerate}
		\item Even more computation.
	\end{enumerate}
	%
	\algoHead{Computation:}
	\begin{algorithmic}[1]
		\State Do things. 
		\State A long an complicated step which requires many words to describe. At some point we have a line break.
		\While{Condition}
		\State In the loop.
		\EndWhile
	\end{algorithmic}
	
\end{systembox}


\begin{gsystembox}{$\Func[example]$}
	
\end{gsystembox}


\begin{simulatorbox}{$\sigma$}
	
\end{simulatorbox}

\end{comment}