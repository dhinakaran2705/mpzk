\section{Preliminaries}\label{sec:prelim}
\subsection{Linear Codes}
\begin{definition}\label{defn:lincode}
For positive integers $n,k$ and a finite field $\FF$, a $k$-dimensional
subspace $L$ of $\FF^n$ is called an $[n,k]$ linear code. Elements of $L$ are
conventionally called {\em codewords}. 
\end{definition}

For codewords $x,y\in L$ where $x=(x_1,\ldots,x_n)$ and $y=(y_1,\ldots,y_n)$ we
define the hamming distance $\dham(x,y)=|\{i\in [n]: x_i\neq y_i\}|$. It is
easily checked that $\dham$ defines a metric on $L$. The minimum distance of
the code $L$, denoted by $\delta(L)$ is defined as $\min\{\dham(x,y):x,y\in L,
x\neq y\}$.

For an $[n,k]$ code $L$, a $n\times k$ matrix $\mc{G}$ is called a {\em generator
matrix} iff (i) $\mc{G} x\in L$ for all $x\in\FF^k$ and (ii) $\mc{G} x\neq
\mc{G} y$ for $x\neq
y$. Clearly, such a matrix $\mc{G}$ has rank $k$. Similarly an $n\times (n-k)$
matrix $\mc{H}$ such that $y^T\mc{H}=0$ for all $y\in L$ is called a {\em parity check}
matrix for $L$. It is easily seen that the above two matrices exist for any
$[n,k]$ linear code $L$. We will assume that description of the linear code $L$
includes a generator matrix $\mc{G}$ and a parity check matrix $\mc{H}$.

\begin{definition}[Interleaved Code]\label{defn:interleavedcode}
For an $[n,k]$-linear code $L$ and a positive integer $m$, we define a {\em row
interleaved code} $\ric{L}{m}$ to be the set of $m\times n$ matrices $A$ such that
each row of $A$ is a codeword in $L$. Similarly, we define a {\em column
interleaved code} $\cic{L}{m}$ to be the set of $n\times m$ matrices $B$ such
that each column of $B$ is a codeword in $L$.
\end{definition}

For a linear $[n,k]$-code $L$ over the field $\FF$, we observe that
$\ric{L}{m}$ forms an $[n,k]$-code over the field $\FF^m$ by viewing each
column of the codeword $A\in \ric{L}{m}$ as a symbol in the field $\FF^m$.
Similarly, $\cic{L}{m}$ forms an $[n,k]$ code over $\FF^m$ by viewing each row
of the codeword $B\in \cic{L}{m}$ as a symbol in $\FF^m$. For $A,A'\in
\ric{L}{m}$, we define the distance $\dham(A,A')=|\{i\in [n]: A[.,i]\neq
A'[.,i]\}|$ where the notation $X[.,i]$ denotes the $i^{th}$ column of the
matrix $X$. Similarly for $B,B'\in \cic{L}{m}$ we define $\dham(B,B')=|\{i\in
[n]: B[i,.]\neq B'[i,.]\}|$.
 
\begin{definition}[Product Code]\label{defn:productcode}
Let $L_i$ be an $[n_i,k_i]$-linear code for $i=1,2$. We define the product code
$L_1\oplus L_2$ to be the code consisting of $n_2\times n_1$ matrices $A$ such
that each row of $A$ is a codeword in $L_1$ and each column of $A$ is a
codeword in $L_2$. 
\end{definition}

Note that by definition, the product code $L_1\oplus L_2$ is a row interleaved
code of $L_1$ and a column interleaved code of $L_2$, i.e $L_1\oplus L_2 =
\ric{L_1}{n_2}\cap \cic{L_2}{n_1}$. For $A,A'\in L_1\oplus L_2$, we define
$\dham_1(A,A')=|\{i\in [n_1]: A[.,i]\neq A'[.,i]\}|$ and $\dham_2(A,A')=|\{i\in
[n_2]: A[i,.]\neq A'[i,.]\}|$. The distance $\dham_1$ corresponds to distance
function of the code $\ric{L}{n_2}$, where we view $A,A'$ as codewords in
$\ric{L}{n_2}$. Similarly, the distance $\dham_2$ corresponds to the distance
function of the code $\cic{L}{n_1}$.

\begin{definition}[Reed Solomon Code]\label{defn:rscode}
An $[n,k]$-Reed Solomon Code $L\subseteq \FF^n$ consists of vectors
$(p(\eta_1),\ldots,p(\eta_n))$ for polynomials $p\in \FF[x]$ of degree less
than $k$ where $\eta_1,\ldots,\eta_n$ are distinct points in $\FF$. We will use
$\rsc{\eta}{k}$ to denote the Reed Solomon code with
$\bm{\eta}=(\eta_1,\ldots,\eta_n)$ and $deg(p)<k$.
\end{definition}
   
The following lemma will be useful to us later. Intuitively it states that for
a code $L$, if a
matrix is ``far'' from an interleaved code of $L$, then a random linear combination of its
rows is also likely to be far from the code $L$. 

\begin{lemma}\label{lem:proximitytest}
Let $L$ be an $[n,k]$ linear code and let $m$ be a positive integer. Let
$U^\ast\in \FF^{m\times n}$ be such that $\dham(U^\ast,\ric{L}{m})>e$ for
$e<\delta(L)/3$. Then $\prob{\dham(r^TU^\ast,L)\leq e}<\delta(L)/|\FF|$ where
$r$ is sampled uniformly from $\FF^m$. 
\end{lemma}
The above lemma is proved in \cite{Ligero2017} for $e=\delta(L)/4$. We provide
a self-contained proof of the above for $e=\delta(L)/3$ in Appendix.


