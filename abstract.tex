\begin{abstract}
Traditional zero-knowledge protocols have been studied and optimized for the
setting where a single prover holds the complete witness and tries to convince a
verifier about a predicate on the witness, without revealing any additional
information to the verifier. This work initiates the study of distributed proof
generation ($\DPZK$) for arbitrary predicates where the witness is shared among
multiple mutually distrusting provers and they want to convince a verifier that
their shares together satisfy the predicate. We provide a new security
definition, discuss new efficiency parameters, and propose a state-of-art
construction for this paradigm: (i) we propose a new MPC-style security
definition to capture the adversarial settings possible for different collusions
between the provers and the verifier, (ii) we discuss new efficiency parameters
for distributed proof generation on the number of rounds $\prrounds$ and the
amount of communication $\prcomm$ among the provers, (iii) we propose a new
single-prover zero-knowledge protocol \name{} in the Interactive
Probabilistically Checkable Proofs (IPCP) paradigm which admits $O(N^{1/c})$
proof size. And, building on it, our $\DPZK$ protocol \dpname{}, with $\Num$
provers, generate the same proof size with a communication complexity $\prcomm$
of $O(\Num \cdot N^{1-2/c}+ \Num^2\cdot \max(N_s,N^{1-1/c})$  and  a round complexity $\prrounds $ of 2, where $N_s$ denotes the number of multiplication gates that takes witnesses of more than one provers,  in the arithmetic circuit representing the prover's computation in \name{}. 

%=======
%Traditional zero-knowledge protocols have been studied and optimized for the setting where a single prover holds the complete witness and trying to convince a verifier about a predicate on the witness, without revealing any additional information to the verifier. This work initiates the study of distributed proof generation where the witness is shared among multiple mutually distrusting provers and they want to convince a verifier that their shares together satisfies the predicate. We start by defining a new MPC-style security definition to capture the possible adversarial settings, and proposing new efficiency parameters for distributed proof generation on the number of rounds $\prrounds$ and the amount of communication $\prcomm$ among the provers. We then propose a new zero-knowledge protocol \name{} in the IPCP paradigm which admits $O(N^{1/c})$ proof size. The provers in \name{} have a total communication complexity $O(\Num \cdot N^{1-2/c}+ N)$ when the proof is generated distributively among $\Num$ provers.
%\end{comment}
%	Zero knowledge is an intriguing area of research from the prospect of theory as well as application, where a party, prover \textit{P}, tries to convince verifier \textit{V} that a statement $x$( instance of an NP language) is correct using a witness $w$ such that $M(x,w)=1$. Let $C$ be the circuit representation of $M$ and $|C|$ is the size of $C$ i.e. the no. of gates in $C$. The work of Ligero in this problem proposed a solution where proof size is $O(|C|^{1/2})$. They solve the problem by converting a NP language to it's corresponding circuit satisfiability problem. $w\in \mathbb{F}^{ml}$ be the extended witness which is the secret input to the prover. Prover \textit{P} encodes $w$ to a matrix $U$ of size $m\times n$, where $m, n, l = O(\sqrt{|C|})$ and $n>l$. In Ligero's protocol first check is for to ensure that the matrix $U$ is a correct encoding, that is done by testing interleaved. Then they check that all the outcomes of all the gates( addition and multiplication) are correct or not, and to do that they introduce two checks, linear constraint and quadratic constraint. We are using the same approach of Ligero, but our encoding is different. Instead of matrix we will encode $w$ to a box( matrix of 3 dimensions), to keep the familiarity of notation we will consider $w\in \mathbb{F}^{pml}$ and the size of the box $U$ is $p\times m\times n$, where $p, m, n, l = O(\sqrt[3]{|C|})$ and $n>l$. To encode $w$ we are going to construct $pm$ many polynomials(univariate) such that each of them has degree at most $s$, where $s$ is some suitable number between $l$ and $n$. 
%\\

%\mycomment{Adding a version below.}
%The work of Ligero[citation] proposed ... In Ligero, a witness $x$ is encoded into a matrix $U$ of dimension $m \times n$, where $m$ and $n$ denote - and - respectively. Our idea is similar to theirs apart from the witness being encoded to a {\em box}. A box is a 3-dimensional matrix ($m \times n \times p$) where --(add technical details of box here). This ensures linearity of (what?) along both rows and columns. We hope that the aforementioned encoding along with homomorphic commitments lead to a ZK proof, whose size will be $O(|C|^{\frac{1}{3}})$.
\end{abstract}
