\documentclass[11pt]{article}
%%----------a------------
%\usepackage{adjustbox}
\usepackage{amsfonts}
\usepackage{amsmath}
\usepackage{amssymb}
%\usepackage{amsthm}
%\usepackage{algorithmic}
%\usepackage[ruled,linesnumbered,noend,algo2e]{algorithm2e}
%\usepackage{array}
%----------b------------
\usepackage{bm}
\usepackage{bbm}
%\usepackage{booktabs}
%\usepackage{breakcites}
%----------c------------
%\usepackage[skip=2pt]{caption}
\usepackage{color}
\usepackage{comment}
%\usepackage{cuted}
\usepackage{cite}
\usepackage[lambda,
	advantage,
	operators,
	sets, 
	adversary, 
	landau,
	probability, 
	notions, 
	logic,
	ff,
	mm,
	primitives, 
	events, 
	complexity, 
	asymptotics, 
	keys]{cryptocode}
%----------d------------
%\usepackage{dsfont}
%----------e------------
%\usepackage{endnotes}
\usepackage{enumerate}
%\usepackage{enumitem}
%\usepackage{epsfig}
%\usepackage{epstopdf}
%\epstopdfsetup{update}
%\usepackage{etoolbox}
%----------f------------
%\usepackage{float}
%\usepackage[T1]{fontenc}
\usepackage{framed}
\usepackage{float}
%----------g------------
%\usepackage[a4paper,total={6in,8in}]{geometry}
\usepackage{graphicx}
%----------h------------
%\usepackage{hhline}
\usepackage{hyperref}
%\hypersetup{
%	colorlinks   = true,
%	citecolor    = blue
%}
%----------l------------
%\usepackage{latexsym}
%----------m------------
%\usepackage{mathtools}
\usepackage{multirow}
\usepackage{multicol}
%\usepackage{mdframed}
%\usepackage{mwe} 
\usepackage{makecell}
%----------n------------
\usepackage{nicefrac}
%----------p------------
\usepackage{pgfplots}
\usepackage{placeins}
%\usepackage[section]{placeins}
%\usepackage{pifont}
%----------r------------
%\usepackage{rotating}
%----------s------------
\usepackage{subfig}
%\usepackage{subcaption}
%----------t------------
\usepackage{tabu}
\usepackage{tabularx}
\usepackage{threeparttable}
\usepackage{tikz}  
	%\usetikzlibrary{shapes.geometric, arrows}
	\usetikzlibrary{shapes, shadows, arrows}
	%\tikzstyle{line} = [draw,-latex']
	%\tikzstyle{block}=[draw, rectangle]
\usepackage{textcomp}
%\usepackage{times}
%----------u------------
%\usepackage[normalem]{ulem}
\usepackage{url}
\usepackage{upgreek}
%----------v------------
%\usepackage{verbatim}
%----------w------------
%\usepackage{wrapfig}
%----------x------------
\usepackage{xcolor}
\usepackage{xspace}
%-----------------------
%----------For New Style------
%general macros
%\usepackage[english]{babel}
%\usepackage[utf8]{inputenc}
%\usepackage{fullpage}
%used for the boxes
\usepackage[framemethod=tikz]{mdframed} 
%use for the examples
\usepackage{algpseudocode}
%\usepackage{nccmath}
%\usepackage{bbm}
%\usepackage[b]{esvect}
%\usepackage{mathrsfs}
%\usepackage[bold-style=ISO]{unicode-math}
%\captionsetup[figure]{position=below}

%\newtheorem{theorem}{Theorem}
%\newtheorem{definition}{Definition}
%\newtheorem{lemma}[theorem]{Lemma}


% Nitin's macros

\newcommand{\RS}[2]{\mathsf{RS}[#1,#2]}
\newcommand{\RSC}[3]{\mathsf{RS}[#1,#2,{\bm{#3}}]}
\newcommand{\rsc}[2]{\mathsf{RS}[\bm{#1},#2]}
\newcommand{\KK}{\mathbb{K}}
\newcommand{\comm}{\mathsf{Com}}
\newcommand{\csetup}{\mathsf{Gen}}
\newcommand{\pip}{\mathcal{P}_\mathrm{ip}}
\newcommand{\vip}{\mathcal{V}_\mathrm{ip}}
\newcommand{\setup}{\mathsf{Setup}}
\newcommand{\mc}[1]{\mathcal{#1}}
\newcommand{\innp}[2]{\langle #1,#2\rangle}
\newcommand{\piplog}{\mathcal{P}_\mathrm{log}}
\newcommand{\viplog}{\mathcal{V}_\mathrm{log}}
\newcommand{\pipsq}{\mathcal{P}_\mathrm{sq}}
\newcommand{\vipsq}{\mathcal{V}_\mathrm{sq}}
\newcommand{\rsoracle}{\mathcal{U}^{\mathsf{RS}}}
\newcommand{\comoracle}{\mathcal{U}^{\mathsf{Com}}}
\newcommand{\decode}{\mathsf{Dec}}
\newcommand{\dham}{\Delta}
\newcommand{\ric}[2]{\mathsf{RIC}(#1,#2)}
\newcommand{\cic}[2]{\mathsf{CIC}(#1,#2)}


% 		Calligraphic and blackboard type letters.

\def\cA{{\cal A}}
\def\cB{{\cal B}}
\def\cC{{\cal C}}
\def\cD{{\cal D}}
\def\cE{{\cal E}}
\def\cF{{\cal F}}
\def\cG{{\cal G}}
\def\cH{{\cal H}}
\def\cI{{\cal I}}
\def\cJ{{\cal J}}
\def\cK{{\cal K}}
\def\cL{{\cal L}}
\def\cM{{\cal M}}
\def\cN{{\cal N}}
\def\cO{{\cal O}}
\def\cP{{\cal P}}
\def\cQ{{\cal Q}}
\def\cR{{\cal R}}
\def\cS{{\cal S}}
\def\cT{{\cal T}}
\def\cU{{\cal U}}
\def\cV{{\cal V}}
\def\cW{{\cal W}}
\def\cX{{\cal X}}
\def\cY{{\cal Y}}
\def\cZ{{\cal Z}}
\def\cl{{\cal l}}
\def\cq{{\cal q}}
%%%%%%%%%%%%%%%%%
\def\bbC{{\mathbb C}}
\def\bbE{{\mathbb E}}
\def\bbF{{\mathbb F}}
\def\bbG{{\mathbb G}}
\def\bbM{{\mathbb M}}
\def\bbN{{\mathbb N}}
\def\bbQ{{\mathbb Q}}
\def\bbR{{\mathbb R}}
\def\bbV{{\mathbb V}}
\def\bbZ{{\mathbb Z}}

\def\Zq{\bbZ_q}

%%%%%%%%%%%%%%%%%

% 		Rounding commands

%\newcommand{\ceil}[1]{\left\lceil #1 \right\rceil}
%\newcommand{\floor}[1]{\left\lfloor #1 \right\rfloor}
\newcommand{\round}[1]{\left\lfloor #1 \right\rceil}

%%%%%%%%%%%%%%%%%

% 		Algorithms

\def\Exec{\mathsf{Exec}}
\def\Setup{\mathsf{Setup}}

% 		Notions

\def\negl{\mathrm{negl}}

\def\inp{\mathsf{in}}
\def\st{\mathsf{st}}
\def\secp{{\lambda}}



\newcommand{\Defn}[1]{\textbf{{\textit{#1}}}}
\newcommand{\Adv}{\mathcal{A}}
\newcommand{\com}{\mathsf{com}}
\newcommand{\vc}[1]{\textbf{{#1}}}
%\newcommand{\bb}[1]{\mathbb{#1}}
%\newcommand{\poly}[1][2]{#1(#2)}
%\newcommand*{\poly}[1][2]{#1(#2)}
\newcommand{\qb}{q_{blind}}
\newcommand{\qbj}{q_{{blind}_j}}
\newcommand{\boldm}[1]{\mathversion{bold}#1\mathversion{normal}}
%\newcommand{\trans}[1]{\textbf{\tau_{#1}}}
\def\T{T} % Matrix T
\def\A{A} % Matrix A
\def\B{B} % Block B
\def\AT{A\cdot T} %Matrix A.T
\def\tc{\tilde{c}}
\def\tU{\widetilde{U}} 
\def\tv{\tilde{v}}
\def\vecx{\textbf{x}}
\def\oQ{\overline{Q}}
\def\oq{\overline{q}}
\def\oU{\overline{U}}
\def\q1b{q_{1_{blind}}}
\def\q1bj{q_{1_{blind_j}}}
\def\P{\textit{P}} %Prover
\def\V{\textit{V}} %Verifier
\def\vecx{$\textbf{\x}$}







%----------Comments-----------
\newcommand{\mycomment}[1] {\textcolor{blue}  {{\sl{#1}}}}

% Author notes
\def\ShowAuthNotes{1}
\ifnum\ShowAuthNotes=1
%\newcommand{\authnote}[2]{{\textbf{$\mathbf{\big[}$~#1's note:}} \textbf{\em\small #2}~{$\mathbf{\big]}$}}
\newcommand{\authnote}[2]{\\ \textcolor{red}{\parbox{0.9\linewidth}{[{\footnotesize {\bf #1:} { {#2}}}]}}\\}
\else
\newcommand{\authnote}[2]{}
\fi

\newcommand{\dnote}[1]{\authnote{VD}{#1}}
\newcommand{\pnote}[1]{\authnote{PKP}{#1}}
\newcommand{\nnote}[1]{\authnote{NS}{#1}}
%%--------------------------------------------------------------
%User Defined Theorems
%--------------------------------------------------------------

\newtheorem{terminology}{Terminology}
%\newtheorem{theorem}{Theorem}[section]
%\newtheorem{corollary}[theorem]{Corollary}
%\newtheorem{lemma}[theorem]{Lemma}
%\newtheorem{conjecture}[theorem]{Conjecture}
%\newtheorem{proposition}[theorem]{Proposition}
%\theoremstyle{definition}
%\newtheorem{definition}[theorem]{Definition}
%\newtheorem{example}[theorem]{Example}
%\newtheorem{remark}[theorem]{Remark}
%\newtheorem{claim}[theorem]{Claim}
\newtheorem{notation}[theorem]{Notation}
\newtheorem{openproblem}[theorem]{Open Problem}


%--------------------------------------------------------------
% Macros for Operators
%--------------------------------------------------------------

%\newcommand{\xor}{\oplus}
\newcommand{\Xor}{\bigoplus}
\newcommand{\band}{\odot}
\newcommand{\bAnd}{\bigodot}
%\newcommand{\abs}[1]{| #1 |}
\newcommand{\Order}{\mathcal{O}}
\newcommand{\BigO}[1]{\ensuremath{\operatorname{O}\left(#1\right)}}
\newcommand{\cmark}{\ding{51}}
\newcommand{\xmark}{\ding{55}}
\newcommand{\defined}{\ensuremath{\stackrel{def}{=} }}
\newcommand{\iseq}{\ensuremath{\stackrel{?}{=} }}

%--------------------------------------------------------------
%Mathematical Objects
%--------------------------------------------------------------

\newcommand{\bitset}{\{0,1\}}
\newcommand{\Prob}{\ensuremath{\mathsf{Pr}}}
%\newcommand{\floor}[1]{\lfloor #1 \rfloor}

%--------------------------------------------------------------
%Fields and Rings
%--------------------------------------------------------------

\newcommand{\GF}[1]{\ensuremath{\mbox{GF}(2^ #1)}}
\newcommand{\Z}[1]{\ensuremath{\mathbb{Z}}_{2^{#1}}}
\newcommand{\AS}{\ensuremath{\mathbb{S}}}
\newcommand{\ZP}{\ensuremath{\mathbb{Z}}_{\mathsf{p}}}
\newcommand{\F}{\ensuremath{\mathbb{F}}}
\newcommand{\FP}{\ensuremath{\mathbb{F}}_{\mathsf{p}}}
\newcommand{\N}{\ensuremath{\mathbb{N}}}


%--------------------------------------------------------------
%MPC SubProtocols
%--------------------------------------------------------------

%\newcommand{\Adv}{\ensuremath{\mathcal{A}}\xspace}
\newcommand{\Sim}{\ensuremath{\mathcal{S}}}

%--------------------------------------------------------------
%MPC Variables
%--------------------------------------------------------------

\newcommand{\sparam}{\ensuremath{s}}
\newcommand{\abort}{\ensuremath{\mathtt{abort}}}
\newcommand{\flag}{\ensuremath{\mathsf{flag}}}
\newcommand{\MPC}{\ensuremath{\mbox{MPC}}}
\newcommand{\ctr}{\ensuremath{\mathsf{ctr}}}
%\newcommand{\secparam}{\ensuremath{\kappa}\xspace}

\newcommand{\Com}{\ensuremath{\mathsf{Com}}} %Commitment

%--------------------------------------------------------------
%Circuit Variables
%--------------------------------------------------------------

\newcommand{\PS}{\ensuremath{\mathsf{A}}} 
\newcommand{\IS}{\ensuremath{\mathsf{I}}}
\newcommand{\OS}{\ensuremath{\mathsf{O}}}  
\newcommand{\MS}{\ensuremath{\mathsf{M}}} 
\newcommand{\DF}{\ensuremath{\mathsf{D}}}
\newcommand{\VL}{\ensuremath{\mathsf{l}}}

\newcommand{\ckt}{\ensuremath{\mathsf{ckt}}}

\newcommand{\plusleft}{\ensuremath{\mathsf{x}}}
\newcommand{\plusright}{\ensuremath{\mathsf{y}}}
\newcommand{\plusoutput}{\ensuremath{\mathsf{z}}}
\newcommand{\multleft}{\ensuremath{\mathbf{x}}}
\newcommand{\multright}{\ensuremath{\mathbf{y}}}
\newcommand{\multoutput}{\ensuremath{\mathbf{z}}}
\newcommand{\multr}{\ensuremath{\mathbf{r}}}
\newcommand{\multrt}{\ensuremath{\trunc{\multr}}}
\newcommand{\multu}{\ensuremath{\mathbf{u}}}
\newcommand{\multv}{\ensuremath{\mathbf{v}}}


%--------------------------------------------------------------
% ML - Macros for Vectors and Matrices
%--------------------------------------------------------------

\newcommand{\va}{\mathbf{a}}
\newcommand{\vb}{\mathbf{b}}
\newcommand{\vp}{\mathbf{p}}
\newcommand{\vq}{\mathbf{q}}
\newcommand{\vu}{\mathbf{u}}
%\newcommand{\vv}{\mathbf{v}}
\newcommand{\vw}{\mathbf{w}}
\newcommand{\vx}{\mathbf{x}}
\newcommand{\vy}{\mathbf{y}}
\newcommand{\vz}{\mathbf{z}}


\newcommand{\Va}{\mathsf{a}}
\newcommand{\Vb}{\mathsf{b}}
\newcommand{\Vh}{\mathsf{h}}
\newcommand{\Vp}{\mathsf{p}}
\newcommand{\Vq}{\mathsf{q}}
\newcommand{\Vr}{\mathsf{r}}
\newcommand{\Vs}{\mathsf{s}}
\newcommand{\Vu}{\mathsf{u}}
\newcommand{\Vv}{\mathsf{v}}
\newcommand{\Vw}{\mathsf{w}}
\newcommand{\Vz}{\mathsf{z}}
\newcommand{\Vrs}{\mathsf{rs}}







\setlist[description]{style=unboxed,leftmargin=0cm}

%--------------------------------------------------------------
%Itemize, Enumerate, Description with less space
%--------------------------------------------------------------
\newenvironment{compactlist}{
	\begin{list}{{$\bullet$}}{
			\setlength\partopsep{0pt}
			\setlength\parskip{0pt}
			\setlength\parsep{0pt}
			\setlength\topsep{0pt}
			\setlength\itemsep{0pt}
			\setlength{\itemindent}{0.4pt}
			\setlength{\leftmargin}{10pt}
		}
	}{
	\end{list}
}


\newenvironment{myitemize}{
	\begin{list}{{$\bullet$}}{
			\setlength\partopsep{0pt}
			\setlength\parskip{0pt}
			\setlength\parsep{0pt}
			\setlength\topsep{0pt}
			\setlength\itemsep{0pt}
			\setlength{\itemindent}{0.4pt}
			\setlength{\leftmargin}{10pt}
		}
	}{
	\end{list}
}


\newenvironment{myitemizeold}
{\begin{list}{$\bullet$}{ %\itemindent=-.3cm \listparindent=.6cm
			\itemindent=-0.1in
			\itemsep=0.0in
			\parsep=0.0in
			\topsep=0.0in
			\partopsep=0.0in}}{\end{list}}
\newcounter{itemcount}

\newenvironment{myenumerate}
{\setcounter{itemcount}{0}\begin{list}
{\arabic{itemcount}.}{\usecounter{itemcount} \itemindent=-0.2cm
\itemsep=0.0in
\parsep=0.0in
\topsep=5pt
\partopsep=0.0in}}{\end{list}}

\newenvironment{mydescription}
{\setcounter{itemcount}{0}\begin{list}
{\arabic{itemcount}.}{\usecounter{itemcount} \itemindent=-0.5cm
\itemsep=0.0in
\parsep=0.0in
\topsep=5pt
\partopsep=0.0in}}{\end{list}}

%--------------------------------------------------------------
%Commands for figures and tables and references
%--------------------------------------------------------------

\newcommand{\tabref}[1]{Table~\protect\ref{tab:#1}}
\newcommand{\secref}[1]{Section~\protect\ref{sec:#1}}
\newcommand{\lemref}[1]{Lemma~\protect\ref{lem:#1}}
\newcommand{\figref}[1]{Figure~\ref{fig:#1}}
\newcommand{\boxref}[1]{Figure~\ref{#1}}
\newcommand{\figlab}[1]{\label{fig:#1}}
\newcommand{\refeqn}[1]{Equation~\eqref{#1}}

\newenvironment{boxfig}[2]{% {#1}{#2} = {Caption}{label}
	\begin{figure}[ht!]
		\newcommand{\FigCaption}{#1}
		\newcommand{\FigLabel}{#2}
		\vspace{-.15cm}
		\begin{center}
			\begin{small}
				\begin{tabular}{@{}|@{~~}l@{~~}|@{}}
					\hline
					%\rule[-1ex]{0pt}{1ex}\begin{minipage}[!htb]{\textwidth}   
					\rule[-1.5ex]{0pt}{1ex}\begin{minipage}[b]{.97\linewidth}
						\vspace{1ex}
						\smallskip
					}{%
					\end{minipage}\\
					\hline
				\end{tabular}
			\end{small}
			\vspace{-0.3cm}
			\caption{\FigCaption}
			\figlab{\FigLabel}
		\end{center}
		\vspace{-0.4cm}
	\end{figure}
}


\newenvironment{boxfig*}[2]{% {#1}{#2} = {Caption}{label}
	\begin{figure*}[h!]		
		\fontsize{5}{5}\selectfont
		\newcommand{\FigCaption}{#1}
		\newcommand{\FigLabel}{#2}
		\vspace{-.05cm}
		\begin{center}
			\begin{small}			 
				\begin{adjustbox}{max width=\textwidth}
					\begin{tabular}{@{}|@{~~}l@{~~}|@{}}
						\hline
						%\rule[-1ex]{0pt}{1ex}\begin{minipage}[!htb]{\textwidth}   
						\rule[-1ex]{0pt}{1ex}\begin{minipage}[b]{.95\linewidth}
							\vspace{1ex}	
						}{%
						\end{minipage}\\
						\hline
					\end{tabular}	
				\end{adjustbox}		
			\end{small}
			\vspace{-0.1cm}
			\caption{\FigCaption}
			\figlab{\FigLabel}
		\end{center}
		\vspace{-.38cm}
	\end{figure*}
}

\newenvironment{myboxfig}[2]{% {#1}{#2} = {Caption}{label}
	\vspace{-0.5cm}
	\begin{figure}[htb!]		
		\fontsize{5}{5}\selectfont
		\newcommand{\FigCaption}{#1}
		\newcommand{\FigLabel}{#2}
		\vspace{-.15cm}
		\begin{center}
			\caption{\FigCaption}
			\begin{small}			 
				\begin{adjustbox}{max width=\textwidth}
					\begin{tabular}{@{}|@{~~}l@{~~}|@{}}
						\hline
						%\rule[-1ex]{0pt}{1ex}\begin{minipage}[!htb]{\textwidth}   
						\rule[-1ex]{0pt}{1ex}\begin{minipage}[b]{.95\linewidth}
							\vspace{1ex}	
						}{%
						\end{minipage}\\
						\hline
					\end{tabular}	
				\end{adjustbox}		
			\end{small}
			%	\vspace{-0.25cm}
			\figlab{\FigLabel}
		\end{center}
		\vspace{-.38cm}
	\end{figure}
}


\newenvironment{myboxfig*}[2]{% {#1}{#2} = {Caption}{label}
	\begin{figure*}[!htb]		
		\fontsize{5}{5}\selectfont
		\newcommand{\FigCaption}{#1}
		\newcommand{\FigLabel}{#2}
		\vspace{-.10cm}
		\begin{center}
			\caption{\FigCaption}
			\begin{small}			 
				\begin{adjustbox}{max width=\textwidth}
					\begin{tabular}{@{}|@{~~}l@{~~}|@{}}
						\hline
						%\rule[-1ex]{0pt}{1ex}\begin{minipage}[!htb]{\textwidth}   
						\rule[-1ex]{0pt}{1ex}\begin{minipage}[b]{.95\linewidth}
							\vspace{1ex}	
						}{%
						\end{minipage}\\
						\hline
					\end{tabular}	
				\end{adjustbox}		
			\end{small}
			\vspace{-0.25cm}
			\figlab{\FigLabel}
		\end{center}
		\vspace{-.38cm}
	\end{figure*}
}



%--------------------------------------------------------------
%New Style boxes
%--------------------------------------------------------------

%----- Box Environment ---------------------------------------------------------
\RequirePackage{mdframed}

%Basic box structure with title box
\newenvironment{titlebox}[5]
{\mdfsetup{
		style=#2,
		innertopmargin=1.1\baselineskip,
		skipabove={\dimexpr0.7\baselineskip+\topskip\relax},
		skipbelow={1em},needspace=3\baselineskip,
		singleextra={\node[#3,right=10pt,overlay] at (P-|O){~{\sffamily\bfseries #1 }};},%
		firstextra={\node[#3,right=10pt,overlay] at (P-|O) {~{\sffamily\bfseries #1 }};},
		frametitleaboveskip=9em,
		innerrightmargin=5pt
	}
	\newcommand{\TitleCaption}{#4}
	\newcommand{\TitleLabel}{#5}
	\begin{mdframed}[font=\small]
		\setlist[itemize]{leftmargin=13pt}\setlist[enumerate]{leftmargin=13pt}\raggedright% 
	}
	{\end{mdframed}
	\vspace{-2em}
	{\captionof{figure}{\normalfont \TitleCaption}\label{\TitleLabel}}
	\medskip
}

%title box style
\tikzstyle{normal} = [thick, fill=white, text=black, draw, rounded corners, rectangle, minimum height=.7cm, inner sep=3pt]
\tikzstyle{gray} = [thick, fill=gray!90, text=white, rounded corners, rectangle, minimum height=.7cm, inner sep=3pt]

%box style
\mdfdefinestyle{commonbox}{%
	align=center, middlelinewidth=1.1pt,userdefinedwidth=\linewidth
	innerrightmargin=0pt,innerleftmargin=5pt,innertopmargin=5pt,
	splittopskip=15pt,splitbottomskip=15pt
}
\mdfdefinestyle{roundbox}{style=commonbox,roundcorner=5pt,userdefinedwidth=\linewidth}

\newenvironment{systembox}[3]
{\vspace{\baselineskip}\begin{titlebox}{Functionality \normalfont #1}{roundbox}{normal}{#2}{#3}}
	{\end{titlebox}}

\newenvironment{gsystembox}[3]
{\vspace{\baselineskip}\begin{titlebox}{Global Functionality \normalfont #1}{roundbox}{normal}{#2}{#3}}
	{\end{titlebox}}

\newenvironment{protocolbox}[3]
{\begin{titlebox}{Protocol \normalfont #1}{commonbox}{normal}{#2}{#3}}
	{\end{titlebox}}

\newenvironment{algobox}[3]
{\begin{titlebox}{Algorithm \normalfont #1}{commonbox}{normal}{#2}{#3}}
	{\end{titlebox}}

\newenvironment{reductionbox}[3]
{\begin{titlebox}{Reduction \normalfont #1}{commonbox}{normal}{#2}{#3}}
	{\end{titlebox}}

\newenvironment{gamebox}[3]
{\begin{titlebox}{Game \normalfont #1}{commonbox}{gray}{#2}{#3}}
	{\end{titlebox}}

\newenvironment{simulatorbox}[3]
{\begin{titlebox}{Simulator \normalfont #1}{commonbox}{normal}{#2}{#3}}
	{\end{titlebox}}
%-------------------------------------------------------------------------------



%-------------------------------------------------------------------------------
% Macros used for the examples
%-------------------------------------------------------------------------------

%----- Algorithm Environment ---------------------------------------------------
%Header for Algorithms/Functionalities
\newcommand{\algoHead}[1]{\vspace{0.2em} \underline{\textbf{#1}} \vspace{0.3em}}
\newcommand{\algoHeadExt}[2]{\vspace{0.2em} \underline{\textbf{#1} #2} \vspace{0.3em}}

%Multiline Algo-States
\makeatletter
\algnewcommand{\ExtendedState}[1]{\State
	\parbox[t]{\dimexpr\linewidth-\ALG@thistlm}{\hangindent=\algorithmicindent\strut\hangafter=3#1\strut}}
\makeatother

%Algorithms States
\algnewcommand\algorithmicinput{\textbf{Input:}}
\algnewcommand\Input{\item[\algorithmicinput]}
\renewcommand{\algorithmicensure}{\textbf{Output:}}

%Algo Comments
\algrenewcommand{\algorithmiccomment}[1]{{\color{gray}// #1}}

%  Font and Notation
%-------------------------------------------------------------------------------
\newcommand{\xmath}[1]{\ensuremath{#1}\xspace}

\newcommand{\command}[1]{\xmath{\textsc{#1}}}
\newcommand{\algorithm}[1]{\xmath{\mathsf{#1}}}
\newcommand{\variable}[1]{\xmath{\mathtt{#1}}}
\newcommand{\parameter}[1]{\xmath{\mathtt{#1}}}

%Nice empty set
\let\oldemptyset\emptyset
\let\emptyset\varnothing

%Functionalities
\newcommand{\Func}[1][\relax]{\xmath{\mathcal{F}_{\textsc{#1}}}}
%-------------------------------------------------------------------------------

%--------------------------------------------------------------
%New Style boxes - Some Examples for reference
%--------------------------------------------------------------

\begin{comment}

%Basic box structure with title box
\newenvironment{titlebox}[3]
{\mdfsetup{
		style=#2,
		innertopmargin=1.1\baselineskip,
		skipabove={\dimexpr0.7\baselineskip+\topskip\relax},
		skipbelow={1em},needspace=3\baselineskip,
		singleextra={\node[#3,right=10pt,overlay] at (P-|O){~{\sffamily\bfseries #1 }};},%
		firstextra={\node[#3,right=10pt,overlay] at (P-|O) {~{\sffamily\bfseries #1 }};},
		frametitleaboveskip=9em,
		innerrightmargin=5pt
	}
	\begin{mdframed}[font=\small]\setlist[itemize]{leftmargin=13pt}\setlist[enumerate]{leftmargin=13pt}\raggedright% 
	}
	{\end{mdframed}}

\newenvironment{systembox}[1]
{\vspace{\baselineskip}\begin{titlebox}{Functionality \normalfont #1}{roundbox}{normal}}
	{\end{titlebox}}

\newenvironment{gsystembox}[1]
{\vspace{\baselineskip}\begin{titlebox}{Global Functionality \normalfont #1}{roundbox}{normal}}
	{\end{titlebox}}

\newenvironment{protocolbox}[1]
{\begin{titlebox}{Protocol \normalfont #1}{commonbox}{normal}}
	{\end{titlebox}}

\newenvironment{algobox}[1]
{\begin{titlebox}{Algorithm \normalfont #1}{commonbox}{normal}}
	{\end{titlebox}}

\newenvironment{reductionbox}[1]
{\begin{titlebox}{Reduction \normalfont #1}{commonbox}{normal}}
	{\end{titlebox}}

\newenvironment{gamebox}[1]
{\begin{titlebox}{Game \normalfont #1}{commonbox}{gray}}
	{\end{titlebox}}

\newenvironment{simulatorbox}[1]
{\begin{titlebox}{Simulator \normalfont #1}{commonbox}{normal}}
	{\end{titlebox}}


\begin{protocolbox}{$\algorithm{MyProtocol}$}
	\algoHead{Computation:}
	\begin{algorithmic}[1]
		\State Do things. 
		\State A long an complicated step which requires many words to describe. At some point we have a line break.
		\While{Condition}
		\State In the loop.
		\EndWhile
	\end{algorithmic}
\end{protocolbox}

\begin{systembox}{$\Func[example]$}
	Some text describing example functionality.\\[2ex]
	
	\algoHead{Initialization:}
	\begin{enumerate}
		\item In the first step do:
		\begin{enumerate}
			\item This
			\item and that.
		\end{enumerate}
		\item Even more computation.
	\end{enumerate}
	%
	\algoHead{Computation:}
	\begin{algorithmic}[1]
		\State Do things. 
		\State A long an complicated step which requires many words to describe. At some point we have a line break.
		\While{Condition}
		\State In the loop.
		\EndWhile
	\end{algorithmic}
	
\end{systembox}


\begin{gsystembox}{$\Func[example]$}
	
\end{gsystembox}


\begin{simulatorbox}{$\sigma$}
	
\end{simulatorbox}

\end{comment}
%\usepackage[a4paper,total={6in,8in}]{geometry}




\newtheorem{theorem}{Theorem}
\newtheorem{definition}{Definition}

\newtheorem{lemma}[theorem]{Lemma}

%\newcommand{\mc}{\mathcal}
\title{Distributed prover proof generation for bulletproofs}

\begin{document}
\maketitle


\section{Distributed prover proof generation for R1CS}
\begin{comment}
We will give two constructions of distributed proof generation for R1CS with zero knowledge property. In the first construction we will start with reducing the proof for single prover R1CS to single prover zero knowledge inner product argument, then we will give a transformation for zero knwoledge inner product argument from single prover to multiple provers, which will directly imply the proof generation system for multiprover R1CS. In the second construction we will  
\end{comment}
Consider the setting of distributed prover, i.e. that witness is distributed among the provers, say N provers together hold the witness. Here we will discuss the distributed prover version of the bulletproofs for R1CS circuits. We will start with describing the distributed prover version of zero knowledge innerproduct, then we will use the protocol to design a protocol for R1CS.

We will start with reducing a R1CS circuit into multiple inner product. Then we will use the zero knowledge inner product to construct the proof of R1CS circuit. Consider that the circuit has $n$ multiplcation gates. corresponding to the $j^{th}$ wire let $x_j, y_j$ and $z_j$ are the left input, right input and output wires respectively. Then the vectors $\bm{x}, \bm{y}, \bm{z}\in \bbF^n$ can be obtained from the extended witness $\bm{w}$, by multiplying public matrices $A, B, C$ which is depended only on the circuit. Therefore we have 
\begin{align*}
A\bm{w}=\bm{x} \cdots (1)\\
B\bm{w}=\bm{y} \cdots (2)\\
C\bm{w}=\bm{z} \cdots (3)
\end{align*}
and $\bm{x}, \bm{y}, \bm{z}$ will satisfy the following eqution: $\bm{x} o \bm{y} =\bm{z} \cdots (4)$ 

For any $\bm{r}\in \bbF^n$, from equation (1), (2), (3) and (4), the following conditions are true:
\begin{align*}
\langle \bm{r}^T \cdot A, \bm{w}\rangle = \langle \bm{r}, \bm{x} \rangle =v_1 \\
\langle \bm{r}^T \cdot B, \bm{w}\rangle = \langle \bm{y}, \bm{r} \rangle =v_2 \\
\langle \bm{r}^T \cdot C, \bm{w}\rangle = \langle \bm{r}, \bm{z} \rangle =v_3 \\
\langle \bm{r} o \bm{x}, \bm{y} \rangle = \langle \bm{r}, \bm{z} \rangle =v_4 \\
\end{align*} 
We have zero knowledge inner product argument protocol for the following instance:\\
$$\{(P,V,\bm{g},\bm{h},g,h);(\bm{a},\bm{b},\sigma,\delta): P=h^{\sigma}\bm{g}^{\bm{a}}\bm{h}^{\bm{b}}, V=h^{\delta}g^{\langle \bm{a},\bm{b}\rangle}\} \cdots (5)$$
where the public part is $\{(P,V,\bm{g},\bm{h},g,h)$ and secret to the prover is $(\bm{a},\bm{b},\sigma,\delta)$.\\
Define: $v=\langle \bm{a},\bm{b}\rangle$\\
%Finally proving the circuit satisfiability for R1CS can be viewed as 
%$$\{(P_1,P_2,P_3,P_4,P_5,P_6,P_7,P_8, V_1,V_2,V_3,V_4,V_5,V_6,V_7,V_8, P_1=)\}$$
So if we apply the above protocol on these 8 inner products that gives a proof of circuit satisfiability of R1CS, for example consider $\langle \bm{r}^T \cdot A, \bm{w}\rangle = v_1$, in this case consider $\bm{a}=\bm{r}^T\cdot A$, $\bm{b}=\bm{w}$ and $v=v_1$, and use the protocol (5) in these setting.\\
\subsection{Multiprover version of zero knowledge inner product argument}
Now we will discuss the multiprover version of the zero knowledge inner product proof: where there will an aggregator $\cA$(need not be trusted) who will iinteract with the verifier.\\
Consider $\cP_1,\ldots, \cP_N$ are provers for the inner product statement i.e. $\{(P,V,\bm{g},\bm{h},g,h)\}$ and the corresponding witness is $(\bm{a},\bm{b}, \sigma, \delta)$, where $P= h^{\sigma} \bm{g}^{\bm{a}} \bm{h}^{\bm{b}}$, $V=h^{\delta}g^{v}$. Let party $\cP_i$ has $(\bm{a}_i,\bm{b}_i,\sigma_i,\delta_i)$ such that
\begin{align*}
	\sum\limits_{i=1}^{N}\bm{a}_i=\bm{a},\text{ }
	\sum\limits_{i=1}^{N}\bm{b}_i=\bm{b},\text{ }
	\sum\limits_{i=1}^{N}\sigma_i=\sigma,\text{ }
	\sum\limits_{i=1}^{N}\delta_i=\delta \text{ }\cdots(5)
\end{align*}
\begin{enumerate}
	\item In this step each party chooses their blinding vectors and commits to that vector. $\cP_i$ samples $\bm{s}^i_L, \bm{s}^i_r \leftarrow_\$ \bbZ_p^n$ and $\rho_i\leftarrow \bbZ_p$, then computes $S^i=h^{\rho_i}\bm{g}^{\bm{s}^i_L}\bm{h}^{\bm{s}^i_R}$ and sends $S^i$ to $\cA$.
	\item $\cA$ computes $S = \prod_{i=1}^{N}S^i = h^{\sum_{i=1}^N\rho_i}\bm{g}^{\sum_{i=1}^N\bm{s}_L^i}\bm{h}^{\sum_{i=1}^N\bm{s}^i_R} = h^{\rho}\bm{g}^{\bm{s}_L}\bm{h}^{\bm{s}_R}$, where $\rho = \sum_{i=1}^{N}\rho_i$, $\bm{s}_L =\sum\limits \bm{s}^i_L,  \bm{s}_R=\sum_{i=1}^{N}\bm{s}^i_R$, and $\cA$ sends $S$ to the verifier $\cV$. In this step $\cA$ computes the commitment of the blinding vector, which is sum of all the provers blinding vector.
	\item Prover $\cP_i$ constructs a vector polynomial $l_i(X)=\bm{a}_i + X. \bm{s}^i_L$ and $r_i(X)=\bm{b}_i + X. \bm{s}^i_R$. All the provers involve in an MPC to obtain the shares of the polynomial $T(X)=\langle \sum_{i=1}^N l_i(X),\sum_{i=1}^N r_i(X)\rangle$. Let $\cP_i$ gets the share $T_i(X)$ of $T(X)$, where $T(X)=\sum_{i=1}^{N}T_i(X)$. Let $T_i(X)= t^i_0+t^i_1.X+t^i_2.X^2$, then $\sum_{i=1}^N t^i_0=\langle \bm{a}, \bm{b} \rangle$. Now $\cP_i$ commits to $t^i_j$ using randomness $\tau^i_j$ i.e. $T^i_j=h^{\tau^i_j}g^{t^i_j}$ and sends $T^i_j$ to $\cA$ for $j\in \{1,2\}$.
	\item $\cA$ computes $T_j=\prod_{i=1}^{N} T^i_j$ which gives the commitment of $t_j \forall j\in\{1,2\}$ and sends $T_1,T_2$ to the verifier $\cV$.
	\item $\cV$ samples $x\leftarrow_\$\bbZ^*_p$ and sends it to the aggregator $\cA$.
	\item $\cA$ sends the $x$ to the provers. 
	\item $\cP$ computes $\bm{l}_i=l_i(x), \bm{r}_i=r_i(x)$, $\mu_i= \sigma_i+\rho_i.x$, $\tau^i_x= \delta_i+\tau^i_1.x+\tau^i_2.x^2$ and sends $\bm{l},\bm{r}, \mu_i, \tau^i_x$ to $\cA$.
	\item $\cA$ defines $\bm{l}=\sum_{i=1}^N\bm{l_i}$ and $\bm{r}=\sum_{i=1}^N\bm{r_i}$ and $\hat{t}=\langle \bm{l},\bm{r}\rangle =T(x)$. $\cA$ computes $Q=\bm{g^lh^r}$, $\mu=\sum_{i=1}^N \mu_i$ and $\tau_x= \sum_{i=1}^{N}\tau_x^i$. And finally $\cA$ sends $Q, \mu, \tau_x, \hat{t}$ to $\cV$.
\end{enumerate}
$\cV$ checks: 
\begin{enumerate}
	\item $h^{\tau_x}g^{\hat{t}}=?V\cdot T_1^x\cdot T_2^{x^2}$ to check the correct evaluation of the committed polynomial $T$ at $x$.
	\item $P\cdot S^x =? h^{\mu}\cdot Q$. 
	\item Runs the inner product proof from \cite{Bulletproofs}, which need not be zero knowledge on the input $(Q, \hat{t}, \bm{g}, \bm{h}, g)$ with the aggregator $\cA$.
\end{enumerate}
The verifier $\cV$ accepts the proof if the checks succeed and the inner product argument accepts.

\begin{comment}
\subsection{Completeness} Consider the provers are honest i.e. $\cP_1,\ldots, \cP_N$ together hold the witness i.e. equation (5) is true, then the checks succeed. We will see one by one how all the checks succeed.\\
The first check succeeds because:
\begin{align*}
	h^{\tau_x}g^{\hat{t}} &= h^{\delta+\tau_1 x+ \tau_2 x^2}\cdot g^{\langle \bm{a},\bm{b}\rangle + t_1x+t_2x^2}\\
	&= h^{\sum\limits_{i=1}^N(\delta_i+\tau^i_1x+\tau^i_2x^2)}\cdot g^{\sum\limits_{i=1}^N (t^i_0+t^i_1x+t^i_2x^2)}\\
	&= \prod\limits_{i=1}^N h^{\delta_i}.h^{\tau^i_1x}.h^{\tau^i_2x^2}\prod\limits_{i=1}^N g^{t_0}.g^{t^i_1x}.g^{t^i_2x^2}\\
	&= (h^{\delta}\cdot g^{\langle \bm{a},\bm{b}\rangle})\cdot(h^{\tau_1}g^{t_1})^x\cdot(h^{\tau_2}g^{t_2})^{x^2}\\
	&= V\cdot T_1^x \cdot T_2^{x^2}
\end{align*} 
The second check succeeds because:
\begin{align*}
	P\cdot S^x &= h^{\sigma}\bm{g}^{\bm{a}}\bm{h}^{\bm{b}}\cdot (h^{\rho}\bm{g}^{\bm{s}_L}\bm{h}^{\bm{s}_R})^x\\
	&= h
\end{align*}
\subsection{Soundness}
\end{comment}
The completeness, soundness and zero knowledge hold and the proof is similar to zero knowledge inner product argument.\\
We can argue the privacy by saying that provers are interacting only to get the shares of $T(X)$ and that is being done using secure MPC, and remaining all the interactions are done with the aggregator $\cA$ only. Who is learning nothing more than whatever $\cV$ learns in zero knowledge inner produt argument. 
$\cA$ has commitments of the shares of $s_L$ and $s_R$, and computes the commitments of $s_L$ and $s_R$, hiding property of the commitment ensures that $\cA$ learns no information about the secrets $s_L, s_R$ as well as their shares.

$\cA$ receives the commitments of the coefficients of the shares of the polynomial $T(X)$, and reconstructs the commitments of the coefficients of $T(X)$, again hiding property of the commitment ensures that $\cA$ learns no information about the polynomial $T(X)$.

In the next step, $\cA$ gets $\hat{l}, \hat{r}, \mu, \tau_x$, which verifier gets in single prover zero knowledge inner product argument, the zero knowledge property of the above ensures that $\hat{l}, \hat{r}, \mu, \tau_x$ is not leaking any information about the secrets.

Therefore A is not learning anything new. Which implies privacy of the provers are preserved.

\subsection{optimization}
We can combine the inner product $\langle \bm{r}^T \cdot A, \bm{w}\rangle = \langle \bm{r}, \bm{x} \rangle =v_1$ and $\langle \bm{r}^T \cdot C, \bm{w}\rangle = \langle \bm{r}, \bm{z} \rangle =v_3$ which can be represented as $\langle \bm{r}^T \cdot(\alpha.A+\gamma.C), \bm{w}\rangle = \langle \bm{r}, (\alpha.\bm{x}+\gamma.\bm{z}) \rangle =v_3$ where $\alpha, \gamma$ are randomly chosen by the verifier $\cV$. In the commitment we are considering that if $\langle \bm{a},\bm{b}\rangle= c$ is in the statement of the inner product, then the commitment of $\bm{a}$ and $\bm{b}$ will be of the form $h^{\sigma} \bm{g}^{\bm{a}} \bm{h}^{\bm{b}}$. Note that in the commitment there should not be any known relation between the generators and for the equation $\langle \bm{r} o \bm{x}, \bm{y} \rangle = \langle \bm{r}, \bm{z} \rangle =v_4$ there should not be the known relation between the generator $\bm{g}$ which is used to commit to $\bm{x}$ with the generator which is used to commit to $\bm{y}$, for that reason the commitment for $\bm{y}$ is considered in left in $\langle \bm{y}, \bm{r} \rangle = v_2$. For that reason this inner product we could not batch with the prior inner products. In this case we need to give 6 inner product proofs instead of 8. 

\subsection{Alternate proof for R1CS}
In the beginning we just presented the reduction of R1CS satisfiability to 8 innerproduct argument, where each of them are inner products of vectors of size $n$. Another simple way of combining the inner products of vectors of size $n$ into a single inner product of size of $8n$, (for further optimization, this $8n$ can be reduced to $6n$) were given in \cite{Bulletproofs} in their range proofs. 
For each inner product choose a different set of generators $(\bm{g}^{(k)},\bm{h}^{(k)})_{k=1}^{8}$ and define $\bm{g}$ as the interleaved concatenation of all $\bm{g}^{(k)}$ such that $g_i=g_{\lceil \frac{i}{8} \rceil}^{(i \mod 9)}$. Define $\bm{h}$ in the similar way. The private vectors are also combined in the similar way. Therefore the final inner product will look like:
$$\{(P,V,\bm{g},\bm{h},g,h);(\bm{a},\bm{b},\sigma,\delta): P=h^{\sigma}\bm{g}^{\bm{a}}\bm{h}^{\bm{b}}, V=h^{\delta}g^{\langle \bm{a},\bm{b}\rangle}\}$$
where $a_i= a_{\lceil \frac{i}{8} \rceil}^{(i \mod 9)}$ similarly define $\bm{b}$. $\langle\bm{a},\bm{b}\rangle=\sum_{k=1}^{4}2\times v_i$
In this way proof size will be $2\log (8n) + 2$.
\bibliographystyle{plain}
\bibliography{references}
\end{document}

 

