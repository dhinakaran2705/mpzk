\section{Introduction} \label{sec:intro}
In a zero-knowledge proof protocol, a verifier tries to verify that an instance $x$ is in a language $L$. A prover with a witness $w$ to $x \in L$ tries to convince the verifier on this without revealing any additional information to the verifier than what he already knows. Be it proving that a user is single-spending a coin that he has earned before in a private manner \cite{zerocash} or validating sensitive web browser data reported during telemetry \cite{MozillaPrio}, ZK protocols have garnered immense interest over the last few years. Still in the dream of a decentralized world, there are multiple co-opetitive entities interacting with each other to obtain insights and maximize their goals. A desirable goal is to enable these mutually distrusting entities to prove a predicate on their joint data. The traditional ZK setting is restrictive here since there is a single prover holding the entire witness required to prove the predicate. 

In this work, we study the general setting where multiple co-opetitive i.e., collaborating but mutually distrusting entities each possessing their own secret data want to prove to a verifier that their secret data together satisfies a predicate of common interest. This is done without revealing any information about their sensitive data to each other, more than what they know about it beforehand. 
-joint loan application-
-proof of joint stake- and use it for joint auction-
-verifiably publishing cross-industry statistics in trade logistics networks-
In abstract terms, we have multiple provers $P_1, \ldots, P_p$ respectively possessing witnesses $w_1, \ldots, w_p$. For a predicate $C$ from the supported class of predicates $\cC$, the provers want to prove to a verifier that $C(w_1,\cdots,w_p) = 1$.

A natural solution for distributed proof generation is to start with the prover code from a single prover protocol and run this code between multiple provers using multi-party computation. Pedersen \cite{Ped92} proposed this notion and discussed this generic construction using MPC.
%Each round of proof generation is implemented by an MPC among the provers to generate the message to be sent to the verifier.
This generic construction is optimal in proof size and verifier complexity in that it retains these complexities from the single-prover version irrespective of the number of provers. But, the complexity of proof generation suffers when the single prover protocol is not constructed with distributed proof generation in mind. Even an optimal $R$-round proof generation algorithm with $O(N)$ complexity might result in a distributed proof generation algorithm with $\Omega(R \cdot N)$ communication.
Prior works have improved the complexity of distributed proof generation but only when considering simple predicates as in threshold signatures \cite{DDS} and simple sigma protocols \cite{EfficientTZ}, or when working with a weaker adversarial model \cite{trinocchio}

The goal of this work is to study zero-knowledge protocols supporting \textit{arbitrary} predicates where the proof generation algorithm is ``MPC-friendly''. %The proof generation algorithm is usually measured for efficiency in terms of number of group operations and number of rounds of interaction with the verifier.
Our work first identifies new parameters that a ZK protocol should optimize for to admit efficient distributed proof generation. We then construct a ZK protocol which is the state-of-art in these parameters.

\subsection{Efficiency parameters for DPZK}
Our discussion will focus on honest-verifier protocols with transparent setup, though most of this would also apply to protocols with a trusted setup. %producing a fixed or an updatable structured reference strings (SRS).
In a single-prover setting, the efficiency of a proof generation algorithm is usually measured in terms of the number of group operations. Additionally, the number of rounds of interaction between prover and verifier is also considered an efficiency parameter when a non-interactive prover from the interactive version in a provably secure manner \cite{BCS16}. But, these parameters do not capture the core bottleneck in the setting of multiple provers. For instance, Spartan \cite{spartan} has a very efficient proof generation algorithm, but due to the high multiplicative complexity of proof generation its distributed version requires $O(N^2)$ MPC multiplications where $N$ is the number of gates in the circuit representation of the predicate being proved. %Similar arguments will be made in a future section for the other state-of-art protocols like Aurora \cite{aurora}, Bulletproofs \cite{bulletproofs}. \dnote{mention the asymptotics in the above claims}

As a first step, we identify parameters with significant impact in distributed proof generation. 
As indicated earlier, the first parameter of interest is \textit{cross-multiplicative complexity} of the proof generation process. 
This would be the number of multiplications between the witness bits from different parties, since an affine function over the witnesses and a multiplication between witness bits from the same prover can be computed without any interaction in a secret-sharing based MPC.
%The complexity of an MPC protocol would grow with this parameter. 
\dnote{fine tune the last two sentences. currently, they allude to secret-sharing based MPC. Is it okay to include a separate sentence to capture GC based MPC protocols?}

In a multi-round interactive protocol, the next parameter of interest is the \textit{number of MPC rounds}. This denotes the number of rounds in the prover-verifier protocol which require multiplication of witness bits from different parties. For instance, even though Bulletproofs \cite{bulletproofs} is optimal in terms of prover complexity, each of the $\log N$ rounds of protocol involves multiplications between witness bits from different provers. And this requires an MPC to be performed for each round of prover message generation.\dnote{1) can we say anything non-trivial here from the NI-MPC literature?\\
2) do we have a stronger motivation, in terms of efficiency, for optimizing the number of rounds rather than just optimizing our first parameter, the multiplicative complexity for the end-to-end prover protocol?}
This parameter is also of cryptographic interest. If there is more than one round of prover message generation which require MPC, care has to be taken to ensure a secure composition of the individual MPCs to prove the complete protocol secure.

\subsection{Subtlety in proving zero-knowledge}
Before we delve into our DPZK construction, we would provide a subtle note regarding the zero-knowledge property of DPZK protocols. This would shed some light on the rationale for some of the artefacts of DPZK protocols. In an DPZK protocol, the verifier is allowed to collude with a subset of provers to learn the witnesses of the other provers. A strong definition of zero-knowledge should ensure that this collusion does not learn any additional information than known before the protocol. This includes the number of other provers involved in the proof generation. \dnote{is it even feasible to achieve this notion?!} In our protocols, we will allow all the provers to know the total number of provers involved. This will lead to a relaxed definition where a verifier colluding with a prover will additionally learn the total number of provers. This still does not let a standalone verifier learn the total number of provers during its interaction with the provers.

This subtle note also impacts the size of the oracles for the protocols in the IOP framework \cite{aurora, ligero}. The oracle and the answers to the oracle queries should be independent of the total number of provers. Hence, for the IOP-based protocols, the two parameters that we discussed earlier should also be optimized for the oracle generation and the query answering.\footnote{In principle, generating the oracle or answering a query would form a part of the prover message generation when IOP is viewed in the lens of an arbitrary interactive protocol.}
The IOP-based protocols in the literature \cite{aurora, ligero} market their non-use of public-key cryptography and thus a potential post-quantum security feature. To achieve zero-knowledge in the distributed prover setting while retaining these features, at a high level, they have to incur the overhead of using information-theoretic MPC to construct the oracle and answer oracle queries, in addition to using it for generating the prover messages.

\subsection{\name{}: a DPZK protocol with linear cross-multiplications and one MPC round}
The main contribution of our work is an MPC-friendly single-prover protocol \name{} with $O(N)$ cross-multiplications all in the same round of the prover-verifier interaction. These cross-multiplications can also be done concurrently, and hence the MPC is performed on an $N$ mult-gates circuit of depth 1. \name{} achieves these properties while achieving proof size and verification complexity competitive with the state-of-art zero-knowledge protocols which do not admit efficient distributed proof generation. In particular, \name{} achieves a proof size of $O(N^{1/c})$ for arbitrary $c\geq 2$, and a verification complexity of $O(N^{1-2/c})$ public key operations and $O(N)$ field operations. \name{} achieves the zero-knowledge property while supporting a threat model of choice - semi-honest/malicious and threshold - to be plugged in for the provers.

\subsection{Circuit share complexity}
Consider a predicate where the witness is \textit{partitioned} among the provers in the DPZK setting. We define a new notion of efficiency for distributed proof generation which would become a prominent efficiency measure for a class of real-world applications.
Let \textit{shared circuit} denote the part of the circuit representing the predicate whose wire values are functions of inputs from more than one prover. 
We introduce the notion of \textit{circuit share complexity} to denote the size of this shared circuit.
Consider the class of applications where hashes (or any other commitment) of data from different parties are stored on a blockchain and a zero-knowledge proof has to be generated on an aggregation of all the data. This could be: 
\begin{enumerate}
\item Finance network: different bank account holders proving to a loan provider that the sum of their account balances is greater than the required threshold.
\item Trade logistic network: different logistic providers proving to a customer or regulator that the average delay of shipment along a particular route is less than a specified time.
\end{enumerate}
In all the applications in this class, each party participating in the aggregation first proves the \textit{relevance} of his/her data before they all run a protocol to prove the aggregated value on their data together. The proof of relevance involves proving that the data corresponds to a commitment in the blockchain network by proving the knowledge of the opening of a commitment in the blockchain. 
For instance, when using verifying a hash commitment within a proof system which works with circuit representation of the relation, the proofs of relevance i.e. the hash verifications of the individual data take up most of the gates in the circuit. But each such verification rely only on data from a single party. The aggregation does involve data from multiple parties but this step requires a relatively smaller number of gates. Hence, the circuit share complexity is very small for such applications, compared to the total size of the circuit. A DPZK protocol with its complexities proportional to the circuit share complexity instead of the total circuit complexity can attain major savings for such applications.%In general, the compatibility of the commitment with function representation of the proof system plays a part in the above discussion. We will discuss the relevant work on this compatibility in more detail in Section \ref{sec:relatedwork}.
Looking ahead, the complexity of the interaction between the provers in our DPZK protocol will be linear only in the circuit share complexity of the circuit predicate to be proven, and not in the total size of the circuit. 





%their models avoid communication among the provers - but it is necessary(???) when proving a general class of predicates?
%Difference in definition when involving multiple rounds with more than one round involving communication among provers -- the input to provers in Round i+1 is a function of protocol execution from previous rounds.
%

%Applications: 
%Proof of joint stake: 
%
%Joint loan application
%
%
%Positioning:  Distributed protocols
%- Threshold signature
%- Sigma Protocols
%- Trusted Setup (Trinocchio)
%- Transparent Setup (Bulletproof - Range proofs)
%
%Efficiency Desiderata:  
%- Transparent setup 
%- Minimum interaction between the provers (\# of rounds of MPC) 
%- Total communication complexity of the msgs exchanged
%- verifiers view is indistinguishable from a single prover setting
%- How efficient is underlying zkp protocol (public key operations, size of proof, prover/ verification time)
%- Applicability to a general class of computation
%
%
%
%
%
%
%Applications of ZK proofs - mention those which naturally extend to multiple provers holding witness - think about applications both in the real-world setting, and also in crypto settings (eg. does our setting lead to more efficient MPC protocols?)\\
%
%A typical zero-knowledge protocol...\\
%In this work, we study the setting where the witness is present among multiple provers - secret shared / partitioned \\
%Motivate by extending above use-cases\\
%Create new use-cases which do not exist with single prover protocols?\\
%
%What are our main selling points?\\
%-- the notion of distributed proof generation (fine-tune it based on state-of-art? with concrete efficiency for general predicates)\\
%-- single-prover protocol with ?? (what are the exact parameters we should highlight? concrete prover complexity, transparency)\\
%
%Related work for the above mentioned claims
%\dnote{CHECK: our protocol needs MPC in only one of the rounds among the log n rounds. How does it get better than others? }

\subsection{Technical overview of \name{}}


%\subsection*{Distributed proof generation}
%--- Informal notion of Distributed Provers Zero Knowledge Protocol $\innp{\Pi}{\verifier}$--- 
%
%A na\"ive approach to build a $\DPZK$ protocol is to take a single-prover ZK protocol and execute a multi-party computation among the provers of each message to be generated by the prover in the single-prover protocol. This satisfies the required security properties, but suffers in concrete efficiency when MPC is used over the proof generation as a black box. 
%
%\subsubsection*{Need for homomorphic commitments} 
%(Oracle access for IOP).
%The (non-interactive) zero-knowledge proof generation protocols based on the Interactive oracle proofs (IOP) or the Interactive PCP (IPCP) paradigms make use of \textit{oracles}. The prover establishes an oracle based on her witness. Once established, the IOP/IPCP protocols ensure that a prover will not modify the oracle further in the protocol. The verifier queries the oracle later in the protocol to verify some the claims made by the prover. When IOP/IPCP based proof systems are converted into the non-interactive setting, the establishment of the oracle is done by producing a commitment to the oracle entries.
%
%We will now run through a mental experiment on achieving distributed proof generation following the blueprint of the single-prover proof generation algorithm. We desire to not have the proof length depend on the number of parties. As we would discuss in depth later, we would also avoid running an MPC protocol on the proof generation circuit(s) of some zero-knowledge protocol. This would trivially, in theory, satisfy our goals, but it would be very inefficient in practice. The way we satisfy these goals for the establishment of the oracle is by requiring the oracle entries and their commitments satisfy homomorphic properties.
%\dnote{self: Sounds somewhat weak. Make it more compelling.}
%
%\subsection*{Technical summary}
%We will now provide a high-level summary of our work and along the way mention our core ideas.
%
%%Our contributions are two-fold: first in defining and constructing a zero-knowledge protocol with distributed proof generation and the second in proposing the first protocol with (expected) sub-linear verifier complexity.
%
%Our main contribution is defining and constructinf a zero-knowledge protocol with distributed proof generation.
%
%\paragraph{DPZK}
%We design a framework for constructing DPZK protocols starting from a single-prover ZK protocol.
%%To construct our DPZK protocol \name{} we follow the strategy of designing a (public coin honest verifier) single-prover ZK protocol whose proofs are ``aggregatable'' at minimal cost. 
%Let the parties $\prover_1, \ldots, \prover_{\Num}$ hold their respective witnesses $\wit_1, \ldots, \wit_{\Num}$ to the relation $R(\stmt, (\wit_1, \ldots,\wit_{\Num}))$. 
%Starting with a (public coin honest verifier) single-prover ZK protocol, there are two steps involved in our proof generation.
%\begin{itemize}
%\item The first step involves performing a secret sharing based MPC protocol on $R$ between the provers. As a result, each prover obtains her share of the extended witness, which includes all the wires (correspondingly variables) of a circuit (correspondingly R1CS) representation of R. 
%\item Next, for each message from the prover to the verifier, the parties perform the following two steps:
%\begin{itemize}
%\item The parties run an MPC protocol to obtain their share of the message. This step has a major implication on the efficiency of distributed proof generation. Depending on the single-prover protocol, some messages already posses homomorphism properties. Hence, the output of a local run of the single-prover algorithm for this round on a prover's shares of extended witness is already her share of the message in the distributed proof generation. When the message is not homomorphic, we design custom efficient protocols to obtain the shares of the message.
%\item One of the prover parties or an external entity trusted with the correctness of execution is chosen as an ``aggregator''. Aggregator is not trusted with the privacy of the inputs. The aggregator receives the shares of the messages from all the parties and performs the steps of the single-prover protocol to be performed with the aggregated message. Also, a prover acting as an aggregator does not have an incentive to deviate from executing the correct protocol during this step. This is because she could have caused the protocol to abort in any of the previous steps, and there is no additional information that she gains in the intermediate steps that aids her deviation in the aggregation step.
%\end{itemize}
%\end{itemize}
%
%Starting with the single-prover protocol of \cite{ours}, we design custom protocols so that the only additional interaction between the parties during the second step involves MPC for $\intpartofC$ concurrent multiplications, where $\intpartofC$ is the number of wires in the \textit{interactive} part of the circuit. Consider the check $x+y \stackrel{?}{=} z$ when given their respective hashes. If the inputs are present with three different parties, the corresponding circuit for the hash check followed by the addition and equality check involves extensive local computations for the hash before the interaction happens. \dnote{self: Provide a compelling real world use-case here} The additional communication complexity of our $\DPZK$ protocol grows only with the size of the interactive part. To enable this, we do.... start with \cite{ours} --- arrange the rows accordingly in the witness matrix --- 
%
%proof size independent of number of parties --- Oracle queries outputs should be independent of the number of parties --- we do it having parties generate a homomorphic oracle starting with their shares on the extended witness --- we would achieve indistinguishability between proof generated in a distributed manner and the proof generated by a single party ---
%In this, we use homomorphic commitments --- we follow from \cite{ours} --- 


\subsection{Related Work}\label{sec:relatedwork}
%\commentA{this section does not look comprehensive}
As mentioned earlier in the introduction, Pedersen \cite{Ped92} defines the
notion of distributed proof genration and proposes a generic construction using
MPC. Over the years, works have discussed distributed proof generation for a
restricted class of predicates: Desmedt et al. \cite{DDB94} proposing proofs for
graph isomorphism, various works \cite{King05, DDS, Desmedt2011} for proposing
threshold signatures, Keller et al. \cite{EfficientTZ} for a class of sigma
protocols. The Bulletproofs paper \cite{bulletproofs} also presents a
``distributed'' protocol to produce an aggregation of range proofs on inputs
held by different parties. However, their completeness property is weaker than
ours, i.e, a set of provers with valid witness can only prove the statement only if each input gate is exclusive to one prover. 
Trinocchio \cite{trinocchio} proposes a distributed proof generation method for
the popular Pinocchio protocol \cite{pinnochio_PHGR} but only when there is a
honest majority of semi-honest parties. Moreover, it needs the trusted setup
model. The work DIZK \cite{dizk} provides a distributed implementation of the proof generation of a non-interactive zero-knowledge protocol (zkSNARK).


%Our single-prover version of the protocol is also competitive in the state-of-art of such protocols \cite{ligero, aurora, bulletproofs}. 
%We provide a brief summary of ``direct'' transformations of the state-of-art zero-knowledge protocols in our setting into their $\DPZK$ versions.
%The IPCP protocol Ligero \cite{ligero} requires no public key operations during its proof generation and admits efficient proof generation, in practice \cite{diogenes}, for a single prover. In asymptotics, it admits $O(N \log N)$ prover complexity and $O(\sqrt{N})$ proof size. Another oracle based construction Aurora \cite{aurora} admits $O(\log^2 N)$ proof size, but its prover algorithm is expensive in practice due to its high usage of memory even for moderate sized circuits with a million gates.\dnote{@Nitin: check the wording of the sentence}
%Ligero and Aurora also do not directly admit distributed proof generation unless a generic MPC is applied for constructing their oracle and answering its queries. The Bulletproofs protocol \cite{InnerProductDLS,bulletproofs} has one of the best proof sizes of $O(\log N)$, but its prover and verifier algorithms involves $O(N)$ public key operations which makes it expensive. 
%\dnote{once we get concrete number for our protocol for a million and a billion gate circuits, make the related work section more concrete, especially for the single prover protocol comparison}
Table \ref{tab:SPZK} captures the summary.
\begin{table*}[ht] 
	\centering
	\resizebox{\textwidth}{!}{
		%\begin{threeparttable}
		\begin{tabular}{|c|c|c|c|c|c|c|} 
			\hline
			Protocols & Argument size & $\zkrounds$ & Prover's complexity & Verifier's complexity & $\prrounds$ & $\prcomm$ \\
			\hline
			D-Bulletproofs & $\log (N)$ & $\log(N)$ & $O(N)$E & $O(N)$E & 2 & $\Num N + \Num^2\cdot N_s$ \\
			\hline
			\name & $N^{1/c}$ & $\log(N)$ & $O(\frac{N}{\log(N)})E + O(N\log(N))M$ & $O(N^{1-2/c})E + O(N)M$ & 2 & $\Num \cdot N^{1-2/c} + \Num^2\cdot N_s$ \\
			\hline
			%&computation & communi- cation & MPC rounds& & & \\ 
			%\hline
			%Ligero & $O(n\log n)M$ & NA & NA &$O(n)M$ & $O(\sqrt{n})$ & 4\\
			%\hline
			%Bulletproofs & $O(nE+q(n+m)M)$ & $O(2N-1)$ & $O(\log n)$& $O(nE+$ $q(n+m)M)$ & $O(\log n)$ & $O(\log n)$  \\
			%\hline
			%Spartan & & & & & &\\
			%\hline
			%Aurora & & & & & &\\
			%\hline
			%\name2D & $O(n \log n)M$  $+nE)$ & $O(2N-1)$ & 1 &$O((nM +$ $ n^{1-1/c}E))$ & $O(n^{1/c})$ & $O(\log n)$ \\
			%\hline
			%\name3D & $O(nM+nE)$ & $O(2N-1)$& 1 & $O(nM+$ $n^{1-2/c}E)$ & $O(n^{1/c})$ & $O(\log n) $\\
			%\hline
		\end{tabular}
		%\end{threeparttable}
	}
	\caption{Comparison table}\label{tab:SPZK}
	%\pnote{Add trinocchio in the table}
\end{table*}
%.\dnote{should section and subsection titles have caps only for the first letter of the first word, or in the first letter of every word?}
%Ligero \cite{ligero}, a zero-knowledge protocol was given, which attains sublinear proof size, without any trusted setup and assuming that finding the collision for a hash function is difficult. Then in \cite{bulletproofs} zero-knowledge argument of logarithmic size without trusted setup, assuming the discrete log is hard. And then \cite{aurora} gave an IOP based protocol which is better than \cite{ligero} in terms of proof size, but not as good as \cite{bulletproofs}, but hardness assumption is a collision-resistant hash function. \cite{spartan} provides an argument which attains sublinear proof as well as sublinear verification, on the hardness assumption of the discrete log. Verification of \cite{bulletproofs} requires the exponentiation operation of linear order, which is very expensive in terms of practice. Whereas \cite{aurora} does not require exponentiation, but as all its messages are constructed as oracles, it takes too much memory in practice. \cite{spartan} is better in terms of verification. In our work, we provide a zero-knowledge protocol which has a sublinear proof size, same as \cite{spartan} and in verification number of required field operations is linear in size of the circuit but the number of exponentiations is very less, lesser than \cite{spartan}, in practice which can outperform all the above works.

%In \cite{DDS} notion of distributed prover protocol was introduced, where provers jointly provide a signature. Generalization of this work came in 2012 by Marcel et al. \cite{EfficientTZ}, which gave a generic notion of the distributed proof generation system. In this paper, the protocols were restricted to $\Sigma$-protocol.
%%Another contribution of our paper is to define the notion of distributed proof generation for zero-knowledge.
%In our work, we are revisiting the definition and providing an efficient protocol for general arithmetic circuits, which supports this notion. In \cite{bulletproofs}, it was given that the range proof supports the notion of distributed proof generation. In \cite{trinocchio}, a zero-knowledge protocol provided in \cite{pinnochio_PHGR} was converted into a distributed proof generation zero-knowledge protocol, but these protocols are based on the trusted setup. Later in this paper, we will describe how state-of-the-art zero-knowledge protocols lack efficiency while converting into distributed proof generation zero-knowledge protocol. \cite{spartan} requires an MPC of quadratic order in the size of the verification circuit. \cite{ligero, aurora} can not be transformed into $\DPZK$ directly. These works can be made $\DPZK$; in that case, the oracle constructions are needed to be changed such that it must have homomorphic property. This makes $\DPZK$ from \cite{aurora} very inefficient because every message in this protocol is constructed as an oracle. We describe a different approach to obtain $\DPZK$ from the inner product adapted from \cite{bulletproofs}.
%\pnote{Include Marcel Keller's paper \cite{EfficientTZ}}


%Let $P$ be the predicate that is being proved. An aggregation of multiple proofs on the same $P$ produces a proof.  Let us consider the interactive version of the protocol. A proof consists of the messages sent by the prover to the verifier and, additionally in IOPs, the replies to the oracle queries of the verifier.
%where each message from the prover is first identified whether it is homomorphic or not. If 
%each prover can produce an additive share of the proof.  find a way to \textit{aggregate} the proof parts from each prover. identify the prover messages which are homomorphic.
%
%the goals that we set for our construction
%
%the theme that we will have - require every message from prover to verifier be homomorphic.
%
%concrete starting point - ligero --- 