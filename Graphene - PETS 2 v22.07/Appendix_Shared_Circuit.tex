\subsection{Reducing MPC Overhead for Small Shared Circuits}\label{sec:sharedcircuitopti}
Let $C$ be a circuit with $N$ wires (we assume a unique incoming wire for each
input gate), let $\wit=(w_1,\ldots,w_N)$ denote the vector of wire values in a
satisfying assignment according to some ordering on the wires. We consider the
case, when each input wire is ``assigned'' to a unique prover. Let $M$ be the
number of multiplication gates in $C$, and let $A$, $B$, $C$ 
 be $M\times N$ matrices such that $\wit_x=A\wit$, $\wit_y=B\wit$
and $\wit_z=C\wit$ are vectors of {\em left}, {\em right} and {\em out} values
of multiplication gates. For $i\in [m]$, we say that multiplication gate $i$ is
{\em isolated} if $\wit_x[i],\wit_y[i]$ and $\wit_z[i]$ depend on inputs of only
one prover. We call the remaining multiplication gates as {\em shared}. Let
$N_s$ be the number of shared gates. Without loss of generality, we can assume
that the gates $1,\ldots,N_s$ are shared (rows of  $A$, $B$, $C$ can be permuted
suitably). Now an additive sharing of the extended witness may be obtained by
(i) each prover locally computing the wires for isolated gates from their
inputs, while setting the shares of other parties for these wires to be $0$ and
(ii) running a secret sharing MPC on the subcircuit containing shared gates.
Let $p_s=\lceil N_s/ms \rceil$. Then the multiplication MPC $\mathsf{Mult}$ in
distributed quadratic check can be restricted to obtaining shares
$\shr{U_x[i,j,k].U_y[i,j,k]}$ for $i\in [p_s]$, as each prover can canonically 
the shares of other slices of $\ewit_x$ and $\ewit_y$. For slices, which depend
on the provers inputs, the prover computes randomized encoding of the slice, for
other slices (where its share is $0$) it computes a deterministic encoding by
setting the buffer columns to $0$. This gives an MPC of depth 1 on circuit of
size $O(\max(N_s, mn))$.