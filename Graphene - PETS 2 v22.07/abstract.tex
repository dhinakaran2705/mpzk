\begin{abstract}
{Traditional zero-knowledge protocols have been studied and optimized for the setting where a single prover holds the complete witness and tries to convince a verifier about a predicate on the witness, without revealing any additional information to the verifier. In this work, we study the notion of distributed proof generation ($\DPZK$) for arbitrary predicates where the witness is shared among multiple mutually distrusting provers and they want to convince a verifier that their shares together satisfy the predicate. We make the following contributions to the notion of distributed proof generation: (i) we propose a new MPC-style security definition to capture the adversarial settings possible for different collusion models between the provers and the verifier, (ii) we discuss new efficiency parameters for distributed proof generation such as the number of rounds of interaction and the amount of communication among the provers, and (iii) we propose a compiler that realizes distributed proof generation from the zero-knowledge protocols in the Interactive Oracle Proofs (IOP) paradigm. 
%We present two efficient instantiations of $\DPZK$ using the compiler. The first is D-Ligero which is obtained from Ligero (Ames et al. CCS 2017).
Our compiler can be used to obtain $\DPZK$ from arbitrary IOP protocols, but the concrete efficiency overheads are substantial in general.
To this end, we present a new zero-knowledge IOP \name{} which can be compiled into an efficient $\DPZK$ protocol. %which admits $O(N^{1/c})$ proof size. This is achieved by compiling a Ligero-style protocol followed by several optimizations.
%The $\DPZK$ protocol \dpname{}, with $\Num$ provers, admits $O(N^{1/c})$ proof size with a communication complexity $\prcomm$ of $O(\Num \cdot N^{1-2/c}+ \Num^2\cdot \max(N_s,N^{1-2/c}))$, where $N_s$ denotes the number of multiplication gates that takes witnesses of more than one provers,  in the arithmetic circuit representing the prover's computation in \name{}. 
The $\DPZK$ protocol \dpname{}, with $\Num$ provers, admits $O(N^{1/c})$ proof size with a communication complexity of $O(\Num \cdot N^{1-2/c}+ \Num^2\cdot N^{1-2/c})$, where $N$ is the number of gates in the arithmetic circuit representing the predicate where the number of wires that depends on 2 or more parties is less. 
Significantly, the distributed proof generation in \dpname{} requires only a single round of interaction among the provers. 
\dpname{} compares favourably with respect to the $\DPZK$ protocols obtained from the state-of-art zero-knowledge protocols, even those not modelled as IOPs. }
\end{abstract}
%=======
%Traditional zero-knowledge protocols have been studied and optimized for the setting where a single prover holds the complete witness and trying to convince a verifier about a predicate on the witness, without revealing any additional information to the verifier. This work initiates the study of distributed proof generation where the witness is shared among multiple mutually distrusting provers and they want to convince a verifier that their shares together satisfies the predicate. We start by defining a new MPC-style security definition to capture the possible adversarial settings, and proposing new efficiency parameters for distributed proof generation on the number of rounds $\prrounds$ and the amount of communication $\prcomm$ among the provers. We then propose a new zero-knowledge protocol \name{} in the IPCP paradigm which admits $O(N^{1/c})$ proof size. The provers in \name{} have a total communication complexity $O(\Num \cdot N^{1-2/c}+ N)$ when the proof is generated distributively among $\Num$ provers.
%\end{comment}
%	Zero knowledge is an intriguing area of research from the prospect of theory as well as application, where a party, prover \textit{P}, tries to convince verifier \textit{V} that a statement $x$( instance of an NP language) is correct using a witness $w$ such that $M(x,w)=1$. Let $C$ be the circuit representation of $M$ and $|C|$ is the size of $C$ i.e. the no. of gates in $C$. The work of Ligero in this problem proposed a solution where proof size is $O(|C|^{1/2})$. They solve the problem by converting a NP language to it's corresponding circuit satisfiability problem. $w\in \mathbb{F}^{ml}$ be the extended witness which is the secret input to the prover. Prover \textit{P} encodes $w$ to a matrix $U$ of size $m\times n$, where $m, n, l = O(\sqrt{|C|})$ and $n>l$. In Ligero's protocol first check is for to ensure that the matrix $U$ is a correct encoding, that is done by testing interleaved. Then they check that all the outcomes of all the gates( addition and multiplication) are correct or not, and to do that they introduce two checks, linear constraint and quadratic constraint. We are using the same approach of Ligero, but our encoding is different. Instead of matrix we will encode $w$ to a box( matrix of 3 dimensions), to keep the familiarity of notation we will consider $w\in \mathbb{F}^{pml}$ and the size of the box $U$ is $p\times m\times n$, where $p, m, n, l = O(\sqrt[3]{|C|})$ and $n>l$. To encode $w$ we are going to construct $pm$ many polynomials(univariate) such that each of them has degree at most $s$, where $s$ is some suitable number between $l$ and $n$. 
%\\

%\mycomment{Adding a version below.}
%The work of Ligero[citation] proposed ... In Ligero, a witness $x$ is encoded into a matrix $U$ of dimension $m \times n$, where $m$ and $n$ denote - and - respectively. Our idea is similar to theirs apart from the witness being encoded to a {\em box}. A box is a 3-dimensional matrix ($m \times n \times p$) where --(add technical details of box here). This ensures linearity of (what?) along both rows and columns. We hope that the aforementioned encoding along with homomorphic commitments lead to a ZK proof, whose size will be $O(|C|^{\frac{1}{3}})$.

