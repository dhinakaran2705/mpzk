%------------------------------
A zero knowledge protocol tries to convince a verifier about the truth of a
statement, without revealing any additional information. %Availability of
%practically efficient zero knowledge protocols is enabling their real-life application. 
These include proving the correctness of
transactions in cryptocurrencies such as \cite{zerocash} or validating sensitive web browser data reported during
telemetry \cite{prio, MozillaPrio}. Still in the dream of a decentralized world, there are
multiple co-opetitive entities i.e.,
collaborating but mutually distrusting entities interacting with each other to obtain insights
and maximize their goals. We envisage applications of zero knowledge
proofs to enable these mutually
distrusting entities to prove a claim on their joint data. In this setting, the 
traditional zero knowledge protocols are restrictive since they require
a single prover to have the entire witness needed to generate the proof. 

%\pdnote{do we want to motivate it in a way such that the honest prover setting we are looking at will suffice for the listed applications? }
In this work, we study the general setting of \textit{distributed prover
zero-knowledge protocols} ($\DPZK$) where multiple co-opetitive entities each possessing their own secret
data want to prove to a verifier that their secret data together satisfies a
predicate of common interest. This is to be done without revealing any information
about their sensitive data to each other or to the verifier. We illustrate
through some real-world applications: %potential
%------
\begin{itemize}
	%-----
	\item[-] A simple but pertinent scenario is of a \textit{joint loan application} by an association of companies from a particular industry. The loan issuer has a set of financial requirements that it wants the association to satisfy, but there is no single trusted entity to act as the prover whom all the companies are willing to provide their sensitive business information with.
	%-----
	\item[-] In cryptocurrency settings \cite{bitcoin, ethereum, zerocash}, this would enable a \textit{multi-wallet transaction} where the wallets are held by different parties or a \textit{proof of joint stake} where the stake is held by different parties. These in turn enable secure collaboration applications by design on a blockchain network. % use it for joint auction-
	%-----
	\item[-] In trade logistics business networks \cite{scbn, e2open, tradelens}, a major reason for businesses to enter these networks is to benefit from cross-industry statistics. Publishing these statistics in a publicly verifiable manner without having a single trusted entity is another embodiment of this setting.
	%-----
%	\item[-] In privacy-preserving server outsourced computation, it is assumed that the number of corrupted parties can be at most $t$, where $t<n/2$, for the honest majority setting and $t<n-1$ for the dishonest majority setting and $n$ is the number of servers. The protocols in these settings do not consider the scenario where the corruption exceeds $t$. In such a case, neither correctness nor privacy is ensured. So, suppose a user gets the output from the servers. In that case, there is no way to verify the correctness of the output, especially if the function is evaluated on inputs of multiple parties. In such a scenario, protocols can be enhanced by attaching a proof produced by all the servers stating the correctness of the computation. Here the statement would be the evaluated function, and the transcripts of each server become the share of the global witness. If the proof is valid, then the user is ensured that the output is indeed correct.
	%-----
	\item[-] More generally, in privacy-preserving server outsourced computations on sensitive data from data owner(s), it is typically assumed that the number of corrupt servers $t$ is $<n/2$ in the honest majority setting and $<n-1$ in the dishonest majority setting with $n$ being the number of servers. The protocols in these settings do not consider the scenario where the number of corrupt servers exceeds $t$, with neither correctness nor privacy ensured in that case. Hence, when a user gets the output from the servers there is no way to verify the correctness of the output, especially if the function is evaluated on inputs of multiple data owners. In such a scenario, protocols can be enhanced by attaching a proof produced by all the servers stating the correctness of the computation. Here the statement would be the evaluated function, and the transcripts of each server become the share of the global witness. If the proof is valid, then the user is ensured that the output is indeed correct.
\end{itemize}

Formally, we have multiple provers $\prover_1, \ldots, \prover_{\Num}$
respectively possessing witnesses $\wit_1, \ldots, \wit_{\Num}$. For a predicate
$C$ the provers wish to prove to a verifier that $C(\wit_1,\cdots,\wit_{\Num}) = 1$.  
A natural solution for distributed
proof generation is to start with the prover algorithm of a single prover
protocol and run this algorithm between multiple provers using multi-party
computation (MPC). This generic construction was discussed by Pedersen~\cite{Ped92}. 
However, the generic construction is likely to be concretely
inefficient, as implementing high level abstractions such as fields, groups and
generic algorithms like $\fft$ used in modern zero-knowledge proof protocols to compute the next prover message as an
arithmetic circuit is challenging, and would lead to large circuit
representations. Running multiparty computation over such circuits with several
parties would incur more overheads.
%\nnote{Any numbers to support the above?}
%Each round of proof generation is implemented by an MPC among the provers to
%generate the message to be sent to the verifier.
%The generic construction is optimal in proof size and verifier complexity in that it retains these complexities from the single-prover version irrespective of the number of provers. But, the complexity of proof generation suffers when the single prover protocol is not constructed with distributed proof generation in mind. Even an optimal $\round$-round proof generation algorithm with $O(N)$ multiplicative complexity of the prover's computation might result in a distributed proof generation algorithm with $\Omega(R \cdot N)$ communication among the provers.
To get around the inefficiency of a generic construction, some prior works have
proposed efficient distributed proof generation in restricted settings. These
include simple predicates involved in threshold signatures ~\cite{DDS}, some
sigma protocols ~\cite{EfficientTZ}, combined range proofs ~\cite{bulletproofs},
or under trusted setup with a weaker adversarial model of majority of parties
being honest ~\cite{trinocchio}. The goal of this paper is to address the
aforementioned shortcomings.

\paragraph*{Our contributions}
\begin{itemize}
%----------
\item We provide a formal definition of distributed prover zero-knowledge (DPZK) in the real-ideal world paradigm. Furthermore, we identify and motivate relevant efficiency parameters to measure the efficiency of a DPZK protocol. %In the next section, we elaborate more on the new efficiency parameters.
%----------
\item We present a compiler that takes any IOP-based zero-knowledge protocol and converts it to an ``MPC-friendly'' zero-knowledge that supports DPZK.
%----------
\item Finally, we produce a protocol, \name, obtained via compiling a Ligero-style proof system using our compiler. It is optimized further to attain comparable efficiency with state-of-the-art protocols while achieving DPZK.
%----------
The discussions will focus on {\em public-coin honest-verifier} protocols with a transparent setup, which is essentially the setting for real-world applications we envisage: 1) protocols with such verifiers can be converted to succinct non-interactive arguments (zk-SNARGs) via the Fiat-Shamir and like transforms \cite{FS86, BCS16}, and 2) generating the parameters of the protocol should ideally {\em not} involve a trusted third party. In this work, we provide an honest verifier zero-knowledge protocol motivated by applications, however using known transformations, such as~\cite{hubavcek2018efficiency}, we can obtain a zero-knowledge protocol while preserving the round complexity and the efficiency of both prover and verifier.
%----------
\end{itemize}
%\pnote{old version below}
%We design zero knowledge protocols which enable efficient distributed proof generation, informally an ``MPC-friendly'' proof generation, while supporting \textit{arbitrary} predicates. %The proof generation algorithm is usually measured for efficiency in terms of number of group operations and number of rounds of interaction with the verifier.
%Our work first identifies and motivates relevant efficiency parameters that a
%zero knowledge protocol should meet to admit efficient distributed proof generation. 
%We then construct a zero knowledge 
%protocol which is the state-of-art in these parameters. The discussions will
%focus on {\em public-coin honest-verifier} protocols with a transparent setup,
%which is essentially the setting for real-world applications we envisage: 1) protocols with
%such verifiers can be converted to succinct non-interactive  arguments
%(zk-SNARGs)  via the Fiat-Shamir and like transforms \cite{FS86, BCS16}, and  2)
%generating the parameters of the protocol should ideally {\em not} involve a
%trusted third party. The next few subsections elaborate on the new efficiency
%parameters, formal definition and our new construction for $\DPZK$ that leverages 
%a new zero-knowledge argument constructed in this paper, which may be of
%independent interest. 


