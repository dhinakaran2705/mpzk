We now describe distributed protocol to produce a $\name$ proof for a
statement, when the witness is shared between $\Num$ provers $\prover_1,\ldots,\prover_{\Num}$. For $\xi\in [\Num]$,
let $\shr{\wit}$ denote the prover $\distprover$'s share of the witness $\wit$.
We assume that the sharing is additive, i.e, $\sum_{\xi\in [\Num]} \shr{\wit} =
\wit$. Recall from Section~\ref{sec:security model}, that there is an algorithm $\Ag$ 
which aggregates the messages received
from provers $\prover_1,\ldots,\prover_\Num$ and constructs the message to be
sent to  $\verifier$. We assume one of the provers executes $\Ag$.
%$\verifier$'s  messages and messages used by $\Ag$ are assumed to be available on an authenticated broadcast channel\pnote{messages to $\Ag$ does not require broadcast channel}. 
We specify the algorithm $\Ag$ implicitly by describing the construction
of message to $\verifier$ from the provers' messages for each round.
We first discuss a protocol secure against semi-honest provers and then 
briefly discuss how to ensure the privacy of the honest provers when the corrupt 
provers are malicious (barring the one who runs $\Ag$). 

\subsection{Distributed Oracle Setup:} In distributed setting, each prover
$\distprover$ encodes her share $\shr{\wit}$ as $\shr{\ewit}=\enc(\shr{\wit})$
and computes the commitment $\shr{\comoracle}=\ocom(\shr{\ewit})$. The provers
then share $\shr{\comoracle}$ with the aggregator $\Ag$ which sets the oracle
$\pi$ as $\pi := \combine(\shr{\comoracle})$, where $\combine$ simply multiplies the corresponding commitments. The homomorphism and the fact that the witnesses are additively shared ensure that $\pi$ contains commitment to the witness.
%------------------------------------------------------------------

\subsection{Distributed Linear Check:} The messages sent by the prover to the
verifier in the linear check protocol include:
\begin{itemize}
	\item Commitments $\tilde{\bm{c}} = (\tilde{c}_1,\ldots,\tilde{c}_\ell)$ to the matrix $\tilde{\ewit}=
	\sum_{i\in [p]}\rho_i\ewit[i,\cdot,\cdot]$ for verifier's challenge $\rho\sample
	\FF^p$.
	\item Commitments $c_0,\ldots,c_{s+\ell-1}$ where $c_0$ is a commitment to random
	vector $P_0\in \FF^{2m-1}$ satisfying\\ $\innp{1^m||0^{m-1}}{P_0}=0$ and
	$c_1,\ldots,c_{s+\ell-1}$ are commitments to the $h\times n$ matrix $P$.
	\item The vectors $X_u=\ewit[\cdot,j_u,k_u]$ for $u\in [t]$, for verifier's
	query $Q=\{(j_u,k_u):u\in [t]\}$.
\end{itemize}
We see that given verifier's challenges, each of the messages is linear function
of the encoding (which itself is a linear function of the witness). Hence, the
provers compute the respective messages on their shares, which can be trivially
combined by $\Ag$. In addition to above messages, we also want $\Ag$ to receive
witnesses to the inner-product protocols namely, the vectors
$\overline{P}[\cdot,k_u]$, $W_u=\sum_{i\in [p]}\delta_i\ewit[i,\cdot,k_u]$ for
$u\in [t]$, $z=\beta P_0+\overline{P}\varphi$ and the randomness used to commit
the vectors. Each of these can again be obtained by combining the respective shares.
Note that the share $\shr{z}$ leaks $r^TA\shr{\wit}=\innp{1^m||0^{m-1}}{\shr{z}}$, which
is non-trivial knowledge about an individual witness share. Thus provers use a
random share $\shr{0^{2m-1}}$ to randomize their share of $z$, and send
$\shr{z}=\beta\shr{P_0}+\shr{\overline{P}}\varphi+\shr{0^{2m-1}}$. 
We provide the complete distributed linear protocol in Figure
\ref{fig:distlincheck}.% in Appendix ~\ref{app:protocolboxes}. 
%---------------------------------------------------------
%--------------------------------------------------------


\subsection{Distributed Quadratic Check:} Here the distributed variant requires an additional interaction among the provers. Recall that in response to $\verifier$'s challenge $r\in \FF^p$, the provers need to compute $P$ as:
%--------------------------------------------------------
\begin{align*}
	P[j,k] & =\sum_{i\in [p]} r_i\big(Q_x^i(\alpha_j,\eta_k).Q_y^i(\alpha_j,\eta_k)-Q_z^i(\alpha_j,\eta_k)\big)\\ 
	& = \sum_{i\in [p]}r_i(\ewit_x[i,j,k].\ewit_y[i,j,k] - \ewit_z[i,j,k])
\end{align*}    
%--------------
where $\ewit_x$, $\ewit_y$ and $\ewit_z$ are the encodings of the witness
vectors $\wit_x$, $\wit_y$ and $\wit_z$ respectively. 
Since the matrix $P$ above is completely determined by it's first $2m-1$ rows
and first $2\ell-1$ columns, the provers need to obtain the shares
$\shr{\ewit_x[i,j,k].\ewit_y[i,j,k]}$ for $i\in [p],j\in [2m-1],k\in [2\ell-1]$.
From the shares of witness $\shr{\wit_x},\shr{\wit_y}$ and $\shr{\wit_z}$ the
provers can locally compute shares $\shr{\ewit_x},\shr{\ewit_y},\shr{\ewit_z}$
of $\ewit_x,\ewit_y$ and $\ewit_z$. Now, the provers call $\FMult$ on inputs $\shr{\ewit_x},\; \shr{\ewit_y}$ and obtain $\shr{\ewit_x[i,j,k].\ewit_y[i,j,k]}$.
We can instantiate $\FMult$ with any state-of-the-art dishonest majority protocol for arithmetic circuits.
%can perform an MPC with
This requires evaluation of $p.(2m-1).(2\ell-1)\approx 4N$ multiplication gates, and depth 1, to obtain the
shares $\shr{\ewit_x[i,j,k].\ewit_y[i,j,k]}$ for $i\in [p],j\in [2m-1],k\in [2\ell-1]$.
%Concretely, we assume an MPC $\mathsf{Mult}$ with following input/output for
%a prover $\distprover$:     
%%\begin{align*}    
%$\shr{\ewit_x[i,j,k].\ewit_y[i,j,k]}\leftarrow
%\mathsf{Mult}(\shr{\ewit_x[i,j,k]},\shr{\ewit_y[i,j,k]})$    
%%\end{align*}    
Thereafter, each prover obtains a share of matrix $P$, and the remaining protocol proceeds as
the distributed linear check protocol. 
The complete protocol for
distributed quadratic check appears in Figure \ref{fig:distquadcheck} %in Appendix \ref{app:protocolboxes}. 
%In Appendix ~\ref{sec:sharedcircuitopti}
, we discuss how to optimize MPC overhead when the size of the shared circuit (see Section~\ref{sec:efficiencyparams}) is small.
%--------------------------------------------------------

%--------------------------------------------------------
We show that the messages received by the aggregator in both the distributed protocols can be efficiently simulated, independent of
their views in the preceding MPC protocols, provided the parties follow the
protocol and jointly possess a valid witness. We present the proofs of (i) Soundness with Witness Extraction (SoWE), (ii) Zero-Knowledge (Zk), and (iii) Witness-Hiding with Collusion (WHwC), which is the same as the plain Witness Hiding in our setting, in Appendix~\ref{app:dp_grapehene_securityproofs}. 
%---------------------------------------------------------------